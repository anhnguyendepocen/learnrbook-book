\chapter{Introduction}

\dictum[Ursula K. Le Guin, \textit{Buffalo Gals and other Animal Presences}, 1985]{Although I whined and tried to hide under the rug, my inexorable publisher demanded an introduction\ldots}\vskip2ex

\section{R's built-in help}

To access help pages through the command prompt we use function \texttt{help()} or a question mark. Every object exported by an R package (functions, methods, classes, data) is documented. Sometimes a single help page documents several R objects. Usually at the end of the help pages some us examples are given.

\begin{knitrout}
\definecolor{shadecolor}{rgb}{0.969, 0.969, 0.969}\color{fgcolor}\begin{kframe}
\begin{alltt}
\hlkwd{help}\hlstd{(}\hlstr{"sum"}\hlstd{)}
\hlopt{?}\hlstd{sum}
\end{alltt}
\end{kframe}
\end{knitrout}

\section{Obtaining help from on-line forums}

Netiquette

StackOverflow

\section{Online webbinars and courses}

\section{R, editors and IDEs}

\section{Packages and repositories}

\section{Other tools}
