% koma_env.tex
\documentclass[a4paper]{scrlttr2}
%\usepackage{amsmath,amssymb,amsthm}
\RequirePackage{unicode-math}

\RequirePackage{fontspec}
\setmainfont{Lucida Bright OT}
\setsansfont{Lucida Sans OT}
\setmonofont{Lucida Console DK}
\setmathfont{Lucida Bright Math OT}
\linespread{1.1} % increase line spacing as we use in-line math and code
\usepackage{url}


\setkomavar{fromname}{Pedro J. Aphalo}
\setkomavar{fromaddress}{Dep.\ Biosciences\\ P.O. Box 65 (Biocenter 3, Viikkinkaari 1)\\ 00014 University of Helsinki, Finland}
\setkomavar{fromphone}{+358 50 3721 504}

\begin{document}

\begin{letter}{Rob Calver\\Publisher --- Mathematics and Statistics\\ Chapman \& Hall/CRC\\ Taylor \& Francis Group}

\KOMAoptions{fromphone=true,fromfax=false}
\setkomavar{subject}{Book manuscript: \emph{Learn R ...as you learnt your mother tongue}}
\opening{Dear Rob,}

It is now a little over two years since we first met at UseR2015! and discussed about the possible publication of a different book manuscript, ``Using R for photobiology'', of which I am the lead author. That project is going at a slower pace than the one we recently discussed about through e-mail. Being the sole author of \emph{Learn R ...as you learnt your mother tongue} and in addition aiming at getting this second manuscript in good shape for use in a course I will teach starting next month, has helped with making progress. So, I am here submitting for your consideration the manuscript for the book \emph{Learn R ...as you learnt your mother tongue} of which I am the sole author. A brief justification of the need for a new book on R follows.

\textbf{Aim:} To teach how to use the R environment for data analysis. Not just the language, but the \emph{modus operandi}. Expected outcome: allow the reader/student to become confident and reasonably independent in the use of R when faced with new challenges. They will learn their way around in the R "ecosystem".

\textbf{Readership:} students and practitioners of research in biology, ecology, engineering, who have been exposed to statistics and data analysis but not to R as a platform for a data analysis workflow. Readers may use or have used R or some other software for ANOVA, t-test and linear model fitting, but not for scripting or complex statistical analyses. They may have produced some simple plots but not publication ready or complex exploratory ones using a scripting language. A few may be familiar with scripting in R but not familiar with the `tidyverse' and/or the nuances of plotting with R. However, the main characteristic of the envisioned readers is that they are not independent in their use of R: they get stuck easily with coding problems or when needing to find alternative or new approaches or when needing to automate a data analysis workflow. In other words, \emph{they do not know their way around in R}, and frequently they \emph{lack self-confidence} in relation to their use of R.

\textbf{Competing book titles:} I do not think there is currently available any book targeted at exactly the same readership. There are many books about R available, an some are no doubt also of interest to the same potential readers. However, I think my proposed book does not directly compete with them, as the overlap is partial, and the approach different. One can also expect within a broad readership, some readers to prefer a differently paced or a different approach to learning than those of the currently available books.

I have targeted a readership of R users who are not professional data analysts or statisticians, but who would greatly benefit by using a modern approach to data analysis in their work. There is a population of researchers and advanced students who need to comply with increasingly strict requirements of reproducibilty for academic publication and reporting and have not received training on how to use the available tools to achieve this. By targeting a different readership than most existing books, I expect my book to find its place in an otherwise crowded market. In addition by using a slower paced approach than some other books and frequently encouraging readers to explore the R environment, I hope to help train future researchers and practitioners in the life sciences and other experimental sciences to move away from a recipe-based approach to data analysis and be freer to flexibly explore and analyse the data they collect so as to efficiently and reliably extract information relevant to the questions they have under study.

I will not list other R books here, as it is difficult to be comprehensive given the number of titles currently available. In the first section of most chapters, where I present the aims of the chapter, I mention existing books that can complement the material presented. In a way, these same books could be thought as competitors.

In conclusion, I am not aware of any other R book that would fully overlap with the contents of the proposed book. More importantly, no other book uses the same hands-on approach or stresses as much as I do, the aspect of developing independence and confidence in the use of R. The need to stress these aspects stems from targeting a readership for which R and data analysis are tools from outwith their own everyday work tasks, but which are becoming central to their success as professionals or students.

\textbf{\emph{Caveats}:} I do not yet consider the manuscript ready, and a very few sections are signaled as incomplete. Neither the text nor the examples are yet revised (''polished'') to my complete satisfaction. The other question is that a few examples in the book are very close to examples in the documentation of the packages described. For copyright reasons, I expect these code examples will need to be edited or replaced before commercial publication of the book.

I am keeping the book source files at Bitbucket: \url{https://bitbucket.org/aphalo/learnr-book}. The book manuscript is written using \LaTeX\ and knitr. I am developing a companion R package for the book, available through CRAN. An earlier version of the manuscript is available at \url{https://leanpub.com/learnr} as a rendered PDF file.

Please, let me know if a more formal book proposal is needed. I would be happy to write it if you provide further instructions or a form.

I hope you and the series editors will find my manuscript of interest. I look forward to your reply.

\closing{Sincerely,}

\encl{Book manuscript: \texttt{learnr-012.pdf}}

\end{letter}

\end{document} 