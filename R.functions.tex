\chapter{R built-in functions}


\section{Loading data}

To start with we need some data. Here we use \texttt{cars}, a data set included in base R.

\begin{knitrout}
\definecolor{shadecolor}{rgb}{0.969, 0.969, 0.969}\color{fgcolor}\begin{kframe}
\begin{alltt}
\hlkwd{data}\hlstd{(cars)}
\end{alltt}
\end{kframe}
\end{knitrout}

\section{Looking at the data}

\begin{knitrout}
\definecolor{shadecolor}{rgb}{0.969, 0.969, 0.969}\color{fgcolor}\begin{kframe}
\begin{alltt}
\hlkwd{names}\hlstd{(cars)}
\end{alltt}
\begin{verbatim}
## [1] "speed" "dist"
\end{verbatim}
\begin{alltt}
\hlkwd{head}\hlstd{(cars)}
\end{alltt}
\begin{verbatim}
##   speed dist
## 1     4    2
## 2     4   10
## 3     7    4
## 4     7   22
## 5     8   16
## 6     9   10
\end{verbatim}
\begin{alltt}
\hlkwd{tail}\hlstd{(cars)}
\end{alltt}
\begin{verbatim}
##    speed dist
## 45    23   54
## 46    24   70
## 47    24   92
## 48    24   93
## 49    24  120
## 50    25   85
\end{verbatim}
\begin{alltt}
\hlkwd{str}\hlstd{(cars)}
\end{alltt}
\begin{verbatim}
## 'data.frame':	50 obs. of  2 variables:
##  $ speed: num  4 4 7 7 8 9 10 10 10 11 ...
##  $ dist : num  2 10 4 22 16 10 18 26 34 17 ...
\end{verbatim}
\end{kframe}
\end{knitrout}

The \texttt{cars} data are stored as a data frame. Data frames consist in columns of equal length. The different columns can be different types (e.g.\ numeric and character). With \texttt{data()} we load it; with \texttt{names()} we obtain the names of the variables or columns. With head with can see the top several lines, and with tail the lines at the end. In general \texttt{data()} is used load R objects saved in a file format used by R. Text files con be read with functions \texttt{scan()}, \texttt{read.table()}, \texttt{read.csv()} and their variants.

It is also possible to `import' data saved in files of \textit{foreign} formats, defined by other programs. Packages such as 'foreign', 'readr', 'readxl', 'RNetCDF', 'jsonlite', etc.\ allow importing data from other statistic and data analysis applications and from standard data exchange formats. It is also good to keep in mind that in R urls are accepted as arguments to the \texttt{file} argument.

\section{Plotting}

The built-in generic function \texttt{plot} can be used to plot data. It is a generic function, that has suitable methods for different kinds of objects. In this section we only very briefly demonstrate the use of the most common base R's graphics functions.

\begin{knitrout}
\definecolor{shadecolor}{rgb}{0.969, 0.969, 0.969}\color{fgcolor}\begin{kframe}
\begin{alltt}
\hlkwd{plot}\hlstd{(dist} \hlopt{~} \hlstd{speed,} \hlkwc{data}\hlstd{=cars)}
\end{alltt}
\end{kframe}
\includegraphics[width=\maxwidth]{figure/plot-2-1} 

\end{knitrout}

\section{Fitting linear models}

One important thing to remember is that model `formulas' are used in different contexts: plotting, fitting of models, and tests like $t$-test. The basic syntax is rather consistently followed, although there are some exceptions.

\subsection{Regression}

The R function \texttt{lm} is used next to fit linear models. If the explanatory variable is continuous, the fit is a regression. In the example below, \texttt{speed} is a numeric variable (floating point in this case). In the ANOVA table calculated for the model fit, in this case a linear regression, we can see that the term for \texttt{speed} has only one degree of freedom (df) for the denominator.

We first fit the model and save the output as \texttt{fm1} (A name I invented to remind myself that this is the first fitted-model in this chapter.

\begin{knitrout}
\definecolor{shadecolor}{rgb}{0.969, 0.969, 0.969}\color{fgcolor}\begin{kframe}
\begin{alltt}
\hlstd{fm1} \hlkwb{<-} \hlkwd{lm}\hlstd{(dist} \hlopt{~} \hlstd{speed,} \hlkwc{data}\hlstd{=cars)}
\end{alltt}
\end{kframe}
\end{knitrout}

The next step is diagnosis of the fit. Are assumptions of the linear model procedure used reasonably fulfilled? In R it is most common to use plots to this end. We show here only one of the four plots normally produced. This quantile vs.\ quantile plot allows to assess how much the residuals deviate from being normally distributed.

\begin{knitrout}
\definecolor{shadecolor}{rgb}{0.969, 0.969, 0.969}\color{fgcolor}\begin{kframe}
\begin{alltt}
\hlkwd{plot}\hlstd{(fm1,} \hlkwc{which} \hlstd{=} \hlnum{2}\hlstd{)}
\end{alltt}
\end{kframe}
\includegraphics[width=\maxwidth]{figure/models-1a-1} 

\end{knitrout}

In the case of a regression, calling \texttt{summary()} with the fitted model object as argument is most useful as it provides a table of coefficient estimates and their errors. \texttt{anova()} applied to the same fitted object, returns the ANOVA table.

\begin{knitrout}
\definecolor{shadecolor}{rgb}{0.969, 0.969, 0.969}\color{fgcolor}\begin{kframe}
\begin{alltt}
\hlkwd{summary}\hlstd{(fm1)} \hlcom{# we inspect the results from the fit}
\end{alltt}
\begin{verbatim}
## 
## Call:
## lm(formula = dist ~ speed, data = cars)
## 
## Residuals:
##     Min      1Q  Median      3Q     Max 
## -29.069  -9.525  -2.272   9.215  43.201 
## 
## Coefficients:
##             Estimate Std. Error t value Pr(>|t|)    
## (Intercept) -17.5791     6.7584  -2.601   0.0123 *  
## speed         3.9324     0.4155   9.464 1.49e-12 ***
## ---
## Signif. codes:  0 '***' 0.001 '**' 0.01 '*' 0.05 '.' 0.1 ' ' 1
## 
## Residual standard error: 15.38 on 48 degrees of freedom
## Multiple R-squared:  0.6511,	Adjusted R-squared:  0.6438 
## F-statistic: 89.57 on 1 and 48 DF,  p-value: 1.49e-12
\end{verbatim}
\begin{alltt}
\hlkwd{anova}\hlstd{(fm1)} \hlcom{# we calculate an ANOVA}
\end{alltt}
\begin{verbatim}
## Analysis of Variance Table
## 
## Response: dist
##           Df Sum Sq Mean Sq F value   Pr(>F)    
## speed      1  21186 21185.5  89.567 1.49e-12 ***
## Residuals 48  11354   236.5                     
## ---
## Signif. codes:  0 '***' 0.001 '**' 0.01 '*' 0.05 '.' 0.1 ' ' 1
\end{verbatim}
\end{kframe}
\end{knitrout}

Let's look at each argument separately: \texttt{dist ~ speed} is the specification of the model to be fitted. The intercept is always implicitly included. To `remove' this implicit intercept from the earlier model we can use \texttt{dist ~ speed - 1}. In what follows we fit a straight line through the origin ($x = 0$, $y = 0$).

\begin{knitrout}
\definecolor{shadecolor}{rgb}{0.969, 0.969, 0.969}\color{fgcolor}\begin{kframe}
\begin{alltt}
\hlstd{fm2} \hlkwb{<-} \hlkwd{lm}\hlstd{(dist} \hlopt{~} \hlstd{speed} \hlopt{-} \hlnum{1}\hlstd{,} \hlkwc{data}\hlstd{=cars)}
\hlkwd{plot}\hlstd{(fm2,} \hlkwc{which} \hlstd{=} \hlnum{2}\hlstd{)}
\end{alltt}
\end{kframe}
\includegraphics[width=\maxwidth]{figure/models-2-1} 
\begin{kframe}\begin{alltt}
\hlkwd{summary}\hlstd{(fm2)}
\end{alltt}
\begin{verbatim}
## 
## Call:
## lm(formula = dist ~ speed - 1, data = cars)
## 
## Residuals:
##     Min      1Q  Median      3Q     Max 
## -26.183 -12.637  -5.455   4.590  50.181 
## 
## Coefficients:
##       Estimate Std. Error t value Pr(>|t|)    
## speed   2.9091     0.1414   20.58   <2e-16 ***
## ---
## Signif. codes:  0 '***' 0.001 '**' 0.01 '*' 0.05 '.' 0.1 ' ' 1
## 
## Residual standard error: 16.26 on 49 degrees of freedom
## Multiple R-squared:  0.8963,	Adjusted R-squared:  0.8942 
## F-statistic: 423.5 on 1 and 49 DF,  p-value: < 2.2e-16
\end{verbatim}
\begin{alltt}
\hlkwd{anova}\hlstd{(fm2)}
\end{alltt}
\begin{verbatim}
## Analysis of Variance Table
## 
## Response: dist
##           Df Sum Sq Mean Sq F value    Pr(>F)    
## speed      1 111949  111949  423.47 < 2.2e-16 ***
## Residuals 49  12954     264                      
## ---
## Signif. codes:  0 '***' 0.001 '**' 0.01 '*' 0.05 '.' 0.1 ' ' 1
\end{verbatim}
\end{kframe}
\end{knitrout}

We now we fit a second degree polynomial.

\begin{knitrout}
\definecolor{shadecolor}{rgb}{0.969, 0.969, 0.969}\color{fgcolor}\begin{kframe}
\begin{alltt}
\hlstd{fm3} \hlkwb{<-} \hlkwd{lm}\hlstd{(dist} \hlopt{~} \hlstd{speed} \hlopt{+} \hlkwd{I}\hlstd{(speed}\hlopt{^}\hlnum{2}\hlstd{),} \hlkwc{data}\hlstd{=cars)} \hlcom{# we fit a model, and then save the result}
\hlkwd{plot}\hlstd{(fm3,} \hlkwc{which} \hlstd{=} \hlnum{3}\hlstd{)} \hlcom{# we produce diagnosis plots}
\end{alltt}
\end{kframe}
\includegraphics[width=\maxwidth]{figure/models-3-1} 
\begin{kframe}\begin{alltt}
\hlkwd{summary}\hlstd{(fm3)} \hlcom{# we inspect the results from the fit}
\end{alltt}
\begin{verbatim}
## 
## Call:
## lm(formula = dist ~ speed + I(speed^2), data = cars)
## 
## Residuals:
##     Min      1Q  Median      3Q     Max 
## -28.720  -9.184  -3.188   4.628  45.152 
## 
## Coefficients:
##             Estimate Std. Error t value Pr(>|t|)
## (Intercept)  2.47014   14.81716   0.167    0.868
## speed        0.91329    2.03422   0.449    0.656
## I(speed^2)   0.09996    0.06597   1.515    0.136
## 
## Residual standard error: 15.18 on 47 degrees of freedom
## Multiple R-squared:  0.6673,	Adjusted R-squared:  0.6532 
## F-statistic: 47.14 on 2 and 47 DF,  p-value: 5.852e-12
\end{verbatim}
\begin{alltt}
\hlkwd{anova}\hlstd{(fm3)} \hlcom{# we calculate an ANOVA}
\end{alltt}
\begin{verbatim}
## Analysis of Variance Table
## 
## Response: dist
##            Df  Sum Sq Mean Sq F value    Pr(>F)    
## speed       1 21185.5 21185.5  91.986 1.211e-12 ***
## I(speed^2)  1   528.8   528.8   2.296    0.1364    
## Residuals  47 10824.7   230.3                      
## ---
## Signif. codes:  0 '***' 0.001 '**' 0.01 '*' 0.05 '.' 0.1 ' ' 1
\end{verbatim}
\end{kframe}
\end{knitrout}

The ``same'' fit using an orthogonal polynomial. Higher degrees can be obtained by supplying as second argument to \texttt{poly()} a different positive integer value.

\begin{knitrout}
\definecolor{shadecolor}{rgb}{0.969, 0.969, 0.969}\color{fgcolor}\begin{kframe}
\begin{alltt}
\hlstd{fm3a} \hlkwb{<-} \hlkwd{lm}\hlstd{(dist} \hlopt{~} \hlkwd{poly}\hlstd{(speed,} \hlnum{2}\hlstd{),} \hlkwc{data}\hlstd{=cars)} \hlcom{# we fit a model, and then save the result}
\hlkwd{plot}\hlstd{(fm3a,} \hlkwc{which} \hlstd{=} \hlnum{3}\hlstd{)} \hlcom{# we produce diagnosis plots}
\end{alltt}
\end{kframe}
\includegraphics[width=\maxwidth]{figure/models-3a-1} 
\begin{kframe}\begin{alltt}
\hlkwd{summary}\hlstd{(fm3a)} \hlcom{# we inspect the results from the fit}
\end{alltt}
\begin{verbatim}
## 
## Call:
## lm(formula = dist ~ poly(speed, 2), data = cars)
## 
## Residuals:
##     Min      1Q  Median      3Q     Max 
## -28.720  -9.184  -3.188   4.628  45.152 
## 
## Coefficients:
##                 Estimate Std. Error t value Pr(>|t|)    
## (Intercept)       42.980      2.146  20.026  < 2e-16 ***
## poly(speed, 2)1  145.552     15.176   9.591 1.21e-12 ***
## poly(speed, 2)2   22.996     15.176   1.515    0.136    
## ---
## Signif. codes:  0 '***' 0.001 '**' 0.01 '*' 0.05 '.' 0.1 ' ' 1
## 
## Residual standard error: 15.18 on 47 degrees of freedom
## Multiple R-squared:  0.6673,	Adjusted R-squared:  0.6532 
## F-statistic: 47.14 on 2 and 47 DF,  p-value: 5.852e-12
\end{verbatim}
\begin{alltt}
\hlkwd{anova}\hlstd{(fm3a)} \hlcom{# we calculate an ANOVA}
\end{alltt}
\begin{verbatim}
## Analysis of Variance Table
## 
## Response: dist
##                Df Sum Sq Mean Sq F value    Pr(>F)    
## poly(speed, 2)  2  21714 10857.1  47.141 5.852e-12 ***
## Residuals      47  10825   230.3                      
## ---
## Signif. codes:  0 '***' 0.001 '**' 0.01 '*' 0.05 '.' 0.1 ' ' 1
\end{verbatim}
\end{kframe}
\end{knitrout}

We can also compare two models, to test whether one of models describes the data better than the other.

\begin{knitrout}
\definecolor{shadecolor}{rgb}{0.969, 0.969, 0.969}\color{fgcolor}\begin{kframe}
\begin{alltt}
\hlkwd{anova}\hlstd{(fm2, fm1)}
\end{alltt}
\begin{verbatim}
## Analysis of Variance Table
## 
## Model 1: dist ~ speed - 1
## Model 2: dist ~ speed
##   Res.Df   RSS Df Sum of Sq      F  Pr(>F)  
## 1     49 12954                              
## 2     48 11354  1    1600.3 6.7655 0.01232 *
## ---
## Signif. codes:  0 '***' 0.001 '**' 0.01 '*' 0.05 '.' 0.1 ' ' 1
\end{verbatim}
\end{kframe}
\end{knitrout}

Or three or more models. But be careful, as the order of the arguments matters.

\begin{knitrout}
\definecolor{shadecolor}{rgb}{0.969, 0.969, 0.969}\color{fgcolor}\begin{kframe}
\begin{alltt}
\hlkwd{anova}\hlstd{(fm2, fm1, fm3, fm3a)}
\end{alltt}
\begin{verbatim}
## Analysis of Variance Table
## 
## Model 1: dist ~ speed - 1
## Model 2: dist ~ speed
## Model 3: dist ~ speed + I(speed^2)
## Model 4: dist ~ poly(speed, 2)
##   Res.Df   RSS Df Sum of Sq      F  Pr(>F)  
## 1     49 12954                              
## 2     48 11354  1   1600.26 6.9482 0.01133 *
## 3     47 10825  1    528.81 2.2960 0.13640  
## 4     47 10825  0      0.00                 
## ---
## Signif. codes:  0 '***' 0.001 '**' 0.01 '*' 0.05 '.' 0.1 ' ' 1
\end{verbatim}
\end{kframe}
\end{knitrout}

We can use different criteria to choose the best model: significance based on $P$-values or information criteria (AIC, BIC). AIC and BIC penalize the resulting `goodness' based on the number of parameters in the fitted model. In the case of AIC and BIC, a smaller value is better, and values returned can be either positive or negative, in which case more negative is better.

\begin{knitrout}
\definecolor{shadecolor}{rgb}{0.969, 0.969, 0.969}\color{fgcolor}\begin{kframe}
\begin{alltt}
\hlkwd{BIC}\hlstd{(fm2, fm1, fm3, fm3a)}
\end{alltt}
\begin{verbatim}
##      df      BIC
## fm2   2 427.5739
## fm1   3 424.8929
## fm3   4 426.4202
## fm3a  4 426.4202
\end{verbatim}
\begin{alltt}
\hlkwd{AIC}\hlstd{(fm2, fm1, fm3, fm3a)}
\end{alltt}
\begin{verbatim}
##      df      AIC
## fm2   2 423.7498
## fm1   3 419.1569
## fm3   4 418.7721
## fm3a  4 418.7721
\end{verbatim}
\end{kframe}
\end{knitrout}

One can see above that these three criteria not necessarily agree on which is the model to be chosen.

\begin{description}
\item[anova] \code{fm1}
\item[BIC] \code{fm1}
\item[AIC] \code{fm3}
\end{description}

\subsection{Analysis of variance, ANOVA}\label{sec:anova}

We use as the \texttt{InsectSpray} data set, giving insect counts in plots sprayed with different insecticides. In these data \texttt{spray} is a factor with six levels.

\begin{knitrout}
\definecolor{shadecolor}{rgb}{0.969, 0.969, 0.969}\color{fgcolor}\begin{kframe}
\begin{alltt}
\hlstd{fm4} \hlkwb{<-} \hlkwd{lm}\hlstd{(count} \hlopt{~} \hlstd{spray,} \hlkwc{data} \hlstd{= InsectSprays)}
\end{alltt}
\end{kframe}
\end{knitrout}

\begin{knitrout}
\definecolor{shadecolor}{rgb}{0.969, 0.969, 0.969}\color{fgcolor}\begin{kframe}
\begin{alltt}
\hlkwd{plot}\hlstd{(fm4,} \hlkwc{which} \hlstd{=} \hlnum{2}\hlstd{)}
\end{alltt}
\end{kframe}
\includegraphics[width=\maxwidth]{figure/model-6a-1} 

\end{knitrout}

\begin{knitrout}
\definecolor{shadecolor}{rgb}{0.969, 0.969, 0.969}\color{fgcolor}\begin{kframe}
\begin{alltt}
\hlkwd{anova}\hlstd{(fm4)}
\end{alltt}
\begin{verbatim}
## Analysis of Variance Table
## 
## Response: count
##           Df Sum Sq Mean Sq F value    Pr(>F)    
## spray      5 2668.8  533.77  34.702 < 2.2e-16 ***
## Residuals 66 1015.2   15.38                      
## ---
## Signif. codes:  0 '***' 0.001 '**' 0.01 '*' 0.05 '.' 0.1 ' ' 1
\end{verbatim}
\end{kframe}
\end{knitrout}

\subsection{Analysis of covariance, ANCOVA}

When a linear model includes both explanatory factors and continuous explanatory variables, we say that analysis of covariance (ANCOVA) is used. The formula syntax is the same for all linear models, what determines the type of analysis is the nature of the explanatory variable(s). Conceptually a factor (an unordered categorical variable) is very different from a continuous variable.

\section{Generalized linear models}

Linear models make the assumption of normally distributed residuals. Generalized linear models are more flexible, and allow the assumed distribution to be selected as well as the link function.
For the analysis of the \texttt{InsectSpray} data set, above (section \ref{sec:anova} on page \pageref{sec:anova}) the Normal distribution is not a good approximation as count data deviates from it. This was visible in the quantile--quantile plot above.

For count data GLMs provide a better alternative. In the example below we fit the same model as above, but we assume a quasi-Poisson distribution instead of the Normal.

\begin{knitrout}
\definecolor{shadecolor}{rgb}{0.969, 0.969, 0.969}\color{fgcolor}\begin{kframe}
\begin{alltt}
\hlstd{fm10} \hlkwb{<-} \hlkwd{glm}\hlstd{(count} \hlopt{~} \hlstd{spray,} \hlkwc{data} \hlstd{= InsectSprays,} \hlkwc{family} \hlstd{= quasipoisson)}
\hlkwd{plot}\hlstd{(fm10,} \hlkwc{which} \hlstd{=} \hlnum{2}\hlstd{)}
\end{alltt}
\end{kframe}
\includegraphics[width=\maxwidth]{figure/model-10-1} 
\begin{kframe}\begin{alltt}
\hlkwd{anova}\hlstd{(fm10,} \hlkwc{test} \hlstd{=} \hlstr{"F"}\hlstd{)}
\end{alltt}
\begin{verbatim}
## Analysis of Deviance Table
## 
## Model: quasipoisson, link: log
## 
## Response: count
## 
## Terms added sequentially (first to last)
## 
## 
##       Df Deviance Resid. Df Resid. Dev      F    Pr(>F)    
## NULL                     71     409.04                     
## spray  5   310.71        66      98.33 41.216 < 2.2e-16 ***
## ---
## Signif. codes:  0 '***' 0.001 '**' 0.01 '*' 0.05 '.' 0.1 ' ' 1
\end{verbatim}
\end{kframe}
\end{knitrout}
