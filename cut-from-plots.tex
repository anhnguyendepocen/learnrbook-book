\begin{warningbox}
The aesthetics and data given as \code{ggplot()}'s arguments become the defaults for all the geoms, but geoms also accept aesthetics and data as arguments, which when supplied locally override the whole-plot defaults. In the example below, we override the default colour of the points.
\end{warningbox}

If we \emph{set} the \code{color} aesthetic to a constant value, \code{"red"}, all points are plotted in red.

<<scatter-05>>=
ggplot(data = mtcars,
       aes(x = disp, y = mpg, color = factor(cyl))) +
  geom_point(color = "red")
@

\begin{playground}
Does the code chunk below produces exactly the same plot as that above this box? Consider how the two mappings differ, and make sure that you understand the reasons behind the difference or lack of difference in output by trying different variations of these examples
<<scatter-06,eval=eval_playground>>=
ggplot(data = mtcars,
       aes(x = disp, y = mpg)) +
  geom_point(color = "red")
@
\end{playground}

As with any R function it is possible to pass arguments by position to \code{aes} when
mapping variables to \emph{aesthetics} but this
makes the code more difficult to read and less tolerant to possible changes to the
definitions of functions. It is not
recommended to use this terse style in scripts or package coding. However, it can
be used by experienced users at the command prompt usually without problems.

Mapping passing arguments by \emph{name} to \code{aes}.

<<>>=
ggplot(data = mtcars, aes(x = disp, y = mpg)) +
  geom_point()
@

\begin{playground}
If we swap the order of the arguments do we still obtain the same plot?
<<scatter-07,eval=eval_playground>>=
ggplot(data = mtcars, aes(y = mpg, x = disp)) +
  geom_point()
@
\end{playground}

Mapping passing arguments by \emph{position} to \code{aes}.

<<scatter-08>>=
ggplot(mtcars, aes(disp, mpg)) +
  geom_point()
@

\begin{playground}
If we swap the order of the arguments do we obtain a different plot?
<<scatter-09,eval=eval_playground>>=
ggplot(mtcars, aes(mpg, disp)) +
  geom_point()
@
\end{playground}


\section{Bar}
\index{plots!bar plot|(}
Next we show an example of a bar plot, in which the columns represent counts, as a transition to the next section where we will describe how to plots summaries calculated on-the-fly within the ggplot code.

In this bar plot, each bar shows the number of observations in each \code{class} of car in the data set. We use a data set included in \ggplot for this example based on the package documentation.
<<>>=
ggplot(mpg, aes(class)) + geom_bar()
@

We can easily get stacked bars of counts, grouped by the number of cylinders of the engine.

<<>>=
ggplot(mpg, aes(class, fill = factor(cyl))) +
   geom_bar()
@

We can as above plot side by side bars, which shows a different aspect of the data.

<<>>=
ggplot(mpg, aes(x = factor(cyl), fill = class)) +
   geom_bar()
@

The default palette used for \code{fill} is rather ugly, so we also show the same plot with another scale for fill. In addition with \code{color = "black"} we make the borders of the bars black.

<<>>=
ggplot(mpg, aes(x = factor(cyl), fill = class)) +
  geom_bar(color = "black") +
  scale_fill_brewer() +
  theme_bw()
@

\begin{explainbox}
The order of stacking\index{plots!stacked bar plot} in \ggplot is consistent and depends on the order of the levels in the corresponding factor by default. The different approaches to reordering factor levels are described in section \ref{sec:data:levels:order} on page \pageref{sec:data:levels:order}. It must be noted, however, that aesthetics can be mapped not only to factors, but also to character variables which are converted on the fly into factors. When a character variable is converted into a factor, by default the levels are ordered alphabetically. As a rule of thumb, if all figures that will be produced from the same data frame require or allow the same ordering, it is best to reorder the levels of the factor in the data frame. If only a single figure requires a special ordering of levels, then the approach shown below is preferable.

If it is not desired to modify the data, the reordering of the stacking can be done in the ggplot. The simplest case, shown next, is to reverse the stacking order, however, this does not reverse the order in the key. This approach is useful when the behaviour of earlier versions of \ggplot needs to the restored.

<<>>=
ggplot(mpg, aes(x = factor(cyl), fill = class)) +
  geom_bar(color = "black", position = position_stack(reverse = TRUE))
@

An arbitrary reordering can be achieved by setting \code{breaks} in a discrete scale. A simple example follows. See section \ref{sec:plot:scales} on page \pageref{sec:plot:scales} for additional information on \emph{scales}. In the case when \code{position} is set to \code{"dodge"}, the ordering will be reflected in the order of the different side-by-side bars within the groups mapped to the $x$ aesthetic.

<<>>=
ggplot(mpg, aes(x = factor(cyl), fill = class)) +
  geom_bar(color = "black") +
  scale_fill_discrete(breaks = c("2seater", "compact",  "subcompact",
                                 "midsize",  "suv",  "minivan", "pickup"))
@

The approaches discussed in this box apply to all \emph{geometries} that allow stacking of groups---i.e.\ \gggeom{geom\_area()}, and \gggeom{geom\_line()}, in addition to \gggeom{geom\_col()} and \gggeom{geom\_bar()}, and in relation to the mapping of factor levels to aesthetics values, to all discrete scales.
\end{explainbox}

\index{plots!bar plot|)}

\section{Function}

\begin{explainbox}
In some cases we may want to tweak some aspects of the plot to better match the properties of the mathematical function. Here we use a predefined function for which the default $x$-axis breaks (tick positions) are not the best. We first show how the plot looks using defaults.

<<function-plot-04, eval=eval_plots_all>>=
ggplot(data.frame(x=c(0, 2 * pi)), aes(x=x)) +
  stat_function(fun=sin)
@

Next we change the $x$-axis scale to better match the sine function and the use of radians as angular units.

<<function-plot-05, eval=eval_plots_all>>=
ggplot(data.frame(x = c(0, 2 * pi)), aes(x = x)) +
  stat_function(fun = sin) +
  scale_x_continuous(
    breaks = c(0, 0.5, 1, 1.5, 2) * pi,
    labels = c("0", expression(0.5~pi), expression(pi),
             expression(1.5~pi), expression(2~pi))) +
  labs(y = "sin(x)")
@

There are three things in the above code that you need to understand: the use of the \Rlang built-in numeric constant \code{pi}, the use of argument `recycling' to avoid having to type \code{pi} many times, and the use of \Rlang \emph{expressions} to construct suitable tick labels for the $x$ axis. Do also consider why \code{pi} is interpreted differently within \code{expression} than within the numeric statements.

The use of \code{expression} is explained in detail in section \ref{sec:plot:plotmath}, an the use of \code{scales} in section \ref{sec:plot:scales}.

\end{explainbox}

To online site?

\begin{explainbox}
Sometimes we would like to include in the title or as an annotation in the plot, the name of the argument passed to \Rfunction{ggplot()}'s \code{data} parameter. To obtain the name of an object as a character string, the usual R ``slang'' is \code{deparse(substitute(x))} where \code{x} is the object (see section \ref{sec:plot:plotmath} on page \pageref{sec:plot:plotmath} for further details).

<<axis-labels-06, eval=eval_plots_all>>=
ggplot(data = Orange,
       aes(x = age, y = circumference, color = Tree)) +
  geom_line() +
  geom_point() +
  expand_limits(y = 0) +
  ggtitle(paste("Data:", deparse(substitute(Orange))))
@

The example above rarely is of much use, as we have anyway to pass the object itself twice, and consequently there is no advantage in effort compare to typing \code{"Data: Orange"} as argument to \Rfunction{ggtitle()}. A more general way to solve this problem is to write a wrapper function.

<<axis-labels-07, eval=eval_plots_all>>=
ggwrapper <- function(data, ...) {
  ggplot(data, ...) +
    ggtitle(paste("Object: ", substitute(data)))
}

ggwrapper(data = Orange,
          mapping = aes(x = age, y = circumference, color = Tree)) +
  geom_line() +
  geom_point() +
  expand_limits(y = 0)
@

This is a bare-bones example, as it does not retain user control over the formatting of the title. The ellipsis (\code{...}) is a catch-all parameter that we use to pass all other arguments to \Rfunction{ggplot()}. Because of the way our wrapper function is defined using ellipsis, we need to always pass \code{mapping} and other arguments that are to be ``forwarded'' to \Rfunction{ggplot()} by name.

Using this function in a loop over a list or vector, will produce output that is not as useful as you may expect. In many cases, the best, although more complex solution is to add case-specific code to the loop itself to generate suitable titles automatically.

We create a suitable set of data frames, build a list with name \code{my.dfs} containing them.

<<axis-labels-08>>=
df1 <- data.frame(x = 1:10, y = (1:10)^2)
df2 <- data.frame(x = 10:1, y = (10:1)^2.5)
my.dfs <- list(first.df = df1, second.df = df2)
@

If we print the output produced by the wrapper function when called in a loop we get always the same title, so this approach is not useful.

<<axis-labels-09>>=
for (df in my.dfs) {
  print(
    ggwrapper(data = df,
              mapping = aes(x = x, y = y)) +
      geom_line()
  )
}
@

\begin{warningbox}
 Automatic printing of objects is disabled within functions and iteration loops, making it necessary to use \Rfunction{print()} explicitly in these cases (see loops above). This `inconsistency' in behaviour is frequently surprising to unexperienced R users, so keep in mind that if some chunk of R code unexpectedly fails to produce visible output, the most frequent reason is that \Rfunction{print()} needs to be included in the code to make the `missing' result visible. Except for base R plotting functions, the norm in R is that printing, either implicitly or explicitly is needed for output to be visible to the user.
\end{warningbox}

As we have given names to the list members, we can use these and enclose the loop in a function. This is a very inflexible approach, and on top the plots are only printed, and the \Rclass{ggplot} objects get discarded once printed.

<<axis-labels-10>>=
plot.dfs <- function(x, ...) {
list.name <- deparse(substitute(x))
member.names <- names(x)
if (is.null(member.names)) {
  member.names <- as.character(seq_along(x))
}

  for (i in seq_along(x)) {
    print(
      ggplot(data = x[[i]], aes(x = x, y = y)) +
        geom_line() +
        ggtitle(paste("Object: ", list.name,
                      '[["', member.names[i], '"]]', sep = ""))
    )
  }

}
@

<<axis-labels-11>>=
plot.dfs(my.dfs)
@

\begin{playground}
Study the output from the two loops, and analyse why the titles differ. This will help not only understand this problem, but the implications of formulating \code{for} loops in these three syntactically correct ways.
\end{playground}

As it should be obvious by now, is that as an object ``moves'' through the function-call stack its visible name changes. Consequently when we nest functions or use loops it becomes difficult to retrieve the name under which the object was saved by the user. After these experiments, is should be clear that saving the titles ``in'' the data frames would be the most elegant approach. It is possible to save additional data in R objects using attributes. R itself uses attributes to keep track of objects' properties like the names of members in a list, or the class of objects.

When one has control over the objects, one can add the desired title as an attribute to the data frame, and then retrieve and use this when plotting. One should be careful, however, as some functions and operators may fail to copy user attributes to their output.

\begin{playground}

As an advanced exercise I suggest implementing this attribute-based solution by tagging the data frames using a function defined as shown below or by directly using \Rfunction{attr()}. You will also need modify the code to use the new attribute when building the \Rclass{ggplot} object.

<<axis-labels-12,eval=eval_playground>>=
add.title.attr <- function(x, my.title) {
  attr(x, "title") <- my.title
  x
}
@

What advantages and disadvantages does this approach have? Can it be used in a loop?

\end{playground}

\end{explainbox}