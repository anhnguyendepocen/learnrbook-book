\documentclass[krantz2]{krantz}\usepackage{knitr}%,ChapterTOCs

%\usepackage[utf8]{inputenc}
\usepackage{color}

\usepackage{polyglossia}
\setdefaultlanguage[variant = british, ordinalmonthday = false]{english}

%\usepackage{gitinfo2} % remember to setup Git hooks

\usepackage{hologo}

\usepackage{csquotes}

\usepackage{graphicx}
\DeclareGraphicsExtensions{.jpg,.pdf,.png}

\usepackage{animate}

%\usepackage{microtype}
\usepackage[style=authoryear-comp,giveninits,sortcites,maxcitenames=2,%
    mincitenames=1,maxbibnames=10,minbibnames=10,backref,uniquename=mininit,%
    uniquelist=minyear,sortgiveninits=true,backend=biber]{biblatex}%,refsection=chapter

\newcommand{\href}[2]{\emph{#2} (\url{#1})}

%\usepackage[unicode,hyperindex,bookmarks,pdfview=FitB,%backref,
%            pdftitle={Learn R ...as you learnt your mother tongue},%
%            pdfkeywords={R, statistics, data analysis, plotting},%
%            pdfsubject={R},%
%            pdfauthor={Pedro J. Aphalo}%
%            ]{hyperref}

%\hypersetup{colorlinks,breaklinks,
%             urlcolor=blue,
%             linkcolor=blue,
%             citecolor=blue,
%             filecolor=blue,
%             menucolor=blue}

\usepackage{framed}

\usepackage{abbrev}
\usepackage{usingr}

\usepackage{imakeidx}

% this is to reduce spacing above and below verbatim, which is used by knitr
% to show returned values
\usepackage{etoolbox}
\makeatletter
\preto{\@verbatim}{\topsep=-5pt \partopsep=-4pt \itemsep=-2pt}
\makeatother

%%% Adjust graphic design

% New float "example" and corresponding "list of examples"
%\DeclareNewTOC[type=example,types=examples,float,counterwithin=chapter]{loe}
%\DeclareNewTOC[name=Box,listname=List of Text Boxes, type=example,types=examples,float,counterwithin=chapter,%
%]{lotxb}

% changing the style of float captions
%\addtokomafont{caption}{\sffamily\small}
%\setkomafont{captionlabel}{\sffamily\bfseries}
%\setcapindent{0em}

% finetuning tocs
%\makeatletter
%\renewcommand*\l@figure{\@dottedtocline{1}{0em}{2.6em}}
%\renewcommand*\l@table{\@dottedtocline{1}{0em}{2.6em}}
%\renewcommand*\l@example{\@dottedtocline{1}{0em}{2.3em}}
%\renewcommand{\@pnumwidth}{2.66em}
%\makeatother
%
%% add pdf bookmarks to tocs
%\makeatletter
%\BeforeTOCHead{%
%  \cleardoublepage
%    \edef\@tempa{%
%      \noexpand\pdfbookmark[0]{\list@fname}{\@currext}%
%    }\@tempa
%}

\setcounter{topnumber}{3}
\setcounter{bottomnumber}{3}
\setcounter{totalnumber}{4}
\renewcommand{\topfraction}{0.90}
\renewcommand{\bottomfraction}{0.90}
\renewcommand{\textfraction}{0.10}
\renewcommand{\floatpagefraction}{0.70}
\renewcommand{\dbltopfraction}{0.90}
\renewcommand{\dblfloatpagefraction}{0.70}

\addbibresource{rbooks.bib}
\addbibresource{references.bib}

\makeindex[title=General index]
\makeindex[name=rindex,title=Alphabetic index of \Rlang names]
\makeindex[name=rcatsidx,title=Index of \Rlang names by category]
\IfFileExists{upquote.sty}{\usepackage{upquote}}{}
\begin{document}

% customize chapter format:
%\KOMAoption{headings}{twolinechapter}
%\renewcommand*\chapterformat{\thechapter\autodot\hspace{1em}}

% customize dictum format:
%\setkomafont{dictumtext}{\itshape\small}
%\setkomafont{dictumauthor}{\normalfont}
%\renewcommand*\dictumwidth{0.7\linewidth}
%\renewcommand*\dictumauthorformat[1]{--- #1}
%\renewcommand*\dictumrule{}

%\extratitle{\vspace*{2\baselineskip}%
%             {\Huge\textsf{\textbf{Learn R}\\ \textsl{\huge\ldots as you learnt your mother tongue}}}}

\title{\Huge{\fontseries{ub}\sffamily Learn R\\{\Large\ldots as you learnt your mother tongue}}}

%\subtitle{Git hash: \gitAbbrevHash; Git date: \gitAuthorIsoDate}

\author{Pedro J. Aphalo}

\date{Helsinki, \today}

%\publishers{Draft, 95\% done\\Available through \href{https://leanpub.com/learnr}{Leanpub}}

%\uppertitleback{\copyright\ 2001--2017 by Pedro J. Aphalo\\
%Licensed under one of the \href{http://creativecommons.org/licenses/}{Creative Commons licenses} as indicated, or when not explicitly indicated, under the \href{http://creativecommons.org/licenses/by-sa/4.0/}{CC BY-SA 4.0 license}.}
%
%\lowertitleback{Typeset with \href{http://www.latex-project.org/}{\hologo{XeLaTeX}}\ in Lucida Bright and \textsf{Lucida Sans} using the KOMA-Script book class.\\
%The manuscript was written using \href{http://www.r-project.org/}{R} with package knitr. The manuscript was edited in \href{http://www.winedt.com/}{WinEdt} and \href{http://www.rstudio.com/}{RStudio}.
%The source files for the whole book are available at \url{https://bitbucket.org/aphalo/using-r}.}

%\frontmatter

% knitr setup

















% \thispagestyle{empty}
% \titleLL
% \clearpage

\frontmatter

\maketitle

%\begin{titlingpage}
%  \maketitle
%\titleLL
%\end{titlingpage}

\setcounter{page}{7} %previous pages will be reserved for frontmatter to be added in later.
\tableofcontents
%\chapter*{Foreword}
I am delighted to introduce the first book on Multimedia Data Mining.  When I came to know about this book project undertaken by two of the most active young researchers in the field, I was pleased that this book is coming in early stage of a field that will need it more than most fields do.  In most emerging research fields, a book can play a significant role in bringing some maturity to the field.  Research fields advance through research papers.  In research papers, however, only a limited perspective could be provided about the field, its application potential, and the techniques required and already developed in the field.  A book gives such a chance.  I liked the idea that there will be a book that will try to unify the field by bringing in disparate topics already available in several papers that are not easy to find and understand.  I was supportive of this book project even before I had seen any material on it.  The project was a brilliant and a bold idea by two active researchers.  Now that I have it on my screen, it appears to be even a better idea.  

Multimedia started gaining recognition in 1990s as a field.  Processing, storage, communication, and capture and display technologies had advanced enough that researchers and technologists started building approaches to combine information in multiple types of signals such as audio, images, video, and  text.  Multimedia computing and communication techniques recognize correlated information in multiple sources as well as insufficiency of information in any individual source.    By properly selecting sources to provide complementary information, such systems aspire, much like human perception system, to create a holistic picture of a situation using only partial information from separate sources.

Data mining is a direct outgrowth of progress in data storage and processing speeds.  When it became possible to store large volume of data and run different statistical computations to explore all possible and even unlikely correlations among data, the field of data mining was born.  Data mining allowed people to hypothesize relationships among data entities and explore support for those.  This field has been put to applications in many diverse domains and keeps getting more applications.  In fact many new fields are direct outgrowth of data mining and it is likely to become a powerful computational tool.\vadjust{\vfill\pagebreak}



\chapter*{Preface}

\begin{VF}
``Suppose that you want to teach the `cat' concept to a very young child. Do you explain that a cat is a relatively small, primarily carnivorous mammal with retractible claws, a distinctive sonic output, etc.? I'll bet not. You probably show the kid a lot of different cats, saying `kitty' each time, until it gets the idea. To put it more generally, generalizations are best made by abstraction from experience.''

\VA{R. P. Boas}{Can we make mathematics intelligible?}
\end{VF}

%\dictum[R. P. Boas (1981) Can we make mathematics intelligible?, \emph{American Mathematical Monthly} \textbf{88:} 727-731.]{"Suppose that you want to teach the `cat' concept to a very young child. Do you explain that a cat is a relatively small, primarily carnivorous mammal with retractible claws, a distinctive sonic output, etc.? I'll bet not. You probably show the kid a lot of different cats, saying `kitty' each time, until it gets the idea. To put it more generally, generalizations are best made by abstraction from experience."}


% Such pauses are not a miss use of our time. To learn a natural language we need to interact with other speakers, we need feedback. In the case of R, we can get feedback both from the outcomes from our utterances to the computer, and from other \Rlang users.}
\noindent
This book covers different aspects of the use of the \Rlang language. Chapters \ref{chap:R:introduction} to \ref{chap:R:functions} describe the \Rlang language itself. Later chapters describe extensions to the \Rlang language available through contributed \emph{packages}, the \emph{grammar of data} and the \emph{grammar of graphics}. In this book, explanations are concise but contain pointers to additional sources of information, so as to encourage the development of a routine of independent exploration. This is not an arbitrary decision, this is the normal \emph{modus operandi} of most of us who use \Rlang regularly for a variety of different problems. Some have called approaches like the one used here, ``learning the hard way'', but I would call it ``learning to be independent''.

I do not discuss in this book statistics or data analysis methods, I describe \Rlang as a language for data manipulation and display. The idea is for you to learn the \Rlang language in a way comparable to how children learn a language: they work-out what the rules are, simply by listening to people speak and trying to utter what they want to tell their parents. Of course, small children receive some guidance, but are not taught a prescriptive set of rules like when learning a second language at school. Instead of listening, you will read code and instead of speaking you will try to execute \Rlang  code statements on a computer---i.e. you will try your hand at using \Rlang to tell a computer what you want it to compute. I do provide explanations and guidance, but the idea of this book is for you to use the numerous examples to find-out by yourself the overall patterns and coding philosophy behind the \Rlang language. Instead of parents being the sound board for your first utterances in \Rlang, the computer will play this role. You will \emph{play} by modifying the examples, see how the computer responds, does \Rlang understand you or not? Using actively a language is the most efficient way of learning it. By using it, I mean actually reading, writing and running scripts or programs (copying and pasting, or typing ready-made examples from books or the internet does not qualify as using a language).

What is a language? A language is a system of communication. \Rlang as a language allows us to communicate with other members of the \Rlang community, and with computers. As most languages in active use, \Rlang evolves. New ``words'' and new ``constructs'' are incorporated into the language, and some earlier frequently used ones are relegated to the fringes of the corpus. I describe current usage and ``modisms'' of the \Rlang language in a way accessible to a readership unfamiliar with computer science but with some background in data analysis as used in Biology, Engineering, or Humanities.

When teaching I tend to lean towards challenging students rather than telling an over-simplified story. There are two reasons for this. First, I prefer as a student, and I learn best myself if the going is not too easy. Second, if I would hide the tricky bits of the \Rlang language, it would make readers' life much more difficult later on. You, will not remember all the details, nobody could. However, you most likely will remember in which situations you need to be careful or should check the details. So, I will expose you not only the usual cases, but also to several exceptions and counterintuitive features of the language. Reading this book will be about exploring a new world, this book aims to be a travel guide, but neither a traveler's account, nor a cookbook of \Rlang recipes.

Keep in mind that it is impossible to remember everything about \Rlang! The \Rlang language in a broad sense is vast because its capabilities can be expanded with independently developed packages. Learning to use \Rlang consists in learning the basics plus developing the skill of finding your way in \Rlang and its documentation.  In 2017 the number packages available in the Comprehensive \Rlang Archive Network (CRAN) broke the 10\,000 barrier. CRAN is the most important, but not only, public repository for \Rlang packages. How good a command of the \Rlang language and packages a user needs depends on the type activities to be carried out. This book attempts to train you in the use of the \Rlang language itself and of popular \Rlang language extensions for data manipulation and graphical display. Given the availability of numerous books on statistical analysis with \Rlang, in the present book I will cover only the bare minimum of this subject. The same is true for package development in \Rlang. This book seats in-between, aiming at teaching programming in-the-small: the use of \Rlang to automate the drudgery of data manipulation from raw data, through data exploration to the production of publication quality illustrations.

As with all ``rich'' languages there are many different ways of doing things in \Rlang, and there is in almost all cases no one-size-fits-all solution to a problem. There is always a compromise involved, usually between time spent by the user and processing time required in the computer. Many of the packages that are most popular nowadays did not exist when I started using \Rlang, and many of these packages make new approaches available. One could write many different \Rlang books with a given aim using substantially different ways of achieving the same results. In this book, I limit myself to packages that are currently popular and/or that I consider elegantly designed. I have in particular tried to limit myself to packages with similar design philosophies, especially in relation to their interfaces. What is elegant design, and in particular what is a friendly user interface depends strongly on each user's preferences and previous experience. Consequently, the contents of the book are strongly biased by my own preferences. I have tried to write examples in ways that execute fast without compromising readability. I encourage readers to take this book as a starting point for exploring the very many packages, styles and approaches which I have not described.

I will appreciate suggestions for further examples, notification of errors and unclear sections. Many of the examples here have been collected from diverse sources over many years and because of this not all sources are acknowledged. If you recognize any example as yours or someone else's please let me know so that I can add a proper acknowledgement. I warmly thank the students that over the years have asked the questions and posed the problems that have helped me write this text and correct the mistakes and voids of previous versions. I have also received help on on-line forums and in person from numerous people, learnt from archived e-mail list messages, blog posts, books, articles, tutorials, webinars, and by struggling to solve some new problems on my own. In many ways this text owes much more to people who are not authors than to myself. However, as I am the one who has written this version and decided what to include and exclude, as author, I take full responsibility for any errors and inaccuracies.

I have been using \Rlang since around 1998 or 1999, but I am still constantly learning new things about \Rlang itself and \Rlang packages. With time it has replaced in my work as a researcher and teacher several other pieces of software: \pgrmname{SPSS}, \pgrmname{Systat}, \pgrmname{Origin}, \pgrmname{Excel}, and it has become a central piece of the tool set I use for producing lecture slides, notes, books and even web pages. This is to say that it is the most useful piece of software and programming language I have ever learnt to use. Of course, in time it will be replaced by something better, but at the moment it is a key language to learn for anybody with a need to analyse and display data.

Why the title ``\emph{Learn R \ldots as you learnt your mother tongue}''? On one hand, because this book is based on exploration and practice. On the other hand, because you will be exposed to current usage and not spared the quirks of the language. When we use our mother tongue in everyday life we do not think about grammar rules or sentence structure, except for the trickier or unfamiliar situations. My aim is for this book to help you grow to use \Rlang in this same way.

\begin{framed}
\noindent\large%
\textbf{I encourage you to approach R, like a child approaches his or hers mother tongue when learning to speak:} Do not struggle, just play! If going gets difficult and frustrating, take a break! If you get a new insight, take a break to enjoy the victory!
\end{framed}

\newpage

\begin{framed}
\noindent
\textbf{Icons used to mark different content.} Throughout the book text boxes marked with icons present different types of information. First of all, we have \emph{playground} boxes indicated with \playicon\ which contain open-ended exercises---ideas and pieces of \Rlang code to play with at the \Rlang console. A few of these will require more time to grasp, and are indicated with \advplayicon. Boxes providing general information, usually not directly related to \Rlang as a language, are indicated with \infoicon. Some boxes highlighted with \ilAttention\ give important bits of information that must be remembered when using \Rlang---i.e.\ explain some unusual feature of the language. Finally, some boxes indicated by \ilAdvanced\ give in depth explanations, that may require you to spend time thinking, which en general can be skipped on first reading, but to which you should return at a later, and peaceful, time with a cup of coffee or tea.
\end{framed}
\newpage

%\newpage
%\begin{infobox}
%\noindent
%\textbf{Status as of 2016-11-23.} I have updated the manuscript to track package updates since the previous version uploaded six months ago, and added several examples of the new functionality added to packages \ggpmisc, \ggrepel, and \ggplot. I have written new sections on packages \viridis, \pkgname{gganimate}, \pkgname{ggstance}, \pkgname{ggbiplot}, \pkgname{ggforce}, \pkgname{ggtern} and \pkgname{ggalt}. Some of these sections are to be expanded, and additional sections are planned for other recently released packages.
%
%With respect to the chapter \textit{Storing and manipulating data with R} I have put it on hold, except for the introduction, until I can see a soon to be published book covering the same subject. Hadley Wickham has named the set of tools developed by him and his collaborators as \textit{tidyverse} to be described in the book titled \textit{R for Data Science} by Grolemund and Wickham (O'Reilly).
%
%An important update to \ggplot was released last week, and it includes changes to the behavior of some existing functions, specially faceting has become extensible through other packages. Several of the new facilities are described in the updated text and code included in this book and this pdf has been generated with up-to-date version of \ggplot and packages as available today from CRAN, except for \pkgname{ggtern} which was downloaded from Bitbucket minutes ago.
%
%The present update adds about 100 pages to the previous versions. I expect to upload a new update to this manuscript in one or two months time.
%
%\textbf{Status as of 2017-01-17.} Added ``playground'' exercises to the chapter describing \ggplot, and converted some of the examples earlier part of the main text into these playground items. Added icons to help readers quickly distinguish playground sections (\textcolor{blue}{\noticestd{"0055}}), information sections (\textcolor{blue}{\modpicts{"003D}}), warnings about things one needs to be specially aware of (\colorbox{yellow}{\typicons{"E136}}) and boxes with more advanced content that may require longer time/more effort to grasp (\typicons{"E04E}). Added to the sections \code{scales} and examples in the \ggplot chapter details about the use of colors in \Rlang and \ggplot2. Removed some redundant examples, and updated the section on \code{plotmath}. Added terms to the alphabetical index. Increased line-spacing to avoid uneven spacing with inline code bits.
%
%\textbf{Status as of 2017-02-09.} Wrote section on ggplot2 themes, and on using system- and Google fonts in ggpplots with the help of package \pkgname{showtext}. Expanded section on \ggplot's \code{annotation}, and revised some sections in the ``R scripts and Programming'' chapter. Started writing the data chapter. Wrote draft on writing and reading text files. Several other smaller edits to text and a few new examples.
%
%\textbf{Status as of 2017-02-14.} Wrote sections on reading and writing MS-Excel files, files from statistical programs such as SPSS, SyStat, etc., and NetCDF files. Also wrote sections on using URLs to directly read data, and on reading HTML and XML files directly, as well on using JSON to retrieve measured/logged data from IoT (internet of things) and similar intelligent physical sensors, micro-controller boards and sensor hubs with network access.
%
%\textbf{Status as of 2017-03-25.} Revised and expanded the chapter on plotting maps, adding a section on the manipulation and plotting of image data. Revised and expanded the chapter on extensions to \pkgname{ggplot2}, so that there are no longer empty sections. Wrote short chapter ``If and when \Rlang needs help''. Revised and expanded the ``Introduction'' chapter. Added index entries, and additional citations to literature.
%
%\textbf{Status as of 2017-04-04.} Revised and expanded the chapter on using \Rpgrm as a calculator. Revised and expanded the ``Scripts'' chapter. Minor edits to ``Functions'' chapter. Continued writing chapter on data, writing a section on \Rlang native apply functions and added preliminary text for a pipes and tees section. Write intro to `tidyverse' and grammar of data manipulation. Added index entries, and a few additional citations to the literature. Spell checking.
%
%\textbf{Status as of 2017-04-08.} Completed writing first draft of chapter on data, writing all the previously missing sections on the ``grammar of data manipulation''. Wrote two extended examples in the same chapter. Add table listing several extensions to \pkgname{ggplot2} not described in the book.
%
%\textbf{Status as of 2017-04-13.} Revised all chapters correcting some spelling mistakes, adding some explanatory text and indexing all functions and operators used. Thoroughly revised the Introduction chapter and the Preface. Expanded section on bar plots (now bar and column plots). Revised section on tile plots. Expanded section on factors in chapter 2, adding examples of reordering of factor labels, and making clearer the difference between the labels of the levels and the levels themselves.
%
%\textbf{Status as of 2017-04-29.} Tested with R 3.4.0. Package \pkgname{gganimate} needs to be installed from Github as the updated version is not yet in CRAN. Function \code{gg\_animate()} has been renamed \code{gganimate().}
%
%\textbf{Status as of 2017-05-14.} Submitted package \pkgname{learnrbook} to CRAN. Revised code in the book
%to use this new package. Small fixes after more testing. Added examples of plotting and labeling based on fits with \code{method = "nls"}, including use of the new \code{ggpmisc::stat\_fit\_tidy()}.
%
%\textbf{Status as of 2017-06-11.} Added sections on R-code bench marking and profiling for performance optimization. Added also an example of explicit compilation of a function defined in the R language. Added section on functions \code{assign()}, \code{get()} and \code{mget()}.
%
%\textbf{Status as of 2017-08-12.} Various edits to all chapters. Expanded section on \pkgname{ggpmisc} to include the new functionality added in version 0.2.15.9002: \code{geom\_table} and \code{stat\_fit\_tb}. Added section on package \pkgname{ggbeeswarm}. Added sections on packages \pkgname{magick} and on using \pgrmname{ImageJ} from \Rpgrm. Improved indexing and cross references.
%
%\textbf{Status as of 2017-10-25.} Edited the chapter on using R as a calculator, adding examples on insertion and deletion of members of lists and vectors, and also of use of \code{gl()} and \code{reorder()}. Edited sections on scale limits and added new section on coordinate limits to explain more thoroughly their differences and uses in chapter on plotting with \pkgname{ggplot2}. Added a section on package \pkgname{ggsignif} to the chapter on extensions to \pkgname{ggplot2}. Expanded section on \pkgname{ggpmisc} in the same chapter describing new functionality added in version 0.2.16.
%\pkgname{ggplo2} $>=$ 2.2.1.9000 is required by the current development version of \pkgname{ggpmisc}.
%
%\textbf{Status as of 2017-10-30.}  Add section on using pipes with \code{ggplot()} and layers.
%\end{infobox} 
\listoffigures
\listoftables
%%%\twocolumn
\chapter*{Contributors}

\begin{multicols}{2}
\contributor{Michael Aftosmis}{NASA Ames Research Center}{Moffett Field, California}

\contributor{Pratul K. Agarwal}{Oak Ridge National Laboratory}{Oak Ridge, Tennessee}

\contributor{Sadaf R. Alam}{Oak Ridge National Laboratory}{Oak Ridge, Tennessee}

\contributor{Gabrielle Allen}{Louisiana State University}{Baton Rouge, Louisiana}

\contributor{Martin Sandve Aln{\ae}s}{Simula Research Laboratory and University of Oslo, Norway}{Norway}

\contributor{Steven F. Ashby} {Lawrence Livermore National Laboratory}{Livermore, California}

\contributor{David A. Bader} {Georgia Institute of Technology}{Atlanta, Georgia}

\contributor{Benjamin Bergen} {Los Alamos National Laboratory}{Los Alamos, New Mexico}

\contributor{Jonathan W. Berry} {Sandia National Laboratories}{Albuquerque, New Mexico}

\contributor{Martin Berzins}{University of Utah}{Salt Lake City, Utah}

\contributor{Abhinav Bhatele}{University of Illinois}{Urbana-Champaign, Illinois}

\contributor{Christian Bischof} {RWTH Aachen University}{Germany}

\contributor{Rupak Biswas} {NASA Ames Research Center}{Moffett Field, California}\vspace*{5pt}

\contributor{Eric Bohm} {University of Illinois}{Urbana-Champaign, Illinois}\vspace*{5pt}

\contributor{James Bordner} {University of California, San Diego}{San Diego, California}\vspace*{5pt}

\contributor{George Bosilca} {University of Tennessee}{Knoxville, Tennessee}\vspace*{5pt}

\contributor{Greg L. Bryan} {Columbia University}{New York, New York}\vspace*{5pt}

\contributor{Marian Bubak} {AGH University of Science and Technology}{
Krak{\'o}w, Poland}\vspace*{5pt}

\contributor{Andrew Canning}{Lawrence Berkeley National
Laboratory}{Berkeley, California}

\contributor{Jonathan Carter} {Lawrence Berkeley National
Laboratory}{Berkeley, California}

\contributor{Zizhong Chen} {Jacksonville State University}{Jacksonville,
Alabama}

\contributor{Joseph R. Crobak} {Rutgers, The State University of New
Jersey}{Piscataway, New Jersey}

\contributor{Roxana E. Diaconescu} {Yahoo! Inc.}{Burbank, California}

\contributor{Peter Diener}
{Louisiana State University}{Baton Rouge, Louisiana}

\contributor{Jack J. Dongarra} {University of Tennessee, Knoxville, 
Oak Ridge National Laboratory, and}{University of Manchester}

\contributor{John B. Drake} {Oak Ridge National Laboratory}{Oak Ridge,
Tennessee}

\contributor{Kelvin K. Droegemeier} {University of Oklahoma}{Norman,
Oklahoma}

\contributor{St{\'e}phane Ethier} {Princeton University}{Princeton, New
Jersey}

\contributor{Christoph Freundl}
{Friedrich--Alexander--Universit{\"a}t}{Erlangen, Germany}

\contributor{Karl F{\"u}rlinger} {University of Tennessee}{Knoxville,
Tennessee}

\contributor{Al Geist} {Oak Ridge National Laboratory}{Oak Ridge,
Tennessee}

\contributor{Michael Gerndt} {Technische Universit{\"a}t
M{\"u}nchen}{Munich, Germany}

\contributor{Tom Goodale}
{Louisiana State University}{Baton Rouge, Louisiana}

\contributor{Tobias Gradl}
{Friedrich--Alexander--Universit{\"a}t}{Erlangen, Germany}

\contributor{William D. Gropp} {Argonne National Laboratory}{Argonne,
Illinois}

\contributor{Robert Harkness} {University of California, San
Diego}{San Diego, California}

\contributor{Albert Hartono} {Ohio State University}{Columbus, Ohio}

\contributor{Thomas C. Henderson} {University of Utah}{Salt Lake City,
Utah}

\contributor{Bruce A. Hendrickson} {Sandia National
Laboratories}{Albuquerque, New Mexico}

\contributor{Alfons G. Hoekstra} {University of Amsterdam}{Amsterdam,
The Netherlands}

\contributor{Philip W. Jones} {Los Alamos National Laboratory}{Los
Alamos, New Mexico}

\contributor{Laxmikant Kal{\'e}} {University of
Illinois}{Urbana-Champaign, Illinois}

\contributor{Shoaib Kamil} {Lawrence Berkeley National
Laboratory}{Berkeley, California}

\contributor{Cetin Kiris} {NASA Ames Research Center}{Moffett Field,
California}

\contributor{Uwe K{\"u}ster} {University of Stuttgart}{Stuttgart,
Germany}

\contributor{Julien Langou} {University of Colorado}{Denver, Colorado}

\contributor{Hans Petter Langtangen}
{Simula Research Laboratory and}{University of Oslo, Norway}

\contributor{Michael Lijewski} {Lawrence Berkeley National
Laboratory}{Berkeley, California}

\contributor{Anders Logg}
{Simula Research Laboratory and}{University of Oslo, Norway}

\contributor{Justin Luitjens} {University of Utah}{Salt Lake City, Utah}

\contributor{Kamesh Madduri} {Georgia Institute of Technology}{Atlanta,
Georgia}

\contributor{Kent-Andre Mardal}
{Simula Research Laboratory and}{University of Oslo, Norway}

\contributor{Satoshi Matsuoka} {Tokyo Institute of Technology}{Tokyo,
Japan}

\contributor{John M. May} {Lawrence Livermore National
Laboratory}{Livermore, California}

\contributor{Celso L. Mendes} {University of Illinois}{Urbana-Champaign,
Illinois}

\contributor{Dieter an Mey} {RWTH Aachen University}{Germany}

\contributor{Tetsu Narumi} {Keio University}{Japan}

\contributor{Michael L. Norman} {University of California, San
Diego}{San Diego, California}

\contributor{Boyana Norris} {Argonne National Laboratory}{Argonne,
Illinois}

\contributor{Yousuke Ohno} {Institute of Physical and Chemical Research
(RIKEN)}{Kanagawa, Japan}

\contributor{Leonid Oliker} {Lawrence Berkeley National
Laboratory}{Berkeley, California}

\contributor{Brian O'Shea} {Los Alamos National Laboratory}{Los Alamos,
New Mexico}

\contributor{Christian D. Ott}
{University of Arizona}{Tucson, Arizona}

\contributor{James C. Phillips} {University of
Illinois}{Urbana-Champaign, Illinois}

\contributor{Simon Portegies Zwart} {University of
Amsterdam,}{Amsterdam, The Netherlands}

\contributor{Thomas Radke}
{Albert-Einstein-Institut}{Golm, Germany}

\contributor{Michael Resch} {University of Stuttgart}{Stuttgart,
Germany}

\contributor{Daniel Reynolds} {University of California, San Diego}{San
Diego, California}

\contributor{Ulrich R{\"u}de}
{Friedrich--Alexander--Universit{\"a}t}{Erlangen, Germany}

\contributor{Samuel Sarholz}
{RWTH Aachen University}{Germany}

\contributor{Erik Schnetter}
{Louisiana State University}{Baton Rouge, Louisiana}

\contributor{Klaus Schulten} {University of Illinois}{Urbana-Champaign,
Illinois}

\contributor{Edward Seidel}
{Louisiana State University}{Baton Rouge, Louisiana}

\contributor{John Shalf} {Lawrence Berkeley National
Laboratory}{Berkeley, California}

\contributor{Bo-Wen Shen} {NASA Goddard Space Flight Center}{Greenbelt,
Maryland}

\contributor{Ola Skavhaug}
{Simula Research Laboratory and}{University of Oslo, Norway}

\contributor{Peter M.A. Sloot} {University of Amsterdam}{Amsterdam, The
Netherlands}

\contributor{Erich Strohmaier} {Lawrence Berkeley National
Laboratory}{Berkeley, California}

\contributor{Makoto Taiji} {Institute of Physical and Chemical Research
(RIKEN)}{Kanagawa, Japan}

\contributor{Christian Terboven}
{RWTH Aachen University,}{Germany}

\contributor{Mariana Vertenstein} {National Center for Atmospheric
Research}{Boulder, Colorado}

\contributor{Rick Wagner} {University of California, San Diego}{San
Diego, California}

\contributor{Daniel Weber} {University of Oklahoma}{Norman, Oklahoma}

\contributor{James B. White, III} {Oak Ridge National Laboratory}{Oak
Ridge, Tennessee}

\contributor{Terry Wilmarth} {University of Illinois}{Urbana-Champaign,
Illinois}

\end{multicols}
%\chapter*{Symbols}
\begin{symbollist}{000000}
\symbolentry{$\alpha$}{To solve the generator maintenance scheduling, in the  past, several mathematical techniques have  been applied.}
\symbolentry{$\sigma^2$}{These include integer programming, integer linear programming, dynamic programming, branch and bound etc.}
\symbolentry{$\sum$}{Several heuristic search algorithms have also been developed. In recent years expert systems,}
\symbolentry{$abc$}{fuzzy approaches, simulated annealing and genetic algorithms have also been tested.}
\symbolentry{$\theta\sqrt{abc}$}{This paper presents a survey of the literature}
\symbolentry{$\zeta$}{ over the past fifteen years in the generator}
\symbolentry{$\partial$}{maintenance scheduling. The objective is to}
\symbolentry{sdf}{present a clear picture of the available recent literature}
\symbolentry{ewq}{of the problem, the constraints and the other aspects of}
\symbolentry{bvcn}{the generator maintenance schedule.}
\end{symbollist}

\mainmatter


%\part{The \Rlang Language}











%\part{The Grammar of Data}



%\part{The Grammar of Graphics}


% !Rnw root = appendix.main.Rnw



\chapter{Grammar of graphics}\label{chap:R:plotting}

\begin{VF}
The commonality between science and art is in trying to see profoundly---to develop strategies of seeing and showing.

\VA{Edward Tufte}{}
\end{VF}

%\dictum[Edward Tufte]{The commonality between science and art is in trying to see profoundly---to develop strategies of seeing and showing.}

\index{geometries ('ggplot2')|see{plots, geometries}}
\index{geom@\texttt{geom}|see{plots, geometries}}
\index{functions!geom@\texttt{geom}|see{plots, geometries}}
\index{statistics ('ggplot2')|see{plots, statistics}}
\index{stat@\texttt{stat}|see{plots, statistics}}
\index{functions!stat@\texttt{stat}|see{plots, statistics}}
\index{scales ('ggplot2')|see{plots, scales}}
\index{scale@\texttt{scale}|see{plots, scales}}
\index{functions!scale@\texttt{scale}|see{plots, scales}}
\index{coordinates ('ggplot2')|see{plots, coordinates}}
\index{themes ('ggplot2')|see{plots, themes}}
\index{theme@\texttt{scale}|see{plots, themes}}
\index{function!theme@\texttt{scale}|see{plots, themes}}
\index{facets ('ggplot2')|see{plots, facets}}
\index{annotations ('ggplot2')|see{plots, annotations}}
\index{aesthetics ('ggplot2')|see{plots, aesthetics}}

\section{Aims of this chapter}

Three main plotting systems are available to \Rlang users: base \Rlang, package \pkgname{lattice} \autocite{Sarkar2008} and package \pkgname{ggplot2} \autocite{Wickham2016}, being the last one the most recent and currently most popular system available in \Rlang for plotting data. Even two different sets of graphics primitives are available in \Rlang, that in base \Rlang and a newer one in the \pkgname{grid} package \autocite{Murrell2011}.

In this chapter you will learn the concepts of the grammar of graphics, on which package \pkgname{ggplot2} is based. You will as well learn how to do many of the data plots that can be produced with package \pkgname{ggplot2}. We will focus only on the grammar of graphics, as it is currently the most used plotting approach in \Rlang. As a consequence of this popularity and its flexibility, many extensions to \pkgname{ggplot2} have been developed and deposited in public repositories. As previous chapters, this chapter is intended to be read in whole.

This chapter focuses mainly on how to construct different types of graphical data displays using the grammar of graphics. We also discuss how to alter the ``graphical design'' of the plots produced, but in less depth, mostly leaving for the reader to try by herself/himself the different combinations of types of plots and themes and color palettes described. The book \citebooktitle{Burchell2016} \autocite{Burchell2016} has a strong focus on the control of how plots look, and can be a good source of worked out examples. For a cook book with a broader scope and detailed explanations consult \citebooktitle{Chang2013} \autocite{Chang2013}. The contents of the current chapter to some extent overlap with that of Chang's book, but using a different approach for presentation. Deeper explanations of technical aspects are available in the book \citebooktitle{Murrell2011} \autocite{Murrell2011}. Finally, the book \citebooktitle{Wickham2016} \autocite{Wickham2016} written by the developers of package \pkgname{ggplot2} is the main reference, and describes the grammar of graphics in more detail than we have space here for. In particular, the hands-on approach followed here makes this chapter a good complement to \citebooktitle{Wickham2016}.

\section{Packages used in this chapter}

\begin{knitrout}\footnotesize
\definecolor{shadecolor}{rgb}{0.969, 0.969, 0.969}\color{fgcolor}\begin{kframe}
\begin{alltt}
\hlkwd{citation}\hlstd{(}\hlkwc{package} \hlstd{=} \hlstr{"ggplot2"}\hlstd{)}
\end{alltt}
\begin{verbatim}
## 
## To cite ggplot2 in publications, please use:
## 
##   H. Wickham. ggplot2: Elegant Graphics for Data Analysis.
##   Springer-Verlag New York, 2016.
## 
## A BibTeX entry for LaTeX users is
## 
##   @Book{,
##     author = {Hadley Wickham},
##     title = {ggplot2: Elegant Graphics for Data Analysis},
##     publisher = {Springer-Verlag New York},
##     year = {2016},
##     isbn = {978-3-319-24277-4},
##     url = {https://ggplot2.tidyverse.org},
##   }
\end{verbatim}
\end{kframe}
\end{knitrout}

If the packages used in this chapter are not yet installed in your computer, you can install them with, as long as package \pkgname{learnrbook} is already installed.

\begin{knitrout}\footnotesize
\definecolor{shadecolor}{rgb}{0.969, 0.969, 0.969}\color{fgcolor}\begin{kframe}
\begin{alltt}
\hlkwd{install.packages}\hlstd{(learnrbook}\hlopt{::}\hlstd{pkgs_ch_ggplot)}
\end{alltt}
\end{kframe}
\end{knitrout}

For executing the examples listed in this chapter you need first to load the following packages from the library:

\begin{knitrout}\footnotesize
\definecolor{shadecolor}{rgb}{0.969, 0.969, 0.969}\color{fgcolor}\begin{kframe}
\begin{alltt}
\hlkwd{library}\hlstd{(learnrbook)}
\hlkwd{library}\hlstd{(scales)}
\hlkwd{library}\hlstd{(ggplot2)}
\hlkwd{library}\hlstd{(ggrepel)}
\hlkwd{library}\hlstd{(gginnards)}
\hlkwd{library}\hlstd{(ggpmisc)}
\hlkwd{library}\hlstd{(ggbeeswarm)}
\hlkwd{library}\hlstd{(ggforce)}
\hlkwd{library}\hlstd{(tikzDevice)}
\hlkwd{library}\hlstd{(lubridate)}
\hlkwd{library}\hlstd{(tidyverse)}
\end{alltt}
\end{kframe}
\end{knitrout}

We set a font of larger size than the default
\begin{knitrout}\footnotesize
\definecolor{shadecolor}{rgb}{0.969, 0.969, 0.969}\color{fgcolor}\begin{kframe}
\begin{alltt}
\hlkwd{theme_set}\hlstd{(}\hlkwd{theme_grey}\hlstd{(}\hlnum{14}\hlstd{))}
\end{alltt}
\end{kframe}
\end{knitrout}



\section{Introduction}

Being \Rlang extensible, in addition to the built-in plotting functions, there are several alternatives provided by packages. Of the general purpose ones, the most extensively used are \pkgname{Lattice} \autocite{Sarkar2008} and \ggplot \autocite{Wickham2016}. There are additional packages that add extra functionality to these packages, many of them available through CRAN.

In the examples in this chapter we describe the of use package \ggplot. We start with an introduction to the `grammar of graphics' and \ggplot. There is ample literature on the use of \ggplot, including the very good reference documentation at \url{http://docs.ggplot2.org/}. The book titled \citebooktitle{Wickham2016} \autocite{Wickham2016} is the authoritative reference, as it is authored by the developers of \ggplot. The book `R Graphics Cookbook' \autocite{Chang2013} is very useful as a reference as it contains many worked out examples. Some of the literature available at this time is for older versions of \ggplot but we here describe version 3.1.0. Consistent with the title of this book, we focus on the Grammar of Graphics, and describe it as a language. We use examples to demonstrate the features of the language and how they can be used.

\section{Grammar}

What separates \ggplot from base \Rlang and trellis/lattice plotting functions is the use of a grammar of graphics\index{grammar of graphics} (the reason behind `gg' in the name of the package). What is meant by grammar in this case is that plots are assembled piece by piece using different `nouns' and `verbs' \autocite{Cleveland1985}. Instead of using a single function with many arguments, plots are assembled by combining different elements with operators \code{+} and \verb|%+%|. Furthermore, the construction is mostly semantic-based and to a large extent how the plot looks when is printed, displayed or exported to a bitmap or vector-graphics file is controlled by themes.

We can think of plotting as representing the observations or data in a graphical language. We use the properties of graphical objects to represent different aspects of our data. An observation can consist in multiple values recorded. Say an observation of air temperature may be defined by a position in 3-dimensional space and a point in time, in addition to the temperature itself. An observation for the size and shape of a plant can consist in height, stem diameter, number of leaves, size of individual leaves, length of roots, fresh mass, dry mass, etc. If we are interested in the relationship between height and stem diameter, we may want to use cartesian coordinates, \emph{mapping} stem diameter to the $x$ dimension of the plot and the height to the $y$ dimension. The observations could be represented on the plot by points and/or joined by lines.

The grammar of graphics allows us to design plots by combining various elements in ways that are nearly orthogonal. In other words, the majority of the possible combinations of ``words'' yield valid plots as long as we assemble them respecting the rules of the grammar. This flexibility makes \ggplot extremely powerful as we can build plots and even types of plots which were not even considered while designing the \ggplot package.

When a plot is built the whole plot and its components are created as \Rlang objects that can be saved in the workspace or written to a file as objects. The graphical representation is generated when the object is printed, explicitly or automatically. The same ggplot object can be rendered into different bitmap and vector graphic formats for display or printing.

Even if we do not explicitly add them all, default elements may be used. The production of a rendered graphic with package \pkgname{ggplot2} can be represented as a flow of information:
\textsf{data $\to$ scale $\to$ statistic $\to$ aesthetic $\to$ geometry $\to$ coordinate $\to$ ggplot $\to$ theme $\to$ rendered graphic}

\subsection{Data}

The data to be plotted must be available as a \code{data.frame} or \code{tibble}, with data stored so that each row represents a single observation event, and the columns different values observed in that single event. In other words, as so-called ``tidy data'' as described in Chapter \ref{chap:R:data}. The variables to be plotted can be \code{numeric}, \code{factor}, \code{character}, and time or date stored as \code{POSIXct}.

\subsection{Mapping}

When we design a plot, we need to map data variables to aesthetics\index{plots!aesthetics} (or graphic `properties'). Most plots will have an $x$ dimension, which is considered an aesthetic, and a variable containing numbers mapped to it. The position on a 2D plot of say a point will be determined by $x$ and $y$ aesthetics, while in a 3D plot, three aesthetics need to be mapped $x$, $y$ and $z$. Many aesthetics are not related to coordinates, they are properties, like color, size, shape, line type or even rotation angle, which add an additional dimension on which to represent the values of variables and/or constants.

\subsection{Geometries}

\sloppy%
Geometries\index{plots!geometries} describe the graphics representation of the data: for example, \gggeom{geom\_point()}, plots a `point' or symbol for each observation, while \gggeom{geom\_line()}, draws line segments between observations. Some geometries rely by default on statistics, but most `geoms' default to the identity statistics. Each time a \emph{geometry} is used to add a graphical representation of data to a plot, we say that a new \emph{layer} has been added. The name \emph{layer} reflects the fact that each new layer added is plotted on top of the layers already present in the plot, or rather when a plot is printed the layers will be generated in the order they were added to the ggplot object. For example, one layer in a plot can display the observations, another layer a regression line fitted to them, and a third one may contain annotations such an equation or a text label.

\subsection{Statistics}

Statistics\index{plots!statistics} are `words' that represent calculation of summaries or some other operation on the values from the data, and these summary values can be plotted with a geometry. For example \ggstat{stat\_smooth()} fits a smoother, and \ggstat{stat\_summary()} applies a summary function. Statistics are applied automatically by group when data has been grouped by mapping additional aesthetics such as color to a factor. When \emph{statistics} are used for a computation, the returned value is passed directly to a \emph{geometry}, and consequently adding an \emph{statistics} also adds a layer to the plot.

\subsection{Scales}

Scales\index{plots!scales} give the relationship between data values and the aesthetic values to be actually plotted. Mapping a variable to the `color' aesthetic only tells that different values stored in the mapped variable will be represented by different colors. A scale, such as \ggscale{scale\_color\_continuous()} will determine which color in the plot corresponds to which value in the variable. The observations falling outside the limits of a scale are ignored rather than passed to the next step (statistics or geometries)---this can easily take place unintentionally when only summaries are included in a plot and a user does not pay attention to warning messages. Scales can also define transformations on the data, which are used to map data values to values to be plotted, while retaining the original values for tick labels.  Scales are used for continuous variables, such as numbers, and for categorical ones such as factors.

\subsection{Coordinate systems}

The most frequently used coordinate system\index{plots!coordinates} when plotting data is the cartesian system, which is the default for most \emph{geometries}. In the cartesian system, $x$ and $y$ are represented as distances on two orthogonal (at 90$^\circ$) axes. In the polar system of coordinates, angles around a central point are used instead of distances on a straight line. However, package \pkgname{ggtern} adds a ternary system of coordinates, to allow the extension of the grammar to allow the construction of ternary plots. Setting limits to a coordinate system changes the region of the plotting space visible in the plot, but does not discard observations. In other words, when using \emph{statistics}, observations located outside the coordinate limits---i.e.\ not visible in the rendered plot---, will still be included in computations.

\subsection{Themes}

How the plots look when displayed or printed can be altered by means of themes\index{plots!themes}. A plot can be saved without adding a theme and then printed or displayed using different themes. Also individual theme elements can be changed, and whole new themes defined. This adds a lot of flexibility and helps in the separation of the data representation aspects from those related to the graphical design.

\subsection{Building a plot}

We have described above the components of the grammar of graphics: aesthetics (\code{aes}) as for example color, geometric elements \code{geom\_\ldots} such as lines, and points, statistics \code{stat\_\ldots}, scales \code{scale\_\ldots}, \code{coordinate} systems and themes \code{theme\_\ldots}. In this section we will see how plots can assembled from these elements.

When we do not explicitly add these elements, default elements will be used, in some cases defaults that result in empty plots.

As the workings and use of the grammar are easier to show by example than to explain with words, we will show how to build plots of increasing complexity. We start with the simplest possible plot, an empty plot. We use function \code{ggplot()} to create the skeleton for a plot.

\begin{knitrout}\footnotesize
\definecolor{shadecolor}{rgb}{0.969, 0.969, 0.969}\color{fgcolor}\begin{kframe}
\begin{alltt}
\hlkwd{ggplot}\hlstd{()}
\end{alltt}
\end{kframe}

{\centering \includegraphics[width=.7\textwidth]{figure/pos-ggplot-basics-01-1} 

}



\end{knitrout}

The plot above is of little use, without any data, so we next pass a data frame object, in this case \code{mtcars}---\code{mtcars} is a data set included in \Rlang; to learn more about this data set, type \code{help("mtcars")} at the \Rlang command prompt.

\begin{knitrout}\footnotesize
\definecolor{shadecolor}{rgb}{0.969, 0.969, 0.969}\color{fgcolor}\begin{kframe}
\begin{alltt}
\hlkwd{ggplot}\hlstd{(}\hlkwc{data} \hlstd{= mtcars)}
\end{alltt}
\end{kframe}

{\centering \includegraphics[width=.7\textwidth]{figure/pos-ggplot-basics-02-1} 

}



\end{knitrout}

Once the data are available, we need to \emph{map} the quantities in the data onto graphical features in the plot, or \emph{aesthetics}. When plotting in two dimensions, we need to map variables in the data to at least the $x$ and $y$ aesthetics. This mapping can be seen in the chunk below by its effect on the plotting area ranges that now match the ranges (plus a margin) of the mapped variables. The axis labels also reflect the names of the mapped variables, however, there is no graphical element yet displayed for the individual observations.

\begin{knitrout}\footnotesize
\definecolor{shadecolor}{rgb}{0.969, 0.969, 0.969}\color{fgcolor}\begin{kframe}
\begin{alltt}
\hlkwd{ggplot}\hlstd{(}\hlkwc{data} \hlstd{= mtcars,}
       \hlkwd{aes}\hlstd{(}\hlkwc{x} \hlstd{= disp,} \hlkwc{y} \hlstd{= mpg))}
\end{alltt}
\end{kframe}

{\centering \includegraphics[width=.7\textwidth]{figure/pos-ggplot-basics-03-1} 

}



\end{knitrout}

To make observations visible in the plot we need to add a suitable \emph{geometry} or \code{geom} to the plot. Here we display the observations as points using \gggeom{geom\_point()}

\begin{knitrout}\footnotesize
\definecolor{shadecolor}{rgb}{0.969, 0.969, 0.969}\color{fgcolor}\begin{kframe}
\begin{alltt}
\hlkwd{ggplot}\hlstd{(}\hlkwc{data} \hlstd{= mtcars,}
       \hlkwd{aes}\hlstd{(}\hlkwc{x} \hlstd{= disp,} \hlkwc{y} \hlstd{= mpg))} \hlopt{+}
  \hlkwd{geom_point}\hlstd{()}
\end{alltt}
\end{kframe}

{\centering \includegraphics[width=.7\textwidth]{figure/pos-ggplot-basics-04-1} 

}



\end{knitrout}

\begin{warningbox}
  In the examples above the plots were printed automatically, which is the default at the R console. However, as any other R objects, ggplots can be given a name, stored,
\begin{knitrout}\footnotesize
\definecolor{shadecolor}{rgb}{0.969, 0.969, 0.969}\color{fgcolor}\begin{kframe}
\begin{alltt}
\hlstd{p} \hlkwb{<-} \hlkwd{ggplot}\hlstd{(}\hlkwc{data} \hlstd{= mtcars,}
            \hlkwd{aes}\hlstd{(}\hlkwc{x} \hlstd{= disp,} \hlkwc{y} \hlstd{= mpg))} \hlopt{+}
       \hlkwd{geom_point}\hlstd{()}
\end{alltt}
\end{kframe}
\end{knitrout}

and printed at a later time.
\begin{knitrout}\footnotesize
\definecolor{shadecolor}{rgb}{0.969, 0.969, 0.969}\color{fgcolor}\begin{kframe}
\begin{alltt}
\hlkwd{print}\hlstd{(p)}
\end{alltt}
\end{kframe}
\end{knitrout}
\end{warningbox}

\begin{advplayground}
Above we have seen how to build a plot, layer by layer, using the grammar of graphics. We have also seen how to save a ggplot. We can peep into the innards of this object using \code{summary()}.
\begin{knitrout}\footnotesize
\definecolor{shadecolor}{rgb}{0.969, 0.969, 0.969}\color{fgcolor}\begin{kframe}
\begin{alltt}
\hlkwd{summary}\hlstd{(p)}
\end{alltt}
\end{kframe}
\end{knitrout}
We can also obtain view into the structure by means of \code{str()}.

Package \pkgname{gginnards} provides methods \code{str()}, \code{num\_layers()}, \code{top\_layer()} and  \code{mapped\_vars()}. Use these methods to explore ggplot objects with different numbers of layers or mappings. You will see that the plot elements that were added to the plot are stored as members of a list with nested lists conforming a tree-like structure.
\end{advplayground}

Although \emph{aesthetics} can be mapped to variables in the data, they can also be set to constant values, but only within layers, not as whole-plot defaults.

\begin{knitrout}\footnotesize
\definecolor{shadecolor}{rgb}{0.969, 0.969, 0.969}\color{fgcolor}\begin{kframe}
\begin{alltt}
\hlkwd{ggplot}\hlstd{(}\hlkwc{data} \hlstd{= mtcars,}
       \hlkwd{aes}\hlstd{(}\hlkwc{x} \hlstd{= disp,} \hlkwc{y} \hlstd{= mpg))} \hlopt{+}
  \hlkwd{geom_point}\hlstd{(}\hlkwc{colour} \hlstd{=} \hlstr{"red"}\hlstd{,} \hlkwc{shape} \hlstd{=} \hlstr{"square"}\hlstd{)}
\end{alltt}
\end{kframe}

{\centering \includegraphics[width=.7\textwidth]{figure/pos-ggplot-basics-04a-1} 

}



\end{knitrout}

While a geometry directly constructs a graphical representation of the observations in the data, a \emph{statistics} or \code{stat} ``sits'' in-between the data and a \code{geom}, applying some computation, usually but not always, to produce a statistical summary of the data. Here we add a fitted line using \code{stat\_smooth()} with its output added to the plot using \gggeom{geom\_line()} passed by name with \code{"line"} as argument to \code{stat\_smooth}. We fit a linear regression, using \code{lm()} passed as method.

\begin{knitrout}\footnotesize
\definecolor{shadecolor}{rgb}{0.969, 0.969, 0.969}\color{fgcolor}\begin{kframe}
\begin{alltt}
\hlkwd{ggplot}\hlstd{(}\hlkwc{data} \hlstd{= mtcars,}
       \hlkwd{aes}\hlstd{(}\hlkwc{x} \hlstd{= disp,} \hlkwc{y} \hlstd{= mpg))} \hlopt{+}
  \hlkwd{geom_point}\hlstd{()} \hlopt{+}
  \hlkwd{stat_smooth}\hlstd{(}\hlkwc{geom} \hlstd{=} \hlstr{"line"}\hlstd{,} \hlkwc{method} \hlstd{=} \hlstr{"lm"}\hlstd{)}
\end{alltt}
\end{kframe}

{\centering \includegraphics[width=.7\textwidth]{figure/pos-ggplot-basics-05-1} 

}



\end{knitrout}

We haven't added yet some of the elements of the grammar described above: \emph{scales}, \emph{coordinates} and \emph{themes}. The plots were anyway rendered because these elements have defaults which are used when we do not set them explicitly. We next will see examples in which they are explicitly set. We start with a scale using a logarithmic transformation. This works like plotting by hand using graph paper with rulings spaced according to a logarithmic scale. Tick marks, continue being expressed in the original units, but statistics are applied to the transformed data. In other words, a transformed scale affects the values before they are passed to \emph{statistics}, and the linear regression will be fitted to \code{log10()} transformed $y$ values and the original $x$ values.

\begin{knitrout}\footnotesize
\definecolor{shadecolor}{rgb}{0.969, 0.969, 0.969}\color{fgcolor}\begin{kframe}
\begin{alltt}
\hlkwd{ggplot}\hlstd{(}\hlkwc{data} \hlstd{= mtcars,}
       \hlkwd{aes}\hlstd{(}\hlkwc{x} \hlstd{= disp,} \hlkwc{y} \hlstd{= mpg))} \hlopt{+}
  \hlkwd{geom_point}\hlstd{()} \hlopt{+}
  \hlkwd{stat_smooth}\hlstd{(}\hlkwc{geom} \hlstd{=} \hlstr{"line"}\hlstd{,} \hlkwc{method} \hlstd{=} \hlstr{"lm"}\hlstd{)} \hlopt{+}
  \hlkwd{scale_y_log10}\hlstd{()}
\end{alltt}
\end{kframe}

{\centering \includegraphics[width=.7\textwidth]{figure/pos-ggplot-basics-06-1} 

}



\end{knitrout}

The range limits of a scale can be set manually, instead of automatically as by default. Scales limits function as \emph{a window into the data}: observations outside the scale limits are not mapped---these observations are not included in the graphical representation. Furthermore, when using \code{stats} the computations are only applied to observations that fall within the limits. These limits \emph{indirectly} also affect the plotting area when the plotting area is automatically set based on the range of the (within limits) data---even the range of values mapped to other aesthetics may change when the data are subset.

In contrast to \emph{scale limits}, \emph{coordinates} function as a \emph{zoomed view} into the plotting area, and do not affect which observations are used by \code{stats}. The coordinate system, as expected, is also determined by this grammar element---here we use cartesian coordinates which are the default, but we manually set $y$ limits.

\begin{knitrout}\footnotesize
\definecolor{shadecolor}{rgb}{0.969, 0.969, 0.969}\color{fgcolor}\begin{kframe}
\begin{alltt}
\hlkwd{ggplot}\hlstd{(}\hlkwc{data} \hlstd{= mtcars,}
       \hlkwd{aes}\hlstd{(}\hlkwc{x} \hlstd{= disp,} \hlkwc{y} \hlstd{= mpg))} \hlopt{+}
  \hlkwd{geom_point}\hlstd{()} \hlopt{+}
  \hlkwd{stat_smooth}\hlstd{(}\hlkwc{geom} \hlstd{=} \hlstr{"line"}\hlstd{,} \hlkwc{method} \hlstd{=} \hlstr{"lm"}\hlstd{)} \hlopt{+}
  \hlkwd{coord_cartesian}\hlstd{(}\hlkwc{ylim} \hlstd{=} \hlkwd{c}\hlstd{(}\hlnum{15}\hlstd{,} \hlnum{25}\hlstd{))}
\end{alltt}
\end{kframe}

{\centering \includegraphics[width=.7\textwidth]{figure/pos-ggplot-basics-07-1} 

}



\end{knitrout}

The next example uses a coordinate system transformation. When the transformation is applied to the coordinate system, it affects only the plotting---it sits between the geom and the rendering of the plot. The transformation is applied to the values returned by any \emph{statistics}. The straight line fitted is plotted on the transformed coordinates as a curve, because the model was fit to the untransformed data, this fitted model automatically used to obtain the predicted values needed to plot the line and the transformation applied to them before rendering the plot. We have here described only cartesian coordinate systems while other coordinate systems are described in sections \ref{sec:plot:sf} and \ref{sec:plot:circular} on pages \pageref{sec:plot:sf} and \pageref{sec:plot:circular}, respectively.

\begin{knitrout}\footnotesize
\definecolor{shadecolor}{rgb}{0.969, 0.969, 0.969}\color{fgcolor}\begin{kframe}
\begin{alltt}
\hlkwd{ggplot}\hlstd{(}\hlkwc{data} \hlstd{= mtcars,}
       \hlkwd{aes}\hlstd{(}\hlkwc{x} \hlstd{= disp,} \hlkwc{y} \hlstd{= mpg))} \hlopt{+}
  \hlkwd{geom_point}\hlstd{()} \hlopt{+}
  \hlkwd{stat_smooth}\hlstd{(}\hlkwc{geom} \hlstd{=} \hlstr{"line"}\hlstd{,} \hlkwc{method} \hlstd{=} \hlstr{"lm"}\hlstd{)} \hlopt{+}
  \hlkwd{coord_trans}\hlstd{(}\hlkwc{y} \hlstd{=} \hlstr{"log10"}\hlstd{)}
\end{alltt}
\end{kframe}

{\centering \includegraphics[width=.7\textwidth]{figure/pos-ggplot-basics-08-1} 

}



\end{knitrout}

Themes affect the rendering of plots at the time of printing---they can be thought as style sheets defining the graphic design. A complete theme can override the default gray theme. The plot is the same, the observations are represented in the same way, the limits of the axes are the same and all text is the same. On the other hand how these elements are rendered by different themes can be drastically different.

\begin{knitrout}\footnotesize
\definecolor{shadecolor}{rgb}{0.969, 0.969, 0.969}\color{fgcolor}\begin{kframe}
\begin{alltt}
\hlkwd{ggplot}\hlstd{(}\hlkwc{data} \hlstd{= mtcars,}
       \hlkwd{aes}\hlstd{(}\hlkwc{x} \hlstd{= disp,} \hlkwc{y} \hlstd{= mpg))} \hlopt{+}
  \hlkwd{geom_point}\hlstd{()} \hlopt{+}
  \hlkwd{theme_classic}\hlstd{()}
\end{alltt}
\end{kframe}

{\centering \includegraphics[width=.7\textwidth]{figure/pos-ggplot-basics-09-1} 

}



\end{knitrout}

We can also override the base font size and font family. This affects the size of all text elements, as their size is defined relative to the base size. Here we add the same theme as used in the previous examples, but with a different base point size for text.

\begin{knitrout}\footnotesize
\definecolor{shadecolor}{rgb}{0.969, 0.969, 0.969}\color{fgcolor}\begin{kframe}
\begin{alltt}
\hlkwd{ggplot}\hlstd{(}\hlkwc{data} \hlstd{= mtcars,}
       \hlkwd{aes}\hlstd{(}\hlkwc{x} \hlstd{= disp,} \hlkwc{y} \hlstd{= mpg))} \hlopt{+}
  \hlkwd{geom_point}\hlstd{()} \hlopt{+}
  \hlkwd{theme_classic}\hlstd{(}\hlnum{20}\hlstd{,} \hlkwc{base_family} \hlstd{=} \hlstr{"serif"}\hlstd{)}
\end{alltt}
\end{kframe}

{\centering \includegraphics[width=.7\textwidth]{figure/pos-ggplot-basics-10-1} 

}



\end{knitrout}

The details of how to set axis labels, tick positions and tick labels will be discussed in depth in section \ref{sec:plot:scales}. Meanwhile, we will use function \code{labs()} which is a convenience function allowing to easily set the title and subtitle of a plot and to replace the default \code{name} of scales---this default is the name of the mapped variable, as we saw in the examples above and is used for labels. When setting the \code{name} of scales with \code{labs()}, we use as parameters the names of aesthetics and pass as argument a character string, or an R expression. Here we use \code{x} and \code{y}, the names of two \emph{aesthetics} to which we have mapped two variables in \code{data}, \code{disp} and \code{mpg}.

\begin{knitrout}\footnotesize
\definecolor{shadecolor}{rgb}{0.969, 0.969, 0.969}\color{fgcolor}\begin{kframe}
\begin{alltt}
\hlkwd{ggplot}\hlstd{(}\hlkwc{data} \hlstd{= mtcars,}
       \hlkwd{aes}\hlstd{(}\hlkwc{x} \hlstd{= disp,} \hlkwc{y} \hlstd{= mpg))} \hlopt{+}
  \hlkwd{geom_point}\hlstd{()} \hlopt{+}
  \hlkwd{labs}\hlstd{(}\hlkwc{x} \hlstd{=} \hlstr{"Engine displacement (cubic inches)"}\hlstd{,}
       \hlkwc{y} \hlstd{=} \hlstr{"Fuel use efficiency\textbackslash{}n(miles per gallon)"}\hlstd{,}
       \hlkwc{title} \hlstd{=} \hlstr{"Motor Trend Car Road Tests"}\hlstd{,}
       \hlkwc{subtitle} \hlstd{=} \hlstr{"Source: 1974 Motor Trend US magazine"}\hlstd{)}
\end{alltt}
\end{kframe}

{\centering \includegraphics[width=.7\textwidth]{figure/pos-ggplot-basics-11-1} 

}



\end{knitrout}

\begin{infobox}
As elsewhere in \Rlang, when a value is expected, either a value stored in a variable or an statement returning a suitable value can be passed as an argument to be mapped to an \emph{aesthetic}. In other words, the values to be plotted do not need to be stored in the data frame passed as argument to parameter \code{data} as a variable, they can also be computed from these variables. Here we plot miles-per-gallon, \code{mpg} on the engine displacement per cylinder by dividing \code{disp} by \code{cyl} within the call to \code{aes()}.

\begin{knitrout}\footnotesize
\definecolor{shadecolor}{rgb}{0.969, 0.969, 0.969}\color{fgcolor}\begin{kframe}
\begin{alltt}
\hlkwd{ggplot}\hlstd{(}\hlkwc{data} \hlstd{= mtcars,} \hlkwd{aes}\hlstd{(}\hlkwc{x} \hlstd{= disp} \hlopt{/} \hlstd{cyl,} \hlkwc{y} \hlstd{= mpg))} \hlopt{+}
  \hlkwd{geom_point}\hlstd{()}
\end{alltt}
\end{kframe}

{\centering \includegraphics[width=.7\textwidth]{figure/pos-ggplot-basics-info-01-1} 

}



\end{knitrout}

\end{infobox}

Each of the elements of the grammar exemplified above has several different members, and many of the individual \emph{geometries} and \emph{statistics} accept arguments that can be used to modify their behaviour. There are also more \emph{aesthetics} than those shown above. Multiple data objects as well as multiple mappings can coexist within a single ggplot object. Packages and user code can define new \emph{geometries}, \emph{statistics} and even \emph{aesthetics}. Individual elements in a theme can be also modified and new complete themes created, re-used and shared. We will describe in the remaining sections of this chapter how to use the grammar of graphics to construct other types of graphical presentations including more complex plots than those in the examples above. We will start by seeing what are the consequences of \code{"gg"} objects being regular \Rlang objects and of graphical rendering being a separate step from the creation of the plot object.

\subsection{Data and mappings in layers}

In the case of simple plots, based on data contained in a single data frame, the usual style is to code a plot as described above, passing an argument, \code{mtcars} in these examples, to the \code{data} parameter of \Rfunction{ggplot()}. Data passed in this way becomes the default for all layers in the plot. The same applies to the argument passed to \code{mapping}.

\begin{knitrout}\footnotesize
\definecolor{shadecolor}{rgb}{0.969, 0.969, 0.969}\color{fgcolor}\begin{kframe}
\begin{alltt}
\hlkwd{ggplot}\hlstd{(}\hlkwc{data} \hlstd{= mtcars,}
       \hlkwc{mapping} \hlstd{=} \hlkwd{aes}\hlstd{(}\hlkwc{x} \hlstd{= disp,} \hlkwc{y} \hlstd{= mpg))} \hlopt{+}
  \hlkwd{geom_point}\hlstd{()}
\end{alltt}
\end{kframe}
\end{knitrout}

However, the grammar of graphics contemplates the possibility of data and mappings restricted to individual layers. In which case the those passed as arguments to \Rfunction{ggplot()}, if present,  are overridden by arguments passed to individual layers. Making it possible to code the same plot as follows.

\begin{knitrout}\footnotesize
\definecolor{shadecolor}{rgb}{0.969, 0.969, 0.969}\color{fgcolor}\begin{kframe}
\begin{alltt}
\hlkwd{ggplot}\hlstd{()} \hlopt{+}
  \hlkwd{geom_point}\hlstd{(}\hlkwc{data} \hlstd{= mtcars,}
             \hlkwc{mapping} \hlstd{=} \hlkwd{aes}\hlstd{(}\hlkwc{x} \hlstd{= disp,} \hlkwc{y} \hlstd{= mpg))}
\end{alltt}
\end{kframe}
\end{knitrout}

The default mapping can also be added directly with the \code{+} operator, instead of being passed as an argument to \Rfunction{ggplot()}.
\begin{knitrout}\footnotesize
\definecolor{shadecolor}{rgb}{0.969, 0.969, 0.969}\color{fgcolor}\begin{kframe}
\begin{alltt}
\hlkwd{ggplot}\hlstd{(}\hlkwc{data} \hlstd{= mtcars)} \hlopt{+}
  \hlkwd{aes}\hlstd{(}\hlkwc{x} \hlstd{= disp,} \hlkwc{y} \hlstd{= mpg)} \hlopt{+}
  \hlkwd{geom_point}\hlstd{()}
\end{alltt}
\end{kframe}
\end{knitrout}

It is possible even to have a default mapping for the whole plot, but no default data.
\begin{knitrout}\footnotesize
\definecolor{shadecolor}{rgb}{0.969, 0.969, 0.969}\color{fgcolor}\begin{kframe}
\begin{alltt}
\hlkwd{ggplot}\hlstd{()} \hlopt{+}
  \hlkwd{aes}\hlstd{(}\hlkwc{x} \hlstd{= disp,} \hlkwc{y} \hlstd{= mpg)} \hlopt{+}
  \hlkwd{geom_point}\hlstd{(}\hlkwc{data} \hlstd{= mtcars)}
\end{alltt}
\end{kframe}
\end{knitrout}

In these examples, the plot remains unchanged, but this flexibility in the grammar allows in plots containing multiple layers, for each layer to use different data or a different mapping.

\begin{explainbox}
The argument passed to parameter \code{data} of a layer function, can be a function instead of a data frame, if the plot contains default data. In this case the function is applied to the default data and must return a data frame containing data to be used in the layer.

\begin{knitrout}\footnotesize
\definecolor{shadecolor}{rgb}{0.969, 0.969, 0.969}\color{fgcolor}\begin{kframe}
\begin{alltt}
\hlkwd{ggplot}\hlstd{(}\hlkwc{data} \hlstd{= mtcars,}
       \hlkwc{mapping} \hlstd{=} \hlkwd{aes}\hlstd{(}\hlkwc{x} \hlstd{= disp,} \hlkwc{y} \hlstd{= mpg))} \hlopt{+}
  \hlkwd{geom_point}\hlstd{(}\hlkwc{size} \hlstd{=} \hlnum{4}\hlstd{)} \hlopt{+}
  \hlkwd{geom_point}\hlstd{(}\hlkwc{data} \hlstd{=} \hlkwa{function}\hlstd{(}\hlkwc{x}\hlstd{)\{}\hlkwd{subset}\hlstd{(x, cyl} \hlopt{==} \hlnum{4}\hlstd{)\},} \hlkwc{colour} \hlstd{=} \hlstr{"yellow"}\hlstd{,} \hlkwc{size} \hlstd{=} \hlnum{1.5}\hlstd{)}
\end{alltt}
\end{kframe}
\end{knitrout}
\end{explainbox}

\subsection{Plots as \Rlang objects}

We can manipulate ggplot objects and their components in the same way as other \Rlang objects. We can operate on them using the operators and methods defined for the \code{"gg"} class they belong to. As shown above we start by saving a ggplot into a variable.

\begin{knitrout}\footnotesize
\definecolor{shadecolor}{rgb}{0.969, 0.969, 0.969}\color{fgcolor}\begin{kframe}
\begin{alltt}
\hlstd{p} \hlkwb{<-} \hlkwd{ggplot}\hlstd{(}\hlkwc{data} \hlstd{= mtcars,}
       \hlkwd{aes}\hlstd{(}\hlkwc{x} \hlstd{= disp,} \hlkwc{y} \hlstd{= mpg))} \hlopt{+}
  \hlkwd{geom_point}\hlstd{()}
\end{alltt}
\end{kframe}
\end{knitrout}

\begin{warningbox}
  The separation of plot construction and rendering is possible, because \code{"gg"} objects are self contained. Most importantly a copy of the data object passed as argument is saved within the plot object. In the example above, \code{p} by itself could be saved to a file on disk and loaded into a clean R session, even on another computer, and rendered as long as package \ggplot and its dependencies are available. Another consequence of a copy of the data being stored in the plot object, is that editing the data used to create a \code{"gg"} object after its creation does \emph{not} affect rendered plots unless we recreate the "gg" object.

  With \code{str()} we can explore the structure of any \Rlang object, including those of class \code{"gg"}. We use \code{max.level = 1} to reduce the length of output, but to see deeper into the nested list you can increase the value passed as argument to \code{max.level} or simply accept its default.

% the next chuck works but it leads to stack overflow in LaTeX
% there is something wrong with how knitr handles the output of str()
% eval_playground
\begin{knitrout}\footnotesize
\definecolor{shadecolor}{rgb}{0.969, 0.969, 0.969}\color{fgcolor}\begin{kframe}
\begin{alltt}
\hlkwd{str}\hlstd{(p,} \hlkwc{max.level} \hlstd{=} \hlnum{1}\hlstd{)}
\end{alltt}
\end{kframe}
\end{knitrout}
\end{warningbox}

When we used in the previous section operator \code{+} to assemble the plots we were operating on ``anonymous'' \Rlang objects. In the same way we can operate on saved or ``named'' objects.

\begin{knitrout}\footnotesize
\definecolor{shadecolor}{rgb}{0.969, 0.969, 0.969}\color{fgcolor}\begin{kframe}
\begin{alltt}
\hlstd{p} \hlopt{+}
  \hlkwd{stat_smooth}\hlstd{(}\hlkwc{geom} \hlstd{=} \hlstr{"line"}\hlstd{,} \hlkwc{method} \hlstd{=} \hlstr{"lm"}\hlstd{)}
\end{alltt}
\end{kframe}

{\centering \includegraphics[width=.7\textwidth]{figure/pos-ggplot-objects-02-1} 

}



\end{knitrout}

\begin{playground}
  Reproduce the examples in the previous section, using \code{p} defined above as a basis instead of building each plot from scratch.
\end{playground}

\begin{infobox}
  In the examples above we have been adding elements one by one, using the \code{+} operator. It is also possible to add multiple components in a single operation using a list. This is useful, when we want to save sets of components in a variable so as to reuse them in multiple plots. This saves typing, ensures consistency and can make alterations to a set of similar plots much easier.

\begin{knitrout}\footnotesize
\definecolor{shadecolor}{rgb}{0.969, 0.969, 0.969}\color{fgcolor}\begin{kframe}
\begin{alltt}
\hlstd{my.layers} \hlkwb{<-} \hlkwd{list}\hlstd{(}
    \hlkwd{stat_smooth}\hlstd{(}\hlkwc{geom} \hlstd{=} \hlstr{"line"}\hlstd{,} \hlkwc{method} \hlstd{=} \hlstr{"lm"}\hlstd{),}
    \hlkwd{scale_x_log10}\hlstd{())}
\end{alltt}
\end{kframe}
\end{knitrout}

\begin{knitrout}\footnotesize
\definecolor{shadecolor}{rgb}{0.969, 0.969, 0.969}\color{fgcolor}\begin{kframe}
\begin{alltt}
\hlstd{p} \hlopt{+} \hlstd{my.layers}
\end{alltt}
\end{kframe}

{\centering \includegraphics[width=.7\textwidth]{figure/pos-ggplot-objects-info-02-1} 

}



\end{knitrout}

\end{infobox}

\section{Geometries}\label{sec:plot:geometries}
\index{plots!geometries|(}

Different geometries support different \emph{aesthetics}. While \gggeom{geom\_point()} supports \code{shape}, and \gggeom{geom\_line()} supports \code{linetype}, both support \code{x}, \code{y}, \code{colour} and \code{size}. In this section we will describe the different \code{geometries} available in package \ggplot and some examples from packages that extend \ggplot. The graphic output from most code examples will not be shown, with the expectation that readers will run them to see the plots.

Mainly for historical reasons, \emph{geometries} accept a \emph{statistic} as argument, in the same way as \emph{statistics} accept a \emph{geometry} as argument. In this section we will only describe \emph{geometries} which have as default \emph{statistic} \code{stat\_identity} which passes values directly as mapped. The \emph{geometries} that have other \emph{statistics} as default are described in section \ref{sec:plot:stat:summaries} together with the corresponding \emph{statistics}.

\subsection{Point}\label{sec:plot:geom:point}
\index{plots!point geometry|(}

As shown earlier in this chapter, \gggeom{geom\_point()}, can be used to add a layer with observations represented by ``points'' or symbols. Variable \code{cyl} describes the numbers of cylinders in the engines of the cars. It is a numeric variable, and when mapped to colour, a continuous colour scale is used to represent this variable.

\index{plots!scatter plot|(}The first examples build scatter plots, because numeric variables are mapped to both \code{x} and \code{y}.
Some scales, like those for \code{colour}, exist in two `flavours', one suitable for numeric variables (continuous) and another for factors (discrete).

\begin{knitrout}\footnotesize
\definecolor{shadecolor}{rgb}{0.969, 0.969, 0.969}\color{fgcolor}\begin{kframe}
\begin{alltt}
\hlkwd{ggplot}\hlstd{(}\hlkwc{data} \hlstd{= mtcars,}
       \hlkwd{aes}\hlstd{(}\hlkwc{x} \hlstd{= disp,} \hlkwc{y} \hlstd{= mpg,} \hlkwc{colour} \hlstd{= cyl))} \hlopt{+}
  \hlkwd{geom_point}\hlstd{()}
\end{alltt}
\end{kframe}

{\centering \includegraphics[width=.7\textwidth]{figure/pos-scatter-01-1} 

}



\end{knitrout}

If we convert \code{cyl} into a factor, a discrete colour scale is used instead of a continuous one.

\begin{knitrout}\footnotesize
\definecolor{shadecolor}{rgb}{0.969, 0.969, 0.969}\color{fgcolor}\begin{kframe}
\begin{alltt}
\hlkwd{ggplot}\hlstd{(}\hlkwc{data} \hlstd{= mtcars,}
       \hlkwd{aes}\hlstd{(}\hlkwc{x} \hlstd{= disp,} \hlkwc{y} \hlstd{= mpg,} \hlkwc{color} \hlstd{=} \hlkwd{factor}\hlstd{(cyl)))} \hlopt{+}
  \hlkwd{geom_point}\hlstd{()}
\end{alltt}
\end{kframe}
\end{knitrout}

If we convert \code{cyl} into an ordered factor, a different discrete colour scale is used by default.

\begin{knitrout}\footnotesize
\definecolor{shadecolor}{rgb}{0.969, 0.969, 0.969}\color{fgcolor}\begin{kframe}
\begin{alltt}
\hlkwd{ggplot}\hlstd{(}\hlkwc{data} \hlstd{= mtcars,}
       \hlkwd{aes}\hlstd{(}\hlkwc{x} \hlstd{= disp,} \hlkwc{y} \hlstd{= mpg,} \hlkwc{color} \hlstd{=} \hlkwd{ordered}\hlstd{(cyl)))} \hlopt{+}
  \hlkwd{geom_point}\hlstd{()}
\end{alltt}
\end{kframe}
\end{knitrout}

\begin{playground}
Try a different mapping: \code{disp} $\rightarrow$ \code{color}, \code{cyl} $\rightarrow$ \code{x}. Continue by using \code{help(mtcars)} and/or \code{names(mtcars)} to see what variables are available, and then try the combinations that trigger your curiosity---i.e.\ explore the data.
\end{playground}

The mapping between data values and aesthetic values is controlled by scales. Different colour scales, and even palettes within a given scale, provide different mappings between data values and rendered colours.

\begin{knitrout}\footnotesize
\definecolor{shadecolor}{rgb}{0.969, 0.969, 0.969}\color{fgcolor}\begin{kframe}
\begin{alltt}
\hlkwd{ggplot}\hlstd{(}\hlkwc{data} \hlstd{= mtcars,}
       \hlkwd{aes}\hlstd{(}\hlkwc{x} \hlstd{= disp,} \hlkwc{y} \hlstd{= mpg,} \hlkwc{color} \hlstd{=} \hlkwd{factor}\hlstd{(cyl)))} \hlopt{+}
  \hlkwd{geom_point}\hlstd{()} \hlopt{+}
  \hlkwd{scale_color_brewer}\hlstd{(}\hlkwc{type} \hlstd{=} \hlstr{"qual"}\hlstd{,} \hlkwc{palette} \hlstd{=} \hlnum{2}\hlstd{)}
\end{alltt}
\end{kframe}
\end{knitrout}

Neither the data, nor the aesthetics mappings or geometries are different than in earlier code; to alter how the plot looks we have changed only the scale and palette used for the color aesthetic. Conceptually it is still exactly the same plot we earlier created, except for the colours used. This is a very important point to understand, because it allows us to separate two different concerns: the semantic structure and the graphic design.

\begin{playground}
Try the different palettes available through the brewer scale. You can play directly with the palettes using function \code{brewer\_pal()} from package \pkgname{scales} together with \code{show\_col()}).
\begin{knitrout}\footnotesize
\definecolor{shadecolor}{rgb}{0.969, 0.969, 0.969}\color{fgcolor}\begin{kframe}
\begin{alltt}
\hlkwd{show_col}\hlstd{(}\hlkwd{brewer_pal}\hlstd{()(}\hlnum{3}\hlstd{))}
\hlkwd{show_col}\hlstd{(}\hlkwd{brewer_pal}\hlstd{(}\hlkwc{type} \hlstd{=} \hlstr{"qual"}\hlstd{,} \hlkwc{palette} \hlstd{=} \hlnum{2}\hlstd{,} \hlkwc{direction} \hlstd{=} \hlnum{1}\hlstd{)(}\hlnum{3}\hlstd{))}
\end{alltt}
\end{kframe}
\end{knitrout}
Once you have found a suitable palette for these data, redo the plot above with the chosen palette.
\end{playground}

When not relying on colors, the most common way of distinguishing groups of observations in scatter plots is to use the \code{shape} of the points as an \emph{aesthetic}. We need to change a single ``word'' in the code statement to achieve this different mapping.

\begin{knitrout}\footnotesize
\definecolor{shadecolor}{rgb}{0.969, 0.969, 0.969}\color{fgcolor}\begin{kframe}
\begin{alltt}
\hlkwd{ggplot}\hlstd{(}\hlkwc{data} \hlstd{= mtcars,} \hlkwd{aes}\hlstd{(}\hlkwc{x} \hlstd{= disp,} \hlkwc{y} \hlstd{= mpg,} \hlkwc{shape} \hlstd{=} \hlkwd{factor}\hlstd{(cyl)))} \hlopt{+}
  \hlkwd{geom_point}\hlstd{()}
\end{alltt}
\end{kframe}
\end{knitrout}

We can use \code{scale\_shape\_manual} to choose each shape to be used. We set three ``open'' shapes that we will see later are very useful as they obey both \code{color} and \code{fill} \emph{aesthetics}.\label{chunk:filled:symbols}

\begin{knitrout}\footnotesize
\definecolor{shadecolor}{rgb}{0.969, 0.969, 0.969}\color{fgcolor}\begin{kframe}
\begin{alltt}
\hlkwd{ggplot}\hlstd{(}\hlkwc{data} \hlstd{= mtcars,} \hlkwd{aes}\hlstd{(}\hlkwc{x} \hlstd{= disp,} \hlkwc{y} \hlstd{= mpg,} \hlkwc{shape} \hlstd{=} \hlkwd{factor}\hlstd{(cyl)))} \hlopt{+}
  \hlkwd{geom_point}\hlstd{()} \hlopt{+}
  \hlkwd{scale_shape_manual}\hlstd{(}\hlkwc{values} \hlstd{=} \hlkwd{c}\hlstd{(}\hlnum{21}\hlstd{,} \hlnum{22}\hlstd{,} \hlnum{23}\hlstd{))}
\end{alltt}
\end{kframe}
\end{knitrout}

It is also possible to use characters as shapes. The character is centred on the position of the observation. As the numbers used as symbols are self-explanatory, we suppress the default guide or key.
\begin{knitrout}\footnotesize
\definecolor{shadecolor}{rgb}{0.969, 0.969, 0.969}\color{fgcolor}\begin{kframe}
\begin{alltt}
\hlkwd{ggplot}\hlstd{(}\hlkwc{data} \hlstd{= mtcars,} \hlkwd{aes}\hlstd{(}\hlkwc{x} \hlstd{= disp,} \hlkwc{y} \hlstd{= mpg,} \hlkwc{shape} \hlstd{=} \hlkwd{factor}\hlstd{(cyl)))} \hlopt{+}
  \hlkwd{geom_point}\hlstd{(}\hlkwc{size} \hlstd{=} \hlnum{2.5}\hlstd{)} \hlopt{+}
  \hlkwd{scale_shape_manual}\hlstd{(}\hlkwc{values} \hlstd{=} \hlkwd{c}\hlstd{(}\hlstr{"4"}\hlstd{,} \hlstr{"6"}\hlstd{,} \hlstr{"8"}\hlstd{),} \hlkwc{guide} \hlstd{=} \hlnum{FALSE}\hlstd{)}
\end{alltt}
\end{kframe}

{\centering \includegraphics[width=.7\textwidth]{figure/pos-scatter-12-1} 

}



\end{knitrout}

\begin{infobox}
  One variable in the data can be mapped to more than one aesthetic allowing redundant aesthetics. This may seem wasteful, but it is extremely useful as it allows one to produce figures that even when produced in color, can still be read if reproduced as black and white images.

\begin{knitrout}\footnotesize
\definecolor{shadecolor}{rgb}{0.969, 0.969, 0.969}\color{fgcolor}\begin{kframe}
\begin{alltt}
\hlkwd{ggplot}\hlstd{(}\hlkwc{data} \hlstd{= mtcars,} \hlkwd{aes}\hlstd{(}\hlkwc{x} \hlstd{= disp,} \hlkwc{y} \hlstd{= mpg,}
                          \hlkwc{shape} \hlstd{=} \hlkwd{factor}\hlstd{(cyl),}
                          \hlkwc{color} \hlstd{=} \hlkwd{factor}\hlstd{(cyl)))} \hlopt{+}
  \hlkwd{geom_point}\hlstd{()}
\end{alltt}
\end{kframe}
\end{knitrout}
\end{infobox}

\index{plots!scatter plot|)}
\index{plots!dot plot|(}Dot plots are similar to scatter plots but a factor is mapped to either the \code{x} or \code{y} \emph{aesthetic}. Dot plots are prone to have overlapping observations, and one way of making these points visible is to make them partly transparent by setting a constant value smaller than one for the \code{alpha} \emph{aesthetic}.

\begin{knitrout}\footnotesize
\definecolor{shadecolor}{rgb}{0.969, 0.969, 0.969}\color{fgcolor}\begin{kframe}
\begin{alltt}
\hlkwd{ggplot}\hlstd{(}\hlkwc{data} \hlstd{= mtcars,} \hlkwd{aes}\hlstd{(}\hlkwc{x} \hlstd{=} \hlkwd{factor}\hlstd{(cyl),} \hlkwc{y} \hlstd{= mpg))} \hlopt{+}
  \hlkwd{geom_point}\hlstd{(}\hlkwc{alpha} \hlstd{=} \hlnum{1}\hlopt{/}\hlnum{3}\hlstd{)}
\end{alltt}
\end{kframe}

{\centering \includegraphics[width=.7\textwidth]{figure/pos-scatter-12a-1} 

}



\end{knitrout}

Instead of making the points semitransparent, we can randomly displace them to avoid overlaps. This is called \emph{jitter}, can be added using \code{position\_jitter()} and the amount of jitter set with \code{width} as a fraction of the distance between adjacent factor levels in the plot.
\begin{knitrout}\footnotesize
\definecolor{shadecolor}{rgb}{0.969, 0.969, 0.969}\color{fgcolor}\begin{kframe}
\begin{alltt}
\hlkwd{ggplot}\hlstd{(}\hlkwc{data} \hlstd{= mtcars,} \hlkwd{aes}\hlstd{(}\hlkwc{x} \hlstd{=} \hlkwd{factor}\hlstd{(cyl),} \hlkwc{y} \hlstd{= mpg))} \hlopt{+}
  \hlkwd{geom_point}\hlstd{(}\hlkwc{position} \hlstd{=} \hlkwd{position_jitter}\hlstd{(}\hlkwc{width} \hlstd{=} \hlnum{0.05}\hlstd{))}
\end{alltt}
\end{kframe}
\end{knitrout}

\index{plots!dot plot|)}
\index{plots!bubble plot|(}We can create a ``bubble'' plot by mapping the \code{size} \emph{aesthetic} to a continuous variable. In this case, one has to think what is visually more meaningful. Although the radius of the shape is frequently mapped, due to how human perception works, mapping a variable to the area of the shape is more useful by being perceptually closer to a linear mapping. For this example we add a new variable to the plot. The weight of the car in tons and map it to the area of the points.

\begin{knitrout}\footnotesize
\definecolor{shadecolor}{rgb}{0.969, 0.969, 0.969}\color{fgcolor}\begin{kframe}
\begin{alltt}
\hlkwd{ggplot}\hlstd{(}\hlkwc{data} \hlstd{= mtcars,} \hlkwd{aes}\hlstd{(}\hlkwc{x} \hlstd{= disp,} \hlkwc{y} \hlstd{= mpg,}
                          \hlkwc{color} \hlstd{=} \hlkwd{factor}\hlstd{(cyl),}
                          \hlkwc{size} \hlstd{= wt))} \hlopt{+}
  \hlkwd{scale_size_area}\hlstd{()} \hlopt{+}
  \hlkwd{geom_point}\hlstd{()}
\end{alltt}
\end{kframe}

{\centering \includegraphics[width=.7\textwidth]{figure/pos-scatter-16-1} 

}



\end{knitrout}

\begin{playground}
If we use a radius-based scale the ``impression'' is different.

\begin{knitrout}\footnotesize
\definecolor{shadecolor}{rgb}{0.969, 0.969, 0.969}\color{fgcolor}\begin{kframe}
\begin{alltt}
\hlkwd{ggplot}\hlstd{(}\hlkwc{data} \hlstd{= mtcars,} \hlkwd{aes}\hlstd{(}\hlkwc{x} \hlstd{= disp,} \hlkwc{y} \hlstd{= mpg,}
                          \hlkwc{color} \hlstd{=} \hlkwd{factor}\hlstd{(cyl),}
                          \hlkwc{size} \hlstd{= wt))} \hlopt{+}
  \hlkwd{scale_size}\hlstd{()} \hlopt{+}
  \hlkwd{geom_point}\hlstd{()}
\end{alltt}
\end{kframe}
\end{knitrout}

Make the plot, look at it carefully. Check the numerical values of some of the weights, and assess if your perception of the plot matches the numbers behind it.
\end{playground}

\index{plots!bubble plot|)}

As a final example summarising the use of \gggeom{geom\_point()} we combine different \emph{aesthetics}, and \emph{scales} in the same scatter plot.

\begin{knitrout}\footnotesize
\definecolor{shadecolor}{rgb}{0.969, 0.969, 0.969}\color{fgcolor}\begin{kframe}
\begin{alltt}
\hlkwd{ggplot}\hlstd{(}\hlkwc{data} \hlstd{= mtcars,} \hlkwd{aes}\hlstd{(}\hlkwc{x} \hlstd{= disp,} \hlkwc{y} \hlstd{= mpg,}
                          \hlkwc{shape} \hlstd{=} \hlkwd{factor}\hlstd{(cyl),}
                          \hlkwc{fill} \hlstd{=} \hlkwd{factor}\hlstd{(cyl),}
                          \hlkwc{size} \hlstd{= wt))} \hlopt{+}
  \hlkwd{geom_point}\hlstd{(}\hlkwc{alpha} \hlstd{=} \hlnum{0.33}\hlstd{,} \hlkwc{color} \hlstd{=} \hlstr{"black"}\hlstd{)} \hlopt{+}
  \hlkwd{scale_size_area}\hlstd{()} \hlopt{+}
  \hlkwd{scale_shape_manual}\hlstd{(}\hlkwc{values} \hlstd{=} \hlkwd{c}\hlstd{(}\hlnum{21}\hlstd{,} \hlnum{22}\hlstd{,} \hlnum{23}\hlstd{))}
\end{alltt}
\end{kframe}

{\centering \includegraphics[width=.7\textwidth]{figure/pos-scatter-18-1} 

}



\end{knitrout}

\begin{playground}
Play with the code in the chunk above. Remove or change each of the mappings and the scale, display the new plot and compare it to the one above. Continue playing with the code until you are sure you understand what each individual element in the code statement creates or controls which graphical element in the plot itself.
\end{playground}
\index{plots!point geometry|(}

It is common to draw error bars together with points representing means or medians of observations and \gggeom{geom\_pointrange()} achieves this based on the values mapped to the \code{x}, \code{y}, \code{ymin} and \code{ymax}, using \code{y} for the position of the point and \code{ymin} and \code{ymax} for the positions of the ends of the line segment representing a range. Two other \emph{geometries}, \gggeom{geom\_range()} and  \gggeom{geom\_errorbar} draw only a segment or a segment with capped ends. They are frequently used together with \emph{statistics} when summaries are calculated on-the-fly, but can be also used directly when the data summaries are stored in a data frame passed as argument to \code{data} (Please, see example on page \pageref{exmpl:plot:errorbar:precalc}).

\subsection{Rug}
\index{plots!rug marging|(}

Rarely rug-plots are used by themselves. Instead they are usually an addition to
scatter plots. An example of the use of \gggeom{geom\_rug()} follows. They make it easier to see the distribution of observations
along the $x$- and $y$-axes.

We generate new fake data by random sampling from the normal distribution. We use \code{set.seed(1234)} to initialize the pseudo-random number generator so that
the same data are generated each time the code is run.

\begin{knitrout}\footnotesize
\definecolor{shadecolor}{rgb}{0.969, 0.969, 0.969}\color{fgcolor}\begin{kframe}
\begin{alltt}
\hlkwd{ggplot}\hlstd{(}\hlkwc{data} \hlstd{= mtcars,}
       \hlkwd{aes}\hlstd{(}\hlkwc{x} \hlstd{= disp,} \hlkwc{y} \hlstd{= mpg,} \hlkwc{color} \hlstd{=} \hlkwd{factor}\hlstd{(cyl)))} \hlopt{+}
  \hlkwd{geom_point}\hlstd{()} \hlopt{+}
  \hlkwd{geom_rug}\hlstd{()}
\end{alltt}
\end{kframe}

{\centering \includegraphics[width=.7\textwidth]{figure/pos-rug-plot-01-1} 

}



\end{knitrout}

\begin{warningbox}
  Rug plots are most useful when the local density of observations is not too high, otherwise rugs become too cluttered and the ``rug threads'' may overlap. When overlap is moderate making the segments semitransparent by setting the \code{alpha} aesthetic to a constant value smaller than one can make the variation in density easier to appreciate. When the number of observations is large, marginal density plots should be preferred.
\end{warningbox}
\index{plots!rug marging|)}

\subsection{Line and area}\label{sec:plot:line}

\index{plots!line geometry|(}
For line plots we use \gggeom{geom\_line()}. The \code{size} of a line is its thickness, and as we had \code{shape} for points, we have \code{linetype} for lines. In a line plot observations in successive rows of the data frame, or the subset corresponding to a group, are joined by straight lines. We use a different data set included in R, Orange, with data on the growth of five orange trees. See the help page for \code{Orange} for details.

\begin{knitrout}\footnotesize
\definecolor{shadecolor}{rgb}{0.969, 0.969, 0.969}\color{fgcolor}\begin{kframe}
\begin{alltt}
\hlkwd{ggplot}\hlstd{(}\hlkwc{data} \hlstd{= Orange,}
       \hlkwd{aes}\hlstd{(}\hlkwc{x} \hlstd{= age,} \hlkwc{y} \hlstd{= circumference,} \hlkwc{color} \hlstd{= Tree))} \hlopt{+}
  \hlkwd{geom_line}\hlstd{()}
\end{alltt}
\end{kframe}

{\centering \includegraphics[width=.7\textwidth]{figure/pos-line-plot-01-1} 

}



\end{knitrout}

\begin{knitrout}\footnotesize
\definecolor{shadecolor}{rgb}{0.969, 0.969, 0.969}\color{fgcolor}\begin{kframe}
\begin{alltt}
\hlkwd{ggplot}\hlstd{(}\hlkwc{data} \hlstd{= Orange,}
       \hlkwd{aes}\hlstd{(}\hlkwc{x} \hlstd{= age,} \hlkwc{y} \hlstd{= circumference,} \hlkwc{linetype} \hlstd{= Tree))} \hlopt{+}
  \hlkwd{geom_line}\hlstd{()}
\end{alltt}
\end{kframe}
\end{knitrout}

Instead of drawing a line joining the successive observations, we may want to draw a disconnected straight-line segment for each observation or row in the data. In this case we use \gggeom{geom\_segment()} which accepts \code{x}, \code{xend}, \code{y} and \code{yend} as mapped aesthetics. \gggeom{geom\_curve()} draws curved lines, and the curvature, control points, and angles can be controlled through additional \emph{aesthetics}. These two \emph{geometries} support arrow heads at their ends. Other \emph{geometries} useful for drawing lines or segments are \gggeom{geom\_path()}, which is similar to \gggeom{geom\_line()} but instead of joining observations according to \code{x} it joins them according to their order in the data, and \gggeom{geom\_spoke()} which is similar to \gggeom{geom\_segment()} but using a polar parametrization, based on \code{x}, \code{y} for origin, and \code{angle} and \code{radius} for the segment. The various other aesthetics that be used with \gggeom{geom\_line()} can be used with \gggeom{geom\_segment()} and \gggeom{geom\_curve()} as well. Finally, \gggeom{geom\_step()} used only vertical and horizontal lines to join the observations, creating a stepped line.

\begin{knitrout}\footnotesize
\definecolor{shadecolor}{rgb}{0.969, 0.969, 0.969}\color{fgcolor}\begin{kframe}
\begin{alltt}
\hlkwd{ggplot}\hlstd{(}\hlkwc{data} \hlstd{= Orange,}
       \hlkwd{aes}\hlstd{(}\hlkwc{x} \hlstd{= age,} \hlkwc{y} \hlstd{= circumference,} \hlkwc{linetype} \hlstd{= Tree))} \hlopt{+}
  \hlkwd{geom_step}\hlstd{()}
\end{alltt}
\end{kframe}

{\centering \includegraphics[width=.7\textwidth]{figure/pos-step-plot-01-1} 

}



\end{knitrout}

\begin{playground}
Using the following toy data make three plots using \code{geom\_line()}, \code{geom\_path()}, and \code{geom\_step} to add a layer.
\begin{knitrout}\footnotesize
\definecolor{shadecolor}{rgb}{0.969, 0.969, 0.969}\color{fgcolor}\begin{kframe}
\begin{alltt}
\hlstd{toy.df} \hlkwb{<-} \hlkwd{data.frame}\hlstd{(}\hlkwc{x} \hlstd{=} \hlkwd{c}\hlstd{(}\hlnum{1}\hlstd{,}\hlnum{3}\hlstd{,}\hlnum{2}\hlstd{,}\hlnum{4}\hlstd{),} \hlkwc{y} \hlstd{=} \hlkwd{c}\hlstd{(}\hlnum{0}\hlstd{,}\hlnum{1}\hlstd{,}\hlnum{0}\hlstd{,}\hlnum{1}\hlstd{))}
\end{alltt}
\end{kframe}
\end{knitrout}
\end{playground}

While \gggeom{geom\_line()} draws a line joining observations, \gggeom{geom\_area()} supports filling the area bellow the drawn line drawn based on the \code{x} and \code{y}, respecting the \code{fill} \emph{aesthetic}. In contrast \gggeom{geom\_ribbon} draws two lines based on the \code{x}, \code{ymin} and \code{ymax} \emph{aesthetics}, with the space between the lines filled according to the \code{fill} \emph{aesthetic}. Finally, \gggeom{geom\_polygom} is similar to \gggeom{geom\_path()} but connects the extreme observations forming a closed polygon that supports \code{fill}.

Much of what was described above for \gggeom{geom\_point} can be adapted to \gggeom{geom\_line}, \gggeom{geom\_ribbon}, area and other geoms described in this section. In some cases, it is useful to stack the areas---e.g.\ when the values represent parts of a bigger whole. In the next, contrived, example, we stack the growth of the different trees by using \code{position = "stack"} instead of the default \code{position = "identity"}. We save the plot for reuse.

\begin{knitrout}\footnotesize
\definecolor{shadecolor}{rgb}{0.969, 0.969, 0.969}\color{fgcolor}\begin{kframe}
\begin{alltt}
\hlkwd{ggplot}\hlstd{(}\hlkwc{data} \hlstd{= Orange,}
       \hlkwd{aes}\hlstd{(}\hlkwc{x} \hlstd{= age,} \hlkwc{y} \hlstd{= circumference,} \hlkwc{fill} \hlstd{= Tree))} \hlopt{+}
  \hlkwd{geom_area}\hlstd{(}\hlkwc{position} \hlstd{=} \hlstr{"stack"}\hlstd{)}
\end{alltt}
\end{kframe}

{\centering \includegraphics[width=.7\textwidth]{figure/pos-area-plot-01-1} 

}



\end{knitrout}

Finally three \emph{geometries} for drawing lines across the whole plotting area: \gggeom{geom\_hline}, \gggeom{geom\_vline} and \gggeom{geom\_abline}. The first two draw horizontal and vertical lines, respectively, while the third one draws straight lines according to the \emph{aesthetics} \code{slope} and \code{intercept} determining the position. The lines drawn with these three geoms extend to the edge of the plotting area.

\gggeom{geom\_hline} and \gggeom{geom\_vline} require a single aesthetic, \code{yintercept} and \code{xintercept}, respectively. Differently to other geoms, the data for these aesthetics can be also passed as constant numeric vectors. The reason for this is that these geoms are most frequently used to annotate plots rather than plotting observations. Let's assume that we want to highlight an event at the age of 1000 days.

\begin{knitrout}\footnotesize
\definecolor{shadecolor}{rgb}{0.969, 0.969, 0.969}\color{fgcolor}\begin{kframe}
\begin{alltt}
\hlkwd{ggplot}\hlstd{(}\hlkwc{data} \hlstd{= Orange,}
       \hlkwd{aes}\hlstd{(}\hlkwc{x} \hlstd{= age,} \hlkwc{y} \hlstd{= circumference,} \hlkwc{fill} \hlstd{= Tree))} \hlopt{+}
  \hlkwd{geom_area}\hlstd{(}\hlkwc{position} \hlstd{=} \hlstr{"stack"}\hlstd{)} \hlopt{+}
  \hlkwd{geom_vline}\hlstd{(}\hlkwc{xintercept} \hlstd{=} \hlnum{1000}\hlstd{,} \hlkwc{colour} \hlstd{=} \hlstr{"gray75"}\hlstd{)} \hlopt{+}
  \hlkwd{geom_vline}\hlstd{(}\hlkwc{xintercept} \hlstd{=} \hlnum{1000}\hlstd{,} \hlkwc{linetype} \hlstd{=} \hlstr{"dotted"}\hlstd{)}
\end{alltt}
\end{kframe}

{\centering \includegraphics[width=.7\textwidth]{figure/pos-area-plot-02-1} 

}



\end{knitrout}

\begin{playground}
  change the order of the three layers in the example above. How did the figure change? What order is best? Would the same order be the best for a scatter plot? and would it be necessary to add two \code{geom\_vline()} layers?
\end{playground}
\index{plots!line plot|)}

\subsection{Column}\label{sec:plot:col}
\index{plots!column plot|(}

The \emph{geometry} \gggeom{geom\_col()} can be used to create \emph{column plots} where each bar represents an observation or case in the data.

\begin{warningbox}
\Rlang users not familiar yet with \ggplot are frequently surprised by the default behaviour of \gggeom{geom\_bar()} as it uses \ggstat{stat\_count()} to produce a histogram, rather than plotting values as is (see section \ref{sec:plot:summaries} on page \pageref{sec:plot:summaries}). \gggeom{geom\_col()} is identical to \gggeom{geom\_bar()} but with \code{"identity"} as default statistic, which passed along the data unaltered.
\end{warningbox}

We create artificial data that we will reuse in multiple variations of the next figure.
\begin{knitrout}\footnotesize
\definecolor{shadecolor}{rgb}{0.969, 0.969, 0.969}\color{fgcolor}\begin{kframe}
\begin{alltt}
\hlkwd{set.seed}\hlstd{(}\hlnum{654321}\hlstd{)}
\hlstd{my.col.data} \hlkwb{<-} \hlkwd{data.frame}\hlstd{(}\hlkwc{treatment} \hlstd{=} \hlkwd{factor}\hlstd{(}\hlkwd{rep}\hlstd{(}\hlkwd{c}\hlstd{(}\hlstr{"A"}\hlstd{,} \hlstr{"B"}\hlstd{,} \hlstr{"C"}\hlstd{),} \hlnum{2}\hlstd{)),}
                          \hlkwc{group} \hlstd{=} \hlkwd{factor}\hlstd{(}\hlkwd{rep}\hlstd{(}\hlkwd{c}\hlstd{(}\hlstr{"male"}\hlstd{,} \hlstr{"female"}\hlstd{),} \hlkwd{c}\hlstd{(}\hlnum{3}\hlstd{,} \hlnum{3}\hlstd{))),}
                          \hlkwc{measurement} \hlstd{=} \hlkwd{rnorm}\hlstd{(}\hlnum{6}\hlstd{)} \hlopt{+} \hlkwd{c}\hlstd{(}\hlnum{5.5}\hlstd{,} \hlnum{5}\hlstd{,} \hlnum{7}\hlstd{))}
\end{alltt}
\end{kframe}
\end{knitrout}

First we plot data for females only, using defaults for all \emph{aesthetics} except $x$ and $y$ which we explicitly map to variables.

\begin{knitrout}\footnotesize
\definecolor{shadecolor}{rgb}{0.969, 0.969, 0.969}\color{fgcolor}\begin{kframe}
\begin{alltt}
\hlkwd{ggplot}\hlstd{(}\hlkwd{subset}\hlstd{(my.col.data, group} \hlopt{==} \hlstr{"female"}\hlstd{),}
       \hlkwd{aes}\hlstd{(}\hlkwc{x} \hlstd{= treatment,} \hlkwc{y} \hlstd{= measurement))} \hlopt{+}
   \hlkwd{geom_col}\hlstd{()}
\end{alltt}
\end{kframe}

{\centering \includegraphics[width=.7\textwidth]{figure/pos-col-plot-02-1} 

}



\end{knitrout}

We play with \emph{aesthetics} to produce a plot with a semi-formal style---e.g.\ suitable for a science popularization article or book. See section \ref{sec:plot:scales} and section \ref{sec:plot:themes} for information on scales and themes, respectively. We set \code{width = 0.5} to make the bars narrower. Setting \code{color = "white"} overrides the default color of the lines bordering the bars.

\begin{knitrout}\footnotesize
\definecolor{shadecolor}{rgb}{0.969, 0.969, 0.969}\color{fgcolor}\begin{kframe}
\begin{alltt}
\hlkwd{ggplot}\hlstd{(my.col.data,} \hlkwd{aes}\hlstd{(}\hlkwc{x} \hlstd{= treatment,} \hlkwc{y} \hlstd{= measurement,} \hlkwc{fill} \hlstd{= group))} \hlopt{+}
     \hlkwd{geom_col}\hlstd{(}\hlkwc{color} \hlstd{=} \hlstr{"white"}\hlstd{,} \hlkwc{width} \hlstd{=} \hlnum{0.5}\hlstd{)} \hlopt{+}
     \hlkwd{scale_fill_grey}\hlstd{()} \hlopt{+} \hlkwd{theme_dark}\hlstd{()}
\end{alltt}
\end{kframe}

{\centering \includegraphics[width=.7\textwidth]{figure/pos-col-plot-03-1} 

}



\end{knitrout}

We next use a formal style, and in addition put the bars side by side by setting \code{position = "dodge"} to override the default \code{position = "stack"}. Setting \code{color = NA} removes the lines bordering the bars.

\begin{knitrout}\footnotesize
\definecolor{shadecolor}{rgb}{0.969, 0.969, 0.969}\color{fgcolor}\begin{kframe}
\begin{alltt}
\hlkwd{ggplot}\hlstd{(my.col.data,} \hlkwd{aes}\hlstd{(}\hlkwc{x} \hlstd{= treatment,} \hlkwc{y} \hlstd{= measurement,} \hlkwc{fill} \hlstd{= group))} \hlopt{+}
     \hlkwd{geom_col}\hlstd{(}\hlkwc{color} \hlstd{=} \hlnum{NA}\hlstd{,} \hlkwc{position} \hlstd{=} \hlstr{"dodge"}\hlstd{)} \hlopt{+}
     \hlkwd{scale_fill_grey}\hlstd{()} \hlopt{+} \hlkwd{theme_classic}\hlstd{()}
\end{alltt}
\end{kframe}

{\centering \includegraphics[width=.7\textwidth]{figure/pos-col-plot-04-1} 

}



\end{knitrout}

\begin{playground}
Change the argument to \code{position}, or let the default be active, until you understand its effect on the figure. What is the difference between \emph{positions} \code{"identity"}, \code{"dodge"} and \code{"stack"}?
\end{playground}

\begin{playground}
Use constants as arguments for \emph{aesthetics} or map variable \code{treatment} to one or more of the \emph{aesthetics} used by \gggeom{geom\_col()}, such as \code{colour}, \code{fill}, \code{linetype}, \code{size}, \code{alpha} and \code{width}.
\end{playground}

\index{plots!column plot|)}

\subsection{Tiles}\label{sec:tileplot}
\index{plots!tile plot|(}
We can draw square or rectangular tiles with \gggeom{geom\_tile()} producing tile plots or simple heat maps.
For the special case of heat maps with marginal cluster trees see section \ref{sec:plot:heatmap} on page \pageref{sec:plot:heatmap}.

We here generate 100 random draws from the $F$ distribution with degrees of freedom $\nu_1 = 5, \nu_2 = 20$.

\begin{knitrout}\footnotesize
\definecolor{shadecolor}{rgb}{0.969, 0.969, 0.969}\color{fgcolor}\begin{kframe}
\begin{alltt}
\hlkwd{set.seed}\hlstd{(}\hlnum{1234}\hlstd{)}
\hlstd{randomf.df} \hlkwb{<-} \hlkwd{data.frame}\hlstd{(}\hlkwc{z} \hlstd{=} \hlkwd{rf}\hlstd{(}\hlnum{100}\hlstd{,} \hlkwc{df1} \hlstd{=} \hlnum{5}\hlstd{,} \hlkwc{df2} \hlstd{=} \hlnum{20}\hlstd{),}
                         \hlkwc{x} \hlstd{=} \hlkwd{rep}\hlstd{(letters[}\hlnum{1}\hlopt{:}\hlnum{10}\hlstd{],} \hlnum{10}\hlstd{),}
                         \hlkwc{y} \hlstd{= LETTERS[}\hlkwd{rep}\hlstd{(}\hlnum{1}\hlopt{:}\hlnum{10}\hlstd{,} \hlkwd{rep}\hlstd{(}\hlnum{10}\hlstd{,} \hlnum{10}\hlstd{))])}
\end{alltt}
\end{kframe}
\end{knitrout}

\gggeom{geom\_tile()} requires aesthetics $x$ and $y$, with no defaults, and \code{width} and \code{height} with defaults that make all tiles of equal size filling the plotting area.

\begin{knitrout}\footnotesize
\definecolor{shadecolor}{rgb}{0.969, 0.969, 0.969}\color{fgcolor}\begin{kframe}
\begin{alltt}
\hlkwd{ggplot}\hlstd{(randomf.df,} \hlkwd{aes}\hlstd{(x, y,} \hlkwc{fill} \hlstd{= z))} \hlopt{+}
  \hlkwd{geom_tile}\hlstd{()}
\end{alltt}
\end{kframe}

{\centering \includegraphics[width=.7\textwidth]{figure/pos-tile-plot-02-1} 

}



\end{knitrout}

We can set \code{color = "grey50"} and \code{size = 1} to make the tile borders more visible as in the example below, or use a contrasting color, to better delineate the borders of the tiles. What to use will depend on whether the individual tiles add meaningful information. In cases like when rows of tiles correspond to individual genes and columns to discrete treatments, contrasting tile borders are preferable. In contrast, in the case when the tiles are an approximation to a continuous surface such as measurements on a regular spatial grid, it is best to suppress the tile borders.

\begin{knitrout}\footnotesize
\definecolor{shadecolor}{rgb}{0.969, 0.969, 0.969}\color{fgcolor}\begin{kframe}
\begin{alltt}
\hlkwd{ggplot}\hlstd{(randomf.df,} \hlkwd{aes}\hlstd{(x, y,} \hlkwc{fill} \hlstd{= z))} \hlopt{+}
  \hlkwd{geom_tile}\hlstd{(}\hlkwc{color} \hlstd{=} \hlstr{"grey50"}\hlstd{,} \hlkwc{size} \hlstd{=} \hlnum{1.33}\hlstd{)}
\end{alltt}
\end{kframe}

{\centering \includegraphics[width=.7\textwidth]{figure/pos-tile-plot-03-1} 

}



\end{knitrout}

\begin{playground}
Play with the arguments passed to parameters \code{color} and \code{size} in the example above, considering what features of the data are most clearly perceived in each of the plots you create.
\end{playground}

Any continuous fill scale can be used to control the appearance. Here we show a tile plot using a grey gradient, with missing values in red.

\begin{knitrout}\footnotesize
\definecolor{shadecolor}{rgb}{0.969, 0.969, 0.969}\color{fgcolor}\begin{kframe}
\begin{alltt}
\hlkwd{ggplot}\hlstd{(randomf.df,} \hlkwd{aes}\hlstd{(x, y,} \hlkwc{fill} \hlstd{= z))} \hlopt{+}
  \hlkwd{geom_tile}\hlstd{(}\hlkwc{color} \hlstd{=} \hlstr{"white"}\hlstd{)} \hlopt{+}
  \hlkwd{scale_fill_gradient}\hlstd{(}\hlkwc{low} \hlstd{=} \hlstr{"grey15"}\hlstd{,} \hlkwc{high} \hlstd{=} \hlstr{"grey85"}\hlstd{,} \hlkwc{na.value} \hlstd{=} \hlstr{"red"}\hlstd{)}
\end{alltt}
\end{kframe}
\end{knitrout}

In contrast to \gggeom{geom\_tile()}, \gggeom{geom\_rect()} draws rectangular tiles based on the position of the corners, mapped to aesthetics \code{xmin}, \code{xmax}, \code{ymin} and \code{ymax}.

\index{plots!tile plot|)}

\subsection{Simple features (sf)}\label{sec:plot:sf}

\ggplot version 3.0.0 or later supports the plotting of shape data similarly as in geographic information systems (GIS) through \gggeom{geom\_sf()} and its companions, \gggeom{geom\_sf\_text()}, \gggeom{geom\_sf\_label()}, and \ggstat{stat\_sf()}. This makes it possible to display data on maps for example using different fill values for different regions. Special \emph{coordinate} \code{coord\_sf()} can be used to select different projections for maps. The \emph{aesthetic} used is called \code{geometry} and contrary to all the other aesthetics we have seen until now, the values to be mapped are of class \code{sfc} containing \emph{simple features} data with multiple components. Manipulation of simple features data is supported by package \pkgname{sf}. This subject exceeds the scope of this book, so a single and very simple example follows.

\begin{knitrout}\footnotesize
\definecolor{shadecolor}{rgb}{0.969, 0.969, 0.969}\color{fgcolor}\begin{kframe}
\begin{alltt}
\hlstd{nc} \hlkwb{<-} \hlstd{sf}\hlopt{::}\hlkwd{st_read}\hlstd{(}\hlkwd{system.file}\hlstd{(}\hlstr{"shape/nc.shp"}\hlstd{,} \hlkwc{package} \hlstd{=} \hlstr{"sf"}\hlstd{),} \hlkwc{quiet} \hlstd{=} \hlnum{TRUE}\hlstd{)}
\hlkwd{ggplot}\hlstd{(nc)} \hlopt{+}
  \hlkwd{geom_sf}\hlstd{(}\hlkwd{aes}\hlstd{(}\hlkwc{fill} \hlstd{= AREA))}
\end{alltt}
\end{kframe}

{\centering \includegraphics[width=.7\textwidth]{figure/pos-sf_plot-01-1} 

}



\end{knitrout}

\subsection{Text}\label{sec:plot:text}
\index{plots!text in|(}
\index{plots!maths in|(}
We can use \gggeom{geom\_text()} or \gggeom{geom\_label()} to add text labels to observations. For \gggeom{geom\_text()} and \gggeom{geom\_label()}, the aesthetic \code{label} provides the text to be plotted and the usual aesthetics \code{x} and \code{y} the location of the labels. As one would expect the \code{color} and \code{size} aesthetics can be also used for the text. In addition \code{angle} and \code{vjust} and \code{hjust} can be used to rotate the label, and adjust its position. The default value of 0.5 for both \code{hjust} and \code{vjust} centres the label positioning the centre of the text at the supplied \code{x} and \code{y} coordinates. `Vertical' and `horizontal' for justification refer to the text, not the plot. This is important when \code{angle} is different from zero. Values smaller than 0.5 shift the label left or down, and values larger than 0.5, right or up. A value of 1 or 0 sets the text so that its edge is at the supplied coordinate. Values outside the range $0\ldots 1$ shift the text even further away, however, still using units based on the length or height of the text string. Recent versions of \pkgname{ggplot2} make possible justification using character constants for alignment: \code{"left"}, \code{"middle"}, \code{"right"}, \code{"bottom"}, \code{"center"} and \code{"top"}, and two special alignments, \code{"inward"} and \code{"outward"} that automatically vary based on the position in the plotting area.

In the case of \gggeom{geom\_label()} the text is enclosed in a box, which obeys the \code{fill} \emph{aesthetic} and takes additional parameters (described starting at page \pageref{start:plot:label}) allowing control on the shape and size of the box. However, it does not support rotation with the \code{angle} aesthetic.

\begin{knitrout}\footnotesize
\definecolor{shadecolor}{rgb}{0.969, 0.969, 0.969}\color{fgcolor}\begin{kframe}
\begin{alltt}
\hlkwd{ggplot}\hlstd{(}\hlkwc{data} \hlstd{= mtcars,} \hlkwd{aes}\hlstd{(}\hlkwc{x} \hlstd{= disp,} \hlkwc{y} \hlstd{= mpg,}
                          \hlkwc{colour} \hlstd{=} \hlkwd{factor}\hlstd{(cyl),}
                          \hlkwc{size} \hlstd{= wt,}
                          \hlkwc{label} \hlstd{= cyl))} \hlopt{+}
  \hlkwd{scale_size}\hlstd{()} \hlopt{+}
  \hlkwd{geom_point}\hlstd{()} \hlopt{+}
  \hlkwd{geom_text}\hlstd{(}\hlkwc{colour} \hlstd{=} \hlstr{"black"}\hlstd{,} \hlkwc{size} \hlstd{=} \hlnum{3}\hlstd{)}
\end{alltt}
\end{kframe}

{\centering \includegraphics[width=.7\textwidth]{figure/pos-text-plot-01-1} 

}



\end{knitrout}

In the remaining examples, with output not shown, we use \gggeom{geom\_text} or \gggeom{geom\_label} together with \gggeom{geom\_point} as they are most frequently used together with this \emph{geometry}.

\begin{knitrout}\footnotesize
\definecolor{shadecolor}{rgb}{0.969, 0.969, 0.969}\color{fgcolor}\begin{kframe}
\begin{alltt}
\hlstd{my.data} \hlkwb{<-}
  \hlkwd{data.frame}\hlstd{(}\hlkwc{x} \hlstd{=} \hlnum{1}\hlopt{:}\hlnum{5}\hlstd{,}
             \hlkwc{y} \hlstd{=} \hlkwd{rep}\hlstd{(}\hlnum{2}\hlstd{,} \hlnum{5}\hlstd{),}
             \hlkwc{label} \hlstd{=} \hlkwd{c}\hlstd{(}\hlstr{"a"}\hlstd{,} \hlstr{"b"}\hlstd{,} \hlstr{"c"}\hlstd{,} \hlstr{"d"}\hlstd{,} \hlstr{"e"}\hlstd{))}

\hlkwd{ggplot}\hlstd{(my.data,} \hlkwd{aes}\hlstd{(x, y,} \hlkwc{label} \hlstd{= label))} \hlopt{+}
  \hlkwd{geom_text}\hlstd{(}\hlkwc{angle} \hlstd{=} \hlnum{45}\hlstd{,} \hlkwc{hjust} \hlstd{=} \hlnum{1.5}\hlstd{,} \hlkwc{size} \hlstd{=} \hlnum{8}\hlstd{)} \hlopt{+}
  \hlkwd{geom_point}\hlstd{()}
\end{alltt}
\end{kframe}
\end{knitrout}

\begin{playground}
Modify the example above to use \gggeom{geom\_label()} instead of \gggeom{geom\_text()} using in addition the \code{fill} aesthetic.
\end{playground}

In the next example we select a different font family, using the same characters in the Roman alphabet.
Base R does not support the use of system fonts in graphics output devices. However, add-on packages allow their use. Of these the simplest to use is package \pkgname{showtext}. The names \code{"sans"} (the default), \code{"serif"} and \code{"mono"} are recognized by all graphics devices on all operating systems. Other fonts like the 35 ``PDF'' fonts are recognised by the \code{pdf()} device and their names can be queried with \code{names(pdfFonts())}.

\begin{knitrout}\footnotesize
\definecolor{shadecolor}{rgb}{0.969, 0.969, 0.969}\color{fgcolor}\begin{kframe}
\begin{alltt}
\hlkwd{ggplot}\hlstd{(my.data,} \hlkwd{aes}\hlstd{(x, y,} \hlkwc{label} \hlstd{= label))} \hlopt{+}
  \hlkwd{geom_text}\hlstd{(}\hlkwc{angle} \hlstd{=} \hlnum{45}\hlstd{,} \hlkwc{hjust} \hlstd{=} \hlnum{1.5}\hlstd{,} \hlkwc{size} \hlstd{=} \hlnum{8}\hlstd{,} \hlkwc{family} \hlstd{=} \hlstr{"serif"}\hlstd{)} \hlopt{+}
  \hlkwd{geom_point}\hlstd{()}
\end{alltt}
\end{kframe}
\end{knitrout}

\begin{playground}
In the examples above the character strings were all of the same length, containing a single character. Redo the plots above with longer character strings of various lengths mapped to the \code{label} \emph{aesthetic}. Do also play with justification of these labels.
\end{playground}

Plotting expressions (mathematical expressions) involves mapping to the \code{label} aesthetic character strings that can be parsed as expressions, and setting \code{parse = TRUE}. We here build the character strings using \Rfunction{paste()} but, of course, they could have also been entered one by one. This use of \Rfunction{paste()} is an example of vectors of different lengths passed as arguments to a function and the shorter ones being recycled.

\begin{knitrout}\footnotesize
\definecolor{shadecolor}{rgb}{0.969, 0.969, 0.969}\color{fgcolor}\begin{kframe}
\begin{alltt}
\hlstd{my.data} \hlkwb{<-}
  \hlkwd{data.frame}\hlstd{(}\hlkwc{x} \hlstd{=} \hlnum{1}\hlopt{:}\hlnum{5}\hlstd{,} \hlkwc{y} \hlstd{=} \hlkwd{rep}\hlstd{(}\hlnum{2}\hlstd{,} \hlnum{5}\hlstd{),}
             \hlkwc{label} \hlstd{=} \hlkwd{paste}\hlstd{(}\hlstr{"alpha["}\hlstd{,} \hlnum{1}\hlopt{:}\hlnum{5}\hlstd{,} \hlstr{"]"}\hlstd{,} \hlkwc{sep} \hlstd{=} \hlstr{""}\hlstd{))}
\hlstd{my.data}\hlopt{$}\hlstd{label}
\end{alltt}
\begin{verbatim}
## [1] alpha[1] alpha[2] alpha[3] alpha[4] alpha[5]
## Levels: alpha[1] alpha[2] alpha[3] alpha[4] alpha[5]
\end{verbatim}
\end{kframe}
\end{knitrout}

\begin{knitrout}\footnotesize
\definecolor{shadecolor}{rgb}{0.969, 0.969, 0.969}\color{fgcolor}\begin{kframe}
\begin{alltt}
\hlkwd{ggplot}\hlstd{(my.data,} \hlkwd{aes}\hlstd{(x, y,} \hlkwc{label} \hlstd{= label))} \hlopt{+}
  \hlkwd{geom_text}\hlstd{(}\hlkwc{hjust} \hlstd{=} \hlopt{-}\hlnum{0.2}\hlstd{,} \hlkwc{parse} \hlstd{=} \hlnum{TRUE}\hlstd{,} \hlkwc{size} \hlstd{=} \hlnum{6}\hlstd{)} \hlopt{+}
  \hlkwd{geom_point}\hlstd{()} \hlopt{+}
  \hlkwd{expand_limits}\hlstd{(}\hlkwc{x} \hlstd{=} \hlnum{5.2}\hlstd{)}
\end{alltt}
\end{kframe}

{\centering \includegraphics[width=.7\textwidth]{figure/pos-text-plot-06-1} 

}



\end{knitrout}

Text and labels do not automatically expand the plotting area past their anchoring coordinates. In the example above we need to use \code{expand\_limits()} to ensure that the text is not clipped at the edge of the plotting area. Plotting maths and other alphabets using \Rlang expressions is discussed in section \ref{sec:plot:plotmath} on page \pageref{sec:plot:plotmath}.

In the examples above we plotted text and expressions present in the data frame passed as argument for \code{data}. It is also possible, and usually preferable, to build suitable labels on-the-fly within \code{aes()} when setting the mapping for \code{label}. Here we use \gggeom{geom\_text()} and with strings parsed on-the-fly into expressions but character strings built in this way can also be used without parsing.

\begin{knitrout}\footnotesize
\definecolor{shadecolor}{rgb}{0.969, 0.969, 0.969}\color{fgcolor}\begin{kframe}
\begin{alltt}
\hlstd{my.data} \hlkwb{<-}
  \hlkwd{data.frame}\hlstd{(}\hlkwc{x} \hlstd{=} \hlnum{1}\hlopt{:}\hlnum{5}\hlstd{,} \hlkwc{y} \hlstd{=} \hlkwd{rep}\hlstd{(}\hlnum{2}\hlstd{,} \hlnum{5}\hlstd{))}

\hlkwd{ggplot}\hlstd{(my.data,} \hlkwd{aes}\hlstd{(x,}
                    \hlstd{y,}
                    \hlkwc{label} \hlstd{=} \hlkwd{paste}\hlstd{(}\hlstr{"alpha["}\hlstd{, x,} \hlstr{"]"}\hlstd{,} \hlkwc{sep} \hlstd{=} \hlstr{""}\hlstd{)))} \hlopt{+}
  \hlkwd{geom_text}\hlstd{(}\hlkwc{hjust} \hlstd{=} \hlopt{-}\hlnum{0.2}\hlstd{,} \hlkwc{parse} \hlstd{=} \hlnum{TRUE}\hlstd{,} \hlkwc{size} \hlstd{=} \hlnum{6}\hlstd{)} \hlopt{+}
  \hlkwd{geom_point}\hlstd{()}
\end{alltt}
\end{kframe}
\end{knitrout}

\begin{playground}
What are the advantages and disadvantages of each approach in relation to making several figures using the same ``labels'' in relation to consistency across figures? Which approach would you prefer if different figures in the same script used different variations of labels derived from the same variables in the data?
\end{playground}

As \gggeom{geom\_label()} obeys the same parameters as \gggeom{geom\_text()} except for \code{angle} we describe below only the additional parameters compared to \gggeom{geom\_text()}.\label{start:plot:label}

\begin{knitrout}\footnotesize
\definecolor{shadecolor}{rgb}{0.969, 0.969, 0.969}\color{fgcolor}\begin{kframe}
\begin{alltt}
\hlstd{my.data} \hlkwb{<-}
  \hlkwd{data.frame}\hlstd{(}\hlkwc{x} \hlstd{=} \hlnum{1}\hlopt{:}\hlnum{5}\hlstd{,} \hlkwc{y} \hlstd{=} \hlkwd{rep}\hlstd{(}\hlnum{2}\hlstd{,} \hlnum{5}\hlstd{),}
             \hlkwc{label} \hlstd{=} \hlkwd{c}\hlstd{(}\hlstr{"one"}\hlstd{,} \hlstr{"two"}\hlstd{,} \hlstr{"three"}\hlstd{,} \hlstr{"four"}\hlstd{,} \hlstr{"five"}\hlstd{))}
\hlkwd{ggplot}\hlstd{(my.data,} \hlkwd{aes}\hlstd{(x, y,} \hlkwc{label} \hlstd{= label))} \hlopt{+}
  \hlkwd{geom_label}\hlstd{(}\hlkwc{hjust} \hlstd{=} \hlopt{-}\hlnum{0.2}\hlstd{)} \hlopt{+}
  \hlkwd{geom_point}\hlstd{()} \hlopt{+}
  \hlkwd{expand_limits}\hlstd{(}\hlkwc{x} \hlstd{=} \hlnum{5.6}\hlstd{)}
\end{alltt}
\end{kframe}

{\centering \includegraphics[width=.7\textwidth]{figure/pos-label-plot-01-1} 

}



\end{knitrout}

We may want to alter the default width of the border line or the color used to \code{fill} the rectangle, or to change the ``roundness'' of the corners. To suppress the border line use \code{label.size = 0}. Corner roundness is controlled by parameter \code{label.r} and the size of the margin around the text with \code{label.padding}.

\begin{knitrout}\footnotesize
\definecolor{shadecolor}{rgb}{0.969, 0.969, 0.969}\color{fgcolor}\begin{kframe}
\begin{alltt}
\hlkwd{ggplot}\hlstd{(my.data,} \hlkwd{aes}\hlstd{(x, y,} \hlkwc{label} \hlstd{= label))} \hlopt{+}
  \hlkwd{geom_label}\hlstd{(}\hlkwc{hjust} \hlstd{=} \hlopt{-}\hlnum{0.2}\hlstd{,} \hlkwc{parse} \hlstd{=} \hlnum{TRUE}\hlstd{,} \hlkwc{size} \hlstd{=} \hlnum{6}\hlstd{,}
             \hlkwc{label.size} \hlstd{=} \hlnum{0L}\hlstd{,}
             \hlkwc{label.r} \hlstd{=} \hlkwd{unit}\hlstd{(}\hlnum{0}\hlstd{,} \hlstr{"lines"}\hlstd{),}
             \hlkwc{label.padding} \hlstd{=} \hlkwd{unit}\hlstd{(}\hlnum{0.15}\hlstd{,} \hlstr{"lines"}\hlstd{),}
             \hlkwc{fill} \hlstd{=} \hlstr{"yellow"}\hlstd{,} \hlkwc{alpha} \hlstd{=} \hlnum{0.5}\hlstd{)} \hlopt{+}
  \hlkwd{geom_point}\hlstd{()} \hlopt{+}
  \hlkwd{expand_limits}\hlstd{(}\hlkwc{x} \hlstd{=} \hlnum{5.6}\hlstd{)}
\end{alltt}
\end{kframe}

{\centering \includegraphics[width=.7\textwidth]{figure/pos-label-plot-02-1} 

}



\end{knitrout}

\begin{playground}
Play with the arguments to the different parameters and with the \emph{aesthetics} to get an idea of what can be done with them. For example, use thicker border lines and increase the padding so that a visually well balanced margin is retained. You may also try mapping the \code{fill} and \code{color} \emph{aesthetics} to factors in the data.
\end{playground}

If the parameter \code{check\_overlap} of \gggeom{geom\_text()} is set to \code{TRUE}, text overlap will be avoided by suppressing the text that would otherwise overlap other text.  \emph{Repulsive} versions of \gggeom{geom\_text} and \gggeom{geom\_label}, \gggeom{geom\_text\_repel} and \gggeom{geom\_label\_repel},  are available in package \pkgname{ggrepel}. These \emph{geometries} avoid overlaps by automatically repositioning the text or labels. Please, read the package documentation for details of how to control the repulsion strength and direction, and the properties of the segments linking the labels to the position of their data coordinates. Nearly all aesthetics supported by \code{geom\_text()} and \code{geom\_label()} are supported by the repulsive versions. However, given that a segment connects the label or text to its anchor point, several properties of these segments can also be controlled with aesthetics or arguments.

\begin{knitrout}\footnotesize
\definecolor{shadecolor}{rgb}{0.969, 0.969, 0.969}\color{fgcolor}\begin{kframe}
\begin{alltt}
\hlkwd{ggplot}\hlstd{(}\hlkwc{data} \hlstd{= mtcars,} \hlkwd{aes}\hlstd{(}\hlkwc{x} \hlstd{= disp,} \hlkwc{y} \hlstd{= mpg,}
                          \hlkwc{colour} \hlstd{=} \hlkwd{factor}\hlstd{(cyl),}
                          \hlkwc{size} \hlstd{= wt,}
                          \hlkwc{label} \hlstd{= cyl))} \hlopt{+}
  \hlkwd{scale_size}\hlstd{()} \hlopt{+}
  \hlkwd{geom_point}\hlstd{(}\hlkwc{alpha} \hlstd{=} \hlnum{1}\hlopt{/}\hlnum{3}\hlstd{)} \hlopt{+}
  \hlkwd{geom_text_repel}\hlstd{(}\hlkwc{colour} \hlstd{=} \hlstr{"black"}\hlstd{,} \hlkwc{size} \hlstd{=} \hlnum{3}\hlstd{,}
                  \hlkwc{min.segment.length} \hlstd{=} \hlnum{0.2}\hlstd{,} \hlkwc{point.padding} \hlstd{=} \hlnum{0.1}\hlstd{)}
\end{alltt}
\end{kframe}

{\centering \includegraphics[width=.7\textwidth]{figure/pos-repel-plot-01-1} 

}



\end{knitrout}

\subsection{Plot insets}\label{sec:plot:insets}

The support for insets in \pkgname{ggplot2} is confined to \code{annotation\_custom()} which was designed to be used for static annotations expected to be the same in each panel of a plot (The use of annotations is described in section \ref{sec:plot:annotate}). Package \pkgname{ggpmisc} provides geoms that mimic \code{geom\_text()} in relation to the \emph{aesthetics} used, but that similarly to \code{geom\_sf()} expect that the column in \code{data} mapped to the \code{label} aesthetics are lists of objects containing multiple pieces of information, rather than atomic vectors. Similarly to \code{geom\_sf()} these geoms do not inherit the plot's default mappings to aesthetics. Three geometries are currently available: \gggeom{geom\_table()}, \gggeom{geom\_plot()} and \gggeom{geom\_grob()}.

\begin{warningbox}
Given that  \gggeom{geom\_table()}, \gggeom{geom\_plot()} and \gggeom{geom\_grob()} will rarely use a mapping inherited from the whole plot, by default they do not inherit it. Either the mapping should be supplied as argument to these functions or their parameter \code{inherit.aes} explicitly set to \code{TRUE}.
\end{warningbox}

The plotting of tables by mapping a list of data frames to the \code{label} \emph{aesthetic} is done with \gggeom{geom\_table}. Positioning, justification and angle work as for \gggeom{geom\_text} and are applied to the whole table. Only \code{tibble} objects (see documentation of package \pkgname{tibble}) can contain as variables lists of data frames, so this \emph{geometry} requires the use of \code{tibble} objects to store the data. The table(s) are created as 'grid' \code{grob} objects, collected in a tree and added to the \code{ggplot} object as a new layer.

We first generate a \code{tibble} containing summaries from the data, formatted as character strings, wrap this tibble in a list and store this list as column in another \code{tibble}. To accomplish this we use functions from the tidyverse described in Chapter \ref{chap:R:data}.

\begin{knitrout}\footnotesize
\definecolor{shadecolor}{rgb}{0.969, 0.969, 0.969}\color{fgcolor}\begin{kframe}
\begin{alltt}
\hlstd{mtcars} \hlopt
  \hlkwd{group_by}\hlstd{(cyl)} \hlopt
  \hlkwd{summarize}\hlstd{(}\hlstr{"mean wt"} \hlstd{=} \hlkwd{format}\hlstd{(}\hlkwd{mean}\hlstd{(wt),} \hlkwc{digits} \hlstd{=} \hlnum{2}\hlstd{),}
            \hlstr{"mean disp"} \hlstd{=} \hlkwd{format}\hlstd{(}\hlkwd{mean}\hlstd{(disp),} \hlkwc{digits} \hlstd{=} \hlnum{0}\hlstd{),}
            \hlstr{"mean mpg"} \hlstd{=} \hlkwd{format}\hlstd{(}\hlkwd{mean}\hlstd{(mpg),} \hlkwc{digits} \hlstd{=} \hlnum{0}\hlstd{))} \hlkwb{->} \hlstd{my.table}
\hlstd{table.tb} \hlkwb{<-} \hlkwd{tibble}\hlstd{(}\hlkwc{x} \hlstd{=} \hlnum{500}\hlstd{,} \hlkwc{y} \hlstd{=} \hlnum{35}\hlstd{,} \hlkwc{table.inset} \hlstd{=} \hlkwd{list}\hlstd{(my.table))}
\end{alltt}
\end{kframe}
\end{knitrout}

\begin{knitrout}\footnotesize
\definecolor{shadecolor}{rgb}{0.969, 0.969, 0.969}\color{fgcolor}\begin{kframe}
\begin{alltt}
\hlkwd{ggplot}\hlstd{(}\hlkwc{data} \hlstd{= mtcars,} \hlkwd{aes}\hlstd{(}\hlkwc{x} \hlstd{= disp,} \hlkwc{y} \hlstd{= mpg,}
                          \hlkwc{colour} \hlstd{=} \hlkwd{factor}\hlstd{(cyl),}
                          \hlkwc{size} \hlstd{= wt,}
                          \hlkwc{label} \hlstd{= cyl))} \hlopt{+}
  \hlkwd{scale_size}\hlstd{()} \hlopt{+}
  \hlkwd{geom_point}\hlstd{()} \hlopt{+}
  \hlkwd{geom_table}\hlstd{(}\hlkwc{data} \hlstd{= table.tb,}
             \hlkwd{aes}\hlstd{(}\hlkwc{x} \hlstd{= x,} \hlkwc{y} \hlstd{= y,} \hlkwc{label} \hlstd{= table.inset),}
             \hlkwc{colour} \hlstd{=} \hlstr{"black"}\hlstd{,} \hlkwc{size} \hlstd{=} \hlnum{3}\hlstd{,} \hlkwc{hjust} \hlstd{=} \hlnum{1}\hlstd{,} \hlkwc{vjust} \hlstd{=} \hlnum{1}\hlstd{)}
\end{alltt}
\end{kframe}

{\centering \includegraphics[width=.7\textwidth]{figure/pos-table-plot-02-1} 

}



\end{knitrout}

The \code{color} and \code{size} aesthetics control the text in the table(s) as a whole.
It is also possible to rotate the table(s) using \code{angle}. As with text labels, justification is interpreted in relation to table-text orientation. We set the \code{y = 0} in \code{data.tb} and then use \code{vjust = 1} to position the top of the table at this coordinate value.

\begin{knitrout}\footnotesize
\definecolor{shadecolor}{rgb}{0.969, 0.969, 0.969}\color{fgcolor}\begin{kframe}
\begin{alltt}
\hlkwd{ggplot}\hlstd{(}\hlkwc{data} \hlstd{= mtcars,} \hlkwd{aes}\hlstd{(}\hlkwc{x} \hlstd{= disp,} \hlkwc{y} \hlstd{= mpg,} \hlkwc{colour} \hlstd{=} \hlkwd{factor}\hlstd{(cyl)))} \hlopt{+}
  \hlkwd{geom_point}\hlstd{()} \hlopt{+}
  \hlkwd{geom_table}\hlstd{(}\hlkwc{data} \hlstd{= table.tb,}
             \hlkwd{aes}\hlstd{(}\hlkwc{x} \hlstd{= x,} \hlkwc{y} \hlstd{= y,} \hlkwc{label} \hlstd{= table.inset),}
             \hlkwc{colour} \hlstd{=} \hlstr{"blue"}\hlstd{,} \hlkwc{size} \hlstd{=} \hlnum{3}\hlstd{,}
             \hlkwc{hjust} \hlstd{=} \hlnum{1}\hlstd{,} \hlkwc{vjust} \hlstd{=} \hlnum{0}\hlstd{,} \hlkwc{angle} \hlstd{=} \hlnum{90}\hlstd{)}
\end{alltt}
\end{kframe}
\end{knitrout}

Parsed text, using R's \emph{plotmath} syntax is supported in the table, with fallback to plain text in case of parsing errors, on a cell by cell basis. We end this section with a simple example, which even if not very useful demonstrates that \gggeom{geom\_table()} behaves like a ``normal'' ggplot \emph{geometry} and that a table can be the only layer in a ggplot if desired. The addition of multiple tables with a single call to \gggeom{geom\_table()} by passing a \code{tibble} with multiple rows as argument for \code{data} is also possible.

\begin{knitrout}\footnotesize
\definecolor{shadecolor}{rgb}{0.969, 0.969, 0.969}\color{fgcolor}\begin{kframe}
\begin{alltt}
\hlstd{tb.pm} \hlkwb{<-} \hlkwd{tibble}\hlstd{(}\hlstr{'x^0'} \hlstd{=} \hlnum{1}\hlstd{,}
                \hlstr{'x^1'} \hlstd{=} \hlnum{1}\hlopt{:}\hlnum{5}\hlstd{,}
                \hlstr{'x^2'} \hlstd{= (}\hlnum{1}\hlopt{:}\hlnum{5}\hlstd{)}\hlopt{^}\hlnum{2}\hlstd{,}
                \hlstr{'x^3'} \hlstd{= (}\hlnum{1}\hlopt{:}\hlnum{5}\hlstd{)}\hlopt{^}\hlnum{3}\hlstd{)}
\hlstd{data.tb} \hlkwb{<-} \hlkwd{tibble}\hlstd{(}\hlkwc{x} \hlstd{=} \hlnum{1}\hlstd{,} \hlkwc{y} \hlstd{=} \hlnum{1}\hlstd{,} \hlkwc{table.inset} \hlstd{=} \hlkwd{list}\hlstd{(tb.pm))}
\hlkwd{ggplot}\hlstd{(data.tb,} \hlkwc{mapping} \hlstd{=} \hlkwd{aes}\hlstd{(x, y,} \hlkwc{label} \hlstd{= table.inset))} \hlopt{+}
  \hlkwd{geom_table}\hlstd{(}\hlkwc{inherit.aes} \hlstd{=} \hlnum{TRUE}\hlstd{,} \hlkwc{size} \hlstd{=} \hlnum{7}\hlstd{,} \hlkwc{parse} \hlstd{=} \hlnum{TRUE}\hlstd{)} \hlopt{+}
  \hlkwd{theme_void}\hlstd{()}
\end{alltt}
\end{kframe}
\end{knitrout}

\begin{explainbox}
  The \emph{geometry} \gggeom{geom\_table()} uses internally functions from package \pkgname{gridExtra} to build a graphical object for the table. The use of table themes was not yet supported by this geometry at the time of writing.
\end{explainbox}

Geometry \gggeom{geom\_plot()} works very similarly to \code{geom\_table()} but instead of expecting a list of data frames or tibbles to be mapped to the \code{label} aesthetics, it expects a list of ggplots (objects of class \code{gg}). This allows adding as inset to a ggplot another ggplot. In the times when plots were hand drafted with India ink on paper, the use of inset plots was more frequent than nowadays. Inset plots can be very useful for zooming-in into parts of a main plot were observations are crowded and for displaying summaries based on the observations shown in the main plot. The inset plots are nested in viewports which control the dimensions of the inset plot, and aesthetics \code{vp.height} and \code{vp.width} control their sizes---with defaults of 1/3 of the height and width of the plotting area of the main plot. Themes can be applied separately to the main and inset plots.

In the first example of inset plots we include one of the summaries shown above as an inset table. We first create a tibble containing the plot to be inset.

\begin{knitrout}\footnotesize
\definecolor{shadecolor}{rgb}{0.969, 0.969, 0.969}\color{fgcolor}\begin{kframe}
\begin{alltt}
\hlstd{mtcars} \hlopt
  \hlkwd{group_by}\hlstd{(cyl)} \hlopt
  \hlkwd{summarize}\hlstd{(}\hlkwc{mean.mpg} \hlstd{=} \hlkwd{mean}\hlstd{(mpg))} \hlopt
  \hlkwd{ggplot}\hlstd{(}\hlkwd{aes}\hlstd{(}\hlkwd{factor}\hlstd{(cyl), mean.mpg,} \hlkwc{fill} \hlstd{=} \hlkwd{factor}\hlstd{(cyl)))} \hlopt{+}
  \hlkwd{scale_fill_discrete}\hlstd{(}\hlkwc{guide} \hlstd{=} \hlnum{FALSE}\hlstd{)} \hlopt{+}
  \hlkwd{labs}\hlstd{(}\hlkwc{y} \hlstd{=} \hlkwa{NULL}\hlstd{)} \hlopt{+}
    \hlkwd{geom_col}\hlstd{()} \hlopt{+}
    \hlkwd{theme_bw}\hlstd{(}\hlnum{8}\hlstd{)} \hlkwb{->} \hlstd{my.plot}
\hlstd{plot.tb} \hlkwb{<-} \hlkwd{tibble}\hlstd{(}\hlkwc{x} \hlstd{=} \hlnum{500}\hlstd{,} \hlkwc{y} \hlstd{=} \hlnum{35}\hlstd{,} \hlkwc{plot.inset} \hlstd{=} \hlkwd{list}\hlstd{(my.plot))}
\end{alltt}
\end{kframe}
\end{knitrout}

\begin{knitrout}\footnotesize
\definecolor{shadecolor}{rgb}{0.969, 0.969, 0.969}\color{fgcolor}\begin{kframe}
\begin{alltt}
\hlkwd{ggplot}\hlstd{(}\hlkwc{data} \hlstd{= mtcars,} \hlkwd{aes}\hlstd{(}\hlkwc{x} \hlstd{= disp,} \hlkwc{y} \hlstd{= mpg,}
                          \hlkwc{colour} \hlstd{=} \hlkwd{factor}\hlstd{(cyl)))} \hlopt{+}
  \hlkwd{geom_point}\hlstd{()} \hlopt{+}
  \hlkwd{geom_plot}\hlstd{(}\hlkwc{data} \hlstd{= plot.tb,}
            \hlkwd{aes}\hlstd{(}\hlkwc{x} \hlstd{= x,} \hlkwc{y} \hlstd{= y,} \hlkwc{label} \hlstd{= plot.inset),}
            \hlkwc{vp.width} \hlstd{=} \hlnum{1}\hlopt{/}\hlnum{2}\hlstd{,}
            \hlkwc{hjust} \hlstd{=} \hlstr{"inward"}\hlstd{,} \hlkwc{vjust} \hlstd{=} \hlstr{"inward"}\hlstd{)}
\end{alltt}
\end{kframe}

{\centering \includegraphics[width=.7\textwidth]{figure/pos-plot-plot-02-1} 

}



\end{knitrout}

In the second example of inset plots we 1) use manually set limits to the coordinates to zoom into a region of the main plot, 2) adjust the \emph{theme} of the inset, 3) remove axis labels as they are the same as in the main plot, 4) add this zoomed version of the same plot as an inset and 5) highlight the zoomed-in region in the main plot. This fairly complex example shows how a new extension to \pkgname{ggplot2} can integrate well into the grammar of graphics paradigm. In this example, only to show an alternative approach, instead of collecting all the data into a data frame, we map constant values directly to the various aesthetics.

\begin{knitrout}\footnotesize
\definecolor{shadecolor}{rgb}{0.969, 0.969, 0.969}\color{fgcolor}\begin{kframe}
\begin{alltt}
\hlstd{p.main} \hlkwb{<-} \hlkwd{ggplot}\hlstd{(}\hlkwc{data} \hlstd{= mtcars,} \hlkwd{aes}\hlstd{(}\hlkwc{x} \hlstd{= disp,} \hlkwc{y} \hlstd{= mpg,} \hlkwc{colour} \hlstd{=} \hlkwd{factor}\hlstd{(cyl)))} \hlopt{+}
  \hlkwd{geom_point}\hlstd{()}
\hlstd{p.inset} \hlkwb{<-} \hlstd{p.main} \hlopt{+}
  \hlkwd{coord_cartesian}\hlstd{(}\hlkwc{xlim} \hlstd{=} \hlkwd{c}\hlstd{(}\hlnum{270}\hlstd{,} \hlnum{330}\hlstd{),} \hlkwc{ylim} \hlstd{=} \hlkwd{c}\hlstd{(}\hlnum{14}\hlstd{,} \hlnum{19}\hlstd{))} \hlopt{+}
  \hlkwd{labs}\hlstd{(}\hlkwc{x} \hlstd{=} \hlkwa{NULL}\hlstd{,} \hlkwc{y} \hlstd{=} \hlkwa{NULL}\hlstd{,} \hlkwc{subtitle} \hlstd{=} \hlstr{"Detail of highlighted region"}\hlstd{)} \hlopt{+}
  \hlkwd{scale_colour_discrete}\hlstd{(}\hlkwc{guide} \hlstd{=} \hlnum{FALSE}\hlstd{)} \hlopt{+}
  \hlkwd{theme_bw}\hlstd{(}\hlnum{8}\hlstd{)} \hlopt{+} \hlkwd{theme}\hlstd{(}\hlkwc{aspect.ratio} \hlstd{=} \hlnum{1}\hlstd{)}
\hlstd{p.main} \hlopt{+}
  \hlkwd{geom_plot}\hlstd{(}\hlkwc{x} \hlstd{=} \hlnum{480}\hlstd{,} \hlkwc{y} \hlstd{=} \hlnum{34}\hlstd{,} \hlkwc{label} \hlstd{=} \hlkwd{list}\hlstd{(p.inset),} \hlkwc{vp.height} \hlstd{=} \hlnum{1}\hlopt{/}\hlnum{2}\hlstd{,}
            \hlkwc{hjust} \hlstd{=} \hlstr{"inward"}\hlstd{,} \hlkwc{vjust} \hlstd{=} \hlstr{"inward"}\hlstd{)} \hlopt{+}
  \hlkwd{annotate}\hlstd{(}\hlkwc{geom} \hlstd{=} \hlstr{"rect"}\hlstd{,} \hlkwc{fill} \hlstd{=} \hlnum{NA}\hlstd{,} \hlkwc{colour} \hlstd{=} \hlstr{"black"}\hlstd{,}
           \hlkwc{xmin} \hlstd{=} \hlnum{270}\hlstd{,} \hlkwc{xmax} \hlstd{=} \hlnum{330}\hlstd{,} \hlkwc{ymin} \hlstd{=} \hlnum{14}\hlstd{,} \hlkwc{ymax} \hlstd{=} \hlnum{19}\hlstd{,}
           \hlkwc{linetype} \hlstd{=} \hlstr{"dotted"}\hlstd{)}
\end{alltt}
\end{kframe}

{\centering \includegraphics[width=.7\textwidth]{figure/pos-plot-plot-03-1} 

}



\end{knitrout}

Geometry \gggeom{geom\_grob()} works very similarly to \code{geom\_table()} and \code{geom\_plot()} but expects a list of \pkgname{grid} graphical objects, called \code{grob} for short. This adds generality at the expense of very frequently having to separately create the grobs either from scratch using \pkgname{grid} or by converting other types of objects into grobs. This geometry is as flexible as \gggeom{annotation\_custom()} with respect to the objects to add to the plot, but behaves as a true ggplot \emph{geometry}. We will show a single example which adds two bitmaps to the plot. The bitmaps are read from PNG files, converted into grobs and added to the plot as a new layer. These bitmaps have a transparent background and consequently the grid of the plotting area shows through.

\begin{knitrout}\footnotesize
\definecolor{shadecolor}{rgb}{0.969, 0.969, 0.969}\color{fgcolor}\begin{kframe}
\begin{alltt}
\hlstd{file1.name} \hlkwb{<-}
  \hlkwd{system.file}\hlstd{(}\hlstr{"extdata"}\hlstd{,} \hlstr{"Isoquercitin.png"}\hlstd{,}
              \hlkwc{package} \hlstd{=} \hlstr{"ggpmisc"}\hlstd{,} \hlkwc{mustWork} \hlstd{=} \hlnum{TRUE}\hlstd{)}
\hlstd{Isoquercitin} \hlkwb{<-} \hlstd{magick}\hlopt{::}\hlkwd{image_read}\hlstd{(file1.name)}

\hlstd{file2.name} \hlkwb{<-}
  \hlkwd{system.file}\hlstd{(}\hlstr{"extdata"}\hlstd{,} \hlstr{"Robinin.png"}\hlstd{,}
              \hlkwc{package} \hlstd{=} \hlstr{"ggpmisc"}\hlstd{,} \hlkwc{mustWork} \hlstd{=} \hlnum{TRUE}\hlstd{)}
\hlstd{Robinin} \hlkwb{<-} \hlstd{magick}\hlopt{::}\hlkwd{image_read}\hlstd{(file2.name)}

\hlstd{grob.tb} \hlkwb{<-} \hlkwd{tibble}\hlstd{(}\hlkwc{x} \hlstd{=} \hlkwd{c}\hlstd{(}\hlnum{0}\hlstd{,} \hlnum{100}\hlstd{),} \hlkwc{y} \hlstd{=} \hlkwd{c}\hlstd{(}\hlnum{10}\hlstd{,} \hlnum{20}\hlstd{),} \hlkwc{height} \hlstd{=} \hlnum{1}\hlopt{/}\hlnum{4}\hlstd{,} \hlkwc{width} \hlstd{=} \hlkwd{c}\hlstd{(}\hlnum{1}\hlopt{/}\hlnum{2}\hlstd{),}
                  \hlkwc{grobs} \hlstd{=} \hlkwd{list}\hlstd{(grid}\hlopt{::}\hlkwd{rasterGrob}\hlstd{(}\hlkwc{image} \hlstd{= Isoquercitin),}
                               \hlstd{grid}\hlopt{::}\hlkwd{rasterGrob}\hlstd{(}\hlkwc{image} \hlstd{= Robinin)))}

\hlkwd{ggplot}\hlstd{()} \hlopt{+}
\hlkwd{geom_grob}\hlstd{(}\hlkwc{data} \hlstd{= grob.tb,}
          \hlkwd{aes}\hlstd{(}\hlkwc{x} \hlstd{= x,} \hlkwc{y} \hlstd{= y,} \hlkwc{label} \hlstd{= grobs,}
              \hlkwc{vp.height} \hlstd{= height,} \hlkwc{vp.width} \hlstd{= width),}
          \hlkwc{hjust} \hlstd{=} \hlstr{"inward"}\hlstd{,} \hlkwc{vjust} \hlstd{=} \hlstr{"inward"}\hlstd{)}
\end{alltt}
\end{kframe}

{\centering \includegraphics[width=.7\textwidth]{figure/pos-plot-grob-01-1} 

}



\end{knitrout}

\begin{warningbox}
You\index{plots!fonts} should be aware that \Rlang and \ggplot support the use of UNICODE\index{UNICODE}, such as UTF8\index{UTF8} character encodings in strings. If your editor or IDE supports their use, then you can type Greek letters and simple maths symbols directly, and they \emph{may} show correctly in labels if a suitable font is loaded and an extended encoding like UTF8 in use by the operating system. Even if UTF8 is in use, text is not fully portable unless the same font is available\index{portability}, as even if the character positions are standardized for many languages, most UNICODE fonts support at most a small number of languages. In principle one can use this mechanism to have labels both using other alphabets and languages like Chinese with their numerous symbols mixed in the same figure. Furthermore, the support for fonts and consequently character sets in \Rlang is output-device dependent. The font encoding used by \Rlang by default depends on the default locale settings of the operating system, which can also lead to garbage printed to the console or wrong characters being plotted running the same code on a different computer from the one where a script was edited. Not all is lost, though, as \Rlang can be coerced to use system fonts and Google fonts with functions provided by packages \pkgname{showtext} and \pkgname{extrafont}. Encoding-related problems, specially in MS-Windows, are common.
\end{warningbox}

\begin{explainbox}
Grid graphics provides the low level functions that both \pkgname{ggplot2} and \pkgname{lattice} use under the hood, support different types of units for expressing the coordinates of positions within the plotting area. All examples outside this text box use \code{"native"} data coordinates, however coordinates can be also given in physical units like \code{"mm"}. More useful when working with scalable plots is to use "npc" normalized parent coordinates, which are expressed as numbers in the range 0 to 1, relative to the dimensions of the sides of the current viewport, with origin at the lower left corner.

Package \pkgname{ggplot2} interprets $x$ and $y$ coordinates in \code{"native"} data coordinates, and trickery seems to be needed to get around this limitation. A rather general solution is provided by package \pkgname{ggpmisc} through \emph{aesthetics} \code{npcx} and \code{npcy} and geometries that support them. At the time of writing, \gggeom{geom\_text\_npc()}, \gggeom{geom\_label\_npc()}, \gggeom{geom\_table\_npc()}, \gggeom{geom\_plot\_npc()} and \gggeom{geom\_grob\_npc()}. These \emph{geometries} are useful for annotating plots and adding insets at positions relative to the plotting area that remain always consistent across different plots, or across panels when using facets with free axis limits. Being geometries they provide freedom in the elements added to different panels and their positions.

\begin{knitrout}\footnotesize
\definecolor{shadecolor}{rgb}{0.969, 0.969, 0.969}\color{fgcolor}\begin{kframe}
\begin{alltt}
\hlkwd{ggplot}\hlstd{(}\hlkwc{data} \hlstd{= mtcars,} \hlkwd{aes}\hlstd{(}\hlkwc{x} \hlstd{= disp,} \hlkwc{y} \hlstd{= mpg,}
                          \hlkwc{colour} \hlstd{=} \hlkwd{factor}\hlstd{(cyl)))} \hlopt{+}
  \hlkwd{geom_point}\hlstd{()} \hlopt{+}
  \hlkwd{annotate}\hlstd{(}\hlkwc{geom} \hlstd{=} \hlstr{"text_npc"}\hlstd{,}
           \hlkwc{npcx} \hlstd{=} \hlnum{0.5}\hlstd{,} \hlkwc{npcy} \hlstd{=} \hlnum{0.9}\hlstd{,} \hlkwc{label} \hlstd{=} \hlstr{"a label"}\hlstd{)}
\end{alltt}
\end{kframe}
\end{knitrout}
\end{explainbox}

\index{plots!text in|)}
\index{plots!maths in|)}

\index{plots!geometries|)}

\section{Statistics}\label{sec:plot:statistics}
\index{plots!statistics|(}

Before learning about \ggplot \emph{statistics} it is important to understand how the mapping of factors to \emph{aesthetic} works. When a factor, for example, is mapped to \code{colour} it creates a new grouping, with the observations matching a given level of the factor, corresponding to a group. Most \emph{statistics} operate on the data for each of these groups separately, returning a summary for each group, for example, the mean of the observations in a group.

\subsection{Functions}\label{sec:plot:function}
\index{plots!plotting functions|(}
\index{plots!statistics!function}
In addition to plotting data from a data frame with variables to map to $x$ and $y$ \emph{aesthetics}, it is possible to have only a variable mapped to $x$ and use \ggstat{stat\_function()} to compute the values to be mapped to $y$ using an \Rlang function. This avoids the need to generate data beforehand as even the number of data points to be generated can be set in \code{geom\_function}. Any \Rlang function, user defined or not, can be used as long as it is vectorised, with the length of the returned vector equal to the length of the vector used as \code{x} argument.

We start with the Normal distribution function.

\begin{knitrout}\footnotesize
\definecolor{shadecolor}{rgb}{0.969, 0.969, 0.969}\color{fgcolor}\begin{kframe}
\begin{alltt}
\hlkwd{ggplot}\hlstd{(}\hlkwd{data.frame}\hlstd{(}\hlkwc{x} \hlstd{=} \hlopt{-}\hlnum{3}\hlopt{:}\hlnum{3}\hlstd{),} \hlkwd{aes}\hlstd{(}\hlkwc{x} \hlstd{= x))} \hlopt{+}
  \hlkwd{stat_function}\hlstd{(}\hlkwc{fun} \hlstd{= dnorm)}
\end{alltt}
\end{kframe}

{\centering \includegraphics[width=.7\textwidth]{figure/pos-function-plot-01-1} 

}



\end{knitrout}

Using a list we can even pass by name additional arguments to a function.

\begin{knitrout}\footnotesize
\definecolor{shadecolor}{rgb}{0.969, 0.969, 0.969}\color{fgcolor}\begin{kframe}
\begin{alltt}
\hlkwd{ggplot}\hlstd{(}\hlkwd{data.frame}\hlstd{(}\hlkwc{x} \hlstd{=} \hlopt{-}\hlnum{3}\hlopt{:}\hlnum{3}\hlstd{),} \hlkwd{aes}\hlstd{(}\hlkwc{x} \hlstd{= x))} \hlopt{+}
  \hlkwd{stat_function}\hlstd{(}\hlkwc{fun} \hlstd{= dnorm,} \hlkwc{args} \hlstd{=} \hlkwd{list}\hlstd{(}\hlkwc{mean} \hlstd{=} \hlnum{1}\hlstd{,} \hlkwc{sd} \hlstd{=} \hlnum{.5}\hlstd{))}
\end{alltt}
\end{kframe}
\end{knitrout}

\begin{playground}
1) Edit the code above so as to plot in the same figure three curves, either for three different values for \code{mean} or for three different values for \code{sd}.

2) Edit the code above to use a different function, say \code{df}, the F distribution, adjusting the argument(s) passed through \code{args} accordingly.
\end{playground}

Of course, user-defined functions (not shown), and anonymous functions (below) can also be used.

\begin{knitrout}\footnotesize
\definecolor{shadecolor}{rgb}{0.969, 0.969, 0.969}\color{fgcolor}\begin{kframe}
\begin{alltt}
\hlkwd{ggplot}\hlstd{(}\hlkwd{data.frame}\hlstd{(}\hlkwc{x} \hlstd{=} \hlnum{0}\hlopt{:}\hlnum{1}\hlstd{),} \hlkwd{aes}\hlstd{(}\hlkwc{x} \hlstd{= x))} \hlopt{+}
  \hlkwd{stat_function}\hlstd{(}\hlkwc{fun} \hlstd{=} \hlkwa{function}\hlstd{(}\hlkwc{x}\hlstd{,} \hlkwc{a}\hlstd{,} \hlkwc{b}\hlstd{)\{a} \hlopt{+} \hlstd{b} \hlopt{*} \hlstd{x}\hlopt{^}\hlnum{2}\hlstd{\},}
                \hlkwc{args} \hlstd{=} \hlkwd{list}\hlstd{(}\hlkwc{a} \hlstd{=} \hlnum{1}\hlstd{,} \hlkwc{b} \hlstd{=} \hlnum{1.4}\hlstd{))}
\end{alltt}
\end{kframe}
\end{knitrout}

\begin{playground}
Edit the code above to use a different function, such as $e^{x + k}$, adjusting the argument(s) passed through \code{args} accordingly. Do this by means of an anonymous function, and by means of an equivalent named function defined by your code.
\end{playground}

\index{plots!plotting functions|)}

\subsection{Summaries}\label{sec:plot:stat:summaries}
\index{plots!statistics|(}
\index{plots!statistics!summary}
The summaries discussed in this section can be superimposed on raw data plots, or plotted on their own. Beware, that if scale limits are manually set, the summaries will be calculated from the subset of observations within these limits. Scale limits can be altered when explicitly defining a scale or by means of functions \Rfunction{xlim()} and \Rfunction{ylim}. See the text box on \pageref{box:plot:coord:lims} for a way of constraining the viewport (the region visible in the plot) by changing coordinate limits while keeping the scale limits on a wider range of $x$ and $y$ values.

It is possible to summarize data on-the-fly when plotting. We describe in the same section the calculation of measures of central position and of variation, as \ggstat{stat\_summary()} allows them to be calculated in the same function call.

For the examples we will generate some normally distributed artificial data.

\begin{knitrout}\footnotesize
\definecolor{shadecolor}{rgb}{0.969, 0.969, 0.969}\color{fgcolor}\begin{kframe}
\begin{alltt}
\hlstd{fake.data} \hlkwb{<-} \hlkwd{data.frame}\hlstd{(}
  \hlkwc{y} \hlstd{=} \hlkwd{c}\hlstd{(}\hlkwd{rnorm}\hlstd{(}\hlnum{10}\hlstd{,} \hlkwc{mean} \hlstd{=} \hlnum{2}\hlstd{,} \hlkwc{sd} \hlstd{=} \hlnum{0.5}\hlstd{),}
        \hlkwd{rnorm}\hlstd{(}\hlnum{10}\hlstd{,} \hlkwc{mean} \hlstd{=} \hlnum{4}\hlstd{,} \hlkwc{sd} \hlstd{=} \hlnum{0.7}\hlstd{)),}
  \hlkwc{group} \hlstd{=} \hlkwd{factor}\hlstd{(}\hlkwd{c}\hlstd{(}\hlkwd{rep}\hlstd{(}\hlstr{"A"}\hlstd{,} \hlnum{10}\hlstd{),} \hlkwd{rep}\hlstd{(}\hlstr{"B"}\hlstd{,} \hlnum{10}\hlstd{)))}
  \hlstd{)}
\end{alltt}
\end{kframe}
\end{knitrout}

We first use scatter plots for the examples, later we give some additional examples for bar plots.
We will reuse a ``base'' plot in a series of examples, so that the differences are easier to appreciate. We first add just the mean. In this case we need to pass as argument to \ggstat{stat\_summary()} the \code{geom} to use, as the default one, \gggeom{geom\_pointrange()}, expects data for plotting error bars in addition to the mean.

\begin{knitrout}\footnotesize
\definecolor{shadecolor}{rgb}{0.969, 0.969, 0.969}\color{fgcolor}\begin{kframe}
\begin{alltt}
\hlkwd{ggplot}\hlstd{(}\hlkwc{data} \hlstd{= fake.data,} \hlkwd{aes}\hlstd{(}\hlkwc{y} \hlstd{= y,} \hlkwc{x} \hlstd{= group))} \hlopt{+}
  \hlkwd{geom_point}\hlstd{(}\hlkwc{shape} \hlstd{=} \hlnum{21}\hlstd{)} \hlopt{+}
  \hlkwd{stat_summary}\hlstd{(}\hlkwc{fun.y} \hlstd{=} \hlstr{"mean"}\hlstd{,} \hlkwc{geom} \hlstd{=} \hlstr{"point"}\hlstd{,} \hlkwc{color} \hlstd{=} \hlstr{"red"}\hlstd{,} \hlkwc{shape} \hlstd{=} \hlstr{"-"}\hlstd{,} \hlkwc{size} \hlstd{=} \hlnum{10}\hlstd{)}
\end{alltt}
\end{kframe}

{\centering \includegraphics[width=.7\textwidth]{figure/pos-summary-plot-02-1} 

}



\end{knitrout}

Then the median, by changing the argument passed to \code{fun.y}. For the next few examples we show only the call to \code{stat\_summary()}.

\begin{knitrout}\footnotesize
\definecolor{shadecolor}{rgb}{0.969, 0.969, 0.969}\color{fgcolor}\begin{kframe}
\begin{alltt}
  \hlkwd{stat_summary}\hlstd{(}\hlkwc{fun.y} \hlstd{=} \hlstr{"median"}\hlstd{,} \hlkwc{geom} \hlstd{=} \hlstr{"point"}\hlstd{,} \hlkwc{colour} \hlstd{=} \hlstr{"red"}\hlstd{,} \hlkwc{shape} \hlstd{=} \hlstr{"-"}\hlstd{,} \hlkwc{size} \hlstd{=} \hlnum{10}\hlstd{)}
\end{alltt}
\end{kframe}
\end{knitrout}

We can add the mean and $p = 0.95$ confidence intervals assuming normality (using the $t$ distribution):

\begin{knitrout}\footnotesize
\definecolor{shadecolor}{rgb}{0.969, 0.969, 0.969}\color{fgcolor}\begin{kframe}
\begin{alltt}
  \hlkwd{stat_summary}\hlstd{(}\hlkwc{fun.data} \hlstd{=} \hlstr{"mean_cl_normal"}\hlstd{,} \hlkwc{colour} \hlstd{=} \hlstr{"red"}\hlstd{,} \hlkwc{size} \hlstd{=} \hlnum{1}\hlstd{,} \hlkwc{alpha} \hlstd{=} \hlnum{0.7}\hlstd{)}
\end{alltt}
\end{kframe}
\end{knitrout}

We can add the means and $p = 0.95$ confidence intervals not assuming normality (using the actual distribution of the data by bootstrapping):

\begin{knitrout}\footnotesize
\definecolor{shadecolor}{rgb}{0.969, 0.969, 0.969}\color{fgcolor}\begin{kframe}
\begin{alltt}
  \hlkwd{stat_summary}\hlstd{(}\hlkwc{fun.data} \hlstd{=} \hlstr{"mean_cl_boot"}\hlstd{,} \hlkwc{colour} \hlstd{=} \hlstr{"red"}\hlstd{,} \hlkwc{size} \hlstd{=} \hlnum{1}\hlstd{,} \hlkwc{alpha} \hlstd{=} \hlnum{0.7}\hlstd{)}
\end{alltt}
\end{kframe}
\end{knitrout}

If needed, we can display less restrictive confidence intervals, at $p = 0.90$ in this example, by means of \code{conf.int = 0.90} passed as a list to the underlying function being called.

\begin{knitrout}\footnotesize
\definecolor{shadecolor}{rgb}{0.969, 0.969, 0.969}\color{fgcolor}\begin{kframe}
\begin{alltt}
  \hlkwd{stat_summary}\hlstd{(}\hlkwc{fun.data} \hlstd{=} \hlstr{"mean_cl_boot"}\hlstd{,}
               \hlkwc{fun.args} \hlstd{=} \hlkwd{list}\hlstd{(}\hlkwc{conf.int} \hlstd{=} \hlnum{0.90}\hlstd{),}
               \hlkwc{colour} \hlstd{=} \hlstr{"red"}\hlstd{,} \hlkwc{size} \hlstd{=} \hlnum{1}\hlstd{,} \hlkwc{alpha} \hlstd{=} \hlnum{0.7}\hlstd{)}
\end{alltt}
\end{kframe}
\end{knitrout}

We can plot error bars corresponding to $\pm$s.e. (standard errors) with the function \Rfunction{"mean\_se"}.

\begin{knitrout}\footnotesize
\definecolor{shadecolor}{rgb}{0.969, 0.969, 0.969}\color{fgcolor}\begin{kframe}
\begin{alltt}
  \hlkwd{stat_summary}\hlstd{(}\hlkwc{fun.data} \hlstd{=} \hlstr{"mean_se"}\hlstd{,}
               \hlkwc{colour} \hlstd{=} \hlstr{"red"}\hlstd{,} \hlkwc{size} \hlstd{=} \hlnum{1}\hlstd{,} \hlkwc{alpha} \hlstd{=} \hlnum{0.7}\hlstd{)}
\end{alltt}
\end{kframe}
\end{knitrout}

\begin{warningbox}\label{box:plot:coord:lims}
\textbf{Scale- and coordinate limits are very different.}\index{limits!coordinate}\index{limits!scale} Scale limits restrict the data used, while coordinate limits restrict the data that are visible. For a scatter plot, the effect of either approach on the resulting plot are equivalent, as no calculations are involved, but when using \emph{statistics} to compute summaries, one should almost always rely on coordinate limits, to make sure that no data are excluded from the calculated summary. \emph{When manually setting scale limits one should never ignore warning messages related to excluded observations.} See section \ref{sec:plot:coord:limits} for details.
\end{warningbox}

Finally we can plot error bars showing $\pm$s.d. (standard deviation).

\begin{knitrout}\footnotesize
\definecolor{shadecolor}{rgb}{0.969, 0.969, 0.969}\color{fgcolor}\begin{kframe}
\begin{alltt}
  \hlkwd{stat_summary}\hlstd{(}\hlkwc{fun.data} \hlstd{=} \hlstr{"mean_sdl"}\hlstd{,} \hlkwc{colour} \hlstd{=} \hlstr{"red"}\hlstd{,} \hlkwc{size} \hlstd{=} \hlnum{1}\hlstd{,} \hlkwc{alpha} \hlstd{=} \hlnum{0.7}\hlstd{)}
\end{alltt}
\end{kframe}
\end{knitrout}

We do not give an example here, but instead of using these functions (from package \Hmisc) it is possible to use user-defined functions. In addition as arguments to any function used, except for the first one containing the actual data, are supplied as a list through formal argument \code{fun.args}, there is a lot of flexibility with respect to what functions can be used.

Finally we plot the means in a scatter plot, with the observations superimposed and $p = 0.95$ confidence interval (the order in which the geoms are added is important: by having \gggeom{geom\_point()} last this layer is plotted on top of the error bars. In this case we set fill, colour and alpha (transparency) to constants, but in more complex data sets mapping them to factors in the data set can be used to distinguish them. Adding \ggstat{stat\_summary()} twice allows us to plot the mean and the error bars using different colors.

\begin{knitrout}\footnotesize
\definecolor{shadecolor}{rgb}{0.969, 0.969, 0.969}\color{fgcolor}\begin{kframe}
\begin{alltt}
\hlkwd{ggplot}\hlstd{(}\hlkwc{data} \hlstd{= fake.data,} \hlkwd{aes}\hlstd{(}\hlkwc{y} \hlstd{= y,} \hlkwc{x} \hlstd{= group))} \hlopt{+}
  \hlkwd{stat_summary}\hlstd{(}\hlkwc{fun.y} \hlstd{=} \hlstr{"mean"}\hlstd{,} \hlkwc{geom} \hlstd{=} \hlstr{"point"}\hlstd{,}
               \hlkwc{fill} \hlstd{=} \hlstr{"white"}\hlstd{,} \hlkwc{colour} \hlstd{=} \hlstr{"black"}\hlstd{)} \hlopt{+}
  \hlkwd{stat_summary}\hlstd{(}\hlkwc{fun.data} \hlstd{=} \hlstr{"mean_cl_boot"}\hlstd{,}
               \hlkwc{geom} \hlstd{=} \hlstr{"errorbar"}\hlstd{,}
               \hlkwc{width} \hlstd{=} \hlnum{0.1}\hlstd{,} \hlkwc{size} \hlstd{=} \hlnum{1}\hlstd{,} \hlkwc{colour} \hlstd{=} \hlstr{"red"}\hlstd{)} \hlopt{+}
  \hlkwd{geom_point}\hlstd{(}\hlkwc{size} \hlstd{=} \hlnum{3}\hlstd{,} \hlkwc{alpha} \hlstd{=} \hlnum{0.3}\hlstd{)}
\end{alltt}
\end{kframe}

{\centering \includegraphics[width=.7\textwidth]{figure/pos-summary-plot-09-1} 

}



\end{knitrout}

Similarly as with scatter plots, we can plot summaries as columns.
If we supply a different argument to \code{geom} we can for example plot the means or medians for a variable, for each \code{class} of car.

\begin{knitrout}\footnotesize
\definecolor{shadecolor}{rgb}{0.969, 0.969, 0.969}\color{fgcolor}\begin{kframe}
\begin{alltt}
\hlkwd{ggplot}\hlstd{(mpg,} \hlkwd{aes}\hlstd{(class, hwy))} \hlopt{+}
  \hlkwd{stat_summary}\hlstd{(}\hlkwc{geom} \hlstd{=} \hlstr{"col"}\hlstd{,} \hlkwc{fun.y} \hlstd{= mean)}
\end{alltt}
\end{kframe}
\end{knitrout}

The ``reverse'' syntax is also valid, as we can add the \emph{geometry} to the plot object and pass the \emph{statistics} as an argument to it. In general in this book we avoid this alternative syntax for the sake of consistency.

\begin{knitrout}\footnotesize
\definecolor{shadecolor}{rgb}{0.969, 0.969, 0.969}\color{fgcolor}\begin{kframe}
\begin{alltt}
\hlkwd{ggplot}\hlstd{(mpg,} \hlkwd{aes}\hlstd{(class, hwy))} \hlopt{+}
  \hlkwd{geom_col}\hlstd{(}\hlkwc{stat} \hlstd{=} \hlstr{"summary"}\hlstd{,} \hlkwc{fun.y} \hlstd{= mean)}
\end{alltt}
\end{kframe}
\end{knitrout}

And we can easily add error bars to the column plot. We use \code{size} to make the lines of the error bars thicker. The default \emph{geometry} in \ggstat{stat\_summary()} is \gggeom{geom\_pointrange()}, so we can pass \code{"linerange"} as argument for \code{geom} to eliminate the point.

\begin{knitrout}\footnotesize
\definecolor{shadecolor}{rgb}{0.969, 0.969, 0.969}\color{fgcolor}\begin{kframe}
\begin{alltt}
  \hlkwd{stat_summary}\hlstd{(}\hlkwc{geom} \hlstd{=} \hlstr{"linerange"}\hlstd{,}
               \hlkwc{fun.data} \hlstd{=} \hlstr{"mean_se"}\hlstd{,} \hlkwc{size} \hlstd{=} \hlnum{1}\hlstd{,}
               \hlkwc{color} \hlstd{=} \hlstr{"red"}\hlstd{)}
\end{alltt}
\end{kframe}
\end{knitrout}

Passing \code{"errorbar"} to \code{geom} results in the use of \gggeom{geom\_errorbar()} resulting in traditional ``capped'' error bars. However, this type of error bars has been criticized as adding unnecessary clutter to plots \autocite{Tufte1983}. We use \code{width} to reduce the width of the caps at the ends of the error bars.

\begin{knitrout}\footnotesize
\definecolor{shadecolor}{rgb}{0.969, 0.969, 0.969}\color{fgcolor}\begin{kframe}
\begin{alltt}
  \hlkwd{stat_summary}\hlstd{(}\hlkwc{geom} \hlstd{=} \hlstr{"errorbar"}\hlstd{,}
               \hlkwc{fun.data} \hlstd{=} \hlstr{"mean_se"}\hlstd{,} \hlkwc{width} \hlstd{=} \hlnum{0.1}\hlstd{,} \hlkwc{size} \hlstd{=} \hlnum{1}\hlstd{,}
               \hlkwc{color} \hlstd{=} \hlstr{"red"}\hlstd{)}
\end{alltt}
\end{kframe}
\end{knitrout}

If we have ready calculated values for the summaries, we can still obtain the same plots. Here we calculate the summaries before plotting, and then redraw the plot immediately above.\label{exmpl:plot:errorbar:precalc}

\begin{knitrout}\footnotesize
\definecolor{shadecolor}{rgb}{0.969, 0.969, 0.969}\color{fgcolor}\begin{kframe}
\begin{alltt}
\hlstd{mpg_g} \hlkwb{<-} \hlstd{dplyr}\hlopt{::}\hlkwd{group_by}\hlstd{(mpg, class)}
\hlstd{mpg_summ} \hlkwb{<-} \hlstd{dplyr}\hlopt{::}\hlkwd{summarise}\hlstd{(mpg_g,} \hlkwc{hwy_mean} \hlstd{=} \hlkwd{mean}\hlstd{(hwy),}
                                    \hlkwc{hwy_se} \hlstd{=} \hlkwd{sd}\hlstd{(hwy)} \hlopt{/} \hlkwd{sqrt}\hlstd{(}\hlkwd{n}\hlstd{()))}

\hlkwd{ggplot}\hlstd{(mpg_summ,} \hlkwd{aes}\hlstd{(}\hlkwc{x} \hlstd{= class,}
                     \hlkwc{y} \hlstd{= hwy_mean,}
                     \hlkwc{ymax} \hlstd{= hwy_mean} \hlopt{+} \hlstd{hwy_se,}
                     \hlkwc{ymin} \hlstd{= hwy_mean} \hlopt{-} \hlstd{hwy_se))} \hlopt{+}
  \hlkwd{geom_col}\hlstd{()} \hlopt{+}
  \hlkwd{geom_errorbar}\hlstd{(}\hlkwc{width} \hlstd{=} \hlnum{0.1}\hlstd{,} \hlkwc{size} \hlstd{=} \hlnum{1}\hlstd{,} \hlkwc{color} \hlstd{=} \hlstr{"red"}\hlstd{)}
\end{alltt}
\end{kframe}
\end{knitrout}

\subsection{Smoothers and models}
\index{plots!smooth curves|(}
\index{plots!fitted curves|(}
\index{plots!statistics!smooth}
The \emph{statistic} \ggstat{stat\_smooth()} fits a smooth curve to observations in the case when the scales for $x$ and $y$ are continuous---the corresponding \emph{geometry} \gggeom{geom\_smooth()} uses this \emph{statistic}, and differs only in how arguments are passed to formal parameters. For the first example, we use \ggstat{stat\_smooth()} with the default smoother, a spline. The type of spline is automatically chosen based on the number of observations.

\begin{knitrout}\footnotesize
\definecolor{shadecolor}{rgb}{0.969, 0.969, 0.969}\color{fgcolor}\begin{kframe}
\begin{alltt}
\hlkwd{ggplot}\hlstd{(}\hlkwc{data} \hlstd{= mtcars,} \hlkwd{aes}\hlstd{(}\hlkwc{x} \hlstd{= disp,} \hlkwc{y} \hlstd{= mpg))} \hlopt{+}
       \hlkwd{stat_smooth}\hlstd{()}
\end{alltt}
\end{kframe}
\end{knitrout}

In most cases we will want to plot the observations as points together with the smoother. We can plot the observation on top of the smoother, as done here, or the smoother on top of the observations.

\begin{knitrout}\footnotesize
\definecolor{shadecolor}{rgb}{0.969, 0.969, 0.969}\color{fgcolor}\begin{kframe}
\begin{alltt}
\hlkwd{ggplot}\hlstd{(}\hlkwc{data} \hlstd{= mtcars,} \hlkwd{aes}\hlstd{(}\hlkwc{x} \hlstd{= disp,} \hlkwc{y} \hlstd{= mpg))} \hlopt{+}
  \hlkwd{stat_smooth}\hlstd{()} \hlopt{+}
  \hlkwd{geom_point}\hlstd{()}
\end{alltt}


{\ttfamily\noindent\itshape\color{messagecolor}{\#\# `geom\_smooth()` using method = 'loess' and formula 'y \textasciitilde{} x'}}\end{kframe}

{\centering \includegraphics[width=.7\textwidth]{figure/pos-smooth-plot-02-1} 

}



\end{knitrout}

Instead of using the default spline, we can fit a different model. In this example we use a linear model as smoother, fitted by \Rfunction{lm()}.

\begin{knitrout}\footnotesize
\definecolor{shadecolor}{rgb}{0.969, 0.969, 0.969}\color{fgcolor}\begin{kframe}
\begin{alltt}
  \hlkwd{stat_smooth}(method=\hlstr{"lm"}) +
\end{alltt}
\end{kframe}
\end{knitrout}

These data are really grouped, so we map variable \code{cyl} to the \code{colour} \emph{aesthetic}. Now we get three groups of points with different colours but also three separate smooth lines.

\begin{knitrout}\footnotesize
\definecolor{shadecolor}{rgb}{0.969, 0.969, 0.969}\color{fgcolor}\begin{kframe}
\begin{alltt}
\hlkwd{ggplot}\hlstd{(}\hlkwc{data} \hlstd{= mtcars,} \hlkwd{aes}\hlstd{(}\hlkwc{x} \hlstd{= disp,} \hlkwc{y} \hlstd{= mpg,} \hlkwc{color} \hlstd{=} \hlkwd{factor}\hlstd{(cyl)))} \hlopt{+}
  \hlkwd{stat_smooth}\hlstd{(}\hlkwc{method} \hlstd{=} \hlstr{"lm"}\hlstd{)} \hlopt{+}
  \hlkwd{geom_point}\hlstd{()}
\end{alltt}
\end{kframe}

{\centering \includegraphics[width=.7\textwidth]{figure/pos-smooth-plot-04-1} 

}



\end{knitrout}

To obtain a single smoother for the three groups, we need to set the mapping of the \code{color} \emph{aesthetic} to a constant within \ggstat{stat\_smooth}. This local value overrides the default for the whole plot set with \code{aes} just for this single \emph{statistic}. We use \code{"black"} but this could be replaced by any other color definition known to \Rlang.

\begin{knitrout}\footnotesize
\definecolor{shadecolor}{rgb}{0.969, 0.969, 0.969}\color{fgcolor}\begin{kframe}
\begin{alltt}
\hlkwd{ggplot}\hlstd{(}\hlkwc{data} \hlstd{= mtcars,} \hlkwd{aes}\hlstd{(}\hlkwc{x} \hlstd{= disp,} \hlkwc{y} \hlstd{= mpg,} \hlkwc{color} \hlstd{=} \hlkwd{factor}\hlstd{(cyl)))} \hlopt{+}
  \hlkwd{stat_smooth}\hlstd{(}\hlkwc{method} \hlstd{=} \hlstr{"lm"}\hlstd{,} \hlkwc{colour} \hlstd{=} \hlstr{"black"}\hlstd{)} \hlopt{+}
  \hlkwd{geom_point}\hlstd{()}
\end{alltt}
\end{kframe}
\end{knitrout}

Instead of using the default \code{formula} for a linear regression as smoother, we pass a different \code{formula} as argument. In this example we use a polynomial of order 2 fitted by \Rfunction{lm()}.

\begin{knitrout}\footnotesize
\definecolor{shadecolor}{rgb}{0.969, 0.969, 0.969}\color{fgcolor}\begin{kframe}
\begin{alltt}
\hlkwd{ggplot}\hlstd{(}\hlkwc{data} \hlstd{= mtcars,} \hlkwd{aes}\hlstd{(}\hlkwc{x} \hlstd{= disp,} \hlkwc{y} \hlstd{= mpg,} \hlkwc{color} \hlstd{=} \hlkwd{factor}\hlstd{(cyl)))} \hlopt{+}
  \hlkwd{stat_smooth}\hlstd{(}\hlkwc{method} \hlstd{=} \hlstr{"lm"}\hlstd{,} \hlkwc{formula} \hlstd{= y} \hlopt{~} \hlkwd{poly}\hlstd{(x,} \hlnum{2}\hlstd{),} \hlkwc{colour} \hlstd{=} \hlstr{"black"}\hlstd{)} \hlopt{+}
  \hlkwd{geom_point}\hlstd{()}
\end{alltt}
\end{kframe}

{\centering \includegraphics[width=.7\textwidth]{figure/pos-smooth-plot-06-1} 

}



\end{knitrout}

It is possible to use other types of models, including GAM and GLM, as smoothers, but we will give only two simple examples of the use of \code{nls()} to fit a model non-linear in its parameters. In the first one we fit a Michaelis-Menten equation to reaction rate (\code{rate}) versus reactant concentration (\code{conc}). \code{Puromycin} is a data set included in the \Rlang distribution. Function \Rfunction{SSmicmen()}
is also from \Rlang, and is a \emph{self starting} implementation of Michaelis-Menten function. Thanks to this, even though the fit is done with an iterative algorithm, we do not need to explicitly provide starting values for the parameters to be fitted. We need to set \code{se = FALSE} because
standard errors are not supported by the \code{predict()} method for \code{nls} fitted models.

\begin{knitrout}\footnotesize
\definecolor{shadecolor}{rgb}{0.969, 0.969, 0.969}\color{fgcolor}\begin{kframe}
\begin{alltt}
\hlkwd{ggplot}\hlstd{(Puromycin,} \hlkwd{aes}\hlstd{(conc, rate,} \hlkwc{colour} \hlstd{= state))} \hlopt{+}
  \hlkwd{geom_point}\hlstd{()} \hlopt{+}
  \hlkwd{geom_smooth}\hlstd{(}\hlkwc{method} \hlstd{=} \hlstr{"nls"}\hlstd{,}
              \hlkwc{formula} \hlstd{=  y} \hlopt{~} \hlkwd{SSmicmen}\hlstd{(x, Vm, K),}
              \hlkwc{se} \hlstd{=} \hlnum{FALSE}\hlstd{)}
\end{alltt}
\end{kframe}
\end{knitrout}

The self-starting models available in R are \code{SSasymp}, \code{SSasympOff}, \code{SSasympOrig}, \code{SSbiexp}, \code{SSfol}, \code{SSfpl}, \code{SSgompertz}, \code{SSlogis}, \code{SSmicmen}, and \code{SSweibull}. Function \code{selfStart} can be used to define new ones. All these functions can be used when fitting models with \Rfunction{nls} or \Rfunction{nlme}. Please, check the respective help pages for details.

In the second example we define the same model directly in the model formula, and provide the starting values explicitly. The names used for the parameters to be fitted can be chosen at will, within the restrictions of the \Rlang language, but of course the names used in \code{formula} and \code{start} must match each other.

\begin{knitrout}\footnotesize
\definecolor{shadecolor}{rgb}{0.969, 0.969, 0.969}\color{fgcolor}\begin{kframe}
\begin{alltt}
\hlkwd{ggplot}\hlstd{(Puromycin,} \hlkwd{aes}\hlstd{(conc, rate,} \hlkwc{colour} \hlstd{= state))} \hlopt{+}
  \hlkwd{geom_point}\hlstd{()} \hlopt{+}
  \hlkwd{geom_smooth}\hlstd{(}\hlkwc{method} \hlstd{=} \hlstr{"nls"}\hlstd{,}
              \hlkwc{method.args} \hlstd{=} \hlkwd{list}\hlstd{(}\hlkwc{formula} \hlstd{=  y} \hlopt{~} \hlstd{(Vmax} \hlopt{*} \hlstd{x)} \hlopt{/} \hlstd{(k} \hlopt{+} \hlstd{x),}
                                 \hlkwc{start} \hlstd{=} \hlkwd{list}\hlstd{(}\hlkwc{Vmax} \hlstd{=} \hlnum{200}\hlstd{,} \hlkwc{k} \hlstd{=} \hlnum{0.05}\hlstd{)),}
              \hlkwc{se} \hlstd{=} \hlnum{FALSE}\hlstd{)}
\end{alltt}
\end{kframe}
\end{knitrout}

\begin{warningbox}
The different geoms and elements can be added in almost any order to a ggplot object, but they will be plotted in the order that they are added. The \code{alpha} (transparency) aesthetic can be mapped to a constant to make underlying layers visible, or \code{alpha} can be mapped to a data variable, for example, making the transparency of points in a plot depend on the number of observations used in its calculation.

\begin{knitrout}\footnotesize
\definecolor{shadecolor}{rgb}{0.969, 0.969, 0.969}\color{fgcolor}\begin{kframe}
\begin{alltt}
\hlkwd{ggplot}\hlstd{(}\hlkwc{data} \hlstd{= mtcars,} \hlkwd{aes}\hlstd{(}\hlkwc{x} \hlstd{= disp,} \hlkwc{y} \hlstd{= mpg,} \hlkwc{colour} \hlstd{=} \hlkwd{factor}\hlstd{(cyl)))} \hlopt{+}
  \hlkwd{geom_smooth}\hlstd{(}\hlkwc{colour} \hlstd{=} \hlstr{"black"}\hlstd{,} \hlkwc{alpha} \hlstd{=} \hlnum{0.7}\hlstd{)} \hlopt{+}
  \hlkwd{geom_point}\hlstd{()} \hlopt{+}
  \hlkwd{theme_bw}\hlstd{()}
\end{alltt}
\end{kframe}
\end{knitrout}
\end{warningbox}

In some cases it is desirable to annotate plots with fitted model equations or fitted parameters. One way of achieving this is by fitting the model and then extracting the parameters to manually construct text strings to use for text or label annotations. However, package \pkgname{ggpmisc} makes it possible to automate such annotations.

\begin{knitrout}\footnotesize
\definecolor{shadecolor}{rgb}{0.969, 0.969, 0.969}\color{fgcolor}\begin{kframe}
\begin{alltt}
\hlstd{my.formula} \hlkwb{<-} \hlstd{y} \hlopt{~} \hlkwd{poly}\hlstd{(x,} \hlnum{2}\hlstd{)}
\hlkwd{ggplot}\hlstd{(}\hlkwc{data} \hlstd{= mtcars,} \hlkwd{aes}\hlstd{(}\hlkwc{x} \hlstd{= disp,} \hlkwc{y} \hlstd{= mpg,} \hlkwc{color} \hlstd{=} \hlkwd{factor}\hlstd{(cyl)))} \hlopt{+}
  \hlkwd{stat_smooth}\hlstd{(}\hlkwc{method} \hlstd{=} \hlstr{"lm"}\hlstd{,} \hlkwc{formula} \hlstd{= my.formula,} \hlkwc{colour} \hlstd{=} \hlstr{"black"}\hlstd{)} \hlopt{+}
  \hlkwd{stat_poly_eq}\hlstd{(}\hlkwc{formula} \hlstd{= my.formula,} \hlkwd{aes}\hlstd{(}\hlkwc{label} \hlstd{= ..eq.label..),}
               \hlkwc{colour} \hlstd{=} \hlstr{"black"}\hlstd{,} \hlkwc{parse} \hlstd{=} \hlnum{TRUE}\hlstd{,} \hlkwc{label.x.npc} \hlstd{=} \hlnum{0.3}\hlstd{)} \hlopt{+}
  \hlkwd{geom_point}\hlstd{()}
\end{alltt}
\end{kframe}

{\centering \includegraphics[width=.7\textwidth]{figure/pos-smooth-plot-12-1} 

}



\end{knitrout}

This same package makes it possible to annotate plots with summary tables from a model fit.
\begin{knitrout}\footnotesize
\definecolor{shadecolor}{rgb}{0.969, 0.969, 0.969}\color{fgcolor}\begin{kframe}
\begin{alltt}
\hlstd{my.formula} \hlkwb{<-} \hlstd{y} \hlopt{~} \hlkwd{poly}\hlstd{(x,} \hlnum{2}\hlstd{)}
\hlkwd{ggplot}\hlstd{(}\hlkwc{data} \hlstd{= mtcars,} \hlkwd{aes}\hlstd{(}\hlkwc{x} \hlstd{= disp,} \hlkwc{y} \hlstd{= mpg,} \hlkwc{color} \hlstd{=} \hlkwd{factor}\hlstd{(cyl)))} \hlopt{+}
  \hlkwd{stat_smooth}\hlstd{(}\hlkwc{method} \hlstd{=} \hlstr{"lm"}\hlstd{,} \hlkwc{formula} \hlstd{= my.formula,} \hlkwc{colour} \hlstd{=} \hlstr{"black"}\hlstd{)} \hlopt{+}
  \hlkwd{stat_fit_tb}\hlstd{(}\hlkwc{method} \hlstd{=} \hlstr{"lm"}\hlstd{,}
              \hlkwc{method.args} \hlstd{=} \hlkwd{list}\hlstd{(}\hlkwc{formula} \hlstd{= my.formula),}
              \hlkwc{colour} \hlstd{=} \hlstr{"black"}\hlstd{,}
              \hlkwc{tb.vars} \hlstd{=} \hlkwd{c}\hlstd{(}\hlkwc{Parameter} \hlstd{=} \hlstr{"term"}\hlstd{,}
                          \hlkwc{Estimate} \hlstd{=} \hlstr{"estimate"}\hlstd{,}
                          \hlstr{"s.e."} \hlstd{=} \hlstr{"std.error"}\hlstd{,}
                          \hlstr{"italic(t)"} \hlstd{=} \hlstr{"statistic"}\hlstd{,}
                          \hlstr{"italic(P)"} \hlstd{=} \hlstr{"p.value"}\hlstd{),}
              \hlkwc{label.y.npc} \hlstd{=} \hlstr{"top"}\hlstd{,} \hlkwc{label.x.npc} \hlstd{=} \hlstr{"right"}\hlstd{,}
              \hlkwc{parse} \hlstd{=} \hlnum{TRUE}\hlstd{)} \hlopt{+}
  \hlkwd{geom_point}\hlstd{()}
\end{alltt}
\end{kframe}

{\centering \includegraphics[width=.7\textwidth]{figure/pos-smooth-plot-13-1} 

}



\end{knitrout}

Package \pkgname{ggpmisc} provides additional \emph{statistics} for the annotation of plots based on fitted models. Please see the package documentation for details.

\index{plots!smooth curves|)}
\index{plots!fitted curves|)}

\subsection{Frequencies and counts}\label{sec:histogram}
\index{plots!histograms|(}
\index{density plots|(}

A different type of summaries are frequencies and empirical density functions. These can be calculated in one or more dimensions. Sometimes instead of being calculated, we rely on the density of graphical elements to convey the density. For example, scatter plots using well chosen values for \code{alpha} can give a satisfactory impression of the density. Rug plots, described below work in a similar way. See sections ....

Histograms are defined by how the plotted values are calculated. Although they are most frequently plotted as bar plots, many bar plots are not histograms. Although rarely done in practice, a histogram could be plotted using a different \emph{geometry} and \code{stat\_bin} the \emph{statistic} used by default by \gggeom{geom\_histogram()}. This \emph{statistic} does binning of observations before computing frequencies, and is suitable for continuous $x$ scales. When a factor is mapped to \code{x}, \code{stat\_count} should be used, which is the default \code{stat} for \gggeom{geom\_bar()}. These two \emph{geometries} are described in this section about statistics, because they default to using statistics different from \code{stat\_identity()} and consequently summarize the data.

\begin{knitrout}\footnotesize
\definecolor{shadecolor}{rgb}{0.969, 0.969, 0.969}\color{fgcolor}\begin{kframe}
\begin{alltt}
\hlkwd{set.seed}\hlstd{(}\hlnum{12345}\hlstd{)}
\hlstd{my.data} \hlkwb{<-}
\hlkwd{data.frame}\hlstd{(}\hlkwc{x} \hlstd{=} \hlkwd{rnorm}\hlstd{(}\hlnum{200}\hlstd{),}
\hlkwc{y} \hlstd{=} \hlkwd{c}\hlstd{(}\hlkwd{rnorm}\hlstd{(}\hlnum{100}\hlstd{,} \hlopt{-}\hlnum{1}\hlstd{,} \hlnum{1}\hlstd{),} \hlkwd{rnorm}\hlstd{(}\hlnum{100}\hlstd{,} \hlnum{1}\hlstd{,} \hlnum{1}\hlstd{)),}
\hlkwc{group} \hlstd{=} \hlkwd{factor}\hlstd{(}\hlkwd{rep}\hlstd{(}\hlkwd{c}\hlstd{(}\hlstr{"A"}\hlstd{,} \hlstr{"B"}\hlstd{),} \hlkwd{c}\hlstd{(}\hlnum{100}\hlstd{,} \hlnum{100}\hlstd{))) )}
\end{alltt}
\end{kframe}
\end{knitrout}

\begin{knitrout}\footnotesize
\definecolor{shadecolor}{rgb}{0.969, 0.969, 0.969}\color{fgcolor}\begin{kframe}
\begin{alltt}
\hlkwd{ggplot}\hlstd{(my.data,} \hlkwd{aes}\hlstd{(x))} \hlopt{+}
  \hlkwd{geom_histogram}\hlstd{(}\hlkwc{bins} \hlstd{=} \hlnum{15}\hlstd{)}
\end{alltt}
\end{kframe}

{\centering \includegraphics[width=.7\textwidth]{figure/pos-histogram-plot-01-1} 

}



\end{knitrout}

\begin{knitrout}\footnotesize
\definecolor{shadecolor}{rgb}{0.969, 0.969, 0.969}\color{fgcolor}\begin{kframe}
\begin{alltt}
\hlkwd{ggplot}\hlstd{(my.data,} \hlkwd{aes}\hlstd{(y,} \hlkwc{fill} \hlstd{= group))} \hlopt{+}
  \hlkwd{geom_histogram}\hlstd{(}\hlkwc{bins} \hlstd{=} \hlnum{15}\hlstd{,} \hlkwc{position} \hlstd{=} \hlstr{"dodge"}\hlstd{)}
\end{alltt}
\end{kframe}
\end{knitrout}

\begin{knitrout}\footnotesize
\definecolor{shadecolor}{rgb}{0.969, 0.969, 0.969}\color{fgcolor}\begin{kframe}
\begin{alltt}
\hlkwd{ggplot}\hlstd{(my.data,} \hlkwd{aes}\hlstd{(y,} \hlkwc{fill} \hlstd{= group))} \hlopt{+}
  \hlkwd{geom_histogram}\hlstd{(}\hlkwc{bins} \hlstd{=} \hlnum{15}\hlstd{,} \hlkwc{position} \hlstd{=} \hlstr{"stack"}\hlstd{)}
\end{alltt}
\end{kframe}
\end{knitrout}

\begin{knitrout}\footnotesize
\definecolor{shadecolor}{rgb}{0.969, 0.969, 0.969}\color{fgcolor}\begin{kframe}
\begin{alltt}
\hlkwd{ggplot}\hlstd{(my.data,} \hlkwd{aes}\hlstd{(y,} \hlkwc{fill} \hlstd{= group))} \hlopt{+}
  \hlkwd{geom_histogram}\hlstd{(}\hlkwc{bins} \hlstd{=} \hlnum{15}\hlstd{,} \hlkwc{position} \hlstd{=} \hlstr{"identity"}\hlstd{,} \hlkwc{alpha} \hlstd{=} \hlnum{0.5}\hlstd{)}
\end{alltt}
\end{kframe}
\end{knitrout}

The \emph{geometry} \gggeom{geom\_bin2d()} by default uses the \emph{statistic} \code{stat\_bin2d} which can be thought as a frequency histogram in two dimensions. The frequency for each rectangular tile is mapped onto a \code{fill} scale.

\begin{knitrout}\footnotesize
\definecolor{shadecolor}{rgb}{0.969, 0.969, 0.969}\color{fgcolor}\begin{kframe}
\begin{alltt}
\hlkwd{ggplot}\hlstd{(my.data,} \hlkwd{aes}\hlstd{(x, y))} \hlopt{+}
  \hlkwd{geom_bin2d}\hlstd{(}\hlkwc{bins} \hlstd{=} \hlnum{8}\hlstd{)}
\end{alltt}
\end{kframe}

{\centering \includegraphics[width=.7\textwidth]{figure/pos-bin2d-plot-01-1} 

}



\end{knitrout}

The \emph{geometry} \gggeom{geom\_hex()} is the equivalent of \gggeom{geom\_bin2d()} using hexagonal tiles instead of square tiles. By default uses the \emph{statistic} \ggstat{stat\_binhex()}. The frequency for each hexagon is mapped onto a \code{fill} scale.

\begin{knitrout}\footnotesize
\definecolor{shadecolor}{rgb}{0.969, 0.969, 0.969}\color{fgcolor}\begin{kframe}
\begin{alltt}
\hlkwd{ggplot}\hlstd{(my.data,} \hlkwd{aes}\hlstd{(x, y))} \hlopt{+}
  \hlkwd{geom_hex}\hlstd{(}\hlkwc{bins} \hlstd{=} \hlnum{8}\hlstd{)}
\end{alltt}
\end{kframe}

{\centering \includegraphics[width=.7\textwidth]{figure/pos-hex-plot-01-1} 

}



\end{knitrout}
\index{plots!histograms|)}

\subsection{Density functions}\label{sec:plot:density}
\index{plots!density plot!1 dimension|(}
\index{plots!statistics!density}
Empirical density functions are the equivalent of a histogram, but are continuous and not calculated using bins. They can be calculated in 1 or 2 dimensions (2d), for $x$ or $x$ and $y$ respectively. As with histograms it is possible to use different \emph{geometries} to visualize them. Examples of the use of \gggeom{geom\_density()} to create 1D density plots follow.

\begin{knitrout}\footnotesize
\definecolor{shadecolor}{rgb}{0.969, 0.969, 0.969}\color{fgcolor}\begin{kframe}
\begin{alltt}
\hlkwd{ggplot}\hlstd{(my.data,} \hlkwd{aes}\hlstd{(y,} \hlkwc{colour} \hlstd{= group))} \hlopt{+}
  \hlkwd{geom_density}\hlstd{()}
\end{alltt}
\end{kframe}

{\centering \includegraphics[width=.7\textwidth]{figure/pos-density-plot-01-1} 

}



\end{knitrout}

A semitransparent fill can be used instead of coloured lines.

\begin{knitrout}\footnotesize
\definecolor{shadecolor}{rgb}{0.969, 0.969, 0.969}\color{fgcolor}\begin{kframe}
\begin{alltt}
\hlkwd{ggplot}\hlstd{(my.data,} \hlkwd{aes}\hlstd{(y,} \hlkwc{fill} \hlstd{= group))} \hlopt{+}
  \hlkwd{geom_density}\hlstd{(}\hlkwc{alpha} \hlstd{=} \hlnum{0.5}\hlstd{)}
\end{alltt}
\end{kframe}
\end{knitrout}
\index{plots!density plot!1 dimension|)}

\index{plots!density plot!2 dimensions|(}
\index{plots!statistics!density 2d}

Examples of the use of \gggeom{stat\_density\_2d()} to create 2D density plots follow. In the first example we use two \emph{geometries} which were earlier described, \code{geom\_point()} and \code{geom\_rug()} to plot the observations in the background. The \emph{statistic} computes and plots a two dimensional density ``map'' plotted using isolines. We map \code{group} to the \code{colour} \emph{aesthetic}.

\begin{knitrout}\footnotesize
\definecolor{shadecolor}{rgb}{0.969, 0.969, 0.969}\color{fgcolor}\begin{kframe}
\begin{alltt}
\hlkwd{ggplot}\hlstd{(my.data,} \hlkwd{aes}\hlstd{(x, y,} \hlkwc{colour} \hlstd{= group))} \hlopt{+}
  \hlkwd{geom_point}\hlstd{()} \hlopt{+}
  \hlkwd{geom_rug}\hlstd{()} \hlopt{+}
  \hlkwd{stat_density_2d}\hlstd{()}
\end{alltt}
\end{kframe}

{\centering \includegraphics[width=.7\textwidth]{figure/pos-density-plot-10-1} 

}



\end{knitrout}

In this case \gggeom{geom\_density\_2d()} is equivalent, and we can replace it in the last line in the chunk above.
\begin{knitrout}\footnotesize
\definecolor{shadecolor}{rgb}{0.969, 0.969, 0.969}\color{fgcolor}\begin{kframe}
\begin{alltt}
  \hlkwd{geom_density_2d}\hlstd{()}
\end{alltt}
\end{kframe}
\end{knitrout}

In the next example we plot the groups in separate panels, and use a \emph{geometry} supporting the \code{fill} \emph{aesthetic} and we map to it the variable \code{level}, computed by \code{stat\_density\_2d()}



\begin{knitrout}\footnotesize
\definecolor{shadecolor}{rgb}{0.969, 0.969, 0.969}\color{fgcolor}\begin{kframe}
\begin{alltt}
\hlkwd{ggplot}\hlstd{(my.data,} \hlkwd{aes}\hlstd{(x, y))} \hlopt{+}
\hlkwd{stat_density_2d}\hlstd{(}\hlkwd{aes}\hlstd{(}\hlkwc{fill} \hlstd{= ..level..),} \hlkwc{geom} \hlstd{=} \hlstr{"polygon"}\hlstd{)} \hlopt{+}
  \hlkwd{facet_wrap}\hlstd{(}\hlopt{~}\hlstd{group)}
\end{alltt}
\end{kframe}

{\centering \includegraphics[width=.7\textwidth]{figure/pos-density-plot-12-1} 

}



\end{knitrout}


\index{plots!density plot!2 dimensions|)}

\subsection{Box and whiskers plots}\label{sec:boxplot}
\index{box plots|see{plots, box and whiskers plot}}
\index{plots!box and whiskers plot|(}

Box and whiskers plots, also very frequently called just boxplots, are also summaries that convey some of the characteristics of a distribution. They are calculated and plotted by means of \gggeom{geom\_boxplot()}. Although they can be calculated and plotted based on just a few observations, they are not useful unless each box plot is based in more than 10 to 15 observations.

\begin{knitrout}\footnotesize
\definecolor{shadecolor}{rgb}{0.969, 0.969, 0.969}\color{fgcolor}\begin{kframe}
\begin{alltt}
\hlkwd{ggplot}\hlstd{(my.data,} \hlkwd{aes}\hlstd{(group, y))} \hlopt{+}
  \hlkwd{geom_boxplot}\hlstd{()}
\end{alltt}
\end{kframe}

{\centering \includegraphics[width=.54\textwidth]{figure/pos-bw-plot-01-1} 

}



\end{knitrout}

As with other \emph{geometries} their appearance obeys both the usual \emph{aesthetics} such as color, and others specific to these type of visual representation.
\index{plots!box and whiskers plot|)}

\subsection{Violin plots}\label{sec:plot:violin}
\index{plots!violin plot|(}

Violin plots are a more recent development than box plots, and usable with relatively large numbers of observations. They could be thought as being a sort of hybrid between an empirical density function (see section \ref{sec:plot:density} on page \pageref{sec:plot:density}) and a box plot (see section \ref{sec:boxplot} on page \pageref{sec:boxplot}). As is the case with box plots, they are particularly useful when comparing distributions of related data, side by side. They can be created with  \gggeom{geom\_violin()} as shown in the examples below.

\begin{knitrout}\footnotesize
\definecolor{shadecolor}{rgb}{0.969, 0.969, 0.969}\color{fgcolor}\begin{kframe}
\begin{alltt}
\hlkwd{ggplot}\hlstd{(my.data,} \hlkwd{aes}\hlstd{(group, y))} \hlopt{+}
  \hlkwd{geom_violin}\hlstd{()}
\end{alltt}
\end{kframe}
\end{knitrout}

\begin{knitrout}\footnotesize
\definecolor{shadecolor}{rgb}{0.969, 0.969, 0.969}\color{fgcolor}\begin{kframe}
\begin{alltt}
\hlkwd{ggplot}\hlstd{(my.data,} \hlkwd{aes}\hlstd{(group, y,} \hlkwc{fill} \hlstd{= group))} \hlopt{+}
  \hlkwd{geom_violin}\hlstd{(}\hlkwc{alpha} \hlstd{=} \hlnum{0.16}\hlstd{)} \hlopt{+}
  \hlkwd{geom_point}\hlstd{(}\hlkwc{alpha} \hlstd{=} \hlnum{0.33}\hlstd{,} \hlkwc{size} \hlstd{=} \hlkwd{rel}\hlstd{(}\hlnum{4}\hlstd{),}
             \hlkwc{colour} \hlstd{=} \hlstr{"black"}\hlstd{,} \hlkwc{shape} \hlstd{=} \hlnum{21}\hlstd{)}
\end{alltt}
\end{kframe}

{\centering \includegraphics[width=.54\textwidth]{figure/pos-violin-plot-02-1} 

}



\end{knitrout}

As with other \emph{geometries} their appearance obeys both the usual \emph{aesthetics} such as colour, and others specific to these type of visual representation.

Other types of displays related to violin plots are \emph{beeswarm} plots and \emph{sina} plots, and can be produced with \emph{geometries} defined in packages \pkgname{ggbeeswarm} and \pkgname{ggforce}, respectively. A minimal example is shown here. See the documentation of the packages for details about the many options in their use.

\begin{knitrout}\footnotesize
\definecolor{shadecolor}{rgb}{0.969, 0.969, 0.969}\color{fgcolor}\begin{kframe}
\begin{alltt}
\hlkwd{ggplot}\hlstd{(my.data,} \hlkwd{aes}\hlstd{(group, y))} \hlopt{+}
  \hlkwd{geom_quasirandom}\hlstd{()}
\end{alltt}
\end{kframe}

{\centering \includegraphics[width=.54\textwidth]{figure/pos-ggbeeswarm-plot-01-1} 

}



\end{knitrout}

\index{plots!violin plot|)}
\index{plots!statistics|)}

\section{Facets}
\index{plots!facets|(}
\index{plots!panels|see{plots, facets}}
Facets are a special kind of plots containing multiple panels in which the panels share some properties.
These sets of coordinated panels are a useful tool for visualizing complex data. These plots became popular through the \code{trellis} graphs in \langname{S}, and the \pkgname{lattice} package in \Rlang. The basic idea is to have row and/or columns of plots with common scales, all plots showing values for the same response variable. This is useful when there are multiple classification factors in a data set. Similarly looking plots but with free scales or with the same scale but a `floating' intercept are sometimes also useful. In \ggplot there are two possible types of facets: facets organized in a grid, and facets along a single `axis' but wrapped into several rows. These are produced by adding \Rfunction{facet\_grid()} or \Rfunction{facet\_wrap()} to a ggplot, respectively. In the examples below we use \gggeom{geom\_point()} but faceting can be used with any \Rclass{ggplot} object.



We create a single-panel plot that we will use through this section to demonstrate how the same plot changes as we add facets using different options. Only some plots are shown and you will need to run the code for the remaining examples to see the output.

\begin{knitrout}\footnotesize
\definecolor{shadecolor}{rgb}{0.969, 0.969, 0.969}\color{fgcolor}\begin{kframe}
\begin{alltt}
\hlstd{p} \hlkwb{<-} \hlkwd{ggplot}\hlstd{(}\hlkwc{data} \hlstd{= mtcars,} \hlkwd{aes}\hlstd{(mpg, wt))} \hlopt{+} \hlkwd{geom_point}\hlstd{()}
\hlstd{p}
\end{alltt}
\end{kframe}

{\centering \includegraphics[width=.7\textwidth]{figure/pos-facets-00-1} 

}



\end{knitrout}

A grid of panels has two dimensions, a vertical and horizontal one. The two dimensions can be assigned to two factors or discrete variables. Until recently a formula syntax was the only available one. Although this notation has been retained, the preferred syntax is currently to use the parameters \code{rows} and \code{cols}. We use \code{cols} in this example. Note that we need to use \code{vars()} to enclose the names of the variables in the data.

\begin{knitrout}\footnotesize
\definecolor{shadecolor}{rgb}{0.969, 0.969, 0.969}\color{fgcolor}\begin{kframe}
\begin{alltt}
\hlstd{p} \hlopt{+} \hlkwd{facet_grid}\hlstd{(}\hlkwc{cols} \hlstd{=} \hlkwd{vars}\hlstd{(cyl))}
\end{alltt}
\end{kframe}

{\centering \includegraphics[width=.7\textwidth]{figure/pos-facets-01-1} 

}



\end{knitrout}

Was in the ``historical notation'' written as
\begin{knitrout}\footnotesize
\definecolor{shadecolor}{rgb}{0.969, 0.969, 0.969}\color{fgcolor}\begin{kframe}
\begin{alltt}
\hlstd{p} \hlopt{+} \hlkwd{facet_grid}\hlstd{(.} \hlopt{~} \hlstd{cyl)}
\end{alltt}
\end{kframe}
\end{knitrout}

By default all panels have the same scale limits and share the plotting scape equally.
\begin{knitrout}\footnotesize
\definecolor{shadecolor}{rgb}{0.969, 0.969, 0.969}\color{fgcolor}\begin{kframe}
\begin{alltt}
\hlstd{p} \hlopt{+} \hlkwd{facet_grid}\hlstd{(}\hlkwc{cols} \hlstd{=} \hlkwd{vars}\hlstd{(cyl),} \hlkwc{scales} \hlstd{=} \hlstr{"free"}\hlstd{)}
\hlstd{p} \hlopt{+} \hlkwd{facet_grid}\hlstd{(}\hlkwc{cols} \hlstd{=} \hlkwd{vars}\hlstd{(cyl),} \hlkwc{scales} \hlstd{=} \hlstr{"free"}\hlstd{,} \hlkwc{space} \hlstd{=} \hlstr{"free"}\hlstd{)}
\end{alltt}
\end{kframe}
\end{knitrout}



To obtain a 2D grid we need to specify both \code{rows} and \code{cols}
\begin{knitrout}\footnotesize
\definecolor{shadecolor}{rgb}{0.969, 0.969, 0.969}\color{fgcolor}\begin{kframe}
\begin{alltt}
\hlstd{p} \hlopt{+} \hlkwd{facet_grid}\hlstd{(}\hlkwc{rows} \hlstd{=} \hlkwd{vars}\hlstd{(vs),} \hlkwc{cols} \hlstd{=} \hlkwd{vars}\hlstd{(am))}
\end{alltt}
\end{kframe}
\end{knitrout}



Margins display an additional column or row of panels with the combined data.
\begin{knitrout}\footnotesize
\definecolor{shadecolor}{rgb}{0.969, 0.969, 0.969}\color{fgcolor}\begin{kframe}
\begin{alltt}
\hlstd{p} \hlopt{+} \hlkwd{facet_grid}\hlstd{(}\hlkwc{cols} \hlstd{=} \hlkwd{vars}\hlstd{(cyl),} \hlkwc{margins} \hlstd{=} \hlnum{TRUE}\hlstd{)}
\end{alltt}
\end{kframe}

{\centering \includegraphics[width=.7\textwidth]{figure/pos-facets-06-1} 

}



\end{knitrout}

We can represent more than one variable per dimension of the grid of plot panels. For this example, we change the default labeller used for the panels by one that includes the name of the variable.
\begin{knitrout}\footnotesize
\definecolor{shadecolor}{rgb}{0.969, 0.969, 0.969}\color{fgcolor}\begin{kframe}
\begin{alltt}
\hlstd{p} \hlopt{+} \hlkwd{facet_grid}\hlstd{(}\hlkwc{cols} \hlstd{=} \hlkwd{vars}\hlstd{(vs, am),} \hlkwc{labeller} \hlstd{= label_both)}
\end{alltt}
\end{kframe}

{\centering \includegraphics[width=.7\textwidth]{figure/pos-facets-07-1} 

}



\end{knitrout}

\begin{explainbox}
Sometimes we may want to have mathematical expressions or Greek letters in the panel headings. The next example shows a way of achieving this. The key is to use a labeller that parses character strings into \Rlang expressions.

\begin{knitrout}\footnotesize
\definecolor{shadecolor}{rgb}{0.969, 0.969, 0.969}\color{fgcolor}\begin{kframe}
\begin{alltt}
\hlstd{mtcars}\hlopt{$}\hlstd{cyl12} \hlkwb{<-} \hlkwd{factor}\hlstd{(mtcars}\hlopt{$}\hlstd{cyl,}
                       \hlkwc{labels} \hlstd{=} \hlkwd{c}\hlstd{(}\hlstr{"alpha"}\hlstd{,} \hlstr{"beta"}\hlstd{,} \hlstr{"sqrt(x, y)"}\hlstd{))}
\hlstd{p1} \hlkwb{<-} \hlkwd{ggplot}\hlstd{(}\hlkwc{data} \hlstd{= mtcars,} \hlkwd{aes}\hlstd{(mpg, wt))} \hlopt{+}
      \hlkwd{geom_point}\hlstd{()} \hlopt{+}
      \hlkwd{facet_grid}\hlstd{(}\hlkwc{cols} \hlstd{=} \hlkwd{vars}\hlstd{(cyl12),} \hlkwc{labeller} \hlstd{= label_parsed)}
\end{alltt}
\end{kframe}
\end{knitrout}

More frequently we may need to include the levels of the factor used in the faceting as part of the labels. Here we use as \code{labeller} function \Rfunction{label\_bquote()} with a special syntax that allows us to use an expression where replacement based on the facet (panel) data takes place. See section \ref{sec:plot:plotmath} for an example of the use of \code{bquote()}, the \Rlang function on which this labeller is built upon.



\begin{knitrout}\footnotesize
\definecolor{shadecolor}{rgb}{0.969, 0.969, 0.969}\color{fgcolor}\begin{kframe}
\begin{alltt}
\hlstd{p} \hlopt{+} \hlkwd{facet_grid}\hlstd{(}\hlkwc{cols} \hlstd{=} \hlkwd{vars}\hlstd{(cyl),} \hlkwc{labeller} \hlstd{=} \hlkwd{label_bquote}\hlstd{(}\hlkwc{cols} \hlstd{=} \hlkwd{.}\hlstd{(cyl)}\hlopt{~}\hlstr{"cylinders"}\hlstd{))}
\end{alltt}
\end{kframe}
\end{knitrout}


\end{explainbox}
%\begin{infobox}
%\sloppy
%In versions of \ggplot before 2.0.0, \code{labeller} was not implemented for \Rfunction{facet\_wrap()}, it was only available for \Rfunction{facet\_grid()}.
%\end{infobox}

A minimal example of a wrapped facet. In this case the number of levels is small, when they are more the row of plots will be wrapped into two or more continuation rows. When using \Rfunction{facet\_wrap()} there is only one dimension, so no `.' is needed before or after the tilde.

\begin{knitrout}\footnotesize
\definecolor{shadecolor}{rgb}{0.969, 0.969, 0.969}\color{fgcolor}\begin{kframe}
\begin{alltt}
\hlstd{p} \hlopt{+} \hlkwd{facet_wrap}\hlstd{(}\hlkwc{facets} \hlstd{=} \hlkwd{vars}\hlstd{(cyl),} \hlkwc{ncol} \hlstd{=} \hlnum{2}\hlstd{)}
\end{alltt}
\end{kframe}

{\centering \includegraphics[width=.7\textwidth]{figure/pos-facets-13-1} 

}



\end{knitrout}

An example showing that even though faceting with \code{facet\_wrap()} is along a single possibly wrapped row, it is possible to produce facets based on more than one variable.



\begin{knitrout}\footnotesize
\definecolor{shadecolor}{rgb}{0.969, 0.969, 0.969}\color{fgcolor}\begin{kframe}
\begin{alltt}
\hlstd{p} \hlopt{+} \hlkwd{facet_wrap}\hlstd{(}\hlkwc{facets} \hlstd{=} \hlkwd{vars}\hlstd{(vs, am),} \hlkwc{ncol}\hlstd{=}\hlnum{2}\hlstd{,} \hlkwc{labeller} \hlstd{= label_both)}
\end{alltt}
\end{kframe}
\end{knitrout}


%In versions of \ggplot before 2.0.0, \code{labeller} was not implemented for
%\code{facet\_wrap()}, it was only available for \code{facet\_grid()}. In the current
%version it is implemented for both.
%
%<<echo=FALSE>>=
%opts_chunk$set(opts_fig_wide)
%@
%
%<<>>=
%p + facet_wrap(~ vs, labeller = label_bquote(alpha ^ .(vs)))
%@
\index{plots!facets|)}

\section{Scales}\label{sec:plot:scales}
\index{plots!scales|(}

Scales map data onto \emph{aesthetics}. There are different types of scales depending on the characteristics of the data being mapped: scales can be continuous or discrete. And of course, there are scales for different attributes of the plotted geometrical object, such as position (\code{x, y, z}), \code{size}, \code{colour}, \code{fill}, \code{alpha} or transparency, \code{angle}, justification, etc. This means that many properties of, for example, the symbols used in a plot can be either set by a constant, or mapped to data. The most direct mapping is \code{identity}, which means that the data is taken at its face value. In a numerical scale, say \ggscale{scale\_x\_continuous()}, this means that for example a `5' in the data is plotted at a position in the plot corresponding to the value `5' along the x-axis. A simple mapping could be a log10 transformation, that we can easily achieve with the pre-defined \code{scale\_x\_log10} in which case the position on the $x$-axis will be based on the logarithm of the original data. A continuous data variable can, be mapped to continuous scale either using an identity mapping or transformation, which for example could be useful if we want to map the value of a variable to the area of the symbol rather than its diameter.

Discrete scales work in a similar way. We can use \ggscale{scale\_colour\_identity()} and have in our data a variable with values that are valid colour names like "red" or "blue". However we can also map the \code{colour} aesthetic to a factor with levels like "control", and "treatment", an these levels will be mapped to colours from a palette or we can use \ggscale{scale\_colour\_manual()} to assign whatever colour we want to each level to be mapped. The same is true for other discrete manual scales like those for  \code{shape} and \code{linetype}. Remember that for example for colour, and `numbers' there are both discrete and continuous scales available. Mapping colour or fill to \code{NA} makes the mapped values invisible. The reverse, mapping \code{NA} values in the data to a specific colour or fill is also possible.

%
%
%\sloppy
%Advanced scale manipulation requires package \code{scales} to be loaded, although \ggplot (2.0.0 and later) re-export several functions from package \code{scales}. Some simple examples follow.

%\begin{infobox}
\subsection{Convenience functions: Axis- and key labels, titles, subtitles and captions}\label{sec:plot:titles}\label{sec:plot:labs}
\index{plots!labels|(}
\index{plots!title|(}
\index{plots!subtitle|(}
\index{plots!caption|(}
We start by describing convenience functions that are very frequently used as they help in keeping the code concise. These can be used both with continuous and discrete scales.
Axis and key labels are given by the \code{name} of scales (see section \ref{sec:plot:scales}). Titles and subtitle can be added with function \Rfunction{ggtitle()} (see section \ref{sec:plot:title}).
Function \Rfunction{labs()} is a convenience function usueful when we use the default scales but we want to manually set axis labels or key titles---the default name of a scale is the name of the variable or expression mapped to the corresponding \emph{aesthetic}. \Rfunction{labs()} accepts arguments named according to the names of the \emph{aesthetics}. In addition it accepts named arguments \code{title}, \code{subtitle}, \code{caption} and \code{tag}. All these can be character strings or \Rlang expressions (see section \ref{sec:plot:plotmath}).

\begin{knitrout}\footnotesize
\definecolor{shadecolor}{rgb}{0.969, 0.969, 0.969}\color{fgcolor}\begin{kframe}
\begin{alltt}
\hlkwd{ggplot}\hlstd{(}\hlkwc{data} \hlstd{= Orange,}
       \hlkwd{aes}\hlstd{(}\hlkwc{x} \hlstd{= age,} \hlkwc{y} \hlstd{= circumference,} \hlkwc{color} \hlstd{= Tree))} \hlopt{+}
  \hlkwd{geom_line}\hlstd{()} \hlopt{+}
  \hlkwd{geom_point}\hlstd{()} \hlopt{+}
  \hlkwd{expand_limits}\hlstd{(}\hlkwc{y} \hlstd{=} \hlnum{0}\hlstd{)} \hlopt{+}
  \hlkwd{labs}\hlstd{(}\hlkwc{title} \hlstd{=} \hlstr{"Growth of orange trees"}\hlstd{,}
       \hlkwc{subtitle} \hlstd{=} \hlstr{"Starting from 1968-12-31"}\hlstd{,}
       \hlkwc{caption} \hlstd{=} \hlstr{"see Draper, N. R. and Smith, H. (1998)"}\hlstd{,}
       \hlkwc{tag} \hlstd{=} \hlstr{"A"}\hlstd{,}
       \hlkwc{x} \hlstd{=} \hlstr{"Time (d)"}\hlstd{,}
       \hlkwc{y} \hlstd{=} \hlstr{"Circumference (mm)"}\hlstd{,}
       \hlkwc{color} \hlstd{=} \hlstr{"Tree\textbackslash{}nnumber"}\hlstd{)}
\end{alltt}
\end{kframe}

{\centering \includegraphics[width=.7\textwidth]{figure/pos-axis-labels-01-1} 

}



\end{knitrout}

There are in addition to \code{labs()} convenience functions for setting the axis labels, \Rfunction{xlab()} and \Rfunction{ylab()}. The code below can replace in part the call to \code{labs()} above.

\begin{knitrout}\footnotesize
\definecolor{shadecolor}{rgb}{0.969, 0.969, 0.969}\color{fgcolor}\begin{kframe}
\begin{alltt}
  \hlkwd{xlab}\hlstd{(}\hlstr{"Time (d)"}\hlstd{)} \hlopt{+}
  \hlkwd{ylab}\hlstd{(}\hlstr{"Circumference (mm)"}\hlstd{)} \hlopt{+}
  \hlkwd{ggtitle}\hlstd{(}\hlstr{"Growth of orange trees"}\hlstd{,}
          \hlkwc{subtitle} \hlstd{=} \hlstr{"Starting from 1968-12-31"}\hlstd{)}
\hlnum{NA}
\end{alltt}
\end{kframe}
\end{knitrout}

\begin{playground}
Make an empty plot (\code{ggplot()}) and add to it as title an expression producing $y = b_0 + b_1 x + b_2 x^2$. (Hint: have a look at the examples for the use of expressions in \code{plotmath} demo in \Rlang by typing \code{demo(plotmath)} at the \Rlang console.
\end{playground}

%\begin{warningbox}
%Check!!
%When setting or updating labels using either \Rfunction{labs()} or \Rfunction{update\_labels()} be aware that even though \code{color} and \code{colour} are synonyms for the same \emph{aesthetics}, the `name' used in the call to \Rfunction{aes()} must match the  `name' used when setting or updating the labels.
%\end{warningbox}
%
%The labels used in keys and axis tick-labels for factor levels can be changed through the different \emph{scales} as described in section \ref{sec:plot:scales} on page \pageref{sec:plot:scales}.
%
\index{plots!caption|)}
\index{plots!subtitle|)}
\index{plots!title|)}
\index{plots!labels|)}

\subsection{Continuous scales}\label{sec:plot:scales:continuous}

We start by listing the most frequently used arguments to the continuous scale functions: \code{name}, \code{breaks}, \code{minor\_breaks}, \code{labels}, \code{limits}, \code{expand}, \code{na.value}, \code{trans}, \code{guide}, and \code{position}. The value of \code{name} is used for axis labels or key title, it defaults to the mapped variable or expression. The arguments to \code{breaks} and \code{minor\_breaks} override the default locations of major and minor ticks and grid lines, setting them to \code{NULL} suppresses the ticks. By default the tick labels are generated from the value of \code{breaks} but an argument to \code{labels} of the same length as \code{breaks} replaces these defaults. The values of \code{limits} determine both the range of values in the data included and the plotting area as described above---by default the off-range observations are replaced by \code{NA} but it is also possible to ``squish'' these observations towards the edge of the plotting area. The argument to \code{expand} determines the size of the margins or padding added to the area delimited by \code{lims} to set the limits of the plotting area. The value passed to \code{na.value} is used as replacement for \code{NA} valued observations---most useful for \code{color} and \code{fill} aesthetics. The transformation object passed as argument to \code{trans} determines the transformation used---the transformation affects the rendering, but breaks and tick label remain expressed in the original data units. The argument to \code{guide} determines the type of key or removes the default key. Finally, \code{position} controls in the axis is located on the \code{"right"}, \code{"left"}, \code{"top"} or \code{"bottom"} of the plot. Depending on the scale in question not all these parameters are used.



We generate new fake data.

\begin{knitrout}\footnotesize
\definecolor{shadecolor}{rgb}{0.969, 0.969, 0.969}\color{fgcolor}\begin{kframe}
\begin{alltt}
\hlstd{fake2.data} \hlkwb{<-}
  \hlkwd{data.frame}\hlstd{(}\hlkwc{y} \hlstd{=} \hlkwd{c}\hlstd{(}\hlkwd{rnorm}\hlstd{(}\hlnum{20}\hlstd{,} \hlkwc{mean} \hlstd{=} \hlnum{20}\hlstd{,} \hlkwc{sd} \hlstd{=} \hlnum{5}\hlstd{),}
                   \hlkwd{rnorm}\hlstd{(}\hlnum{20}\hlstd{,} \hlkwc{mean} \hlstd{=} \hlnum{40}\hlstd{,} \hlkwc{sd} \hlstd{=} \hlnum{10}\hlstd{)),}
             \hlkwc{group} \hlstd{=} \hlkwd{factor}\hlstd{(}\hlkwd{c}\hlstd{(}\hlkwd{rep}\hlstd{(}\hlstr{"A"}\hlstd{,} \hlnum{20}\hlstd{),} \hlkwd{rep}\hlstd{(}\hlstr{"B"}\hlstd{,} \hlnum{20}\hlstd{))),}
             \hlkwc{z} \hlstd{=} \hlkwd{rnorm}\hlstd{(}\hlnum{40}\hlstd{,} \hlkwc{mean} \hlstd{=} \hlnum{12}\hlstd{,} \hlkwc{sd} \hlstd{=} \hlnum{6}\hlstd{))}
\end{alltt}
\end{kframe}
\end{knitrout}
\subsubsection{Limits}

Limits are relevant to all types of scales, but we here exemplify them using the $x$ and $y$ aesthetics.

In addition to the \code{lims()}, \code{xlim()} and \code{ylim()} convenience functions, limits can be set to constant values.
In the next example we set hard limits, which will exclude observations from both the plot and from any calculations of summaries or smoothers.

\begin{knitrout}\footnotesize
\definecolor{shadecolor}{rgb}{0.969, 0.969, 0.969}\color{fgcolor}\begin{kframe}
\begin{alltt}
\hlkwd{ggplot}\hlstd{(fake2.data,} \hlkwd{aes}\hlstd{(z, y))} \hlopt{+} \hlkwd{geom_point}\hlstd{()} \hlopt{+}
  \hlkwd{scale_y_continuous}\hlstd{(}\hlkwc{limits} \hlstd{=} \hlkwd{c}\hlstd{(}\hlnum{0}\hlstd{,} \hlnum{100}\hlstd{))}
\end{alltt}
\end{kframe}
\end{knitrout}

To set only one limit leaving the other free, we use \code{NA}.

\begin{knitrout}\footnotesize
\definecolor{shadecolor}{rgb}{0.969, 0.969, 0.969}\color{fgcolor}\begin{kframe}
\begin{alltt}
\hlkwd{ggplot}\hlstd{(fake2.data,} \hlkwd{aes}\hlstd{(z, y))} \hlopt{+} \hlkwd{geom_point}\hlstd{()} \hlopt{+}
  \hlkwd{scale_y_continuous}\hlstd{(}\hlkwc{limits} \hlstd{=} \hlkwd{c}\hlstd{(}\hlnum{50}\hlstd{,} \hlnum{NA}\hlstd{))}
\end{alltt}
\end{kframe}
\end{knitrout}

Functions \Rfunction{ylim()} and \Rfunction{xlim()} are convenience functions that set the limits to the default scales in use. We here exemplify the use of \Rfunction{ylim()} only, but \Rfunction{xlim()} is identical except for the scale it affects.

\begin{knitrout}\footnotesize
\definecolor{shadecolor}{rgb}{0.969, 0.969, 0.969}\color{fgcolor}\begin{kframe}
\begin{alltt}
\hlkwd{ggplot}\hlstd{(fake2.data,} \hlkwd{aes}\hlstd{(z, y))} \hlopt{+} \hlkwd{geom_point}\hlstd{()} \hlopt{+}
  \hlkwd{ylim}\hlstd{(}\hlnum{50}\hlstd{,} \hlnum{NA}\hlstd{)}
\end{alltt}
\end{kframe}
\end{knitrout}

\begin{warningbox}
  In the examples above we demonstrate on purpose how some observations laying outside the limits we set were removed. In general setting hard limits should be avoided, even though a warning is issued when observations are omitted, as it is easy to unwillingly subset the data being plotted.

In contrast, function \Rfunction{expand\_limits()} makes it possible to safely \emph{expand} the dynamically computed default limits of a scale---in this case the scale limits will grow past the expanded values when needed to accommodate all observations. The arguments to \code{x} and \code{y} are numeric vectors of length one or two each, matching how the limits of the $x$ and $y$ continuous scales are defined. Her we expand the limits to include the origin.

\begin{knitrout}\footnotesize
\definecolor{shadecolor}{rgb}{0.969, 0.969, 0.969}\color{fgcolor}\begin{kframe}
\begin{alltt}
\hlkwd{ggplot}\hlstd{(fake2.data,} \hlkwd{aes}\hlstd{(z, y))} \hlopt{+} \hlkwd{geom_point}\hlstd{()} \hlopt{+} \hlkwd{expand_limits}\hlstd{(}\hlkwc{y} \hlstd{=} \hlnum{0}\hlstd{,} \hlkwc{x} \hlstd{=} \hlnum{0}\hlstd{)}
\end{alltt}
\end{kframe}

{\centering \includegraphics[width=.54\textwidth]{figure/pos-scale-limits-04-1} 

}



\end{knitrout}

\end{warningbox}

The \code{expand} parameter of the scales plays a different role than \Rfunction{expand\_limits()}. It controls how much larger the plotting area is compared to the limits. In other words it adds a ``margin'' to the plotting are outside the limits set either dynamically or manually. Very rarely plots are drawn so that observations are plotted on top of the axes, avoiding this is a key function of \code{expand}. Rug plots and marginal annotations will also require the plotting are to be expanded. In \ggplot the default is to always apply expansion.

We here set the upper limit of the plotting area to be expanded by adding padding to the top and removing the padding from the bottom of the plotting area.
\begin{knitrout}\footnotesize
\definecolor{shadecolor}{rgb}{0.969, 0.969, 0.969}\color{fgcolor}\begin{kframe}
\begin{alltt}
\hlkwd{ggplot}\hlstd{(fake2.data,}
  \hlkwd{aes}\hlstd{(}\hlkwc{fill} \hlstd{= group,} \hlkwc{colour} \hlstd{= group,} \hlkwc{x} \hlstd{= y))} \hlopt{+}
  \hlkwd{stat_density}\hlstd{(}\hlkwc{alpha} \hlstd{=} \hlnum{0.3}\hlstd{)} \hlopt{+}
  \hlkwd{scale_y_continuous}\hlstd{(}\hlkwc{expand} \hlstd{=} \hlkwd{expand_scale}\hlstd{(}\hlkwc{add} \hlstd{=} \hlkwd{c}\hlstd{(}\hlnum{0}\hlstd{,} \hlnum{0.02}\hlstd{)))}
\end{alltt}
\end{kframe}
\end{knitrout}

Here we instead use a multiplier to a similar effect as above.
\begin{knitrout}\footnotesize
\definecolor{shadecolor}{rgb}{0.969, 0.969, 0.969}\color{fgcolor}\begin{kframe}
\begin{alltt}
\hlkwd{ggplot}\hlstd{(fake2.data,}
  \hlkwd{aes}\hlstd{(}\hlkwc{fill} \hlstd{= group,} \hlkwc{colour} \hlstd{= group,} \hlkwc{x} \hlstd{= y))} \hlopt{+}
  \hlkwd{stat_density}\hlstd{(}\hlkwc{alpha} \hlstd{=} \hlnum{0.3}\hlstd{)} \hlopt{+}
  \hlkwd{scale_y_continuous}\hlstd{(}\hlkwc{expand} \hlstd{=} \hlkwd{expand_scale}\hlstd{(}\hlkwc{mult} \hlstd{=} \hlkwd{c}\hlstd{(}\hlnum{0}\hlstd{,} \hlnum{0.1}\hlstd{)))}
\end{alltt}
\end{kframe}
\end{knitrout}

In the case of scales, we cannot reverse their direction through the setting of limits. We need instead to use a transformation as described in the next section.

%%% to be moved
We can also use \code{limits} with discrete scales, listing all or some of the levels of a factor that are to be included in the scale. This works even if the levels are defined in the factor but not present in a given data set, such as after subsetting.

\begin{playground}
Test what is the result if the first limit is larger than the second one. Is it the same as when setting these same values as limits with \code{ylim()} as shown above?

\begin{knitrout}\footnotesize
\definecolor{shadecolor}{rgb}{0.969, 0.969, 0.969}\color{fgcolor}\begin{kframe}
\begin{alltt}
\hlkwd{ggplot}\hlstd{(fake2.data,} \hlkwd{aes}\hlstd{(z, y))} \hlopt{+} \hlkwd{geom_point}\hlstd{()} \hlopt{+}
  \hlkwd{scale_y_continuous}\hlstd{(}\hlkwc{limits} \hlstd{=} \hlkwd{c}\hlstd{(}\hlnum{100}\hlstd{,} \hlnum{0}\hlstd{))}
\end{alltt}
\end{kframe}
\end{knitrout}
\end{playground}

\subsubsection{Ticks and their labels}

Parameter \code{breaks}\index{plots!scales!continuous!tick breaks} is used to set the location of ticks along the axis. Parameter \code{labels}\index{plots!scales!continuous!tick labels} is used to set the tick label to be used at each break. Both can be passed either a vector or a function as argument. The default is to compute good ones based on the limits and format the numbers as strings.

When manually setting breaks, we can keep the default computed labels for the \code{breaks}.
\begin{knitrout}\footnotesize
\definecolor{shadecolor}{rgb}{0.969, 0.969, 0.969}\color{fgcolor}\begin{kframe}
\begin{alltt}
\hlkwd{ggplot}\hlstd{(fake2.data,} \hlkwd{aes}\hlstd{(z, y))} \hlopt{+}
  \hlkwd{geom_point}\hlstd{()} \hlopt{+}
  \hlkwd{scale_y_continuous}\hlstd{(}\hlkwc{breaks} \hlstd{=} \hlkwd{c}\hlstd{(}\hlnum{20}\hlstd{, pi} \hlopt{*} \hlnum{10}\hlstd{,} \hlnum{40}\hlstd{,} \hlnum{60}\hlstd{))}
\end{alltt}
\end{kframe}
\end{knitrout}

We can also set tick labels manually, in parallel to the setting of \code{breaks} by passing as arguments two vectors of equal length. Using an expression we obtain a Greek letter.

\begin{knitrout}\footnotesize
\definecolor{shadecolor}{rgb}{0.969, 0.969, 0.969}\color{fgcolor}\begin{kframe}
\begin{alltt}
\hlkwd{ggplot}\hlstd{(fake2.data,} \hlkwd{aes}\hlstd{(z, y))} \hlopt{+}
  \hlkwd{geom_point}\hlstd{()} \hlopt{+}
  \hlkwd{scale_y_continuous}\hlstd{(}\hlkwc{breaks} \hlstd{=} \hlkwd{c}\hlstd{(}\hlnum{20}\hlstd{, pi} \hlopt{*} \hlnum{10}\hlstd{,} \hlnum{40}\hlstd{,} \hlnum{60}\hlstd{),}
                     \hlkwc{labels} \hlstd{=} \hlkwd{c}\hlstd{(}\hlstr{"20"}\hlstd{,} \hlkwd{expression}\hlstd{(}\hlnum{10}\hlopt{*}\hlstd{pi),} \hlstr{"40"}\hlstd{,} \hlstr{"60"}\hlstd{))}
\end{alltt}
\end{kframe}

{\centering \includegraphics[width=.54\textwidth]{figure/pos-scale-ticks-02-1} 

}



\end{knitrout}

Package \pkgname{scales} provides several functions for the automatic generation of labels. To display tick labels as percentages for data available as decimal fractions, we can use function \code{scales::percent()}.

\begin{knitrout}\footnotesize
\definecolor{shadecolor}{rgb}{0.969, 0.969, 0.969}\color{fgcolor}\begin{kframe}
\begin{alltt}
\hlkwd{ggplot}\hlstd{(fake2.data,} \hlkwd{aes}\hlstd{(z, y} \hlopt{/} \hlkwd{max}\hlstd{(y)))} \hlopt{+}
  \hlkwd{geom_point}\hlstd{()} \hlopt{+}
  \hlkwd{scale_y_continuous}\hlstd{(}\hlkwc{labels} \hlstd{= scales}\hlopt{::}\hlstd{percent)}
\end{alltt}
\end{kframe}
\end{knitrout}

In the case of currency we can use \code{scales::dollar()}, to use commas to separate thousands, millions, and so on, we can use \code{scales::comma()}, for exponents of 10, useful for logarithmic-transformed scales \code{scales::scientific\_format()}.
It is also possible to use user-defined functions both for breaks and labels.

\subsubsection{Transformed scales}

The\index{plots!scales!continuous!transformations} default scale used by the \code{y} aesthetic uses \code{trans = "identity"}, but there are also available predefined scales with transformations.

Although transformations can be passed as argument to \ggscale{scale\_x\_continuous()} and \ggscale{scale\_y\_continuous()}, there are predefined convenience scale functions for \code{log10}, \code{sqrt} and \code{reverse}.

\begin{warningbox}
  Similarly to the maths functions of R, the name of the scales are \ggscale{scale\_x\_log10()} and \ggscale{scale\_y\_log10()} rather than \ggscale{scale\_y\_log()} because in R the function \code{log} returns the natural or Neperian logarithm.
\end{warningbox}

We can use \ggscale{scale\_x\_reverse()} to reverse the direction of a continuous scale,

\begin{knitrout}\footnotesize
\definecolor{shadecolor}{rgb}{0.969, 0.969, 0.969}\color{fgcolor}\begin{kframe}
\begin{alltt}
\hlkwd{ggplot}\hlstd{(fake2.data,} \hlkwd{aes}\hlstd{(z, y))} \hlopt{+} \hlkwd{geom_point}\hlstd{()} \hlopt{+} \hlkwd{scale_x_reverse}\hlstd{()}
\end{alltt}
\end{kframe}

{\centering \includegraphics[width=.54\textwidth]{figure/pos-scale-trans-01-1} 

}



\end{knitrout}

Axis tick-labels display the original values before applying the transformation. The \code{"breaks"} need to be given in the original scale as well. We use \ggscale{scale\_y\_log10()} to apply a $\log_{10}$ transformation to the $y$ values.

\begin{knitrout}\footnotesize
\definecolor{shadecolor}{rgb}{0.969, 0.969, 0.969}\color{fgcolor}\begin{kframe}
\begin{alltt}
\hlkwd{ggplot}\hlstd{(fake2.data,} \hlkwd{aes}\hlstd{(z, y))} \hlopt{+}
  \hlkwd{geom_point}\hlstd{()} \hlopt{+}
  \hlkwd{scale_y_log10}\hlstd{(}\hlkwc{breaks}\hlstd{=}\hlkwd{c}\hlstd{(}\hlnum{10}\hlstd{,}\hlnum{20}\hlstd{,}\hlnum{50}\hlstd{,}\hlnum{100}\hlstd{))}
\end{alltt}
\end{kframe}
\end{knitrout}

In contrast, transforming the data on-the-fly when mapping it to the $x$ \emph{aesthetic}, results in tick-labels expressed in the logarithm of the original data.

\begin{knitrout}\footnotesize
\definecolor{shadecolor}{rgb}{0.969, 0.969, 0.969}\color{fgcolor}\begin{kframe}
\begin{alltt}
\hlkwd{ggplot}\hlstd{(fake2.data,} \hlkwd{aes}\hlstd{(z,} \hlkwd{log10}\hlstd{(y)))} \hlopt{+} \hlkwd{geom_point}\hlstd{()}
\end{alltt}
\end{kframe}
\end{knitrout}

We show next how to specify a transformation to a continuous scale, using a predefined ``transformation'' object.

\begin{knitrout}\footnotesize
\definecolor{shadecolor}{rgb}{0.969, 0.969, 0.969}\color{fgcolor}\begin{kframe}
\begin{alltt}
\hlkwd{ggplot}\hlstd{(fake2.data,} \hlkwd{aes}\hlstd{(z, y))} \hlopt{+} \hlkwd{geom_point}\hlstd{()} \hlopt{+}
  \hlkwd{scale_y_continuous}\hlstd{(}\hlkwc{trans} \hlstd{=} \hlstr{"reciprocal"}\hlstd{)}
\end{alltt}
\end{kframe}
\end{knitrout}

Natural logarithms are important in growth analysis as the slope against time gives the relative growth rate. We show this with the \code{Orange} data set.

\begin{knitrout}\footnotesize
\definecolor{shadecolor}{rgb}{0.969, 0.969, 0.969}\color{fgcolor}\begin{kframe}
\begin{alltt}
\hlkwd{ggplot}\hlstd{(}\hlkwc{data} \hlstd{= Orange,}
       \hlkwd{aes}\hlstd{(}\hlkwc{x} \hlstd{= age,} \hlkwc{y} \hlstd{= circumference,} \hlkwc{color} \hlstd{= Tree))} \hlopt{+}
  \hlkwd{geom_line}\hlstd{()} \hlopt{+}
  \hlkwd{geom_point}\hlstd{()} \hlopt{+}
  \hlkwd{scale_y_continuous}\hlstd{(}\hlkwc{trans} \hlstd{=} \hlstr{"log"}\hlstd{,} \hlkwc{breaks} \hlstd{=} \hlkwd{c}\hlstd{(}\hlnum{20}\hlstd{,} \hlnum{50}\hlstd{,} \hlnum{100}\hlstd{,} \hlnum{200}\hlstd{))}
\end{alltt}
\end{kframe}
\end{knitrout}

\begin{warningbox}
When combining scale transformations and \emph{statistics}, one should be aware of which data are used, transformed or not. The two last examples above are not equivalent, as the transformation applied to the scale does not affect the data.
\end{warningbox}

\subsection{Time and date scales for $x$ and $y$}

\subsubsection{Limits}

Time\index{plots!scales!time!limits} and date scales are conceptually similar to continuous numeric scales, but use special data types and formatting for labels. We can set limits and breaks using constants as time or dates. These are most easily input with the functions in packages \pkgname{lubridate} or \pkgname{anytime}.

Please, see section \ref{sec:ggpmisc:trydf} on page \pageref{sec:ggpmisc:trydf} for examples.

\subsubsection{Axis labels}

By\index{plots!scales!time!axis labels} default the tick labels produced and their formatting is automatically selected based on the extent of the time data. For example, if we have all data collected within a single day, then the tick labels will show hours and minutes. If we plot data for several years, the labels will show the date portion of the time instant. The default is frequently good enough, but it is possible, as for numbers to use different formatter functions to generate the tick labels.

\subsection{Discrete scales for $x$ and $y$}

In\index{plots!scales!limits}\index{plots!scales!discrete} the case of ordered or unordered factors, the tick labels are by default the names of the factor levels. Consequently one roundabout way to obtaining the desired tick labels is to use them as factor levels. This approach is not recommended as in most cases the text of the desired tick labels may not be recognized as a valid name making the code using them difficult to type in scripts or at the command prompt. It is best to use simple mnemonic short names for factor levels and variables, and to set suitable through \emph{scales} when plotting, as we will show here.

We use here once again the \code{mpg} data set.
We can use \ggscale{scale\_x\_discrete()} to reorder and select the columns without altering the data. If we use this approach to subset the data, then to avoid warnings we need to add \code{na.rm = TRUE}. We additionally use \code{scale\_x\_discret} to convert level names to uppercase.

\begin{knitrout}\footnotesize
\definecolor{shadecolor}{rgb}{0.969, 0.969, 0.969}\color{fgcolor}\begin{kframe}
\begin{alltt}
\hlkwd{ggplot}\hlstd{(mpg,} \hlkwd{aes}\hlstd{(class, hwy))} \hlopt{+}
  \hlkwd{stat_summary}\hlstd{(}\hlkwc{geom} \hlstd{=} \hlstr{"col"}\hlstd{,} \hlkwc{fun.y} \hlstd{= mean,} \hlkwc{na.rm} \hlstd{=} \hlnum{TRUE}\hlstd{)} \hlopt{+}
  \hlkwd{scale_x_discrete}\hlstd{(}\hlkwc{limits} \hlstd{=} \hlkwd{c}\hlstd{(}\hlstr{"compact"}\hlstd{,} \hlstr{"subcompact"}\hlstd{,} \hlstr{"midsize"}\hlstd{),}
                   \hlkwc{labels} \hlstd{=} \hlkwd{c}\hlstd{(}\hlstr{"COMPACT"}\hlstd{,} \hlstr{"SUBCOMPACT"}\hlstd{,} \hlstr{"MIDSIZE"}\hlstd{))}
\end{alltt}
\end{kframe}

{\centering \includegraphics[width=.54\textwidth]{figure/pos-scale-discrete-10-1} 

}



\end{knitrout}

If only case needs to be changed, to \Rfunction{toupper()} and \Rfunction{tolower()} allow a more general and less error prone approach.
\begin{knitrout}\footnotesize
\definecolor{shadecolor}{rgb}{0.969, 0.969, 0.969}\color{fgcolor}\begin{kframe}
\begin{alltt}
  \hlkwd{scale_x_discrete}\hlstd{(}\hlkwc{limits} \hlstd{=} \hlkwd{c}\hlstd{(}\hlstr{"compact"}\hlstd{,} \hlstr{"subcompact"}\hlstd{,} \hlstr{"midsize"}\hlstd{),}
                   \hlkwc{labels} \hlstd{= toupper)}
\end{alltt}
\end{kframe}
\end{knitrout}

Alternatively, we can change the order of the columns in the plot by reordering the levels of factor \code{mpg\$class}. This approach makes sense if the ordering needs to be done programmatically. See section \ref{sec:calc:factors} on page \pageref{sec:calc:factors} for details. The example below shows how to reorder the columns based on the \code{mean()} of \code{hwy}.

\begin{knitrout}\footnotesize
\definecolor{shadecolor}{rgb}{0.969, 0.969, 0.969}\color{fgcolor}\begin{kframe}
\begin{alltt}
\hlkwd{ggplot}\hlstd{(mpg,} \hlkwd{aes}\hlstd{(}\hlkwd{reorder}\hlstd{(}\hlkwc{x} \hlstd{=} \hlkwd{factor}\hlstd{(class),} \hlkwc{X} \hlstd{= hwy,} \hlkwc{FUN} \hlstd{= mean), hwy))} \hlopt{+}
  \hlkwd{stat_summary}\hlstd{(}\hlkwc{geom} \hlstd{=} \hlstr{"col"}\hlstd{,} \hlkwc{fun.y} \hlstd{= mean)}
\end{alltt}
\end{kframe}
\end{knitrout}

\subsection{Size}

For\index{plots!scales!size} the \code{size} \emph{aesthetic} several scales are available, both discrete and continuous. They do not differ much from those already described above. \emph{Geometries} \gggeom{geom\_point()}, \gggeom{geom\_line()}, \gggeom{geom\_hline()}, \gggeom{geom\_vline()}, \gggeom{geom\_text()}, \gggeom{geom\_label()} obey \code{size} as expected. In the case of \gggeom{geom\_bar()}, \gggeom{geom\_col()}, \gggeom{geom\_area()} and all other geometric elements bordered by lines, \code{size} is obeyed by these border lines. In fact, other aesthetics natural for lines such as \code{linetype} also apply to these borders.

When using \code{size} scales, \code{breaks} and \code{labels} affect the key or \code{guide}. In scales that produce a key passing \code{guide = FALSE} removes the key corresponding to the scale.

\subsection{Color and fill}

Colour\index{scales!color}\index{scales!fill} and fill scales are similar, but they affect different elements of the plot. All visual elements in a plot obey the \code{color} \emph{aesthetic}, but only elements that have an inner region and a boundary, obey both \code{color} and \code{fill} \emph{aesthetics}. There are separate but equivalent sets of scales available for these two aesthetics. We will describe in more detail the \code{color} \emph{aesthetic} and give only some examples for \code{fill}. We will however, start by reviewing how colors are defined and used in R.

\subsubsection{Color definitions in R}\label{sec:plot:colors}
\index{colour|see{color}}
\index{color!definitions|(}
\index{color!using|(}
Colors can be specified in R not only through character strings with the names of previously defined colors, but also directly as strings describing the RGB components as hexadecimal numbers (on base 16) such as \code{"\#FFFFFF"} for white or \code{"\#000000"} for black, or \code{"\#FF0000"} for the brightest available pure red. The list of color names known to R can be obtained be entering \code{colors()} in the console.

Given the number of colors available, we may want to subset them based on their names. Function \code{colors()} returns a character vector. We can use \code{grep()} or \code{grepl()} to find indexes to the names containing a given character substring, in this example \code{"dark"}.

\begin{knitrout}\footnotesize
\definecolor{shadecolor}{rgb}{0.969, 0.969, 0.969}\color{fgcolor}\begin{kframe}
\begin{alltt}
\hlkwd{grep}\hlstd{(}\hlstr{"dark"}\hlstd{,}\hlkwd{colors}\hlstd{())}
\end{alltt}
\begin{verbatim}
##  [1]  73  74  75  76  77  78  79  80  81  82  83  84  85  86  87  88  89
## [18]  90  91  92  93  94  95  96  97  98  99 100 101 102 103 104 105 106
## [35] 107 108 109 110 111 112 113 114 115
\end{verbatim}
\end{kframe}
\end{knitrout}

\begin{playground}
Replace \Rfunction{grep()} by \Rfunction{grepl()} in the example above. What is the difference in the returned value?
\end{playground}

Although the vector of indexes, or the logical vector, could be used to extract the subset of matching color names with code like,

\begin{knitrout}\footnotesize
\definecolor{shadecolor}{rgb}{0.969, 0.969, 0.969}\color{fgcolor}\begin{kframe}
\begin{alltt}
\hlkwd{colors}\hlstd{()[}\hlkwd{grep}\hlstd{(}\hlstr{"dark"}\hlstd{,} \hlkwd{colors}\hlstd{())]}
\end{alltt}
\begin{verbatim}
##  [1] "darkblue"        "darkcyan"        "darkgoldenrod"  
##  [4] "darkgoldenrod1"  "darkgoldenrod2"  "darkgoldenrod3" 
##  [7] "darkgoldenrod4"  "darkgray"        "darkgreen"      
## [10] "darkgrey"        "darkkhaki"       "darkmagenta"    
## [13] "darkolivegreen"  "darkolivegreen1" "darkolivegreen2"
## [16] "darkolivegreen3" "darkolivegreen4" "darkorange"     
## [19] "darkorange1"     "darkorange2"     "darkorange3"    
## [22] "darkorange4"     "darkorchid"      "darkorchid1"    
## [25] "darkorchid2"     "darkorchid3"     "darkorchid4"    
## [28] "darkred"         "darksalmon"      "darkseagreen"   
## [31] "darkseagreen1"   "darkseagreen2"   "darkseagreen3"  
## [34] "darkseagreen4"   "darkslateblue"   "darkslategray"  
## [37] "darkslategray1"  "darkslategray2"  "darkslategray3" 
## [40] "darkslategray4"  "darkslategrey"   "darkturquoise"  
## [43] "darkviolet"
\end{verbatim}
\end{kframe}
\end{knitrout}

a simpler approach is available.

\begin{knitrout}\footnotesize
\definecolor{shadecolor}{rgb}{0.969, 0.969, 0.969}\color{fgcolor}\begin{kframe}
\begin{alltt}
\hlkwd{grep}\hlstd{(}\hlstr{"dark"}\hlstd{,} \hlkwd{colors}\hlstd{(),} \hlkwc{value} \hlstd{=} \hlnum{TRUE}\hlstd{)}
\end{alltt}
\begin{verbatim}
##  [1] "darkblue"        "darkcyan"        "darkgoldenrod"  
##  [4] "darkgoldenrod1"  "darkgoldenrod2"  "darkgoldenrod3" 
##  [7] "darkgoldenrod4"  "darkgray"        "darkgreen"      
## [10] "darkgrey"        "darkkhaki"       "darkmagenta"    
## [13] "darkolivegreen"  "darkolivegreen1" "darkolivegreen2"
## [16] "darkolivegreen3" "darkolivegreen4" "darkorange"     
## [19] "darkorange1"     "darkorange2"     "darkorange3"    
## [22] "darkorange4"     "darkorchid"      "darkorchid1"    
## [25] "darkorchid2"     "darkorchid3"     "darkorchid4"    
## [28] "darkred"         "darksalmon"      "darkseagreen"   
## [31] "darkseagreen1"   "darkseagreen2"   "darkseagreen3"  
## [34] "darkseagreen4"   "darkslateblue"   "darkslategray"  
## [37] "darkslategray1"  "darkslategray2"  "darkslategray3" 
## [40] "darkslategray4"  "darkslategrey"   "darkturquoise"  
## [43] "darkviolet"
\end{verbatim}
\end{kframe}
\end{knitrout}

To retrieve the RGB values for a color definition we use

\begin{knitrout}\footnotesize
\definecolor{shadecolor}{rgb}{0.969, 0.969, 0.969}\color{fgcolor}\begin{kframe}
\begin{alltt}
\hlkwd{col2rgb}\hlstd{(}\hlstr{"purple"}\hlstd{)}
\end{alltt}
\begin{verbatim}
##       [,1]
## red    160
## green   32
## blue   240
\end{verbatim}
\begin{alltt}
\hlkwd{col2rgb}\hlstd{(}\hlstr{"#FF0000"}\hlstd{)}
\end{alltt}
\begin{verbatim}
##       [,1]
## red    255
## green    0
## blue     0
\end{verbatim}
\end{kframe}
\end{knitrout}

Color definitions in R can contain a \emph{transparency} described by an \code{alpha} value, which by default is not returned.

\begin{knitrout}\footnotesize
\definecolor{shadecolor}{rgb}{0.969, 0.969, 0.969}\color{fgcolor}\begin{kframe}
\begin{alltt}
\hlkwd{col2rgb}\hlstd{(}\hlstr{"purple"}\hlstd{,} \hlkwc{alpha} \hlstd{=} \hlnum{TRUE}\hlstd{)}
\end{alltt}
\begin{verbatim}
##       [,1]
## red    160
## green   32
## blue   240
## alpha  255
\end{verbatim}
\end{kframe}
\end{knitrout}

With function \Rfunction{rgb()} we can define new named or nameless colors.
\begin{knitrout}\footnotesize
\definecolor{shadecolor}{rgb}{0.969, 0.969, 0.969}\color{fgcolor}\begin{kframe}
\begin{alltt}
\hlkwd{rgb}\hlstd{(}\hlnum{1}\hlstd{,} \hlnum{1}\hlstd{,} \hlnum{0}\hlstd{)}
\end{alltt}
\begin{verbatim}
## [1] "#FFFF00"
\end{verbatim}
\begin{alltt}
\hlkwd{rgb}\hlstd{(}\hlnum{1}\hlstd{,} \hlnum{1}\hlstd{,} \hlnum{0}\hlstd{,} \hlkwc{names} \hlstd{=} \hlstr{"my.color"}\hlstd{)}
\end{alltt}
\begin{verbatim}
##  my.color 
## "#FFFF00"
\end{verbatim}
\begin{alltt}
\hlkwd{rgb}\hlstd{(}\hlnum{255}\hlstd{,} \hlnum{255}\hlstd{,} \hlnum{0}\hlstd{,} \hlkwc{names} \hlstd{=} \hlstr{"my.color"}\hlstd{,} \hlkwc{maxColorValue} \hlstd{=} \hlnum{255}\hlstd{)}
\end{alltt}
\begin{verbatim}
##  my.color 
## "#FFFF00"
\end{verbatim}
\end{kframe}
\end{knitrout}

As described above colors can be defined in the RGB \emph{color space}, however, other color models such as HSV (hue, saturation, value) can be also used to define colours.

\begin{knitrout}\footnotesize
\definecolor{shadecolor}{rgb}{0.969, 0.969, 0.969}\color{fgcolor}\begin{kframe}
\begin{alltt}
\hlkwd{hsv}\hlstd{(}\hlkwd{c}\hlstd{(}\hlnum{0}\hlstd{,}\hlnum{0.25}\hlstd{,}\hlnum{0.5}\hlstd{,}\hlnum{0.75}\hlstd{,}\hlnum{1}\hlstd{),} \hlnum{0.5}\hlstd{,} \hlnum{0.5}\hlstd{)}
\end{alltt}
\begin{verbatim}
## [1] "#804040" "#608040" "#408080" "#604080" "#804040"
\end{verbatim}
\end{kframe}
\end{knitrout}

The probably a more useful flavour of HSV colors are those returned by function \Rfunction{hcl()} for hue, chroma and luminance. While the ``value'' and ``saturation'' in HSV are based physical values, the ``chroma'' and ``luminance'' values in HCL are based on human visual perception. Colours with equal luminance will be as equally bright by average human being. In a scale based on different hues but equal chroma and luminance values, as used by package \ggplot, all colours are perceived as equally bright. The hues need to be expressed as angles in degrees, with values between zero and 360.

\begin{knitrout}\footnotesize
\definecolor{shadecolor}{rgb}{0.969, 0.969, 0.969}\color{fgcolor}\begin{kframe}
\begin{alltt}
\hlkwd{hcl}\hlstd{(}\hlkwd{c}\hlstd{(}\hlnum{0}\hlstd{,}\hlnum{0.25}\hlstd{,}\hlnum{0.5}\hlstd{,}\hlnum{0.75}\hlstd{,}\hlnum{1}\hlstd{)} \hlopt{*} \hlnum{360}\hlstd{)}
\end{alltt}
\begin{verbatim}
## [1] "#FFC5D0" "#D4D8A7" "#99E2D8" "#D5D0FC" "#FFC5D0"
\end{verbatim}
\end{kframe}
\end{knitrout}

It is also important to remember that humans can only distinguish a limited set of colours, and even smaller colour gamuts can be reproduced by screens and printers. Furthermore, variation from individual to individual exists in color perception, including different types of colour blindness. It is important to take this into account when using colour in illustrations.
\index{color!definitions|)}

\subsection{Continuous colour-related scales}
\index{plots!scales!color|(}
\index{plots!scales!fill|(}
\sloppy
Scales \ggscale{scale\_color\_continuous()}, \ggscale{scale\_color\_gradient()}, \ggscale{scale\_color\_gradient2()},  \ggscale{scale\_color\_gradientn()}, \ggscale{scale\_color\_date()} and \ggscale{scale\_color\_datetime()}, give a smooth continuous gradient between two or more colours. They are useful for numerical, date and datetime data. A corresponding set of \code{fill} scales is also available.

\subsection{Discrete colour-related scales}

\sloppy
Scales \ggscale{scale\_color\_discrete()}, \ggscale{scale\_color\_hue()}, \ggscale{scale\_color\_grey()} are useful for categorical data stored as factors.

\subsection{Identity scales}

In the case of identity scales the mapping is 1 to 1 to the data. For example, if we map the \code{color} or \code{fill} \emph{aesthetic} to a variable using \ggscale{scale\_color\_identity()} or \ggscale{scale\_fill\_identity()} the variable in the data frame passed as argument for \code{data} must already contain valid color definitions. In the case of mapping \code{alpha} the variable must contain numeric values in the rage 0 to 1.

We create a data frame containing a variable \code{colors} containing character strings interpretable as the names of color definitions known to R. We then use them directly in the plot.

\begin{knitrout}\footnotesize
\definecolor{shadecolor}{rgb}{0.969, 0.969, 0.969}\color{fgcolor}\begin{kframe}
\begin{alltt}
\hlstd{df99} \hlkwb{<-} \hlkwd{data.frame}\hlstd{(}\hlkwc{x} \hlstd{=} \hlnum{1}\hlopt{:}\hlnum{10}\hlstd{,} \hlkwc{y} \hlstd{=} \hlkwd{dnorm}\hlstd{(}\hlnum{10}\hlstd{),} \hlkwc{colors} \hlstd{=} \hlkwd{rep}\hlstd{(}\hlkwd{c}\hlstd{(}\hlstr{"red"}\hlstd{,} \hlstr{"blue"}\hlstd{),} \hlnum{5}\hlstd{))}

\hlkwd{ggplot}\hlstd{(df99,} \hlkwd{aes}\hlstd{(x, y,} \hlkwc{color} \hlstd{= colors))} \hlopt{+}
  \hlkwd{geom_point}\hlstd{()} \hlopt{+}
  \hlkwd{scale_color_identity}\hlstd{()}
\end{alltt}
\end{kframe}

{\centering \includegraphics[width=.54\textwidth]{figure/pos-main-chunk-29-1} 

}



\end{knitrout}

\begin{playground}
How does the plot look, if the identity scale is deleted from the example above? Edit and re-run the example code.
\end{playground}

\begin{playground}
While using the identity scale, how would you need to change the code example above, to produce a plot with green and purple point?
\end{playground}
\index{plots!scales!fill|)}
\index{plots!scales!color|)}
\index{color!using|)}

\subsection{Position of axes}
\index{plots!axis position}

\begin{knitrout}\footnotesize
\definecolor{shadecolor}{rgb}{0.969, 0.969, 0.969}\color{fgcolor}\begin{kframe}
\begin{alltt}
\hlkwd{ggplot}\hlstd{(fake2.data,} \hlkwd{aes}\hlstd{(z, y))} \hlopt{+} \hlkwd{geom_point}\hlstd{()} \hlopt{+}
  \hlkwd{scale_x_continuous}\hlstd{(}\hlkwc{position} \hlstd{=} \hlstr{"top"}\hlstd{)} \hlopt{+}
  \hlkwd{scale_y_continuous}\hlstd{(}\hlkwc{position} \hlstd{=} \hlstr{"right"}\hlstd{)}
\end{alltt}
\end{kframe}

{\centering \includegraphics[width=.54\textwidth]{figure/pos-main-chunk-30-1} 

}



\end{knitrout}

\subsection{Secondary axes}
\index{plots!secondary axes}
\begin{knitrout}\footnotesize
\definecolor{shadecolor}{rgb}{0.969, 0.969, 0.969}\color{fgcolor}\begin{kframe}
\begin{alltt}
\hlkwd{ggplot}\hlstd{(fake2.data,} \hlkwd{aes}\hlstd{(z, y))} \hlopt{+} \hlkwd{geom_point}\hlstd{()} \hlopt{+}
  \hlkwd{scale_y_continuous}\hlstd{(}
    \hlstr{"y"}\hlstd{,}
    \hlkwc{sec.axis} \hlstd{=} \hlkwd{sec_axis}\hlstd{(}\hlopt{~} \hlstd{.} \hlopt{^-}\hlnum{1}\hlstd{,} \hlkwc{name} \hlstd{=} \hlstr{"1/y"}\hlstd{)}
  \hlstd{)}
\end{alltt}
\end{kframe}

{\centering \includegraphics[width=.54\textwidth]{figure/pos-main-chunk-31-1} 

}



\end{knitrout}

\begin{knitrout}\footnotesize
\definecolor{shadecolor}{rgb}{0.969, 0.969, 0.969}\color{fgcolor}\begin{kframe}
\begin{alltt}
\hlkwd{ggplot}\hlstd{(fake2.data,} \hlkwd{aes}\hlstd{(z, y))} \hlopt{+} \hlkwd{geom_point}\hlstd{()} \hlopt{+}
  \hlkwd{scale_y_continuous}\hlstd{(}
    \hlstr{"y"}\hlstd{,}
    \hlkwc{sec.axis} \hlstd{=} \hlkwd{sec_axis}\hlstd{(}\hlopt{~} \hlstd{.,} \hlkwc{name} \hlstd{=} \hlstr{"y"}\hlstd{,} \hlkwc{breaks} \hlstd{=} \hlkwd{c}\hlstd{(}\hlnum{33.2}\hlstd{,} \hlnum{55.4}\hlstd{))}
  \hlstd{)}
\end{alltt}
\end{kframe}

{\centering \includegraphics[width=.54\textwidth]{figure/pos-main-chunk-32-1} 

}



\end{knitrout}
\index{plot!scales|)}

\section{Adding annotations}
\index{plots!annotations}
Annotations use the data coordinates of the plot, but do not `inherit' the default data or mapped aesthetics from the enclosing \Rclass{ggplot} object. They are added to a ggplot with \Rfunction{annotate()}. Annotations frequently make use \code{"text"} or \code{"label"} \emph{geometries} with character strings as data, possibly to be parsed as expressions. However, other \emph{geometries} can also be very useful.

\begin{warningbox}
While layers added to a plot directly using \emph{geometries} and \emph{statistics} follow faceting, annotation layers added with \Rfunction{annotate()} are replicated unchanged in every panel of a faceted plot. The reason is that annotation layers take the data as constant values which are directly mapped to the aesthetics and consequently these data are the same in every panel as no grouping is possible.
\end{warningbox}

We show a simple example using \code{"text"}.

\begin{knitrout}\footnotesize
\definecolor{shadecolor}{rgb}{0.969, 0.969, 0.969}\color{fgcolor}\begin{kframe}
\begin{alltt}
\hlkwd{ggplot}\hlstd{(fake2.data,} \hlkwd{aes}\hlstd{(z, y))} \hlopt{+}
  \hlkwd{geom_point}\hlstd{()} \hlopt{+}
  \hlkwd{annotate}\hlstd{(}\hlkwc{geom} \hlstd{=} \hlstr{"text"}\hlstd{,}
           \hlkwc{label} \hlstd{=} \hlstr{"origin"}\hlstd{,}
           \hlkwc{x} \hlstd{=} \hlnum{0}\hlstd{,} \hlkwc{y} \hlstd{=} \hlnum{0}\hlstd{,}
           \hlkwc{color} \hlstd{=} \hlstr{"blue"}\hlstd{,}
           \hlkwc{size}\hlstd{=}\hlnum{4}\hlstd{)}
\end{alltt}
\end{kframe}

{\centering \includegraphics[width=.54\textwidth]{figure/pos-annotate-01-1} 

}



\end{knitrout}

\begin{playground}
Play with the values of the arguments to \Rfunction{annotate()} to vary the position, size, color, font family, font face, rotation angle and justification of the annotation.
\end{playground}

\index{plots!insets as annotations|(}

It is relatively common to use inset tables, plots, bitmaps or vector plots as annotations. As \Rfunction{annotation\_custom()} accepts a Grob (\pkgname{grid} graphical object) as argument, it can be used to add any graphical object (\code{grob}) to a ggplot. To convert a ggplot object into a Grob we use \Rfunction{ggplotGrob()}. In this example the inset is a zoomed in window into the main plot.

\begin{knitrout}\footnotesize
\definecolor{shadecolor}{rgb}{0.969, 0.969, 0.969}\color{fgcolor}\begin{kframe}
\begin{alltt}
\hlstd{p} \hlopt{+}
  \hlkwd{annotation_custom}\hlstd{(}\hlkwd{ggplotGrob}\hlstd{(p} \hlopt{+} \hlkwd{coord_cartesian}\hlstd{(}\hlkwc{xlim} \hlstd{=} \hlkwd{c}\hlstd{(}\hlnum{12.5}\hlstd{,} \hlnum{17.5}\hlstd{),}
                                                   \hlkwc{ylim} \hlstd{=} \hlkwd{c}\hlstd{(}\hlnum{3.17}\hlstd{,} \hlnum{4.17}\hlstd{))),}
                    \hlkwc{xmin} \hlstd{=} \hlnum{25.5}\hlstd{,} \hlkwc{xmax} \hlstd{=} \hlnum{34.5}\hlstd{,} \hlkwc{ymin} \hlstd{=} \hlnum{2.5}\hlstd{,} \hlkwc{ymax} \hlstd{=} \hlnum{5.5}\hlstd{)}
\end{alltt}
\end{kframe}

{\centering \includegraphics[width=.54\textwidth]{figure/pos-inset-01-1} 

}



\end{knitrout}

This approach has the limitation that if used together with faceting, the inset will be the same for each main panel. See section \ref{sec:plot:insets} on page \pageref{sec:plot:insets} for geometries that can be used to add insets.
\index{plots!insets as annotations|)}

In the next example, in addition to adding expressions as annotations, we also pass expressions as tick labels through the scale. Do notice that we use recycling for setting the breaks, as \code{c(0, 0.5, 1, 1.5, 2) * pi} is equivalent to \code{c(0, 0.5 * pi, pi, 1.5 * pi, 2 * pi}. Annotations are plotted at their own position, unrelated to any observation in the data, but using the same coordinates and units as for plotting the data.

\begin{knitrout}\footnotesize
\definecolor{shadecolor}{rgb}{0.969, 0.969, 0.969}\color{fgcolor}\begin{kframe}
\begin{alltt}
\hlkwd{ggplot}\hlstd{(}\hlkwd{data.frame}\hlstd{(}\hlkwc{x} \hlstd{=} \hlkwd{c}\hlstd{(}\hlnum{0}\hlstd{,} \hlnum{2} \hlopt{*} \hlstd{pi)),} \hlkwd{aes}\hlstd{(}\hlkwc{x} \hlstd{= x))} \hlopt{+}
  \hlkwd{stat_function}\hlstd{(}\hlkwc{fun} \hlstd{= sin)} \hlopt{+}
  \hlkwd{scale_x_continuous}\hlstd{(}
    \hlkwc{breaks} \hlstd{=} \hlkwd{c}\hlstd{(}\hlnum{0}\hlstd{,} \hlnum{0.5}\hlstd{,} \hlnum{1}\hlstd{,} \hlnum{1.5}\hlstd{,} \hlnum{2}\hlstd{)} \hlopt{*} \hlstd{pi,}
    \hlkwc{labels} \hlstd{=} \hlkwd{c}\hlstd{(}\hlstr{"0"}\hlstd{,} \hlkwd{expression}\hlstd{(}\hlnum{0.5}\hlopt{~}\hlstd{pi),} \hlkwd{expression}\hlstd{(pi),}
             \hlkwd{expression}\hlstd{(}\hlnum{1.5}\hlopt{~}\hlstd{pi),} \hlkwd{expression}\hlstd{(}\hlnum{2}\hlopt{~}\hlstd{pi)))} \hlopt{+}
  \hlkwd{labs}\hlstd{(}\hlkwc{y} \hlstd{=} \hlstr{"sin(x)"}\hlstd{)} \hlopt{+}
  \hlkwd{annotate}\hlstd{(}\hlkwc{geom} \hlstd{=} \hlstr{"text"}\hlstd{,}
           \hlkwc{label} \hlstd{=} \hlkwd{c}\hlstd{(}\hlstr{"+"}\hlstd{,} \hlstr{"-"}\hlstd{),}
           \hlkwc{x} \hlstd{=} \hlkwd{c}\hlstd{(}\hlnum{0.5}\hlstd{,} \hlnum{1.5}\hlstd{)} \hlopt{*} \hlstd{pi,} \hlkwc{y} \hlstd{=} \hlkwd{c}\hlstd{(}\hlnum{0.5}\hlstd{,} \hlopt{-}\hlnum{0.5}\hlstd{),}
           \hlkwc{size} \hlstd{=} \hlnum{20}\hlstd{)} \hlopt{+}
  \hlkwd{annotate}\hlstd{(}\hlkwc{geom} \hlstd{=} \hlstr{"point"}\hlstd{,}
           \hlkwc{colour} \hlstd{=} \hlstr{"red"}\hlstd{,}
           \hlkwc{shape} \hlstd{=} \hlnum{21}\hlstd{,}
           \hlkwc{fill} \hlstd{=} \hlstr{"white"}\hlstd{,}
           \hlkwc{x} \hlstd{=} \hlkwd{c}\hlstd{(}\hlnum{0}\hlstd{,} \hlnum{1}\hlstd{,} \hlnum{2}\hlstd{)} \hlopt{*} \hlstd{pi,} \hlkwc{y} \hlstd{=} \hlnum{0}\hlstd{,}
           \hlkwc{size} \hlstd{=} \hlnum{6}\hlstd{)}
\end{alltt}
\end{kframe}

{\centering \includegraphics[width=.54\textwidth]{figure/pos-annotate-03-1} 

}



\end{knitrout}

\begin{playground}
Modify the plot above to show the cosine instead of the sine function, replacing \code{sin} with \code{cos}. This is easy, but the catch is that you will need to relocate the annotations.
\end{playground}

\begin{infobox}
We cannot use \Rfunction{annotate()} with \code{geom = "vline"} or \code{geom = "hline"} as we can use \code{geom = "line"} or \code{geom = "segment"}. Instead \gggeom{geom\_vline()} and/or  \gggeom{geom\_hline()} can be used directly passing constant arguments to them. See section \ref{sec:plot:vhline} on page \pageref{sec:plot:vhline}.
\end{infobox}

\section{Coordinates and circular plots}\label{sec:plot:circular}
\index{plots!circular|(}

In this section I include pie charts and wind-rose plots. Here we add a new "word" to the grammar of graphics, \textit{coordinates}, such as \ggcoordinate{coord\_polar()} in the next examples\index{coordinates!polar}\index{plots!coordinates!polar}. The default coordinate system for $x$ and $y$ \textit{aesthetics} is cartesian.

\subsection{Pie charts}
\index{plots!pie charts|(}

Pie charts are more difficult to read: our brain is more comfortable at comparing
lengths than angles. If used, they should only be used to show composition, or
fractional components that add up to a total. In this case only if the number of
“pie slices” is small (rule of thumb: less than seven).

We make the equivalent of the first bar plot above. As we are still using \gggeom{geom\_bar()} the default is \code{stat\_count}. As earlier we use the brewer scale for nicer colors.

\begin{knitrout}\footnotesize
\definecolor{shadecolor}{rgb}{0.969, 0.969, 0.969}\color{fgcolor}\begin{kframe}
\begin{alltt}
\hlkwd{ggplot}\hlstd{(}\hlkwc{data} \hlstd{= mpg,} \hlkwd{aes}\hlstd{(}\hlkwc{x} \hlstd{=} \hlkwd{factor}\hlstd{(}\hlnum{1}\hlstd{),} \hlkwc{fill} \hlstd{=} \hlkwd{factor}\hlstd{(class)))} \hlopt{+}
  \hlkwd{geom_bar}\hlstd{(}\hlkwc{width} \hlstd{=} \hlnum{1}\hlstd{,} \hlkwc{color} \hlstd{=} \hlstr{"black"}\hlstd{)} \hlopt{+}
  \hlkwd{coord_polar}\hlstd{(}\hlkwc{theta} \hlstd{=} \hlstr{"y"}\hlstd{)} \hlopt{+}
  \hlkwd{scale_fill_brewer}\hlstd{()} \hlopt{+}
  \hlkwd{scale_x_discrete}\hlstd{(}\hlkwc{breaks} \hlstd{=} \hlkwa{NULL}\hlstd{)} \hlopt{+}
  \hlkwd{labs}\hlstd{(}\hlkwc{x} \hlstd{=} \hlkwa{NULL}\hlstd{,} \hlkwc{fill} \hlstd{=} \hlstr{"Vehicle class"}\hlstd{)}
\end{alltt}
\end{kframe}

{\centering \includegraphics[width=.54\textwidth]{figure/pos-main-chunk-33-1} 

}



\end{knitrout}

Even with four slices pie charts can be difficult to read. Compare the following bar plot and pie chart.

\begin{knitrout}\footnotesize
\definecolor{shadecolor}{rgb}{0.969, 0.969, 0.969}\color{fgcolor}\begin{kframe}
\begin{alltt}
\hlkwd{ggplot}\hlstd{(}\hlkwc{data} \hlstd{= mpg,} \hlkwd{aes}\hlstd{(}\hlkwc{x} \hlstd{=} \hlkwd{factor}\hlstd{(cyl),} \hlkwc{fill} \hlstd{=} \hlkwd{factor}\hlstd{(cyl)))} \hlopt{+}
  \hlkwd{geom_bar}\hlstd{(}\hlkwc{color} \hlstd{=} \hlstr{"black"}\hlstd{)} \hlopt{+}
  \hlkwd{scale_fill_grey}\hlstd{()} \hlopt{+}
  \hlkwd{scale_x_discrete}\hlstd{(}\hlkwc{breaks} \hlstd{=} \hlkwa{NULL}\hlstd{)} \hlopt{+}
  \hlkwd{labs}\hlstd{(}\hlkwc{x} \hlstd{=} \hlkwa{NULL}\hlstd{,} \hlkwc{fill} \hlstd{=} \hlstr{"Vehicle class"}\hlstd{)} \hlopt{+}
  \hlkwd{theme_bw}\hlstd{()}

\hlkwd{ggplot}\hlstd{(}\hlkwc{data} \hlstd{= mpg,} \hlkwd{aes}\hlstd{(}\hlkwc{x} \hlstd{=} \hlkwd{factor}\hlstd{(}\hlnum{1}\hlstd{),} \hlkwc{fill} \hlstd{=} \hlkwd{factor}\hlstd{(cyl)))} \hlopt{+}
  \hlkwd{geom_bar}\hlstd{(}\hlkwc{width} \hlstd{=} \hlnum{1}\hlstd{,} \hlkwc{color} \hlstd{=} \hlstr{"black"}\hlstd{)} \hlopt{+}
  \hlkwd{coord_polar}\hlstd{(}\hlkwc{theta} \hlstd{=} \hlstr{"y"}\hlstd{)} \hlopt{+}
  \hlkwd{scale_fill_grey}\hlstd{()} \hlopt{+}
  \hlkwd{scale_x_discrete}\hlstd{(}\hlkwc{breaks} \hlstd{=} \hlkwa{NULL}\hlstd{)} \hlopt{+}
  \hlkwd{labs}\hlstd{(}\hlkwc{x} \hlstd{=} \hlkwa{NULL}\hlstd{,} \hlkwc{fill} \hlstd{=} \hlstr{"Vehicle class"}\hlstd{)} \hlopt{+}
  \hlkwd{theme_bw}\hlstd{()}
\end{alltt}
\end{kframe}

{\centering \includegraphics[width=.54\textwidth]{figure/pos-main-chunk-34-1} 
\includegraphics[width=.54\textwidth]{figure/pos-main-chunk-34-2} 

}



\end{knitrout}

An example comparing pie charts to bar plots is presented in section \ref{sec:plot:pie:bar} on page \pageref{sec:plot:pie:bar}.
\index{plots!pie charts|)}

\subsection{Wind-rose plots}
\index{plots!wind rose|(}
They can be plotted as histograms on polar coordinates, when the data is to be represented by frequencies or, as density plot. A bar plot or a line or points when the values are means calculated with a \emph{statistic} or a single observation is available per quadrat. It also possible to use summaries, or smoothers.

Some types of data are more naturally expressed on polar coordinates than on cartesian coordinates. The clearest example is wind direction, from which the name derives. In some cases of time series data with a strong periodic variation, polar coordinates can be used to highlight any phase shifts or changes in frequency. A more mundane application is to plot variation in a response variable through the day with a clock-face like representation of time-of-day.

We use for the next examples wind speed and direction data, measured once per minute during 24~h, from \pkgname{learnrbook}.
We first show a time series plot, using cartesian coordinates, which demonstrates the problem of using an arbitrary origin at the North for a variable that does not have a scale with true limits: early in the day the predominant direction is just slightly West of 0 degrees North and the cloud of observations gets artificially split. We can also observe a clear change in wind direction soon after solar noon.

\begin{knitrout}\footnotesize
\definecolor{shadecolor}{rgb}{0.969, 0.969, 0.969}\color{fgcolor}\begin{kframe}
\begin{alltt}
\hlkwd{ggplot}\hlstd{(viikki_d29.dat,} \hlkwd{aes}\hlstd{(solar_time, WindDir_D1_WVT))}  \hlopt{+}
\hlkwd{geom_point}\hlstd{()} \hlopt{+}
\hlkwd{scale_x_datetime}\hlstd{(}\hlkwc{date_labels} \hlstd{=} \hlstr{"%H:%M"}\hlstd{)} \hlopt{+}
\hlkwd{labs}\hlstd{(}\hlkwc{x} \hlstd{=} \hlstr{"Time of day (hh:mm)"}\hlstd{,} \hlkwc{y} \hlstd{=} \hlstr{"Wind direction (degrees)"}\hlstd{)}
\end{alltt}
\end{kframe}

{\centering \includegraphics[width=.54\textwidth]{figure/pos-wind-02-1} 

}



\end{knitrout}

No such problem exists with wind speed, and we add a smooth line with \gggeom{geom\_smooth()}.

\begin{knitrout}\footnotesize
\definecolor{shadecolor}{rgb}{0.969, 0.969, 0.969}\color{fgcolor}\begin{kframe}
\begin{alltt}
\hlkwd{ggplot}\hlstd{(viikki_d29.dat,} \hlkwd{aes}\hlstd{(solar_time, WindSpd_S_WVT))}  \hlopt{+}
\hlkwd{geom_point}\hlstd{()} \hlopt{+}
\hlcom{# geom_smooth() +}
\hlkwd{scale_x_datetime}\hlstd{(}\hlkwc{date_labels} \hlstd{=} \hlstr{"%H:%M"}\hlstd{)} \hlopt{+}
\hlkwd{labs}\hlstd{(}\hlkwc{x} \hlstd{=} \hlstr{"Time of day (hh:mm)"}\hlstd{,} \hlkwc{y} \hlstd{=} \hlstr{"Wind speed (m/s)"}\hlstd{)}
\end{alltt}
\end{kframe}

{\centering \includegraphics[width=.54\textwidth]{figure/pos-wind-03-1} 

}



\end{knitrout}

Using a scatter plot with polar coordinates helps to some extent, but having time of day on the radial axis is rather unclear.

\begin{knitrout}\footnotesize
\definecolor{shadecolor}{rgb}{0.969, 0.969, 0.969}\color{fgcolor}\begin{kframe}
\begin{alltt}
\hlkwd{ggplot}\hlstd{(viikki_d29.dat,} \hlkwd{aes}\hlstd{(WindDir_D1_WVT, solar_time))}  \hlopt{+}
  \hlkwd{coord_polar}\hlstd{()} \hlopt{+}
  \hlkwd{geom_point}\hlstd{()} \hlopt{+}
  \hlkwd{scale_x_continuous}\hlstd{(}\hlkwc{breaks} \hlstd{=} \hlkwd{c}\hlstd{(}\hlnum{0}\hlstd{,} \hlnum{90}\hlstd{,} \hlnum{180}\hlstd{,} \hlnum{270}\hlstd{),}
                     \hlkwc{labels} \hlstd{=} \hlkwd{c}\hlstd{(}\hlstr{"N"}\hlstd{,} \hlstr{"E"}\hlstd{,} \hlstr{"S"}\hlstd{,} \hlstr{"W"}\hlstd{),}
                     \hlkwc{limits} \hlstd{=} \hlkwd{c}\hlstd{(}\hlnum{0}\hlstd{,} \hlnum{360}\hlstd{),}
                     \hlkwc{expand} \hlstd{=} \hlkwd{c}\hlstd{(}\hlnum{0}\hlstd{,} \hlnum{0}\hlstd{),}
                     \hlkwc{name} \hlstd{=} \hlstr{"Wind direction"}\hlstd{)} \hlopt{+}
  \hlkwd{scale_y_datetime}\hlstd{(}\hlkwc{date_labels} \hlstd{=} \hlstr{"%H:%M"}\hlstd{,}
                   \hlkwc{name} \hlstd{=} \hlstr{"Time of day (hh:mm)"}\hlstd{,}
                   \hlkwc{date_breaks} \hlstd{=} \hlstr{"6 hours"}\hlstd{,}
                   \hlkwc{date_minor_breaks} \hlstd{=} \hlstr{"3 hours"}\hlstd{,)}
\end{alltt}
\end{kframe}

{\centering \includegraphics[width=.54\textwidth]{figure/pos-wind-04-1} 

}



\end{knitrout}

Most frequently, wind-rose plots use summaries, such as histograms or densities.
Next we plot a circular histogram of wind directions with 15 degrees-wide bins. We use \ggstat{stat\_bin()}.

\begin{knitrout}\footnotesize
\definecolor{shadecolor}{rgb}{0.969, 0.969, 0.969}\color{fgcolor}\begin{kframe}
\begin{alltt}
\hlkwd{ggplot}\hlstd{(viikki_d29.dat,} \hlkwd{aes}\hlstd{(WindDir_D1_WVT))}  \hlopt{+}
  \hlkwd{coord_polar}\hlstd{()} \hlopt{+}
  \hlkwd{stat_bin}\hlstd{(}\hlkwc{color} \hlstd{=} \hlstr{"black"}\hlstd{,} \hlkwc{fill} \hlstd{=} \hlstr{"grey50"}\hlstd{,} \hlkwc{binwidth} \hlstd{=} \hlnum{15}\hlstd{,} \hlkwc{geom} \hlstd{=} \hlstr{"bar"}\hlstd{)} \hlopt{+}
  \hlkwd{scale_x_continuous}\hlstd{(}\hlkwc{breaks} \hlstd{=} \hlkwd{c}\hlstd{(}\hlnum{0}\hlstd{,} \hlnum{90}\hlstd{,} \hlnum{180}\hlstd{,} \hlnum{270}\hlstd{),}
                     \hlkwc{labels} \hlstd{=} \hlkwd{c}\hlstd{(}\hlstr{"N"}\hlstd{,} \hlstr{"E"}\hlstd{,} \hlstr{"S"}\hlstd{,} \hlstr{"W"}\hlstd{),}
                     \hlkwc{limits} \hlstd{=} \hlkwd{c}\hlstd{(}\hlnum{0}\hlstd{,} \hlnum{360}\hlstd{),}
                     \hlkwc{expand} \hlstd{=} \hlkwd{c}\hlstd{(}\hlnum{0}\hlstd{,} \hlnum{0}\hlstd{),}
                     \hlkwc{name} \hlstd{=} \hlstr{"Wind direction"}\hlstd{)} \hlopt{+}
  \hlkwd{scale_y_continuous}\hlstd{(}\hlkwc{name} \hlstd{=} \hlstr{"Frequency"}\hlstd{)}
\end{alltt}


{\ttfamily\noindent\color{warningcolor}{\#\# Warning: Removed 2 rows containing missing values (geom\_bar).}}\end{kframe}

{\centering \includegraphics[width=.54\textwidth]{figure/pos-wind-05-1} 

}



\end{knitrout}

An equivalent plot, using an empirical density, created with \ggstat{stat\_density()}.

\begin{knitrout}\footnotesize
\definecolor{shadecolor}{rgb}{0.969, 0.969, 0.969}\color{fgcolor}\begin{kframe}
\begin{alltt}
\hlkwd{ggplot}\hlstd{(viikki_d29.dat,} \hlkwd{aes}\hlstd{(WindDir_D1_WVT))}  \hlopt{+}
  \hlkwd{coord_polar}\hlstd{()} \hlopt{+}
  \hlkwd{stat_density}\hlstd{(}\hlkwc{color} \hlstd{=} \hlstr{"black"}\hlstd{,} \hlkwc{fill} \hlstd{=} \hlstr{"grey50"}\hlstd{,} \hlkwc{size} \hlstd{=} \hlnum{1}\hlstd{,} \hlkwc{na.rm} \hlstd{=} \hlnum{TRUE}\hlstd{)} \hlopt{+}
  \hlkwd{scale_x_continuous}\hlstd{(}\hlkwc{breaks} \hlstd{=} \hlkwd{c}\hlstd{(}\hlnum{0}\hlstd{,} \hlnum{90}\hlstd{,} \hlnum{180}\hlstd{,} \hlnum{270}\hlstd{),}
                     \hlkwc{labels} \hlstd{=} \hlkwd{c}\hlstd{(}\hlstr{"N"}\hlstd{,} \hlstr{"E"}\hlstd{,} \hlstr{"S"}\hlstd{,} \hlstr{"W"}\hlstd{),}
                     \hlkwc{limits} \hlstd{=} \hlkwd{c}\hlstd{(}\hlnum{0}\hlstd{,} \hlnum{360}\hlstd{),}
                     \hlkwc{expand} \hlstd{=} \hlkwd{c}\hlstd{(}\hlnum{0}\hlstd{,} \hlnum{0}\hlstd{),}
                     \hlkwc{name} \hlstd{=} \hlstr{"Wind direction"}\hlstd{)} \hlopt{+}
  \hlkwd{scale_y_continuous}\hlstd{(}\hlkwc{name} \hlstd{=} \hlstr{"Density"}\hlstd{)}
\end{alltt}
\end{kframe}

{\centering \includegraphics[width=.54\textwidth]{figure/pos-wind-06-1} 

}



\end{knitrout}

As final wind-rose plot examples we do a scatter plot of wind speeds versus wind direction and a two dimensional density plot. In both cases we use \Rfunction{facet\_wrap()} to have separate panel for AM and PM. In the scatter plot we set \code{alpha = 0.1} for better visualization of overlapping points.

\begin{knitrout}\footnotesize
\definecolor{shadecolor}{rgb}{0.969, 0.969, 0.969}\color{fgcolor}\begin{kframe}
\begin{alltt}
\hlkwd{ggplot}\hlstd{(viikki_d29.dat,} \hlkwd{aes}\hlstd{(WindDir_D1_WVT, WindSpd_S_WVT))} \hlopt{+}
  \hlkwd{coord_polar}\hlstd{()} \hlopt{+}
  \hlkwd{geom_point}\hlstd{(}\hlkwc{alpha} \hlstd{=} \hlnum{0.1}\hlstd{,} \hlkwc{shape} \hlstd{=} \hlnum{16}\hlstd{)} \hlopt{+}
  \hlkwd{scale_x_continuous}\hlstd{(}\hlkwc{breaks} \hlstd{=} \hlkwd{c}\hlstd{(}\hlnum{0}\hlstd{,} \hlnum{90}\hlstd{,} \hlnum{180}\hlstd{,} \hlnum{270}\hlstd{),}
                     \hlkwc{labels} \hlstd{=} \hlkwd{c}\hlstd{(}\hlstr{"N"}\hlstd{,} \hlstr{"E"}\hlstd{,} \hlstr{"S"}\hlstd{,} \hlstr{"W"}\hlstd{),}
                     \hlkwc{limits} \hlstd{=} \hlkwd{c}\hlstd{(}\hlnum{0}\hlstd{,} \hlnum{360}\hlstd{),}
                     \hlkwc{expand} \hlstd{=} \hlkwd{c}\hlstd{(}\hlnum{0}\hlstd{,} \hlnum{0}\hlstd{),}
                     \hlkwc{name} \hlstd{=} \hlstr{"Wind direction"}\hlstd{)} \hlopt{+}
  \hlkwd{scale_y_continuous}\hlstd{(}\hlkwc{name} \hlstd{=} \hlstr{"Wind speed (m/s)"}\hlstd{)} \hlopt{+}
  \hlkwd{facet_wrap}\hlstd{(}\hlopt{~}\hlkwd{factor}\hlstd{(}\hlkwd{ifelse}\hlstd{(}\hlkwd{hour}\hlstd{(solar_time)} \hlopt{<} \hlnum{12}\hlstd{,} \hlstr{"AM"}\hlstd{,} \hlstr{"PM"}\hlstd{)))}
\end{alltt}
\end{kframe}

{\centering \includegraphics[width=.54\textwidth]{figure/pos-wind-07-1} 

}



\end{knitrout}

\begin{knitrout}\footnotesize
\definecolor{shadecolor}{rgb}{0.969, 0.969, 0.969}\color{fgcolor}\begin{kframe}
\begin{alltt}
\hlkwd{ggplot}\hlstd{(viikki_d29.dat,} \hlkwd{aes}\hlstd{(WindDir_D1_WVT, WindSpd_S_WVT))} \hlopt{+}
  \hlkwd{coord_polar}\hlstd{()} \hlopt{+}
  \hlkwd{stat_density_2d}\hlstd{()} \hlopt{+}
  \hlkwd{scale_x_continuous}\hlstd{(}\hlkwc{breaks} \hlstd{=} \hlkwd{c}\hlstd{(}\hlnum{0}\hlstd{,} \hlnum{90}\hlstd{,} \hlnum{180}\hlstd{,} \hlnum{270}\hlstd{),}
                     \hlkwc{labels} \hlstd{=} \hlkwd{c}\hlstd{(}\hlstr{"N"}\hlstd{,} \hlstr{"E"}\hlstd{,} \hlstr{"S"}\hlstd{,} \hlstr{"W"}\hlstd{),}
                     \hlkwc{limits} \hlstd{=} \hlkwd{c}\hlstd{(}\hlnum{0}\hlstd{,} \hlnum{360}\hlstd{),}
                     \hlkwc{expand} \hlstd{=} \hlkwd{c}\hlstd{(}\hlnum{0}\hlstd{,} \hlnum{0}\hlstd{),}
                     \hlkwc{name} \hlstd{=} \hlstr{"Wind direction"}\hlstd{)} \hlopt{+}
  \hlkwd{scale_y_continuous}\hlstd{(}\hlkwc{name} \hlstd{=} \hlstr{"Wind speed (m/s)"}\hlstd{)} \hlopt{+}
  \hlkwd{facet_wrap}\hlstd{(}\hlopt{~}\hlkwd{factor}\hlstd{(}\hlkwd{ifelse}\hlstd{(}\hlkwd{hour}\hlstd{(solar_time)} \hlopt{<} \hlnum{12}\hlstd{,} \hlstr{"AM"}\hlstd{,} \hlstr{"PM"}\hlstd{)))}
\end{alltt}
\end{kframe}

{\centering \includegraphics[width=.54\textwidth]{figure/pos-wind-08-1} 

}



\end{knitrout}

\index{plots!wind rose|)}
\index{plots!circular|)}


\section{Themes}\label{sec:plot:themes}
\index{plots!themes|(}
For ggplot2 themes are the equivalent of style sheets for text. They determine how the different elements of a plot are rendered when displayed, printed or saved to a file. They do not alter how the data themselves are displayed, but instead that of text-labels, titles, axes, grids, etc. are formatted. Package \ggplot includes several predefined themes, and some extension packages described in the next chapter define additional ones. In addition to switching between themes, the user can modify the format applied to individual elements, or define totally new themes.

\subsection{Predefined themes}
\index{plots!themes!predefined|(}
The theme used by default is \ggtheme{theme\_grey()}. Themes are defined as functions, with parameters. These parameters allow changing some ``base'' properties. The base size for text elements is given in points, and affects all text elements in a plot (except those produced by \emph{geometries}) as the size of them is by default defined relative to the base size. Another parameter, \code{base\_family}, allows the font family to be set.

Even the default \ggtheme{theme\_grey()} can come in handy because the first parameter to themes is the base point size.


\begin{knitrout}\footnotesize
\definecolor{shadecolor}{rgb}{0.969, 0.969, 0.969}\color{fgcolor}\begin{kframe}
\begin{alltt}
\hlkwd{ggplot}\hlstd{(fake2.data,} \hlkwd{aes}\hlstd{(z, y))} \hlopt{+}
  \hlkwd{geom_point}\hlstd{()} \hlopt{+}
  \hlkwd{theme_grey}\hlstd{(}\hlnum{10}\hlstd{)}
\end{alltt}
\end{kframe}

{\centering \includegraphics[width=.54\textwidth]{figure/pos-themes-01-1} 

}



\end{knitrout}

\begin{knitrout}\footnotesize
\definecolor{shadecolor}{rgb}{0.969, 0.969, 0.969}\color{fgcolor}\begin{kframe}
\begin{alltt}
\hlkwd{ggplot}\hlstd{(fake2.data,} \hlkwd{aes}\hlstd{(z, y))} \hlopt{+}
  \hlkwd{geom_point}\hlstd{()} \hlopt{+}
  \hlkwd{theme_grey}\hlstd{(}\hlnum{15}\hlstd{)}
\end{alltt}
\end{kframe}

{\centering \includegraphics[width=.54\textwidth]{figure/pos-themes-01a-1} 

}



\end{knitrout}

We can also set the base family.

\begin{knitrout}\footnotesize
\definecolor{shadecolor}{rgb}{0.969, 0.969, 0.969}\color{fgcolor}\begin{kframe}
\begin{alltt}
\hlkwd{ggplot}\hlstd{(fake2.data,} \hlkwd{aes}\hlstd{(z, y))} \hlopt{+}
  \hlkwd{geom_point}\hlstd{()} \hlopt{+}
  \hlkwd{theme_grey}\hlstd{(}\hlkwc{base_size} \hlstd{=} \hlnum{15}\hlstd{,}
             \hlkwc{base_family} \hlstd{=} \hlstr{"serif"}\hlstd{)}
\end{alltt}
\end{kframe}

{\centering \includegraphics[width=.54\textwidth]{figure/pos-themes-01b-1} 

}



\end{knitrout}

\begin{playground}
Change the code in the previous chunk to use the \code{"mono"} font family at size 8.
\end{playground}

\begin{playground}
Change the code in the previous chunk to use all the other predefined themes: \ggtheme{theme\_bw()}, \ggtheme{theme\_classic()}, \ggtheme{theme\_minimal()}, \ggtheme{theme\_linedraw()}, \ggtheme{theme\_light()}, \ggtheme{theme\_dark()} and \ggtheme{theme\_void()}.
\end{playground}

A frequent idiom is to create a ggplot without specifying a theme, and then adding the theme when printed.

\begin{knitrout}\footnotesize
\definecolor{shadecolor}{rgb}{0.969, 0.969, 0.969}\color{fgcolor}\begin{kframe}
\begin{alltt}
\hlstd{p} \hlkwb{<-} \hlkwd{ggplot}\hlstd{(fake2.data,} \hlkwd{aes}\hlstd{(z, y))} \hlopt{+}
       \hlkwd{geom_point}\hlstd{()}
\hlstd{p} \hlopt{+} \hlkwd{theme_bw}\hlstd{()}
\end{alltt}
\end{kframe}

{\centering \includegraphics[width=.54\textwidth]{figure/pos-themes-03-1} 

}



\end{knitrout}

\begin{playground}
Play by replacing in the last statement in the previous code chunk the theme used to print the saved ggplot object \code{p}. Do also try the effect of changing the base size and font family.
\end{playground}

It is also possible to set the default theme to be used by all subsequent plots rendered.

\begin{knitrout}\footnotesize
\definecolor{shadecolor}{rgb}{0.969, 0.969, 0.969}\color{fgcolor}\begin{kframe}
\begin{alltt}
\hlstd{p}
\end{alltt}
\end{kframe}

{\centering \includegraphics[width=.54\textwidth]{figure/pos-themes-04-1} 

}



\end{knitrout}

We save the current default theme, so as to be able to restore it. If there is no need to `go back' then saving can be skipped by not including the left hand side and the assignment operator in the first statement below.

\begin{knitrout}\footnotesize
\definecolor{shadecolor}{rgb}{0.969, 0.969, 0.969}\color{fgcolor}\begin{kframe}
\begin{alltt}
\hlstd{old_theme} \hlkwb{<-} \hlkwd{theme_set}\hlstd{(}\hlkwd{theme_bw}\hlstd{(}\hlnum{15}\hlstd{))}
\hlstd{p}
\end{alltt}
\end{kframe}

{\centering \includegraphics[width=.54\textwidth]{figure/pos-themes-05-1} 

}



\end{knitrout}

\begin{knitrout}\footnotesize
\definecolor{shadecolor}{rgb}{0.969, 0.969, 0.969}\color{fgcolor}\begin{kframe}
\begin{alltt}
\hlkwd{theme_set}\hlstd{(old_theme)}
\hlstd{p}
\end{alltt}
\end{kframe}

{\centering \includegraphics[width=.54\textwidth]{figure/pos-themes-06-1} 

}



\end{knitrout}
\index{plots!themes!predefined|)}

\subsection{Modifying a theme}
\index{plots!themes!modifying|(}
Sometimes we would just like to slightly tweak one of the predefined themes. This is also possible. We exemplify this by solving the frequent problem of overlapping $x$-axis tick labels with different approaches. We force this by setting the number ticks to a high value. Usually rotating the text of the labels solves the problem.

\begin{knitrout}\footnotesize
\definecolor{shadecolor}{rgb}{0.969, 0.969, 0.969}\color{fgcolor}\begin{kframe}
\begin{alltt}
\hlkwd{ggplot}\hlstd{(fake2.data,} \hlkwd{aes}\hlstd{(z} \hlopt{+} \hlnum{100}\hlstd{, y))} \hlopt{+}
  \hlkwd{geom_point}\hlstd{()} \hlopt{+}
  \hlkwd{scale_x_continuous}\hlstd{(}\hlkwc{breaks} \hlstd{= scales}\hlopt{::}\hlkwd{pretty_breaks}\hlstd{(}\hlkwc{n} \hlstd{=} \hlnum{20}\hlstd{))} \hlopt{+}
  \hlkwd{theme}\hlstd{(}\hlkwc{axis.text.x} \hlstd{=} \hlkwd{element_text}\hlstd{(}\hlkwc{angle} \hlstd{=} \hlnum{90}\hlstd{,} \hlkwc{hjust} \hlstd{=} \hlnum{1}\hlstd{,} \hlkwc{vjust} \hlstd{=} \hlnum{0.5}\hlstd{))}
\end{alltt}
\end{kframe}

{\centering \includegraphics[width=.54\textwidth]{figure/pos-themes-11-1} 

}



\end{knitrout}

\begin{playground}
Play with the code above, modifying the values used for \code{angle}, \code{hjust} and \code{vjust}. (Angles are expressed in degrees, and justification with values between 0 and 1.
\end{playground}

\begin{warningbox}
When tick labels are rotated one usually needs to set both the horizontal and vertical justification as the default values are no longer suitable. This is due to the fact that justification settings are referenced to the text itself rather than to the plot, i.e.\ vertical justification of $x$-axis tick labels rotated 90 degrees sets their horizontal position with respect to the plot.
\end{warningbox}

Another possibility is to use a smaller font size. Within theme function \Rfunction{rel()} can be used to set size relative to the base size.

\begin{knitrout}\footnotesize
\definecolor{shadecolor}{rgb}{0.969, 0.969, 0.969}\color{fgcolor}\begin{kframe}
\begin{alltt}
\hlkwd{ggplot}\hlstd{(fake2.data,} \hlkwd{aes}\hlstd{(z} \hlopt{+} \hlnum{100}\hlstd{, y))} \hlopt{+}
  \hlkwd{geom_point}\hlstd{()} \hlopt{+}
  \hlkwd{scale_x_continuous}\hlstd{(}\hlkwc{breaks} \hlstd{= scales}\hlopt{::}\hlkwd{pretty_breaks}\hlstd{(}\hlkwc{n} \hlstd{=} \hlnum{20}\hlstd{))} \hlopt{+}
  \hlkwd{theme}\hlstd{(}\hlkwc{axis.text} \hlstd{=} \hlkwd{element_text}\hlstd{(}\hlkwc{color} \hlstd{=} \hlstr{"darkblue"}\hlstd{),}
        \hlkwc{axis.text.x} \hlstd{=} \hlkwd{element_text}\hlstd{(}\hlkwc{size} \hlstd{=} \hlkwd{rel}\hlstd{(}\hlnum{0.6}\hlstd{)))}
\end{alltt}
\end{kframe}

{\centering \includegraphics[width=.54\textwidth]{figure/pos-themes-12-1} 

}



\end{knitrout}

Themes definitions follow a hierarchy, allowing us to modify the formatting of groups of similar elements, as well as of individual elements. In the chunk above we modify the color of the tick labels in both axes, but changed the font size only for the $x$-axis.

\begin{playground}
Modify the example above, so that the tick labels on the $x$-axis are blue and those on the $y$-axis red, and the font size the same for both axes, but changed from the default.
\end{playground}

Formatting of all other text elements can be adjusted in a similar way.

The color of the background, and the properties of the grid lines and other lines can be adjusted thought theme elements. We next change the properties of the lines used for the axes, removing the lines on the top and right margins, and adding arrow heads to the axis lines. See chapter 9 in \citebooktitle{Wickham2016} \autocite{Wickham2016} for additional examples and \citebooktitle{Chang2013} \autocite{Chang2013} for more details.

\begin{warningbox}
If you use a saved theme, and want to modify some elements, then the saved theme should be added to the plot before adding \code{+ theme(...)} as otherwise the changes would be overwritten.
\end{warningbox}

It is also possible to modify the default theme used for rendering all subsequent plots.

\begin{knitrout}\footnotesize
\definecolor{shadecolor}{rgb}{0.969, 0.969, 0.969}\color{fgcolor}\begin{kframe}
\begin{alltt}
\hlstd{p}
\end{alltt}
\end{kframe}

{\centering \includegraphics[width=.54\textwidth]{figure/pos-themes-14-1} 

}



\end{knitrout}

As above, we save the current default theme, so as to be able to restore it.

\begin{knitrout}\footnotesize
\definecolor{shadecolor}{rgb}{0.969, 0.969, 0.969}\color{fgcolor}\begin{kframe}
\begin{alltt}
\hlstd{old_theme} \hlkwb{<-} \hlkwd{theme_update}\hlstd{(}\hlkwc{text} \hlstd{=} \hlkwd{element_text}\hlstd{(}\hlkwc{color} \hlstd{=} \hlstr{"red"}\hlstd{))}
\hlstd{p}
\end{alltt}
\end{kframe}

{\centering \includegraphics[width=.54\textwidth]{figure/pos-themes-15-1} 

}



\end{knitrout}

\begin{knitrout}\footnotesize
\definecolor{shadecolor}{rgb}{0.969, 0.969, 0.969}\color{fgcolor}\begin{kframe}
\begin{alltt}
\hlkwd{theme_set}\hlstd{(old_theme)}
\hlstd{p}
\end{alltt}
\end{kframe}

{\centering \includegraphics[width=.54\textwidth]{figure/pos-themes-16-1} 

}



\end{knitrout}
\index{plots!themes!modifying|)}

\subsection{Defining a new theme}
\index{plots!themes!creating|(}

Themes can be defined both from scratch, or by modifying existing saved themes, and saving the modified version. If we want to preserve the ability to change the base settings, we cannot use \code{theme()} to modify a saved theme and save the resulting theme. We need to create a new theme from scratch. However, unless you are writing a package, the first way of ``creating'' a new theme is enough, and documented in the vignette accompanying package \ggplot. We give an example below.

\begin{knitrout}\footnotesize
\definecolor{shadecolor}{rgb}{0.969, 0.969, 0.969}\color{fgcolor}\begin{kframe}
\begin{alltt}
\hlstd{my_theme} \hlkwb{<-} \hlkwd{theme_bw}\hlstd{()} \hlopt{+} \hlkwd{theme}\hlstd{(}\hlkwc{text} \hlstd{=} \hlkwd{element_text}\hlstd{(}\hlkwc{color} \hlstd{=} \hlstr{"red"}\hlstd{))}
\end{alltt}
\end{kframe}
\end{knitrout}

The default theme remains unchanged.

\begin{knitrout}\footnotesize
\definecolor{shadecolor}{rgb}{0.969, 0.969, 0.969}\color{fgcolor}\begin{kframe}
\begin{alltt}
\hlstd{p}
\end{alltt}
\end{kframe}

{\centering \includegraphics[width=.54\textwidth]{figure/pos-themes-22-1} 

}



\end{knitrout}

But we can use the saved theme when desired.

\begin{knitrout}\footnotesize
\definecolor{shadecolor}{rgb}{0.969, 0.969, 0.969}\color{fgcolor}\begin{kframe}
\begin{alltt}
\hlstd{p} \hlopt{+} \hlstd{my_theme}
\end{alltt}
\end{kframe}

{\centering \includegraphics[width=.54\textwidth]{figure/pos-themes-23-1} 

}



\end{knitrout}

Be aware that our own \code{my\_theme} is not a function, and consequently we do not use parenthesis as with the saved themes included in package \ggplot.

\begin{playground}
It is always good to learn to recognize error messages. One way of doing this is by generating errors on purpose. So do add parentheses to the statement in the code chunk above.
\end{playground}

\begin{explainbox}
How to create a new theme with a behaviour similar to those part of package \ggplot is not documented, as it is usually the case with changes that involve programming. However, you should always remember that the source code is available. Usually typing the name of a function without the parentheses is enough to get a listing of its definition, or if this is not useful, then reading the source file in the package reveals how a function has been defined. We can then use it as a template for writing our own function.
\end{explainbox}

Looking at the definition of \ggtheme{theme\_minimal()} gives us enough information as to proceed to define our own modified theme as a function.

\begin{knitrout}\footnotesize
\definecolor{shadecolor}{rgb}{0.969, 0.969, 0.969}\color{fgcolor}\begin{kframe}
\begin{alltt}
\hlstd{theme_minimal}
\end{alltt}
\begin{verbatim}
## function (base_size = 11, base_family = "", base_line_size = base_size/22, 
##     base_rect_size = base_size/22) 
## {
##     theme_bw(base_size = base_size, base_family = base_family, 
##         base_line_size = base_line_size, base_rect_size = base_rect_size) %+replace% 
##         theme(axis.ticks = element_blank(), legend.background = element_blank(), 
##             legend.key = element_blank(), panel.background = element_blank(), 
##             panel.border = element_blank(), strip.background = element_blank(), 
##             plot.background = element_blank(), complete = TRUE)
## }
## <bytecode: 0x000000002c82c328>
## <environment: namespace:ggplot2>
\end{verbatim}
\end{kframe}
\end{knitrout}

Using \ggtheme{theme\_minimal()} as a model, we will proceed to define our own theme function. Argument \code{complete = TRUE} is essential as it affects the behaviour of the returned theme. A `complete' theme replaces any theme present in the ggplot object clearing all settings, while a theme that is not `complete' adds to the existing the new elements without clearing existing settings not being redefined. Saved themes like \ggtheme{theme\_grey()} are complete themes, while the themes objects returned by \Rfunction{theme()} are by default not complete.

\begin{knitrout}\footnotesize
\definecolor{shadecolor}{rgb}{0.969, 0.969, 0.969}\color{fgcolor}\begin{kframe}
\begin{alltt}
\hlstd{my_theme} \hlkwb{<-}
    \hlkwa{function} \hlstd{(}\hlkwc{base_size} \hlstd{=} \hlnum{11}\hlstd{,} \hlkwc{base_family} \hlstd{=} \hlstr{""}\hlstd{) \{}
        \hlkwd{theme_grey}\hlstd{(}\hlkwc{base_size} \hlstd{= base_size,} \hlkwc{base_family} \hlstd{= base_family)} \hlopt{+}
            \hlkwd{theme}\hlstd{(}\hlkwc{text} \hlstd{=} \hlkwd{element_text}\hlstd{(}\hlkwc{color} \hlstd{=} \hlstr{"red"}\hlstd{),} \hlkwc{complete} \hlstd{=} \hlnum{TRUE}\hlstd{)}
    \hlstd{\}}
\end{alltt}
\end{kframe}
\end{knitrout}

The default theme remains unchanged, as shown earlier. The saved theme is now a function, and accepts arguments. In this example we have kept the function parameters the same as used by the predefined themes---whenever it is possible we should avoid surprising users.

\begin{knitrout}\footnotesize
\definecolor{shadecolor}{rgb}{0.969, 0.969, 0.969}\color{fgcolor}\begin{kframe}
\begin{alltt}
\hlstd{p} \hlopt{+} \hlkwd{my_theme}\hlstd{(}\hlkwc{base_family} \hlstd{=} \hlstr{"serif"}\hlstd{)}
\end{alltt}
\end{kframe}

{\centering \includegraphics[width=.54\textwidth]{figure/pos-themes-33-1} 

}



\end{knitrout}

There is nothing to prevent us from defining a theme function with additional parameters. The example below is fully compatible with the one defined above thanks to the default argument for \code{text.color} but allows changing the color.

\begin{knitrout}\footnotesize
\definecolor{shadecolor}{rgb}{0.969, 0.969, 0.969}\color{fgcolor}\begin{kframe}
\begin{alltt}
\hlstd{my_theme} \hlkwb{<-}
    \hlkwa{function} \hlstd{(}\hlkwc{base_size} \hlstd{=} \hlnum{11}\hlstd{,} \hlkwc{base_family} \hlstd{=} \hlstr{""}\hlstd{,} \hlkwc{text.color} \hlstd{=} \hlstr{"red"}\hlstd{) \{}
        \hlkwd{theme_grey}\hlstd{(}\hlkwc{base_size} \hlstd{= base_size,} \hlkwc{base_family} \hlstd{= base_family)} \hlopt{+}
            \hlkwd{theme}\hlstd{(}\hlkwc{text} \hlstd{=} \hlkwd{element_text}\hlstd{(}\hlkwc{color} \hlstd{= text.color),} \hlkwc{complete} \hlstd{=} \hlnum{TRUE}\hlstd{)}
    \hlstd{\}}
\end{alltt}
\end{kframe}
\end{knitrout}

\begin{knitrout}\footnotesize
\definecolor{shadecolor}{rgb}{0.969, 0.969, 0.969}\color{fgcolor}\begin{kframe}
\begin{alltt}
\hlstd{p} \hlopt{+} \hlkwd{my_theme}\hlstd{(}\hlkwc{text.color} \hlstd{=} \hlstr{"green"}\hlstd{)}
\end{alltt}
\end{kframe}

{\centering \includegraphics[width=.54\textwidth]{figure/pos-themes-33a-1} 

}



\end{knitrout}

\begin{playground}
Define a theme function that instead of \code{color} allows setting the \code{face} (regular, bold, italic) through a user-supplied argument.
\end{playground}

\begin{warningbox}
In the definition of \ggtheme{theme\_minimal()}, \code{\%+replace\%} is used so as to unset all the properties of each theme element, while \code{+} only replaces the properties explicitly given as argument to the \emph{element}-setting function.
\end{warningbox}

\begin{explainbox}
The function \ggtheme{theme\_minimal()} was a good model for the example above, however, it was not the first function I explored. I did list the definition of \ggtheme{theme\_gray()} first, but as this theme is defined from scratch, it was not the best starting point for our problem. Of course, if we had wanted to define a theme from scratch, then it would have been the `model' to use for defining it.
\end{explainbox}

Frequently one needs the same plots differently formatted, e.g.\ for overhead slides and for use in a printed article or book. In such a case, we may even want some elements like titles to be included only in the plots in overhead slides. One could create two different \Rclass{ggplot} objects, one for each occasion, but this can lead to inconsistencies if the code used to create the plot is updated. A better solution is to use themes, more generally, define themes for the different occasions according to one's taste and needs. A simple example is given in the next five code chunks.

\begin{knitrout}\footnotesize
\definecolor{shadecolor}{rgb}{0.969, 0.969, 0.969}\color{fgcolor}\begin{kframe}
\begin{alltt}
\hlstd{theme_ovh} \hlkwb{<-}
    \hlkwa{function} \hlstd{(}\hlkwc{base_size} \hlstd{=} \hlnum{15}\hlstd{,} \hlkwc{base_family} \hlstd{=} \hlstr{""}\hlstd{) \{}
        \hlkwd{theme_grey}\hlstd{(}\hlkwc{base_size} \hlstd{= base_size,} \hlkwc{base_family} \hlstd{= base_family)} \hlopt{+}
            \hlkwd{theme}\hlstd{(}\hlkwc{text} \hlstd{=} \hlkwd{element_text}\hlstd{(}\hlkwc{face} \hlstd{=} \hlstr{"bold"}\hlstd{),} \hlkwc{complete} \hlstd{=} \hlnum{TRUE}\hlstd{)}
    \hlstd{\}}

\hlstd{theme_prn} \hlkwb{<-}
    \hlkwa{function} \hlstd{(}\hlkwc{base_size} \hlstd{=} \hlnum{11}\hlstd{,} \hlkwc{base_family} \hlstd{=} \hlstr{"serif"}\hlstd{) \{}
        \hlkwd{theme_classic}\hlstd{(}\hlkwc{base_size} \hlstd{= base_size,} \hlkwc{base_family} \hlstd{= base_family)} \hlopt{+}
            \hlkwd{theme}\hlstd{(}\hlkwc{plot.title} \hlstd{=} \hlkwd{element_blank}\hlstd{(),}
                  \hlkwc{plot.subtitle} \hlstd{=} \hlkwd{element_blank}\hlstd{(),}
                  \hlkwc{complete} \hlstd{=} \hlnum{TRUE}\hlstd{)}
    \hlstd{\}}
\end{alltt}
\end{kframe}
\end{knitrout}

\begin{knitrout}\footnotesize
\definecolor{shadecolor}{rgb}{0.969, 0.969, 0.969}\color{fgcolor}\begin{kframe}
\begin{alltt}
\hlstd{p1} \hlkwb{<-} \hlstd{p} \hlopt{+} \hlkwd{ggtitle}\hlstd{(}\hlstr{"A Title"}\hlstd{,} \hlkwc{subtitle} \hlstd{=} \hlstr{"with a subtitle"}\hlstd{)}
\end{alltt}
\end{kframe}
\end{knitrout}

\begin{knitrout}\footnotesize
\definecolor{shadecolor}{rgb}{0.969, 0.969, 0.969}\color{fgcolor}\begin{kframe}
\begin{alltt}
\hlstd{p1}
\end{alltt}
\end{kframe}

{\centering \includegraphics[width=.54\textwidth]{figure/pos-themes-37-1} 

}



\end{knitrout}

\begin{knitrout}\footnotesize
\definecolor{shadecolor}{rgb}{0.969, 0.969, 0.969}\color{fgcolor}\begin{kframe}
\begin{alltt}
\hlstd{p1} \hlopt{+} \hlkwd{theme_ovh}\hlstd{()}
\end{alltt}
\end{kframe}

{\centering \includegraphics[width=.54\textwidth]{figure/pos-themes-38-1} 

}



\end{knitrout}

\begin{knitrout}\footnotesize
\definecolor{shadecolor}{rgb}{0.969, 0.969, 0.969}\color{fgcolor}\begin{kframe}
\begin{alltt}
\hlstd{p1} \hlopt{+} \hlkwd{theme_prn}\hlstd{()}
\end{alltt}
\end{kframe}

{\centering \includegraphics[width=.54\textwidth]{figure/pos-themes-39-1} 

}



\end{knitrout}

\begin{playground}
Modify the two themes defined above, so as to suite your own tastes and needs, but first of all, just play around to get a feel of all the possibilities. The help page for function \Rfunction{theme()} describes and exemplifies the use of most if not all the valid theme elements.
\end{playground}
\index{plots!themes!creating|)}
\index{plots!themes|)}

\section[Using plotmath expressions]{Using \code{plotmath} expressions}\label{sec:plot:plotmath}
\index{plotmath}
\index{plots!math expressions|(}
In sections \ref{sec:plot:function} and \ref{sec:plot:text} we gave some simple examples of the use of R expressions in plot. The \code{plotmath} demo and help in R give all the details of using expressions in plots. Composing syntactically correct expressions can be challenging. Expressions are very useful but rather tricky to use because the syntax is unusual. Although expressions are here shown in the context of plotting, they are also used in other contexts in R code.% as described in section \ref{sec:??} on page \pageref{sec:??}.

When constructing a \Rclass{ggplot} object one can either use expressions explicitly, or supply them as character string labels, and tell \code{ggplot} to parse them. For titles, axis-labels, etc. (anything that is defined within \Rfunction{labs()}) the expressions have to be entered explicitly, or saved as such into a variable, and the variable supplied as argument.

When plotting expressions using \gggeom{geom\_text()} \emph{expression} arguments should be supplied as character strings and the optional argument \code{parse = TRUE} used to tell the \emph{geometry} to parse (``convert'') the text labels into expressions.

Finally in the case of facets, panel labels can also be expressions. They can be generated by \emph{labeller} functions to allow them to be dynamic.

Before giving examples using these different mechanisms to add maths to plots, I will describe the syntax used to write expressions. The most difficult thing to remember is how to connect the different parts of the expression. Tilde (\code{\textasciitilde}) adds space in between symbols. Asterisk (\code{*}) can be also used as a connector, and is needed usually when dealing with numbers. Using space is allowed in some situations, but not in others. For a long list of examples have a look a the output and code displayed by \code{demo(plotmath)} at the R command prompt.

\begin{knitrout}\footnotesize
\definecolor{shadecolor}{rgb}{0.969, 0.969, 0.969}\color{fgcolor}\begin{kframe}
\begin{alltt}
\hlkwd{demo}\hlstd{(plotmath)}
\end{alltt}
\end{kframe}
\end{knitrout}

We will use a couple of complex examples to show in each plot how to use expressions for different elements of a plot.

We first create a data frame, using \Rfunction{paste()} to assemble a vector of subscripted $\alpha$ values.
\begin{knitrout}\footnotesize
\definecolor{shadecolor}{rgb}{0.969, 0.969, 0.969}\color{fgcolor}\begin{kframe}
\begin{alltt}
\hlkwd{set.seed}\hlstd{(}\hlnum{54321}\hlstd{)} \hlcom{# make sure we always generate the same data}
\hlstd{my.data} \hlkwb{<-}
  \hlkwd{data.frame}\hlstd{(}\hlkwc{x} \hlstd{=} \hlnum{1}\hlopt{:}\hlnum{5}\hlstd{,}
             \hlkwc{y} \hlstd{=} \hlkwd{rnorm}\hlstd{(}\hlnum{5}\hlstd{),}
             \hlkwc{greek.label} \hlstd{=} \hlkwd{paste}\hlstd{(}\hlstr{"alpha["}\hlstd{,} \hlnum{1}\hlopt{:}\hlnum{5}\hlstd{,} \hlstr{"]"}\hlstd{,} \hlkwc{sep} \hlstd{=} \hlstr{""}\hlstd{))}
\end{alltt}
\end{kframe}
\end{knitrout}

We also use a Greek $\alpha$ character, but with $i$ as subscript, instead of a number. The $y$-axis label uses a superscript for the units. The title is a rather complex expression. In these three cases, we explicitly use \Rfunction{expression()}.

We label each observation with a subscripted $alpha$, offset from the point position and rotated. We finally add an annotation with the same formula as used for the title but in red. Annotations are plotted ignoring the default aesthetics, but still make use of \emph{geometries}. We cannot pass expressions to \emph{geometries} by simply mapping them to the label aesthetic. Instead, we pass character strings that can be parsed into expressions. In simpler terms, a string, that is written using the syntax of expressions but not using the function \Rfunction{expression()}. We need to set \code{parse = TRUE} so that the strings instead of being plotted as is, are parsed into expressions at the time the plot is output. When using \gggeom{geom\_text()}, the argument passed to parameter \code{label} must be a character string. Consequently, expressions to be plotted through this \emph{geometry} need always to be parsed.

\begin{knitrout}\footnotesize
\definecolor{shadecolor}{rgb}{0.969, 0.969, 0.969}\color{fgcolor}\begin{kframe}
\begin{alltt}
\hlkwd{ggplot}\hlstd{(my.data,} \hlkwd{aes}\hlstd{(x, y,} \hlkwc{label} \hlstd{= greek.label))} \hlopt{+}
   \hlkwd{geom_point}\hlstd{()} \hlopt{+}
   \hlkwd{geom_text}\hlstd{(}\hlkwc{angle} \hlstd{=} \hlnum{45}\hlstd{,} \hlkwc{hjust} \hlstd{=} \hlnum{1.2}\hlstd{,} \hlkwc{parse} \hlstd{=} \hlnum{TRUE}\hlstd{)} \hlopt{+}
   \hlkwd{labs}\hlstd{(}\hlkwc{x} \hlstd{=} \hlkwd{expression}\hlstd{(alpha[i]),}
        \hlkwc{y} \hlstd{=} \hlkwd{expression}\hlstd{(Speed}\hlopt{~~}\hlstd{(m}\hlopt{~}\hlstd{s}\hlopt{^}\hlstd{\{}\hlopt{-}\hlnum{1}\hlstd{\})),}
        \hlkwc{title} \hlstd{=} \hlkwd{expression}\hlstd{(}\hlkwd{sqrt}\hlstd{(alpha[}\hlnum{1}\hlstd{]} \hlopt{+} \hlkwd{frac}\hlstd{(beta, gamma)))}
        \hlstd{)} \hlopt{+}
   \hlkwd{annotate}\hlstd{(}\hlstr{"text"}\hlstd{,} \hlkwc{label} \hlstd{=} \hlstr{"sqrt(alpha[1] + frac(beta, gamma))"}\hlstd{,}
           \hlkwc{y} \hlstd{=} \hlnum{2.5}\hlstd{,} \hlkwc{x} \hlstd{=} \hlnum{3}\hlstd{,} \hlkwc{size} \hlstd{=} \hlnum{8}\hlstd{,} \hlkwc{colour} \hlstd{=} \hlstr{"red"}\hlstd{,} \hlkwc{parse} \hlstd{=} \hlnum{TRUE}\hlstd{)} \hlopt{+}
   \hlkwd{expand_limits}\hlstd{(}\hlkwc{y} \hlstd{=} \hlkwd{c}\hlstd{(}\hlopt{-}\hlnum{2}\hlstd{,} \hlnum{4}\hlstd{))}
\end{alltt}
\end{kframe}

{\centering \includegraphics[width=.54\textwidth]{figure/pos-plotmath-02-1} 

}



\end{knitrout}

We can also use a character string stored in a variable, and use \Rfunction{parse()} both explicitly and implicitly by setting \code{parse = TRUE}.

\begin{knitrout}\footnotesize
\definecolor{shadecolor}{rgb}{0.969, 0.969, 0.969}\color{fgcolor}\begin{kframe}
\begin{alltt}
\hlstd{my_eq.char} \hlkwb{<-} \hlstr{"sqrt(alpha[1] + frac(beta, gamma))"}
\hlkwd{ggplot}\hlstd{(my.data,} \hlkwd{aes}\hlstd{(x, y,} \hlkwc{label} \hlstd{= greek.label))} \hlopt{+}
   \hlkwd{geom_point}\hlstd{()} \hlopt{+}
   \hlkwd{geom_text}\hlstd{(}\hlkwc{angle} \hlstd{=} \hlnum{45}\hlstd{,} \hlkwc{hjust} \hlstd{=} \hlnum{1.2}\hlstd{,} \hlkwc{parse} \hlstd{=} \hlnum{TRUE}\hlstd{)} \hlopt{+}
   \hlkwd{labs}\hlstd{(}\hlkwc{x} \hlstd{=} \hlkwd{expression}\hlstd{(alpha[i]),}
        \hlkwc{y} \hlstd{=} \hlkwd{expression}\hlstd{(Speed}\hlopt{~~}\hlstd{(m}\hlopt{~}\hlstd{s}\hlopt{^}\hlstd{\{}\hlopt{-}\hlnum{1}\hlstd{\})),}
        \hlkwc{title} \hlstd{=} \hlkwd{parse}\hlstd{(}\hlkwc{text} \hlstd{= my_eq.char)}
        \hlstd{)} \hlopt{+}
   \hlkwd{annotate}\hlstd{(}\hlstr{"text"}\hlstd{,} \hlkwc{label} \hlstd{= my_eq.char,}
           \hlkwc{y} \hlstd{=} \hlnum{2.5}\hlstd{,} \hlkwc{x} \hlstd{=} \hlnum{3}\hlstd{,} \hlkwc{size} \hlstd{=} \hlnum{8}\hlstd{,} \hlkwc{colour} \hlstd{=} \hlstr{"red"}\hlstd{,} \hlkwc{parse} \hlstd{=} \hlnum{TRUE}\hlstd{)} \hlopt{+}
   \hlkwd{expand_limits}\hlstd{(}\hlkwc{y} \hlstd{=} \hlkwd{c}\hlstd{(}\hlopt{-}\hlnum{2}\hlstd{,} \hlnum{4}\hlstd{))}
\end{alltt}
\end{kframe}

{\centering \includegraphics[width=.54\textwidth]{figure/pos-plotmath-02a-1} 

}



\end{knitrout}

The examples above are moderately complex, but do not use expressions for all the elements in a ggplot that accept them. The next example uses them for scale labels. In the cases of scales, there are alternative approaches. One approach is to use user-supplied expressions.

\begin{knitrout}\footnotesize
\definecolor{shadecolor}{rgb}{0.969, 0.969, 0.969}\color{fgcolor}\begin{kframe}
\begin{alltt}
\hlkwd{ggplot}\hlstd{(my.data,} \hlkwd{aes}\hlstd{(x,y,}\hlkwc{label} \hlstd{= greek.label))} \hlopt{+}
   \hlkwd{geom_point}\hlstd{()} \hlopt{+}
   \hlkwd{geom_text}\hlstd{(}\hlkwc{angle} \hlstd{=} \hlnum{45}\hlstd{,} \hlkwc{hjust} \hlstd{=} \hlnum{1.2}\hlstd{,} \hlkwc{parse} \hlstd{=} \hlnum{TRUE}\hlstd{)} \hlopt{+}
   \hlkwd{labs}\hlstd{(}\hlkwc{x} \hlstd{=} \hlkwa{NULL}\hlstd{,}
        \hlkwc{y} \hlstd{=} \hlkwd{expression}\hlstd{(Speed}\hlopt{~~}\hlstd{(m}\hlopt{~}\hlstd{s}\hlopt{^}\hlstd{\{}\hlopt{-}\hlnum{1}\hlstd{\})),}
        \hlkwc{title} \hlstd{=} \hlkwd{expression}\hlstd{(}\hlkwd{sqrt}\hlstd{(alpha[}\hlnum{1}\hlstd{]} \hlopt{+} \hlkwd{frac}\hlstd{(beta, gamma)))}
        \hlstd{)} \hlopt{+}
   \hlkwd{annotate}\hlstd{(}\hlstr{"text"}\hlstd{,} \hlkwc{label} \hlstd{=} \hlstr{"sqrt(alpha[1] + frac(beta, gamma))"}\hlstd{,}
           \hlkwc{y} \hlstd{=} \hlnum{2.5}\hlstd{,} \hlkwc{x} \hlstd{=} \hlnum{3}\hlstd{,} \hlkwc{size} \hlstd{=} \hlnum{8}\hlstd{,} \hlkwc{colour} \hlstd{=} \hlstr{"red"}\hlstd{,} \hlkwc{parse} \hlstd{=} \hlnum{TRUE}\hlstd{)} \hlopt{+}
   \hlkwd{scale_x_continuous}\hlstd{(}\hlkwc{breaks} \hlstd{=} \hlkwd{c}\hlstd{(}\hlnum{1}\hlstd{,}\hlnum{3}\hlstd{,}\hlnum{5}\hlstd{),}
                      \hlkwc{labels} \hlstd{=} \hlkwd{c}\hlstd{(}\hlkwd{expression}\hlstd{(alpha[}\hlnum{1}\hlstd{]),}
                                 \hlkwd{expression}\hlstd{(alpha[}\hlnum{3}\hlstd{]),}
                                 \hlkwd{expression}\hlstd{(alpha[}\hlnum{5}\hlstd{]))}
                      \hlstd{)} \hlopt{+}
   \hlkwd{expand_limits}\hlstd{(}\hlkwc{y} \hlstd{=} \hlkwd{c}\hlstd{(}\hlopt{-}\hlnum{2}\hlstd{,} \hlnum{4}\hlstd{))}
\end{alltt}
\end{kframe}

{\centering \includegraphics[width=.54\textwidth]{figure/pos-plotmath-03-1} 

}



\end{knitrout}

As \Rfunction{expression()} accepts multiple arguments separated by commas, the labels can be written more concisely using a single call to \Rfunction{expression()}.

\begin{knitrout}\footnotesize
\definecolor{shadecolor}{rgb}{0.969, 0.969, 0.969}\color{fgcolor}\begin{kframe}
\begin{alltt}
\hlkwd{ggplot}\hlstd{(my.data,} \hlkwd{aes}\hlstd{(x, y,} \hlkwc{label} \hlstd{= greek.label))} \hlopt{+}
   \hlkwd{geom_point}\hlstd{()} \hlopt{+}
   \hlkwd{geom_text}\hlstd{(}\hlkwc{angle} \hlstd{=} \hlnum{45}\hlstd{,} \hlkwc{hjust} \hlstd{=} \hlnum{1.2}\hlstd{,} \hlkwc{parse} \hlstd{=} \hlnum{TRUE}\hlstd{)} \hlopt{+}
   \hlkwd{labs}\hlstd{(}\hlkwc{x} \hlstd{=} \hlkwa{NULL}\hlstd{,}
        \hlkwc{y} \hlstd{=} \hlkwd{expression}\hlstd{(Speed}\hlopt{~~}\hlstd{(m}\hlopt{~}\hlstd{s}\hlopt{^}\hlstd{\{}\hlopt{-}\hlnum{1}\hlstd{\})),}
        \hlkwc{title} \hlstd{=} \hlkwd{expression}\hlstd{(}\hlkwd{sqrt}\hlstd{(alpha[}\hlnum{1}\hlstd{]} \hlopt{+} \hlkwd{frac}\hlstd{(beta, gamma)))}
        \hlstd{)} \hlopt{+}
   \hlkwd{annotate}\hlstd{(}\hlstr{"text"}\hlstd{,} \hlkwc{label} \hlstd{=} \hlstr{"sqrt(alpha[1] + frac(beta, gamma))"}\hlstd{,}
           \hlkwc{y} \hlstd{=} \hlnum{2.5}\hlstd{,} \hlkwc{x} \hlstd{=} \hlnum{3}\hlstd{,} \hlkwc{size} \hlstd{=} \hlnum{8}\hlstd{,} \hlkwc{colour} \hlstd{=} \hlstr{"red"}\hlstd{,} \hlkwc{parse} \hlstd{=} \hlnum{TRUE}\hlstd{)} \hlopt{+}
   \hlkwd{scale_x_continuous}\hlstd{(}\hlkwc{breaks} \hlstd{=} \hlkwd{c}\hlstd{(}\hlnum{1}\hlstd{,}\hlnum{3}\hlstd{,}\hlnum{5}\hlstd{),}
                      \hlkwc{labels} \hlstd{=} \hlkwd{expression}\hlstd{(alpha[}\hlnum{1}\hlstd{], alpha[}\hlnum{3}\hlstd{], alpha[}\hlnum{5}\hlstd{])}
                      \hlstd{)} \hlopt{+}
   \hlkwd{expand_limits}\hlstd{(}\hlkwc{y} \hlstd{=} \hlkwd{c}\hlstd{(}\hlopt{-}\hlnum{2}\hlstd{,} \hlnum{4}\hlstd{))}
\end{alltt}
\end{kframe}

{\centering \includegraphics[width=.54\textwidth]{figure/pos-plotmath-03a-1} 

}



\end{knitrout}

A different approach (no example shown) would be to use \Rfunction{parse()} explicitly for each individual label, something that might be needed if the tick labels need to be ``assembled'' programmatically instead of set as constants.

\begin{advplayground}
Instead of this being an exercise for you to write code, you will need to study the code shown below until you are sure understand how it works. It makes use of different things you have learn in the current and previous chapters.

Parsing multiple labels in a scale definition, after assembling them with \Rfunction{paste()}. We want to achieve more generality, looking ahead to a future function to be defined.

\begin{knitrout}\footnotesize
\definecolor{shadecolor}{rgb}{0.969, 0.969, 0.969}\color{fgcolor}\begin{kframe}
\begin{alltt}
\hlstd{labels.char} \hlkwb{<-} \hlkwd{paste}\hlstd{(}\hlstr{"alpha["}\hlstd{,} \hlkwd{as.character}\hlstd{(}\hlkwd{c}\hlstd{(}\hlnum{1}\hlstd{,}\hlnum{3}\hlstd{,}\hlnum{5}\hlstd{)),} \hlstr{"]"}\hlstd{)}
\hlstd{my_parse} \hlkwb{<-} \hlkwa{function}\hlstd{(}\hlkwc{x}\hlstd{,} \hlkwc{...}\hlstd{) \{}\hlkwd{parse}\hlstd{(}\hlkwc{text} \hlstd{= x, ...)\}}
\hlstd{labels.xpr} \hlkwb{<-} \hlkwd{sapply}\hlstd{(labels.char, my_parse)}
\end{alltt}
\end{kframe}
\end{knitrout}

This three lines of code return a vector of expressions that can be used in a scale definition. Before using them, we will make a function out of them.

\begin{knitrout}\footnotesize
\definecolor{shadecolor}{rgb}{0.969, 0.969, 0.969}\color{fgcolor}\begin{kframe}
\begin{alltt}
\hlstd{make_labels} \hlkwb{<-} \hlkwa{function}\hlstd{(}\hlkwc{base_text} \hlstd{=} \hlstr{"alpha"}\hlstd{,} \hlkwc{idxs} \hlstd{=} \hlnum{1}\hlopt{:}\hlnum{5}\hlstd{,} \hlkwc{...}\hlstd{) \{}
    \hlkwd{sapply}\hlstd{(}\hlkwc{X} \hlstd{=} \hlkwd{paste}\hlstd{(base_text,} \hlstr{"["}\hlstd{,} \hlkwd{as.character}\hlstd{(idxs),} \hlstr{"]"}\hlstd{,} \hlkwc{sep} \hlstd{=} \hlstr{""}\hlstd{),}
           \hlkwc{FUN} \hlstd{=} \hlkwa{function}\hlstd{(}\hlkwc{x}\hlstd{,} \hlkwc{...}\hlstd{) \{}\hlkwd{parse}\hlstd{(}\hlkwc{text} \hlstd{= x, ...)\},}
           \hlkwc{USE.NAMES} \hlstd{=} \hlnum{FALSE}\hlstd{)}
\hlstd{\}}
\end{alltt}
\end{kframe}
\end{knitrout}

And now we can use the function in a plot.

\begin{knitrout}\footnotesize
\definecolor{shadecolor}{rgb}{0.969, 0.969, 0.969}\color{fgcolor}\begin{kframe}
\begin{alltt}
\hlstd{breaks} \hlkwb{<-} \hlkwd{c}\hlstd{(}\hlnum{1}\hlstd{,}\hlnum{3}\hlstd{,}\hlnum{5}\hlstd{)}
\hlkwd{ggplot}\hlstd{(my.data,} \hlkwd{aes}\hlstd{(x, y,} \hlkwc{label} \hlstd{= greek.label))} \hlopt{+}
   \hlkwd{geom_point}\hlstd{()} \hlopt{+}
   \hlkwd{geom_text}\hlstd{(}\hlkwc{angle} \hlstd{=} \hlnum{45}\hlstd{,} \hlkwc{hjust} \hlstd{=} \hlnum{1.2}\hlstd{,} \hlkwc{parse} \hlstd{=} \hlnum{TRUE}\hlstd{)} \hlopt{+}
   \hlkwd{labs}\hlstd{(}\hlkwc{x} \hlstd{=} \hlkwa{NULL}\hlstd{,}
        \hlkwc{y} \hlstd{=} \hlkwd{expression}\hlstd{(Speed}\hlopt{~~}\hlstd{(m}\hlopt{~}\hlstd{s}\hlopt{^}\hlstd{\{}\hlopt{-}\hlnum{1}\hlstd{\})),}
        \hlkwc{title} \hlstd{=} \hlkwd{expression}\hlstd{(}\hlkwd{sqrt}\hlstd{(alpha[}\hlnum{1}\hlstd{]} \hlopt{+} \hlkwd{frac}\hlstd{(beta, gamma)))}
        \hlstd{)} \hlopt{+}
   \hlkwd{annotate}\hlstd{(}\hlstr{"text"}\hlstd{,} \hlkwc{label} \hlstd{=} \hlstr{"sqrt(alpha[1] + frac(beta, gamma))"}\hlstd{,}
           \hlkwc{y} \hlstd{=} \hlnum{2.5}\hlstd{,} \hlkwc{x} \hlstd{=} \hlnum{3}\hlstd{,} \hlkwc{size} \hlstd{=} \hlnum{8}\hlstd{,} \hlkwc{colour} \hlstd{=} \hlstr{"red"}\hlstd{,} \hlkwc{parse} \hlstd{=} \hlnum{TRUE}\hlstd{)} \hlopt{+}
   \hlkwd{scale_x_continuous}\hlstd{(}\hlkwc{breaks} \hlstd{= breaks,}
                      \hlkwc{labels} \hlstd{=} \hlkwd{make_labels}\hlstd{(}\hlstr{"alpha"}\hlstd{, breaks)}
                      \hlstd{)} \hlopt{+}
   \hlkwd{expand_limits}\hlstd{(}\hlkwc{y} \hlstd{=} \hlkwd{c}\hlstd{(}\hlopt{-}\hlnum{2}\hlstd{,} \hlnum{4}\hlstd{))}
\end{alltt}
\end{kframe}
\end{knitrout}

As a final task, change the code above so that the labels are subscripted $\beta$s and breaks from 1 to 5 with step 1.

\end{advplayground}

\begin{explainbox}
\textbf{Differences between \Rfunction{parse()} and \Rfunction{expression()}}. Function \Rfunction{parse()} takes as argument a character string. This is very useful as the character string can be created programmatically. When using \code{expression()} this is not possible, except for substitution at execution time of the value of variables into the expression. See help pages for both functions.

Function \Rfunction{expression()} accepts its arguments without any delimiters. Function \Rfunction{parse()} takes a single character string as argument to be parsed, in which case quotation marks need to be \emph{escaped} (using \code{\backslash"} where a literal \code{"} is desired). We can, also in both cases embed a character string by means of one of the functions \Rfunction{plain()}, \Rfunction{italic()}, \Rfunction{bold()} or \Rfunction{bolditalic()} which also affect the font used. The argument to these functions needs sometimes to be a character string delimited by quotation marks.

When using \Rfunction{expression()}, bare quotation marks can be embedded,

\begin{knitrout}\footnotesize
\definecolor{shadecolor}{rgb}{0.969, 0.969, 0.969}\color{fgcolor}\begin{kframe}
\begin{alltt}
\hlkwd{ggplot}\hlstd{(cars,} \hlkwd{aes}\hlstd{(speed, dist))} \hlopt{+}
  \hlkwd{geom_point}\hlstd{()} \hlopt{+}
  \hlkwd{xlab}\hlstd{(}\hlkwd{expression}\hlstd{(x[}\hlnum{1}\hlstd{]}\hlopt{*}\hlstr{"  test"}\hlstd{))}
\end{alltt}
\end{kframe}

{\centering \includegraphics[width=.54\textwidth]{figure/pos-expr-parse-box-01-1} 

}



\end{knitrout}

while in the case of \Rfunction{parse()} they need to be \emph{escaped},

\begin{knitrout}\footnotesize
\definecolor{shadecolor}{rgb}{0.969, 0.969, 0.969}\color{fgcolor}\begin{kframe}
\begin{alltt}
\hlkwd{ggplot}\hlstd{(cars,} \hlkwd{aes}\hlstd{(speed, dist))} \hlopt{+}
  \hlkwd{geom_point}\hlstd{()} \hlopt{+}
  \hlkwd{xlab}\hlstd{(}\hlkwd{parse}\hlstd{(}\hlkwc{text} \hlstd{=} \hlstr{"x[1]*\textbackslash{}"  test\textbackslash{}""}\hlstd{))}
\end{alltt}
\end{kframe}

{\centering \includegraphics[width=.54\textwidth]{figure/pos-expr-parse-box-02-1} 

}



\end{knitrout}

and in some cases will need to be enclosed within a format function.

\begin{knitrout}\footnotesize
\definecolor{shadecolor}{rgb}{0.969, 0.969, 0.969}\color{fgcolor}\begin{kframe}
\begin{alltt}
\hlkwd{ggplot}\hlstd{(cars,} \hlkwd{aes}\hlstd{(speed, dist))} \hlopt{+}
  \hlkwd{geom_point}\hlstd{()} \hlopt{+}
  \hlkwd{xlab}\hlstd{(}\hlkwd{parse}\hlstd{(}\hlkwc{text} \hlstd{=} \hlstr{"x[1]*italic(\textbackslash{}"  test\textbackslash{}")"}\hlstd{))}
\end{alltt}
\end{kframe}

{\centering \includegraphics[width=.54\textwidth]{figure/pos-expr-parse-box-03-1} 

}



\end{knitrout}

We can compare the expressions returned by \Rfunction{expression()} and \Rfunction{parse()} as used above.

\begin{knitrout}\footnotesize
\definecolor{shadecolor}{rgb}{0.969, 0.969, 0.969}\color{fgcolor}\begin{kframe}
\begin{alltt}
\hlkwd{expression}\hlstd{(x[}\hlnum{1}\hlstd{]}\hlopt{*}\hlstr{"  test"}\hlstd{)}
\end{alltt}
\begin{verbatim}
## expression(x[1] * "  test")
\end{verbatim}
\begin{alltt}
\hlkwd{parse}\hlstd{(}\hlkwc{text} \hlstd{=} \hlstr{"x[1]*\textbackslash{}"  test\textbackslash{}""}\hlstd{)}
\end{alltt}
\begin{verbatim}
## expression(x[1] * "  test")
\end{verbatim}
\end{kframe}
\end{knitrout}

A few additional remarks. If \Rfunction{expression()} is passed multiple arguments, \Rfunction{ggplot()} uses only the first one, in the case of axis labels, when a single character string is expected as argument.

\begin{knitrout}\footnotesize
\definecolor{shadecolor}{rgb}{0.969, 0.969, 0.969}\color{fgcolor}\begin{kframe}
\begin{alltt}
\hlkwd{expression}\hlstd{(x[}\hlnum{1}\hlstd{],} \hlstr{"  test"}\hlstd{)}
\end{alltt}
\begin{verbatim}
## expression(x[1], "  test")
\end{verbatim}
\end{kframe}
\end{knitrout}

\begin{knitrout}\footnotesize
\definecolor{shadecolor}{rgb}{0.969, 0.969, 0.969}\color{fgcolor}\begin{kframe}
\begin{alltt}
\hlkwd{ggplot}\hlstd{(cars,} \hlkwd{aes}\hlstd{(speed, dist))} \hlopt{+}
  \hlkwd{geom_point}\hlstd{()} \hlopt{+}
  \hlkwd{xlab}\hlstd{(}\hlkwd{expression}\hlstd{(x[}\hlnum{1}\hlstd{],} \hlstr{"  test"}\hlstd{))}
\end{alltt}
\end{kframe}

{\centering \includegraphics[width=.54\textwidth]{figure/pos-expr-parse-box-06-1} 

}



\end{knitrout}

Depending on the location within a expression, spaces maybe ignored, or even illegal. To juxtapose elements without adding space use \code{*}, to explicitly insert white space, use \code{\textasciitilde}. As shown above spaces are accepted within quoted text.

So the following alternatives can also be used.

\begin{knitrout}\footnotesize
\definecolor{shadecolor}{rgb}{0.969, 0.969, 0.969}\color{fgcolor}\begin{kframe}
\begin{alltt}
\hlkwd{ggplot}\hlstd{(cars,} \hlkwd{aes}\hlstd{(speed, dist))} \hlopt{+}
  \hlkwd{geom_point}\hlstd{()} \hlopt{+}
  \hlkwd{xlab}\hlstd{(}\hlkwd{parse}\hlstd{(}\hlkwc{text} \hlstd{=} \hlstr{"x[1]~~~~\textbackslash{}"test\textbackslash{}""}\hlstd{))}
\end{alltt}
\end{kframe}

{\centering \includegraphics[width=.54\textwidth]{figure/pos-expr-parse-box-07-1} 

}



\end{knitrout}

\begin{knitrout}\footnotesize
\definecolor{shadecolor}{rgb}{0.969, 0.969, 0.969}\color{fgcolor}\begin{kframe}
\begin{alltt}
\hlkwd{ggplot}\hlstd{(cars,} \hlkwd{aes}\hlstd{(speed, dist))} \hlopt{+}
  \hlkwd{geom_point}\hlstd{()} \hlopt{+}
  \hlkwd{xlab}\hlstd{(}\hlkwd{parse}\hlstd{(}\hlkwc{text} \hlstd{=} \hlstr{"x[1]~~~~plain(test)"}\hlstd{))}
\end{alltt}
\end{kframe}

{\centering \includegraphics[width=.54\textwidth]{figure/pos-expr-parse-box-08-1} 

}



\end{knitrout}

However, unquoted white space is discarded.

\begin{knitrout}\footnotesize
\definecolor{shadecolor}{rgb}{0.969, 0.969, 0.969}\color{fgcolor}\begin{kframe}
\begin{alltt}
\hlkwd{ggplot}\hlstd{(cars,} \hlkwd{aes}\hlstd{(speed, dist))} \hlopt{+}
  \hlkwd{geom_point}\hlstd{()} \hlopt{+}
  \hlkwd{xlab}\hlstd{(}\hlkwd{parse}\hlstd{(}\hlkwc{text} \hlstd{=} \hlstr{"x[1]*plain(   test)"}\hlstd{))}
\end{alltt}
\end{kframe}

{\centering \includegraphics[width=.54\textwidth]{figure/pos-expr-parse-box-09-1} 

}



\end{knitrout}

Finally, it can be surprising that trailing zeros in numeric values appearing within an expression or text to be parsed are dropped. To force the trailing zeros to be retained we need to enclose the number in quotation marks so that it is interpreted as a character string.

\begin{knitrout}\footnotesize
\definecolor{shadecolor}{rgb}{0.969, 0.969, 0.969}\color{fgcolor}\begin{kframe}
\begin{alltt}
\hlkwd{ggplot}\hlstd{(cars,} \hlkwd{aes}\hlstd{(speed, dist))} \hlopt{+}
  \hlkwd{geom_point}\hlstd{()} \hlopt{+}
  \hlkwd{annotate}\hlstd{(}\hlkwc{geom} \hlstd{=} \hlstr{"text"}\hlstd{,}
           \hlkwc{x} \hlstd{=} \hlkwd{rep}\hlstd{(}\hlnum{6}\hlstd{,} \hlnum{3}\hlstd{),} \hlkwc{y} \hlstd{=} \hlkwd{c}\hlstd{(}\hlnum{90}\hlstd{,} \hlnum{100}\hlstd{,} \hlnum{110}\hlstd{),}
           \hlkwc{label} \hlstd{=} \hlkwd{c}\hlstd{(}\hlstr{"'1.00'*x^2"}\hlstd{,} \hlstr{"1.00*x^2"}\hlstd{,} \hlstr{"1.01*x^2"}\hlstd{),} \hlkwc{parse} \hlstd{=} \hlnum{TRUE}\hlstd{)}
\end{alltt}
\end{kframe}

{\centering \includegraphics[width=.54\textwidth]{figure/pos-expr-parse-box-11-1} 

}



\end{knitrout}

\end{explainbox}

Above we used paste to insert values stored in a variable, and this combined with \Rfunction{format()}, \Rfunction{sprintf()}, and \Rfunction{strftime()} gives already a lot of flexibility.

\begin{playground}
Study the examples below. If you are familiar with \langname{C} or \langname{C++} the last two functions will be already familiar to you.

\begin{knitrout}\footnotesize
\definecolor{shadecolor}{rgb}{0.969, 0.969, 0.969}\color{fgcolor}\begin{kframe}
\begin{alltt}
\hlkwd{sprintf}\hlstd{(}\hlstr{"%s: %.3g two values formatted and inserted"}\hlstd{,} \hlstr{"test"}\hlstd{,} \hlnum{15234}\hlstd{)}
\hlkwd{sprintf}\hlstd{(}\hlstr{"log(%.3f) = %.3f"}\hlstd{,} \hlnum{5}\hlstd{,} \hlkwd{log}\hlstd{(}\hlnum{5}\hlstd{))}
\hlkwd{sprintf}\hlstd{(}\hlstr{"log(%.3g) = %.3g"}\hlstd{,} \hlnum{5}\hlstd{,} \hlkwd{log}\hlstd{(}\hlnum{5}\hlstd{))}
\end{alltt}
\end{kframe}
\end{knitrout}

Write a function for the second statement in the chunk above. The function should take a single numeric argument through its only formal parameter, and produce equivalent output to the statement above. However, it should be usable with any numeric value.

Do look up the help pages for these three functions and play with them at the console. They are extremely useful.
\end{playground}

It is also possible to substitute the value of variables or, in fact, the result of evaluation, into a new expression, allowing on-the-fly construction of expressions. Such expressions are frequently used as labels in plots. This is achieved through use of \emph{quoting} and \emph{substitution}.

We use \Rfunction{bquote()} to substitute variables or expressions enclosed in \code{.( )} by their value. Be aware that the argument to \Rfunction{bquote()} needs to be written as an expression, in this example we need to use a tilde, \code{\textasciitilde}, to insert a space between words. Furthermore, if the expressions include variables, these will be searched for in the environment.

\begin{knitrout}\footnotesize
\definecolor{shadecolor}{rgb}{0.969, 0.969, 0.969}\color{fgcolor}\begin{kframe}
\begin{alltt}
\hlkwd{ggplot}\hlstd{(cars,} \hlkwd{aes}\hlstd{(speed, dist))} \hlopt{+}
  \hlkwd{geom_point}\hlstd{()} \hlopt{+}
  \hlkwd{labs}\hlstd{(}\hlkwc{title} \hlstd{=} \hlkwd{bquote}\hlstd{(Time}\hlopt{~}\hlstd{zone}\hlopt{:} \hlkwd{.}\hlstd{(}\hlkwd{Sys.timezone}\hlstd{())),}
       \hlkwc{subtitle} \hlstd{=} \hlkwd{bquote}\hlstd{(Date}\hlopt{:} \hlkwd{.}\hlstd{(}\hlkwd{as.character}\hlstd{(}\hlkwd{today}\hlstd{())))}
       \hlstd{)}
\end{alltt}
\end{kframe}

{\centering \includegraphics[width=.54\textwidth]{figure/pos-expr-bquote-01-1} 

}



\end{knitrout}

In the case of \Rfunction{substitute()} we supply what is to used for substitution through a named list.

\begin{knitrout}\footnotesize
\definecolor{shadecolor}{rgb}{0.969, 0.969, 0.969}\color{fgcolor}\begin{kframe}
\begin{alltt}
\hlkwd{ggplot}\hlstd{(cars,} \hlkwd{aes}\hlstd{(speed, dist))} \hlopt{+}
  \hlkwd{geom_point}\hlstd{()} \hlopt{+}
  \hlkwd{labs}\hlstd{(}\hlkwc{title} \hlstd{=} \hlkwd{substitute}\hlstd{(Time}\hlopt{~}\hlstd{zone}\hlopt{:} \hlstd{tz,} \hlkwd{list}\hlstd{(}\hlkwc{tz} \hlstd{=} \hlkwd{Sys.timezone}\hlstd{())),}
       \hlkwc{subtitle} \hlstd{=} \hlkwd{substitute}\hlstd{(Date}\hlopt{:} \hlstd{date,} \hlkwd{list}\hlstd{(}\hlkwc{date} \hlstd{=} \hlkwd{as.character}\hlstd{(}\hlkwd{today}\hlstd{())))}
       \hlstd{)}
\end{alltt}
\end{kframe}

{\centering \includegraphics[width=.54\textwidth]{figure/pos-expr-substitute-01-1} 

}



\end{knitrout}

For example, substitution can be used to assemble an expression within a function based on the arguments passed. One case of interest is to retrieve the name of the object passed as an argument, from within a function.

\begin{knitrout}\footnotesize
\definecolor{shadecolor}{rgb}{0.969, 0.969, 0.969}\color{fgcolor}\begin{kframe}
\begin{alltt}
\hlstd{deparse_test} \hlkwb{<-} \hlkwa{function}\hlstd{(}\hlkwc{x}\hlstd{) \{}
  \hlkwd{print}\hlstd{(}\hlkwd{deparse}\hlstd{(}\hlkwd{substitute}\hlstd{(x)))}
\hlstd{\}}

\hlstd{a} \hlkwb{<-} \hlstr{"saved in variable"}

\hlkwd{deparse_test}\hlstd{(}\hlstr{"constant"}\hlstd{)}
\end{alltt}
\begin{verbatim}
## [1] "\"constant\""
\end{verbatim}
\begin{alltt}
\hlkwd{deparse_test}\hlstd{(}\hlnum{1} \hlopt{+} \hlnum{2}\hlstd{)}
\end{alltt}
\begin{verbatim}
## [1] "1 + 2"
\end{verbatim}
\begin{alltt}
\hlkwd{deparse_test}\hlstd{(a)}
\end{alltt}
\begin{verbatim}
## [1] "a"
\end{verbatim}
\end{kframe}
\end{knitrout}

\index{plots!math expressions|)}

\section{Generating output files}
\index{devices!output|see{graphic output devices}}
\index{graphic output devices|(}
It is possible, when using \RStudio, to directly export the displayed plot to a file. However, if the file will have to be generated again at a later time, or a series of plots need to be produced with consistent format, it is best to include the commands to export the plot in the script.

In R,\index{plots!printing}\index{plots!saving}\index{plots!output to files} files are created by printing to different devices. Printing is directed to a currently open device. Some devices produce screen output, others files. Devices depend on drivers. There are both devices that or part of R, and devices that can be added through packages.

A very\index{plots!PDF output} simple example of PDF output (width and height in inches):

\begin{knitrout}\footnotesize
\definecolor{shadecolor}{rgb}{0.969, 0.969, 0.969}\color{fgcolor}\begin{kframe}
\begin{alltt}
\hlstd{fig1} \hlkwb{<-} \hlkwd{ggplot}\hlstd{(}\hlkwd{data.frame}\hlstd{(}\hlkwc{x} \hlstd{=} \hlopt{-}\hlnum{3}\hlopt{:}\hlnum{3}\hlstd{),} \hlkwd{aes}\hlstd{(}\hlkwc{x} \hlstd{= x))} \hlopt{+}
  \hlkwd{stat_function}\hlstd{(}\hlkwc{fun} \hlstd{= dnorm)}
\hlkwd{pdf}\hlstd{(}\hlkwc{file} \hlstd{=} \hlstr{"fig1.pdf"}\hlstd{,} \hlkwc{width} \hlstd{=} \hlnum{8}\hlstd{,} \hlkwc{height} \hlstd{=} \hlnum{6}\hlstd{)}
\hlkwd{print}\hlstd{(fig1)}
\hlkwd{dev.off}\hlstd{()}
\end{alltt}
\end{kframe}
\end{knitrout}

Encapsulated\index{plots!Postscript output} Postscript output (width and height in inches):

\begin{knitrout}\footnotesize
\definecolor{shadecolor}{rgb}{0.969, 0.969, 0.969}\color{fgcolor}\begin{kframe}
\begin{alltt}
\hlkwd{postscript}\hlstd{(}\hlkwc{file} \hlstd{=} \hlstr{"fig1.eps"}\hlstd{,} \hlkwc{width} \hlstd{=} \hlnum{8}\hlstd{,} \hlkwc{height} \hlstd{=} \hlnum{6}\hlstd{)}
\hlkwd{print}\hlstd{(fig1)}
\hlkwd{dev.off}\hlstd{()}
\end{alltt}
\end{kframe}
\end{knitrout}

There are Graphics devices for\index{plots!bitmap output} BMP, JPEG, PNG and TIFF format bitmap files. In this case the default units for width and height is pixels. For example we can generate TIFF output:

\begin{knitrout}\footnotesize
\definecolor{shadecolor}{rgb}{0.969, 0.969, 0.969}\color{fgcolor}\begin{kframe}
\begin{alltt}
\hlkwd{tiff}\hlstd{(}\hlkwc{file} \hlstd{=} \hlstr{"fig1.tiff"}\hlstd{,} \hlkwc{width} \hlstd{=} \hlnum{1000}\hlstd{,} \hlkwc{height} \hlstd{=} \hlnum{800}\hlstd{)}
\hlkwd{print}\hlstd{(fig1)}
\hlkwd{dev.off}\hlstd{()}
\end{alltt}
\end{kframe}
\end{knitrout}

\section{Building complex data displays}
\index{examples!modular plot construction|(}

In this section we do not refer to those aspects of the design of a plot that can be adjust through themes (see section \ref{sec:plot:themes} on page \pageref{sec:plot:themes}. Whenever this possibility exists, it is the best. Here we refer to aspects that are not really part of the graphical (''artistic'') design, but instead mappings, labels and similar data and metadata related aspects of plots. In many cases scales (see section \ref{sec:plot:scales} on page \pageref{sec:plot:scales}) also fall within the scope of the present section.

\subsection{Using the grammar of graphics for individual plots}\label{sec:plot:composition}

The grammar of graphics\index{grammar of graphics}\index{plots!layers} allows one to build and test plots incrementally. In daily use, it is best to start with a simple design for a plot, print this plot, checking that the output is as expected and the code error-free. Afterwards, one can map additional \emph{aesthetics} and \emph{geometries} and \emph{statistics} gradually. The final steps are then to add \emph{annotations} and the text or expressions used for titles, and axis and key labels.

\begin{playground}
  Build a graphically complex data plot of your interest, step by step. By step by step, I do not refer to using the grammar in the construction of the plot as earlier, but of taking advantage of this modularity to test intermediate version in an iterative design process, first by building up the complex plot in stages as a tool in debugging, and later using iteration in the processes of improving the graphic design of the plot and improving its readability and effectiveness.
\end{playground}

As in any type of script with instructions (for humans or computers), we should avoid unnecessary repetition, as repetition conspires against consistent results and is a major source of errors when the script needs to be modified. Not less important, a shorter script, if well written is easier to read.

One approach is to use user-defined functions\index{plots!consistent format using functions}. One can write simple wrapper functions making use of functions defined in \ggplot, for example, adding/changing the defaults mappings to ones suitable for our application. In the case of \Rfunction{ggplot()}, as it is defined as a generic function, if one's data is stored in objects of a user-defined class, instead of a wrapper we can use a specialization of the generic. Such a specialized methods is almost invisible to users (e.g.\ does not require a different syntax or adding a word to the grammar). At the other extreme of complexity compared to a wrapper function, we could write a function that encapsulates all the code needed to build a specific type of plot. Package \pkgname{ggspectra} uses both approaches.

The graphic style of plots is best adjusted by means of themes. Although themes can be also used as building blocks as described below in the together with layers and scales, in can be better to add them when plots are being rendered as we may want to use different themes when, for example, rendering the same plot on different devices such as a printer or screen.

As \Rclass{ggplot}\index{plots!reusing parts of} objects are composed using operator \code{+} to assemble together the different components, one can also store in a variable these components, or using a list, partial plots, which can be used to compose the final figure.

\subsection{Saving plot layers and scales in variables}

We can assign a ggplot object or a part of it to a variable, and then assemble a new plot from the different pieces.

\begin{knitrout}\footnotesize
\definecolor{shadecolor}{rgb}{0.969, 0.969, 0.969}\color{fgcolor}\begin{kframe}
\begin{alltt}
\hlstd{myplot} \hlkwb{<-} \hlkwd{ggplot}\hlstd{(}\hlkwc{data} \hlstd{= mtcars,}
                 \hlkwd{aes}\hlstd{(}\hlkwc{x} \hlstd{= disp,} \hlkwc{y} \hlstd{= mpg,}
                 \hlkwc{colour} \hlstd{=} \hlkwd{factor}\hlstd{(cyl)))} \hlopt{+}
          \hlkwd{geom_point}\hlstd{()}

\hlstd{mylabs} \hlkwb{<-} \hlkwd{labs}\hlstd{(}\hlkwc{x} \hlstd{=} \hlstr{"Engine displacement)"}\hlstd{,}
               \hlkwc{y} \hlstd{=} \hlstr{"Gross horsepower"}\hlstd{,}
               \hlkwc{colour} \hlstd{=} \hlstr{"Number of\textbackslash{}ncylinders"}\hlstd{,}
               \hlkwc{shape} \hlstd{=} \hlstr{"Number of\textbackslash{}ncylinders"}\hlstd{)}
\end{alltt}
\end{kframe}
\end{knitrout}

And now we can assemble the final plot by putting the saved ``blocks'' together. A \code{"gg"} object should always be on the left side of the \code{+} operator. We add layers or other elements to it.

\begin{knitrout}\footnotesize
\definecolor{shadecolor}{rgb}{0.969, 0.969, 0.969}\color{fgcolor}\begin{kframe}
\begin{alltt}
\hlstd{myplot}
\hlstd{myplot} \hlopt{+} \hlstd{mylabs} \hlopt{+} \hlkwd{theme_bw}\hlstd{(}\hlnum{16}\hlstd{)}
\hlstd{myplot} \hlopt{+} \hlstd{mylabs} \hlopt{+} \hlkwd{theme_bw}\hlstd{(}\hlnum{16}\hlstd{)} \hlopt{+} \hlkwd{ylim}\hlstd{(}\hlnum{0}\hlstd{,} \hlnum{NA}\hlstd{)}
\end{alltt}
\end{kframe}

{\centering \includegraphics[width=.54\textwidth]{figure/pos-main-chunk-40-1} 
\includegraphics[width=.54\textwidth]{figure/pos-main-chunk-40-2} 
\includegraphics[width=.54\textwidth]{figure/pos-main-chunk-40-3} 

}



\end{knitrout}

We can also save intermediate results.

\begin{knitrout}\footnotesize
\definecolor{shadecolor}{rgb}{0.969, 0.969, 0.969}\color{fgcolor}\begin{kframe}
\begin{alltt}
\hlstd{mylogplot} \hlkwb{<-} \hlstd{myplot} \hlopt{+} \hlkwd{scale_y_log10}\hlstd{(}\hlkwc{limits}\hlstd{=}\hlkwd{c}\hlstd{(}\hlnum{8}\hlstd{,}\hlnum{55}\hlstd{))}
\hlstd{mylogplot} \hlopt{+} \hlstd{mylabs} \hlopt{+} \hlkwd{theme_bw}\hlstd{(}\hlnum{16}\hlstd{)}
\end{alltt}
\end{kframe}

{\centering \includegraphics[width=.54\textwidth]{figure/pos-main-chunk-41-1} 

}



\end{knitrout}

\subsection{Saving plot layers and scales in variables in lists}

If the pieces to be put together do not include a \code{"gg"} object, we can group them
into a "list" object. When we later add the list to a \code{"gg"} object, the members of the list are added one by one to the plot.

\begin{knitrout}\footnotesize
\definecolor{shadecolor}{rgb}{0.969, 0.969, 0.969}\color{fgcolor}\begin{kframe}
\begin{alltt}
\hlstd{myparts} \hlkwb{<-} \hlkwd{list}\hlstd{(mylabs,} \hlkwd{theme_bw}\hlstd{(}\hlnum{16}\hlstd{))}
\hlstd{mylogplot} \hlopt{+} \hlstd{myparts}
\end{alltt}
\end{kframe}

{\centering \includegraphics[width=.54\textwidth]{figure/pos-main-chunk-42-1} 

}



\end{knitrout}

\begin{playground}
Revise the code you wrote for the ``playground'' exercise in section \ref{sec:plot:composition}, but this time, pre-building and saving groups of elements that you expect to be useful unchanged when composing a different plot of the same type, or a plot of a different type from the same data.
\end{playground}

\index{examples!modular plot construction|)}

\subsection{Using functions as building blocks}

When the blocks we assemble need to accept arguments when used, we have to define functions instead of saving constant blocks to variables either individually or grouped in lists. The functions we define, have to return either a \code{"gg"} object, a list of components or a single plot component. The simplest use is to alter some defaults in existing functions which return \code{"gg"} objects or layers. The ellipsis (\code{...}) allows passing named arguments to a nested function. In this case, every single passed by name to \code{bw\_ggplot()} will be copied as argument to the nested call the \code{ggplot()}. Be aware, that supplying arguments by position, is not possible, unless, one includes the parameters explicitly in the wrapper function.

\begin{knitrout}\footnotesize
\definecolor{shadecolor}{rgb}{0.969, 0.969, 0.969}\color{fgcolor}\begin{kframe}
\begin{alltt}
\hlstd{bw_ggplot} \hlkwb{<-} \hlkwa{function}\hlstd{(}\hlkwc{...}\hlstd{) \{}
  \hlkwd{ggplot}\hlstd{(...)} \hlopt{+}
  \hlkwd{theme_bw}\hlstd{()}
\hlstd{\}}
\end{alltt}
\end{kframe}
\end{knitrout}

Which could be used as follows.

\begin{knitrout}\footnotesize
\definecolor{shadecolor}{rgb}{0.969, 0.969, 0.969}\color{fgcolor}\begin{kframe}
\begin{alltt}
\hlkwd{bw_ggplot}\hlstd{(}\hlkwc{data} \hlstd{= mtcars,}
          \hlkwd{aes}\hlstd{(}\hlkwc{x} \hlstd{= disp,} \hlkwc{y} \hlstd{= mpg,}
          \hlkwc{colour} \hlstd{=} \hlkwd{factor}\hlstd{(cyl)))} \hlopt{+}
          \hlkwd{geom_point}\hlstd{()}
\end{alltt}
\end{kframe}

{\centering \includegraphics[width=.54\textwidth]{figure/pos-main-chunk-44-1} 

}



\end{knitrout}


\begin{knitrout}\footnotesize
\definecolor{shadecolor}{rgb}{0.969, 0.969, 0.969}\color{fgcolor}\begin{kframe}
\begin{alltt}
\hlkwd{try}\hlstd{(}\hlkwd{detach}\hlstd{(package}\hlopt{:}\hlstd{tidyverse))}
\hlkwd{try}\hlstd{(}\hlkwd{detach}\hlstd{(package}\hlopt{:}\hlstd{lubridate))}
\hlkwd{try}\hlstd{(}\hlkwd{detach}\hlstd{(package}\hlopt{:}\hlstd{tikzDevice))}
\hlkwd{try}\hlstd{(}\hlkwd{detach}\hlstd{(package}\hlopt{:}\hlstd{ggplot2))}
\end{alltt}
\begin{verbatim}
## Error : package 'ggplot2' is required by 'ggforce' so will not be detached
\end{verbatim}
\begin{alltt}
\hlkwd{try}\hlstd{(}\hlkwd{detach}\hlstd{(package}\hlopt{:}\hlstd{scales))}
\hlkwd{try}\hlstd{(}\hlkwd{detach}\hlstd{(package}\hlopt{:}\hlstd{learnrbook))}
\end{alltt}
\end{kframe}
\end{knitrout}



\chapter{Further reading about R}\label{chap:R:readings}

\begin{VF}
Before you become too entranced with gorgeous gadgets and mesmerizing video displays, let me remind you that information is not knowledge, knowledge is not wisdom, and wisdom is not foresight. Each grows out of the other, and we need them all.

\VA{Arthur C. Clarke}{}
\end{VF}

%\dictum[Arthur C. Clarke]{Before you become too entranced with gorgeous gadgets and mesmerizing video displays, let me remind you that information is not knowledge, knowledge is not wisdom, and wisdom is not foresight. Each grows out of the other, and we need them all.}\vskip2ex

\begin{warningbox}
  This list will be expanded and more importantly reorganized and short comments added for book or group of books.
\end{warningbox}

\section{Introductory texts}

\cite{Allerhand2011,Dalgaard2008,Zuur2009,Teetor2011,Peng2017,Paradis2005,Peng2016}

\section{Texts on specific aspects}

\cite{Chang2013,Fox2002,Fox2010,Faraway2004,Faraway2006,Everitt2011,Wickham2017}

\section{Advanced texts}

\cite{Xie2013,Chambers2016,Wickham2015,Wickham2014advanced,Wickham2016,Pinheiro2000,Murrell2011,Matloff2011,Ihaka1996,Venables2000}

\section{Texts for S/R wisdom}

\cite{Burns1998,Burns2011,Burns2012,Bentley1986,Bentley1988}

\backmatter

\printbibliography

\printindex

\printindex[rcatsidx]

\printindex[rindex]

\end{document}

\appendix

\chapter{Build information}

\begin{knitrout}\footnotesize
\definecolor{shadecolor}{rgb}{0.969, 0.969, 0.969}\color{fgcolor}\begin{kframe}
\begin{alltt}
\hlkwd{Sys.info}\hlstd{()}
\end{alltt}
\begin{verbatim}
##        sysname        release        version       nodename        machine 
##      "Windows"       "10 x64"  "build 18362"        "MUSTI"       "x86-64" 
##          login           user effective_user 
##       "aphalo"       "aphalo"       "aphalo"
\end{verbatim}
\end{kframe}
\end{knitrout}



\begin{knitrout}\footnotesize
\definecolor{shadecolor}{rgb}{0.969, 0.969, 0.969}\color{fgcolor}\begin{kframe}
\begin{alltt}
\hlkwd{sessionInfo}\hlstd{()}
\end{alltt}
\begin{verbatim}
## R version 3.6.0 (2019-04-26)
## Platform: x86_64-w64-mingw32/x64 (64-bit)
## Running under: Windows 10 x64 (build 18362)
## 
## Matrix products: default
## 
## locale:
## [1] LC_COLLATE=English_United Kingdom.1252 
## [2] LC_CTYPE=English_United Kingdom.1252   
## [3] LC_MONETARY=English_United Kingdom.1252
## [4] LC_NUMERIC=C                           
## [5] LC_TIME=English_United Kingdom.1252    
## 
## attached base packages:
## [1] tools     stats     graphics  grDevices utils     datasets  methods  
## [8] base     
## 
## other attached packages:
##  [1] forcats_0.4.0      dplyr_0.8.1        purrr_0.3.2       
##  [4] readr_1.3.1        tidyr_0.8.3        tibble_2.1.3      
##  [7] ggforce_0.2.2      ggbeeswarm_0.6.0   ggpmisc_0.3.1.9000
## [10] gginnards_0.0.2    ggrepel_0.8.1      ggplot2_3.2.0     
## [13] svglite_1.2.2      stringr_1.4.0      knitr_1.23        
## 
## loaded via a namespace (and not attached):
##  [1] nlme_3.1-139        sf_0.7-4            lubridate_1.7.4    
##  [4] RColorBrewer_1.1-2  httr_1.4.0          backports_1.1.4    
##  [7] R6_2.4.0            rpart_4.1-15        KernSmooth_2.23-15 
## [10] vipor_0.4.5         Hmisc_4.2-0         DBI_1.0.0          
## [13] lazyeval_0.2.2      colorspace_1.4-1    nnet_7.3-12        
## [16] withr_2.1.2         tidyselect_0.2.5    gridExtra_2.3      
## [19] compiler_3.6.0      cli_1.1.0           rvest_0.3.4        
## [22] formatR_1.7         htmlTable_1.13.1    xml2_1.2.0         
## [25] labeling_0.3        checkmate_1.9.3     scales_1.0.0       
## [28] hexbin_1.27.3       classInt_0.3-3      digest_0.6.19      
## [31] foreign_0.8-71      htmltools_0.3.6     base64enc_0.1-3    
## [34] pkgconfig_2.0.2     highr_0.8           htmlwidgets_1.3    
## [37] rlang_0.3.4         readxl_1.3.1        rstudioapi_0.10    
## [40] farver_1.1.0        generics_0.0.2      tikzDevice_0.12    
## [43] jsonlite_1.6        acepack_1.4.1       magrittr_1.5       
## [46] polynom_1.4-0       Formula_1.2-3       Matrix_1.2-17      
## [49] Rcpp_1.0.1          munsell_0.5.0       gdtools_0.1.9      
## [52] stringi_1.4.3       MASS_7.3-51.4       plyr_1.8.4         
## [55] grid_3.6.0          crayon_1.3.4        lattice_0.20-38    
## [58] haven_2.1.0         splines_3.6.0       hms_0.4.2          
## [61] magick_2.0          pillar_1.4.1        reshape2_1.4.3     
## [64] glue_1.3.1          evaluate_0.14       latticeExtra_0.6-28
## [67] data.table_1.12.2   modelr_0.1.4        tweenr_1.0.1       
## [70] cellranger_1.1.0    gtable_0.3.0        polyclip_1.10-0    
## [73] assertthat_0.2.1    xfun_0.7            broom_0.5.2        
## [76] tidyverse_1.2.1     e1071_1.7-2         filehash_2.4-2     
## [79] class_7.3-15        survival_2.44-1.1   viridisLite_0.3.0  
## [82] beeswarm_0.2.3      units_0.6-3         cluster_2.0.8      
## [85] learnrbook_0.0.2
\end{verbatim}
\end{kframe}
\end{knitrout}

%

\end{document}


