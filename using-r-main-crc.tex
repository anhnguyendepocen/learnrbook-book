\documentclass[krantz2]{krantz}\usepackage{knitr}%,ChapterTOCs

%\usepackage[utf8]{inputenc}
\usepackage{color}

\usepackage{polyglossia}
\setdefaultlanguage[variant = british, ordinalmonthday = false]{english}

%\usepackage{gitinfo2} % remember to setup Git hooks

\usepackage{hologo}

\usepackage{csquotes}

\usepackage{graphicx}
\DeclareGraphicsExtensions{.jpg,.pdf,.png}

\usepackage{animate}

%\usepackage{microtype}
\usepackage[style=authoryear-comp,giveninits,sortcites,maxcitenames=2,%
    mincitenames=1,maxbibnames=10,minbibnames=10,backref,uniquename=mininit,%
    uniquelist=minyear,sortgiveninits=true,backend=biber]{biblatex}%,refsection=chapter

\newcommand{\href}[2]{\emph{#2} (\url{#1})}

%\usepackage[unicode,hyperindex,bookmarks,pdfview=FitB,%backref,
%            pdftitle={Learn R ...as you learnt your mother tongue},%
%            pdfkeywords={R, statistics, data analysis, plotting},%
%            pdfsubject={R},%
%            pdfauthor={Pedro J. Aphalo}%
%            ]{hyperref}

%\hypersetup{colorlinks,breaklinks,
%             urlcolor=blue,
%             linkcolor=blue,
%             citecolor=blue,
%             filecolor=blue,
%             menucolor=blue}

\usepackage{framed}

\usepackage{abbrev}
\usepackage{usingr}

\usepackage{imakeidx}

% this is to reduce spacing above and below verbatim, which is used by knitr
% to show returned values
\usepackage{etoolbox}
\makeatletter
\preto{\@verbatim}{\topsep=-5pt \partopsep=-4pt \itemsep=-2pt}
\makeatother

%%% Adjust graphic design

% New float "example" and corresponding "list of examples"
%\DeclareNewTOC[type=example,types=examples,float,counterwithin=chapter]{loe}
%\DeclareNewTOC[name=Box,listname=List of Text Boxes, type=example,types=examples,float,counterwithin=chapter,%
%]{lotxb}

% changing the style of float captions
%\addtokomafont{caption}{\sffamily\small}
%\setkomafont{captionlabel}{\sffamily\bfseries}
%\setcapindent{0em}

% finetuning tocs
%\makeatletter
%\renewcommand*\l@figure{\@dottedtocline{1}{0em}{2.6em}}
%\renewcommand*\l@table{\@dottedtocline{1}{0em}{2.6em}}
%\renewcommand*\l@example{\@dottedtocline{1}{0em}{2.3em}}
%\renewcommand{\@pnumwidth}{2.66em}
%\makeatother
%
%% add pdf bookmarks to tocs
%\makeatletter
%\BeforeTOCHead{%
%  \cleardoublepage
%    \edef\@tempa{%
%      \noexpand\pdfbookmark[0]{\list@fname}{\@currext}%
%    }\@tempa
%}

\setcounter{topnumber}{3}
\setcounter{bottomnumber}{3}
\setcounter{totalnumber}{4}
\renewcommand{\topfraction}{0.90}
\renewcommand{\bottomfraction}{0.90}
\renewcommand{\textfraction}{0.10}
\renewcommand{\floatpagefraction}{0.70}
\renewcommand{\dbltopfraction}{0.90}
\renewcommand{\dblfloatpagefraction}{0.70}

\addbibresource{rbooks.bib}
\addbibresource{references.bib}

\makeindex[title=General index]
\makeindex[name=rindex,title=Alphabetic index of \Rlang names]
\makeindex[name=rcatsidx,title=Index of \Rlang names by category]
\IfFileExists{upquote.sty}{\usepackage{upquote}}{}
\begin{document}

% customize chapter format:
%\KOMAoption{headings}{twolinechapter}
%\renewcommand*\chapterformat{\thechapter\autodot\hspace{1em}}

% customize dictum format:
%\setkomafont{dictumtext}{\itshape\small}
%\setkomafont{dictumauthor}{\normalfont}
%\renewcommand*\dictumwidth{0.7\linewidth}
%\renewcommand*\dictumauthorformat[1]{--- #1}
%\renewcommand*\dictumrule{}

%\extratitle{\vspace*{2\baselineskip}%
%             {\Huge\textsf{\textbf{Learn R}\\ \textsl{\huge\ldots as you learnt your mother tongue}}}}

\title{\Huge{\fontseries{ub}\sffamily Learn R\\{\Large\ldots as you learnt your mother tongue}}}

%\subtitle{Git hash: \gitAbbrevHash; Git date: \gitAuthorIsoDate}

\author{Pedro J. Aphalo}

\date{Helsinki, \today}

%\publishers{Draft, 95\% done\\Available through \href{https://leanpub.com/learnr}{Leanpub}}

%\uppertitleback{\copyright\ 2001--2017 by Pedro J. Aphalo\\
%Licensed under one of the \href{http://creativecommons.org/licenses/}{Creative Commons licenses} as indicated, or when not explicitly indicated, under the \href{http://creativecommons.org/licenses/by-sa/4.0/}{CC BY-SA 4.0 license}.}
%
%\lowertitleback{Typeset with \href{http://www.latex-project.org/}{\hologo{XeLaTeX}}\ in Lucida Bright and \textsf{Lucida Sans} using the KOMA-Script book class.\\
%The manuscript was written using \href{http://www.r-project.org/}{R} with package knitr. The manuscript was edited in \href{http://www.winedt.com/}{WinEdt} and \href{http://www.rstudio.com/}{RStudio}.
%The source files for the whole book are available at \url{https://bitbucket.org/aphalo/using-r}.}

%\frontmatter

% knitr setup

















% \thispagestyle{empty}
% \titleLL
% \clearpage

\frontmatter

\maketitle

\newpage

%\begin{titlingpage}
%  \maketitle
%\titleLL
%\end{titlingpage}

\setcounter{page}{7} %previous pages will be reserved for frontmatter to be added in later.
\tableofcontents
%\chapter*{Foreword}
I am delighted to introduce the first book on Multimedia Data Mining.  When I came to know about this book project undertaken by two of the most active young researchers in the field, I was pleased that this book is coming in early stage of a field that will need it more than most fields do.  In most emerging research fields, a book can play a significant role in bringing some maturity to the field.  Research fields advance through research papers.  In research papers, however, only a limited perspective could be provided about the field, its application potential, and the techniques required and already developed in the field.  A book gives such a chance.  I liked the idea that there will be a book that will try to unify the field by bringing in disparate topics already available in several papers that are not easy to find and understand.  I was supportive of this book project even before I had seen any material on it.  The project was a brilliant and a bold idea by two active researchers.  Now that I have it on my screen, it appears to be even a better idea.  

Multimedia started gaining recognition in 1990s as a field.  Processing, storage, communication, and capture and display technologies had advanced enough that researchers and technologists started building approaches to combine information in multiple types of signals such as audio, images, video, and  text.  Multimedia computing and communication techniques recognize correlated information in multiple sources as well as insufficiency of information in any individual source.    By properly selecting sources to provide complementary information, such systems aspire, much like human perception system, to create a holistic picture of a situation using only partial information from separate sources.

Data mining is a direct outgrowth of progress in data storage and processing speeds.  When it became possible to store large volume of data and run different statistical computations to explore all possible and even unlikely correlations among data, the field of data mining was born.  Data mining allowed people to hypothesize relationships among data entities and explore support for those.  This field has been put to applications in many diverse domains and keeps getting more applications.  In fact many new fields are direct outgrowth of data mining and it is likely to become a powerful computational tool.\vadjust{\vfill\pagebreak}



\chapter*{Preface}

\begin{VF}
``Suppose that you want to teach the `cat' concept to a very young child. Do you explain that a cat is a relatively small, primarily carnivorous mammal with retractible claws, a distinctive sonic output, etc.? I'll bet not. You probably show the kid a lot of different cats, saying `kitty' each time, until it gets the idea. To put it more generally, generalizations are best made by abstraction from experience.''

\VA{R. P. Boas}{Can we make mathematics intelligible?}
\end{VF}

%\dictum[R. P. Boas (1981) Can we make mathematics intelligible?, \emph{American Mathematical Monthly} \textbf{88:} 727-731.]{"Suppose that you want to teach the `cat' concept to a very young child. Do you explain that a cat is a relatively small, primarily carnivorous mammal with retractible claws, a distinctive sonic output, etc.? I'll bet not. You probably show the kid a lot of different cats, saying `kitty' each time, until it gets the idea. To put it more generally, generalizations are best made by abstraction from experience."}


% Such pauses are not a miss use of our time. To learn a natural language we need to interact with other speakers, we need feedback. In the case of R, we can get feedback both from the outcomes from our utterances to the computer, and from other \Rlang users.}
\noindent
This book covers different aspects of the use of the \Rlang language. Chapters \ref{chap:R:introduction} to \ref{chap:R:functions} describe the \Rlang language itself. Later chapters describe extensions to the \Rlang language available through contributed \emph{packages}, the \emph{grammar of data} and the \emph{grammar of graphics}. In this book, explanations are concise but contain pointers to additional sources of information, so as to encourage the development of a routine of independent exploration. This is not an arbitrary decision, this is the normal \emph{modus operandi} of most of us who use \Rlang regularly for a variety of different problems. Some have called approaches like the one used here, ``learning the hard way'', but I would call it ``learning to be independent''.

I do not discuss in this book statistics or data analysis methods, I describe \Rlang as a language for data manipulation and display. The idea is for you to learn the \Rlang language in a way comparable to how children learn a language: they work-out what the rules are, simply by listening to people speak and trying to utter what they want to tell their parents. Of course, small children receive some guidance, but are not taught a prescriptive set of rules like when learning a second language at school. Instead of listening, you will read code and instead of speaking you will try to execute \Rlang  code statements on a computer---i.e. you will try your hand at using \Rlang to tell a computer what you want it to compute. I do provide explanations and guidance, but the idea of this book is for you to use the numerous examples to find-out by yourself the overall patterns and coding philosophy behind the \Rlang language. Instead of parents being the sound board for your first utterances in \Rlang, the computer will play this role. You will \emph{play} by modifying the examples, see how the computer responds, does \Rlang understand you or not? Using actively a language is the most efficient way of learning it. By using it, I mean actually reading, writing and running scripts or programs (copying and pasting, or typing ready-made examples from books or the internet does not qualify as using a language).

What is a language? A language is a system of communication. \Rlang as a language allows us to communicate with other members of the \Rlang community, and with computers. As most languages in active use, \Rlang evolves. New ``words'' and new ``constructs'' are incorporated into the language, and some earlier frequently used ones are relegated to the fringes of the corpus. I describe current usage and ``modisms'' of the \Rlang language in a way accessible to a readership unfamiliar with computer science but with some background in data analysis as used in Biology, Engineering, or Humanities.

When teaching I tend to lean towards challenging students rather than telling an over-simplified story. There are two reasons for this. First, I prefer as a student, and I learn best myself if the going is not too easy. Second, if I would hide the tricky bits of the \Rlang language, it would make readers' life much more difficult later on. You, will not remember all the details, nobody could. However, you most likely will remember in which situations you need to be careful or should check the details. So, I will expose you not only the usual cases, but also to several exceptions and counterintuitive features of the language. Reading this book will be about exploring a new world, this book aims to be a travel guide, but neither a traveler's account, nor a cookbook of \Rlang recipes.

Keep in mind that it is impossible to remember everything about \Rlang! The \Rlang language in a broad sense is vast because its capabilities can be expanded with independently developed packages. Learning to use \Rlang consists in learning the basics plus developing the skill of finding your way in \Rlang and its documentation.  In 2017 the number packages available in the Comprehensive \Rlang Archive Network (CRAN) broke the 10\,000 barrier. CRAN is the most important, but not only, public repository for \Rlang packages. How good a command of the \Rlang language and packages a user needs depends on the type activities to be carried out. This book attempts to train you in the use of the \Rlang language itself and of popular \Rlang language extensions for data manipulation and graphical display. Given the availability of numerous books on statistical analysis with \Rlang, in the present book I will cover only the bare minimum of this subject. The same is true for package development in \Rlang. This book seats in-between, aiming at teaching programming in-the-small: the use of \Rlang to automate the drudgery of data manipulation from raw data, through data exploration to the production of publication quality illustrations.

As with all ``rich'' languages there are many different ways of doing things in \Rlang, and there is in almost all cases no one-size-fits-all solution to a problem. There is always a compromise involved, usually between time spent by the user and processing time required in the computer. Many of the packages that are most popular nowadays did not exist when I started using \Rlang, and many of these packages make new approaches available. One could write many different \Rlang books with a given aim using substantially different ways of achieving the same results. In this book, I limit myself to packages that are currently popular and/or that I consider elegantly designed. I have in particular tried to limit myself to packages with similar design philosophies, especially in relation to their interfaces. What is elegant design, and in particular what is a friendly user interface depends strongly on each user's preferences and previous experience. Consequently, the contents of the book are strongly biased by my own preferences. I have tried to write examples in ways that execute fast without compromising readability. I encourage readers to take this book as a starting point for exploring the very many packages, styles and approaches which I have not described.

I will appreciate suggestions for further examples, notification of errors and unclear sections. Many of the examples here have been collected from diverse sources over many years and because of this not all sources are acknowledged. If you recognize any example as yours or someone else's please let me know so that I can add a proper acknowledgement. I warmly thank the students that over the years have asked the questions and posed the problems that have helped me write this text and correct the mistakes and voids of previous versions. I have also received help on on-line forums and in person from numerous people, learnt from archived e-mail list messages, blog posts, books, articles, tutorials, webinars, and by struggling to solve some new problems on my own. In many ways this text owes much more to people who are not authors than to myself. However, as I am the one who has written this version and decided what to include and exclude, as author, I take full responsibility for any errors and inaccuracies.

I have been using \Rlang since around 1998 or 1999, but I am still constantly learning new things about \Rlang itself and \Rlang packages. With time it has replaced in my work as a researcher and teacher several other pieces of software: \pgrmname{SPSS}, \pgrmname{Systat}, \pgrmname{Origin}, \pgrmname{Excel}, and it has become a central piece of the tool set I use for producing lecture slides, notes, books and even web pages. This is to say that it is the most useful piece of software and programming language I have ever learnt to use. Of course, in time it will be replaced by something better, but at the moment it is a key language to learn for anybody with a need to analyse and display data.

Why the title ``\emph{Learn R \ldots as you learnt your mother tongue}''? On one hand, because this book is based on exploration and practice. On the other hand, because you will be exposed to current usage and not spared the quirks of the language. When we use our mother tongue in everyday life we do not think about grammar rules or sentence structure, except for the trickier or unfamiliar situations. My aim is for this book to help you grow to use \Rlang in this same way.

\begin{framed}
\noindent\large%
\textbf{I encourage you to approach R, like a child approaches his or hers mother tongue when learning to speak:} Do not struggle, just play! If going gets difficult and frustrating, take a break! If you get a new insight, take a break to enjoy the victory!
\end{framed}

\newpage

\begin{framed}
\noindent
\textbf{Icons used to mark different content.} Throughout the book text boxes marked with icons present different types of information. First of all, we have \emph{playground} boxes indicated with \playicon\ which contain open-ended exercises---ideas and pieces of \Rlang code to play with at the \Rlang console. A few of these will require more time to grasp, and are indicated with \advplayicon. Boxes providing general information, usually not directly related to \Rlang as a language, are indicated with \infoicon. Some boxes highlighted with \ilAttention\ give important bits of information that must be remembered when using \Rlang---i.e.\ explain some unusual feature of the language. Finally, some boxes indicated by \ilAdvanced\ give in depth explanations, that may require you to spend time thinking, which en general can be skipped on first reading, but to which you should return at a later, and peaceful, time with a cup of coffee or tea.
\end{framed}
\newpage

%\newpage
%\begin{infobox}
%\noindent
%\textbf{Status as of 2016-11-23.} I have updated the manuscript to track package updates since the previous version uploaded six months ago, and added several examples of the new functionality added to packages \ggpmisc, \ggrepel, and \ggplot. I have written new sections on packages \viridis, \pkgname{gganimate}, \pkgname{ggstance}, \pkgname{ggbiplot}, \pkgname{ggforce}, \pkgname{ggtern} and \pkgname{ggalt}. Some of these sections are to be expanded, and additional sections are planned for other recently released packages.
%
%With respect to the chapter \textit{Storing and manipulating data with R} I have put it on hold, except for the introduction, until I can see a soon to be published book covering the same subject. Hadley Wickham has named the set of tools developed by him and his collaborators as \textit{tidyverse} to be described in the book titled \textit{R for Data Science} by Grolemund and Wickham (O'Reilly).
%
%An important update to \ggplot was released last week, and it includes changes to the behavior of some existing functions, specially faceting has become extensible through other packages. Several of the new facilities are described in the updated text and code included in this book and this pdf has been generated with up-to-date version of \ggplot and packages as available today from CRAN, except for \pkgname{ggtern} which was downloaded from Bitbucket minutes ago.
%
%The present update adds about 100 pages to the previous versions. I expect to upload a new update to this manuscript in one or two months time.
%
%\textbf{Status as of 2017-01-17.} Added ``playground'' exercises to the chapter describing \ggplot, and converted some of the examples earlier part of the main text into these playground items. Added icons to help readers quickly distinguish playground sections (\textcolor{blue}{\noticestd{"0055}}), information sections (\textcolor{blue}{\modpicts{"003D}}), warnings about things one needs to be specially aware of (\colorbox{yellow}{\typicons{"E136}}) and boxes with more advanced content that may require longer time/more effort to grasp (\typicons{"E04E}). Added to the sections \code{scales} and examples in the \ggplot chapter details about the use of colors in \Rlang and \ggplot2. Removed some redundant examples, and updated the section on \code{plotmath}. Added terms to the alphabetical index. Increased line-spacing to avoid uneven spacing with inline code bits.
%
%\textbf{Status as of 2017-02-09.} Wrote section on ggplot2 themes, and on using system- and Google fonts in ggpplots with the help of package \pkgname{showtext}. Expanded section on \ggplot's \code{annotation}, and revised some sections in the ``R scripts and Programming'' chapter. Started writing the data chapter. Wrote draft on writing and reading text files. Several other smaller edits to text and a few new examples.
%
%\textbf{Status as of 2017-02-14.} Wrote sections on reading and writing MS-Excel files, files from statistical programs such as SPSS, SyStat, etc., and NetCDF files. Also wrote sections on using URLs to directly read data, and on reading HTML and XML files directly, as well on using JSON to retrieve measured/logged data from IoT (internet of things) and similar intelligent physical sensors, micro-controller boards and sensor hubs with network access.
%
%\textbf{Status as of 2017-03-25.} Revised and expanded the chapter on plotting maps, adding a section on the manipulation and plotting of image data. Revised and expanded the chapter on extensions to \pkgname{ggplot2}, so that there are no longer empty sections. Wrote short chapter ``If and when \Rlang needs help''. Revised and expanded the ``Introduction'' chapter. Added index entries, and additional citations to literature.
%
%\textbf{Status as of 2017-04-04.} Revised and expanded the chapter on using \Rpgrm as a calculator. Revised and expanded the ``Scripts'' chapter. Minor edits to ``Functions'' chapter. Continued writing chapter on data, writing a section on \Rlang native apply functions and added preliminary text for a pipes and tees section. Write intro to `tidyverse' and grammar of data manipulation. Added index entries, and a few additional citations to the literature. Spell checking.
%
%\textbf{Status as of 2017-04-08.} Completed writing first draft of chapter on data, writing all the previously missing sections on the ``grammar of data manipulation''. Wrote two extended examples in the same chapter. Add table listing several extensions to \pkgname{ggplot2} not described in the book.
%
%\textbf{Status as of 2017-04-13.} Revised all chapters correcting some spelling mistakes, adding some explanatory text and indexing all functions and operators used. Thoroughly revised the Introduction chapter and the Preface. Expanded section on bar plots (now bar and column plots). Revised section on tile plots. Expanded section on factors in chapter 2, adding examples of reordering of factor labels, and making clearer the difference between the labels of the levels and the levels themselves.
%
%\textbf{Status as of 2017-04-29.} Tested with R 3.4.0. Package \pkgname{gganimate} needs to be installed from Github as the updated version is not yet in CRAN. Function \code{gg\_animate()} has been renamed \code{gganimate().}
%
%\textbf{Status as of 2017-05-14.} Submitted package \pkgname{learnrbook} to CRAN. Revised code in the book
%to use this new package. Small fixes after more testing. Added examples of plotting and labeling based on fits with \code{method = "nls"}, including use of the new \code{ggpmisc::stat\_fit\_tidy()}.
%
%\textbf{Status as of 2017-06-11.} Added sections on R-code bench marking and profiling for performance optimization. Added also an example of explicit compilation of a function defined in the R language. Added section on functions \code{assign()}, \code{get()} and \code{mget()}.
%
%\textbf{Status as of 2017-08-12.} Various edits to all chapters. Expanded section on \pkgname{ggpmisc} to include the new functionality added in version 0.2.15.9002: \code{geom\_table} and \code{stat\_fit\_tb}. Added section on package \pkgname{ggbeeswarm}. Added sections on packages \pkgname{magick} and on using \pgrmname{ImageJ} from \Rpgrm. Improved indexing and cross references.
%
%\textbf{Status as of 2017-10-25.} Edited the chapter on using R as a calculator, adding examples on insertion and deletion of members of lists and vectors, and also of use of \code{gl()} and \code{reorder()}. Edited sections on scale limits and added new section on coordinate limits to explain more thoroughly their differences and uses in chapter on plotting with \pkgname{ggplot2}. Added a section on package \pkgname{ggsignif} to the chapter on extensions to \pkgname{ggplot2}. Expanded section on \pkgname{ggpmisc} in the same chapter describing new functionality added in version 0.2.16.
%\pkgname{ggplo2} $>=$ 2.2.1.9000 is required by the current development version of \pkgname{ggpmisc}.
%
%\textbf{Status as of 2017-10-30.}  Add section on using pipes with \code{ggplot()} and layers.
%\end{infobox} 
\listoffigures
\listoftables
%%%\twocolumn
\chapter*{Contributors}

\begin{multicols}{2}
\contributor{Michael Aftosmis}{NASA Ames Research Center}{Moffett Field, California}

\contributor{Pratul K. Agarwal}{Oak Ridge National Laboratory}{Oak Ridge, Tennessee}

\contributor{Sadaf R. Alam}{Oak Ridge National Laboratory}{Oak Ridge, Tennessee}

\contributor{Gabrielle Allen}{Louisiana State University}{Baton Rouge, Louisiana}

\contributor{Martin Sandve Aln{\ae}s}{Simula Research Laboratory and University of Oslo, Norway}{Norway}

\contributor{Steven F. Ashby} {Lawrence Livermore National Laboratory}{Livermore, California}

\contributor{David A. Bader} {Georgia Institute of Technology}{Atlanta, Georgia}

\contributor{Benjamin Bergen} {Los Alamos National Laboratory}{Los Alamos, New Mexico}

\contributor{Jonathan W. Berry} {Sandia National Laboratories}{Albuquerque, New Mexico}

\contributor{Martin Berzins}{University of Utah}{Salt Lake City, Utah}

\contributor{Abhinav Bhatele}{University of Illinois}{Urbana-Champaign, Illinois}

\contributor{Christian Bischof} {RWTH Aachen University}{Germany}

\contributor{Rupak Biswas} {NASA Ames Research Center}{Moffett Field, California}\vspace*{5pt}

\contributor{Eric Bohm} {University of Illinois}{Urbana-Champaign, Illinois}\vspace*{5pt}

\contributor{James Bordner} {University of California, San Diego}{San Diego, California}\vspace*{5pt}

\contributor{George Bosilca} {University of Tennessee}{Knoxville, Tennessee}\vspace*{5pt}

\contributor{Greg L. Bryan} {Columbia University}{New York, New York}\vspace*{5pt}

\contributor{Marian Bubak} {AGH University of Science and Technology}{
Krak{\'o}w, Poland}\vspace*{5pt}

\contributor{Andrew Canning}{Lawrence Berkeley National
Laboratory}{Berkeley, California}

\contributor{Jonathan Carter} {Lawrence Berkeley National
Laboratory}{Berkeley, California}

\contributor{Zizhong Chen} {Jacksonville State University}{Jacksonville,
Alabama}

\contributor{Joseph R. Crobak} {Rutgers, The State University of New
Jersey}{Piscataway, New Jersey}

\contributor{Roxana E. Diaconescu} {Yahoo! Inc.}{Burbank, California}

\contributor{Peter Diener}
{Louisiana State University}{Baton Rouge, Louisiana}

\contributor{Jack J. Dongarra} {University of Tennessee, Knoxville, 
Oak Ridge National Laboratory, and}{University of Manchester}

\contributor{John B. Drake} {Oak Ridge National Laboratory}{Oak Ridge,
Tennessee}

\contributor{Kelvin K. Droegemeier} {University of Oklahoma}{Norman,
Oklahoma}

\contributor{St{\'e}phane Ethier} {Princeton University}{Princeton, New
Jersey}

\contributor{Christoph Freundl}
{Friedrich--Alexander--Universit{\"a}t}{Erlangen, Germany}

\contributor{Karl F{\"u}rlinger} {University of Tennessee}{Knoxville,
Tennessee}

\contributor{Al Geist} {Oak Ridge National Laboratory}{Oak Ridge,
Tennessee}

\contributor{Michael Gerndt} {Technische Universit{\"a}t
M{\"u}nchen}{Munich, Germany}

\contributor{Tom Goodale}
{Louisiana State University}{Baton Rouge, Louisiana}

\contributor{Tobias Gradl}
{Friedrich--Alexander--Universit{\"a}t}{Erlangen, Germany}

\contributor{William D. Gropp} {Argonne National Laboratory}{Argonne,
Illinois}

\contributor{Robert Harkness} {University of California, San
Diego}{San Diego, California}

\contributor{Albert Hartono} {Ohio State University}{Columbus, Ohio}

\contributor{Thomas C. Henderson} {University of Utah}{Salt Lake City,
Utah}

\contributor{Bruce A. Hendrickson} {Sandia National
Laboratories}{Albuquerque, New Mexico}

\contributor{Alfons G. Hoekstra} {University of Amsterdam}{Amsterdam,
The Netherlands}

\contributor{Philip W. Jones} {Los Alamos National Laboratory}{Los
Alamos, New Mexico}

\contributor{Laxmikant Kal{\'e}} {University of
Illinois}{Urbana-Champaign, Illinois}

\contributor{Shoaib Kamil} {Lawrence Berkeley National
Laboratory}{Berkeley, California}

\contributor{Cetin Kiris} {NASA Ames Research Center}{Moffett Field,
California}

\contributor{Uwe K{\"u}ster} {University of Stuttgart}{Stuttgart,
Germany}

\contributor{Julien Langou} {University of Colorado}{Denver, Colorado}

\contributor{Hans Petter Langtangen}
{Simula Research Laboratory and}{University of Oslo, Norway}

\contributor{Michael Lijewski} {Lawrence Berkeley National
Laboratory}{Berkeley, California}

\contributor{Anders Logg}
{Simula Research Laboratory and}{University of Oslo, Norway}

\contributor{Justin Luitjens} {University of Utah}{Salt Lake City, Utah}

\contributor{Kamesh Madduri} {Georgia Institute of Technology}{Atlanta,
Georgia}

\contributor{Kent-Andre Mardal}
{Simula Research Laboratory and}{University of Oslo, Norway}

\contributor{Satoshi Matsuoka} {Tokyo Institute of Technology}{Tokyo,
Japan}

\contributor{John M. May} {Lawrence Livermore National
Laboratory}{Livermore, California}

\contributor{Celso L. Mendes} {University of Illinois}{Urbana-Champaign,
Illinois}

\contributor{Dieter an Mey} {RWTH Aachen University}{Germany}

\contributor{Tetsu Narumi} {Keio University}{Japan}

\contributor{Michael L. Norman} {University of California, San
Diego}{San Diego, California}

\contributor{Boyana Norris} {Argonne National Laboratory}{Argonne,
Illinois}

\contributor{Yousuke Ohno} {Institute of Physical and Chemical Research
(RIKEN)}{Kanagawa, Japan}

\contributor{Leonid Oliker} {Lawrence Berkeley National
Laboratory}{Berkeley, California}

\contributor{Brian O'Shea} {Los Alamos National Laboratory}{Los Alamos,
New Mexico}

\contributor{Christian D. Ott}
{University of Arizona}{Tucson, Arizona}

\contributor{James C. Phillips} {University of
Illinois}{Urbana-Champaign, Illinois}

\contributor{Simon Portegies Zwart} {University of
Amsterdam,}{Amsterdam, The Netherlands}

\contributor{Thomas Radke}
{Albert-Einstein-Institut}{Golm, Germany}

\contributor{Michael Resch} {University of Stuttgart}{Stuttgart,
Germany}

\contributor{Daniel Reynolds} {University of California, San Diego}{San
Diego, California}

\contributor{Ulrich R{\"u}de}
{Friedrich--Alexander--Universit{\"a}t}{Erlangen, Germany}

\contributor{Samuel Sarholz}
{RWTH Aachen University}{Germany}

\contributor{Erik Schnetter}
{Louisiana State University}{Baton Rouge, Louisiana}

\contributor{Klaus Schulten} {University of Illinois}{Urbana-Champaign,
Illinois}

\contributor{Edward Seidel}
{Louisiana State University}{Baton Rouge, Louisiana}

\contributor{John Shalf} {Lawrence Berkeley National
Laboratory}{Berkeley, California}

\contributor{Bo-Wen Shen} {NASA Goddard Space Flight Center}{Greenbelt,
Maryland}

\contributor{Ola Skavhaug}
{Simula Research Laboratory and}{University of Oslo, Norway}

\contributor{Peter M.A. Sloot} {University of Amsterdam}{Amsterdam, The
Netherlands}

\contributor{Erich Strohmaier} {Lawrence Berkeley National
Laboratory}{Berkeley, California}

\contributor{Makoto Taiji} {Institute of Physical and Chemical Research
(RIKEN)}{Kanagawa, Japan}

\contributor{Christian Terboven}
{RWTH Aachen University,}{Germany}

\contributor{Mariana Vertenstein} {National Center for Atmospheric
Research}{Boulder, Colorado}

\contributor{Rick Wagner} {University of California, San Diego}{San
Diego, California}

\contributor{Daniel Weber} {University of Oklahoma}{Norman, Oklahoma}

\contributor{James B. White, III} {Oak Ridge National Laboratory}{Oak
Ridge, Tennessee}

\contributor{Terry Wilmarth} {University of Illinois}{Urbana-Champaign,
Illinois}

\end{multicols}
%\chapter*{Symbols}
\begin{symbollist}{000000}
\symbolentry{$\alpha$}{To solve the generator maintenance scheduling, in the  past, several mathematical techniques have  been applied.}
\symbolentry{$\sigma^2$}{These include integer programming, integer linear programming, dynamic programming, branch and bound etc.}
\symbolentry{$\sum$}{Several heuristic search algorithms have also been developed. In recent years expert systems,}
\symbolentry{$abc$}{fuzzy approaches, simulated annealing and genetic algorithms have also been tested.}
\symbolentry{$\theta\sqrt{abc}$}{This paper presents a survey of the literature}
\symbolentry{$\zeta$}{ over the past fifteen years in the generator}
\symbolentry{$\partial$}{maintenance scheduling. The objective is to}
\symbolentry{sdf}{present a clear picture of the available recent literature}
\symbolentry{ewq}{of the problem, the constraints and the other aspects of}
\symbolentry{bvcn}{the generator maintenance schedule.}
\end{symbollist}

\mainmatter





% !Rnw root = appendix.main.Rnw



\chapter{The R language: ``words'' and ``sentences''}\label{chap:R:as:calc}

\begin{VF}
The desire to economize time and mental effort in arithmetical computations, and to eliminate human liability to error, is probably as old as the science of arithmetic itself.

\VA{Howard Aiken}{Proposed automatic calculating machine}
\end{VF}

%\dictum[Howard Aiken, \emph{Proposed automatic calculating machine}, presented to IBM in 1937]{The desire to economize time and mental effort in arithmetical computations, and to eliminate human liability to error, is probably as old as the science of arithmetic itself.}\vskip2ex

\section{Aims of this chapter}

In my experience, for those not familiar with computer programming languages, the best first step in learning the \Rlang language is to use it interactively by issuing textual commands at the \emph{console}. This will teach not only the syntax and grammar rules, but also give a glimpse at the advantages and flexibility of this approach to data analysis.

In the first part of the chapter we will use \Rlang to do everyday calculations that should be so easy and familiar that you will not need to think about the operations themselves. This easy start will give you a chance to focus on learning how to issue textual commands at the command prompt.

Later in the chapter, you will gradually need to focus more on the \Rlang language and its grammar and less on how commands are entered. By the end of the chapter you will be familiar with most of the kinds of ``words'' used in the \Rlang language and you will be able to write simple ``sentences'' in \Rlang.

Along the chapter, I will show occasionally the equivalent of the \Rlang code in mathematical notation. If you are not familiar with the mathematical notation, you can safely ignore it, as long as you understand the \Rlang code.

\section{Natural and computer languages}
\index{languages!natural and computer}
Computer languages have strict rules and interpreters and compilers are unforgiving about errors. They will issue error messages, but in contrast to human readers or listeners, will not guess your intentions and continue. However, computer languages have a much smaller set of words than natural languages, such as English. If you are new to computer programming, understanding the parallels between computer and natural languages may be useful.

One can think of constant values and variables (values stored under a name) as nouns and of operators and functions as verbs. A complete command, or statement, is the equivalent of a natural language sentence: ``a comprehensible utterance''. The simple statement \code{a + 1} has three components: \code{a}, a variable, \code{+}, an operator and \code{1} a constant. The statement \code{sqrt(4)} has two components, a function \code{sqrt()} and a numerical constant \code{4}. We say that ``to compute $\sqrt{4}$ we \emph{call} \code{sqrt()} with \code{4} as its \emph{argument}''.

In later chapters you will learn how to write compound statements, the equivalent of natural-language paragraphs, and scripts, the equivalent of essays. You will also learn how to define new verbs, user defined functions and operators, and new nouns, user defined classes.

\section{Numeric values and arithmetic}
\index{classes and modes!numeric, integer, double|(}\index{numbers and their arithmetic|(}\qRclass{numeric}\index{math operators}\index{math functions}\index{numeric values}\qRoperator{+}\qRoperator{-}\qRoperator{*}\qRoperator{/}
When working in \Rlang with arithmetic expressions, the normal mathematical precedence rules are respected, but parentheses can be used to alter this order. Parentheses can be nested, but in contrast to the usual practice in mathematics, the same parenthesis symbol is used at all nesting levels. It must also be remembered that trigonometric \Rlang functions interpret numeric values representing angles passed as arguments as being expressed in radians.

The equivalent of the math expression\qRfunction{exp()}\qRfunction{sin()}\qRconst{pi}
$$
\frac{3 + e^2}{\sin \pi}
$$
is

\begin{knitrout}\footnotesize
\definecolor{shadecolor}{rgb}{0.969, 0.969, 0.969}\color{fgcolor}\begin{kframe}
\begin{alltt}
\hlstd{(}\hlnum{3} \hlopt{+} \hlkwd{exp}\hlstd{(}\hlnum{2}\hlstd{))} \hlopt{/} \hlkwd{sin}\hlstd{(pi)}
\end{alltt}
\begin{verbatim}
## [1] 8.483588e+16
\end{verbatim}
\end{kframe}
\end{knitrout}

It can be seen above that mathematical constants and functions are part of the \Rlang language. One thing to remember when translating complex fractions as above into \Rlang code, is that in arithmetic expressions the bar of the fraction generates a grouping that alters the normal precedence of operations. In contrast, in an \Rlang expression the grouping must be explicitly signaled with additional parentheses.

If you are in doubt about how precedence rules work, you can add parentheses to make sure the order of computations is the one you intend. Redundant parentheses have no effect.

\begin{knitrout}\footnotesize
\definecolor{shadecolor}{rgb}{0.969, 0.969, 0.969}\color{fgcolor}\begin{kframe}
\begin{alltt}
\hlnum{1} \hlopt{+} \hlnum{2} \hlopt{*} \hlnum{3}
\end{alltt}
\begin{verbatim}
## [1] 7
\end{verbatim}
\begin{alltt}
\hlnum{1} \hlopt{+} \hlstd{(}\hlnum{2} \hlopt{*} \hlnum{3}\hlstd{)}
\end{alltt}
\begin{verbatim}
## [1] 7
\end{verbatim}
\begin{alltt}
\hlstd{(}\hlnum{1} \hlopt{+} \hlnum{2}\hlstd{)} \hlopt{*} \hlnum{3}
\end{alltt}
\begin{verbatim}
## [1] 9
\end{verbatim}
\end{kframe}
\end{knitrout}

The number of opening (left side) and closing (right side) parentheses must be balanced, and they must be located so that each enclosed term is a valid mathematical expression. For example while \code{(1 + 2) * 3} is valid, \code{(1 +) 2 * 3} is a syntax error as \code{1 +} is incomplete and cannot be calculated.

\begin{playground}
Here results are not shown. These are examples for you to type at the command prompt. In general you should not skip them, as in many cases, as with the statements highlighted with comments in the code chunk below, they have something to teach or demonstrate. You are also strongly encouraged to \emph{play}, in other words, create new variations of the examples and execute them to explore how \Rlang works.\qRfunction{sqrt()}\qRfunction{sin()}\qRfunction{log(), log10(), log2()}\qRfunction{exp()}

\begin{knitrout}\footnotesize
\definecolor{shadecolor}{rgb}{0.969, 0.969, 0.969}\color{fgcolor}\begin{kframe}
\begin{alltt}
\hlnum{1} \hlopt{+} \hlnum{1}
\hlnum{2} \hlopt{*} \hlnum{2}
\hlnum{2} \hlopt{+} \hlnum{10} \hlopt{/} \hlnum{5}
\hlstd{(}\hlnum{2} \hlopt{+} \hlnum{10}\hlstd{)} \hlopt{/} \hlnum{5}
\hlnum{10}\hlopt{^}\hlnum{2} \hlopt{+} \hlnum{1}
\hlkwd{sqrt}\hlstd{(}\hlnum{9}\hlstd{)}
\hlstd{pi} \hlcom{# whole precision not shown when printing}
\hlkwd{print}\hlstd{(pi,} \hlkwc{digits} \hlstd{=} \hlnum{22}\hlstd{)}
\hlkwd{sin}\hlstd{(pi)} \hlcom{# oops! Read on for explanation.}
\hlkwd{log}\hlstd{(}\hlnum{100}\hlstd{)}
\hlkwd{log10}\hlstd{(}\hlnum{100}\hlstd{)}
\hlkwd{log2}\hlstd{(}\hlnum{8}\hlstd{)}
\hlkwd{exp}\hlstd{(}\hlnum{1}\hlstd{)}
\end{alltt}
\end{kframe}
\end{knitrout}

\end{playground}

Variables\index{variables}\index{assignment} are used to store values. After we \emph{assign} a value to variable, we can use the name of the variable in place of the stored value. The `usual' assignment operator is \Roperator{<-}. In \Rlang all names, including variable names, are case sensitive. Variables \code{a} and \code{A} are two different variables. Variable names can be quite long, but usually it is not a good idea to use very long names. Here I am using very short names, something that is usually also a very bad idea. However, in the examples in this chapter where the stored values have no connection to the real world, simple names emphasize their abstract nature.

\begin{knitrout}\footnotesize
\definecolor{shadecolor}{rgb}{0.969, 0.969, 0.969}\color{fgcolor}\begin{kframe}
\begin{alltt}
\hlstd{a} \hlkwb{<-} \hlnum{1}
\hlstd{a} \hlopt{+} \hlnum{1}
\end{alltt}
\begin{verbatim}
## [1] 2
\end{verbatim}
\begin{alltt}
\hlstd{a}
\end{alltt}
\begin{verbatim}
## [1] 1
\end{verbatim}
\begin{alltt}
\hlstd{b} \hlkwb{<-} \hlnum{10}
\hlstd{b} \hlkwb{<-} \hlstd{a} \hlopt{+} \hlstd{b}
\hlstd{b}
\end{alltt}
\begin{verbatim}
## [1] 11
\end{verbatim}
\begin{alltt}
\hlnum{3e-2} \hlopt{*} \hlnum{2.0}
\end{alltt}
\begin{verbatim}
## [1] 0.06
\end{verbatim}
\end{kframe}
\end{knitrout}

Entering the name of a variable \emph{at the R console} implicitly calls function \code{print()} displaying the stored value on the console. The same applies to any other statement entered \emph{at the R console}: \code{print()} is implicitly called with the result of executing the statement as its argument.

\begin{knitrout}\footnotesize
\definecolor{shadecolor}{rgb}{0.969, 0.969, 0.969}\color{fgcolor}\begin{kframe}
\begin{alltt}
\hlstd{a}
\end{alltt}
\begin{verbatim}
## [1] 1
\end{verbatim}
\begin{alltt}
\hlkwd{print}\hlstd{(a)}
\end{alltt}
\begin{verbatim}
## [1] 1
\end{verbatim}
\begin{alltt}
\hlstd{a} \hlopt{+} \hlnum{1}
\end{alltt}
\begin{verbatim}
## [1] 2
\end{verbatim}
\begin{alltt}
\hlkwd{print}\hlstd{(a} \hlopt{+} \hlnum{1}\hlstd{)}
\end{alltt}
\begin{verbatim}
## [1] 2
\end{verbatim}
\end{kframe}
\end{knitrout}
\begin{playground}
There are some syntactically legal statements that are not very frequently used, but you should be aware that they are valid, as they will not trigger error messages, and may surprise you. The most important thing is to write code consistently. The `backwards' assignment operator \Roperator{->} and resulting code like \code{1 -> a}\index{assignment!leftwise} are valid but less frequently used. The use of the equals sign (\Roperator{=}) for assignment in place of \Roperator{<-} although valid is discouraged. Chaining\index{assignment!chaining} assignments as in the first statement below can be used to signal to the human reader that \code{a}, \code{b} and \code{c} are being assigned the same value.

\begin{knitrout}\footnotesize
\definecolor{shadecolor}{rgb}{0.969, 0.969, 0.969}\color{fgcolor}\begin{kframe}
\begin{alltt}
\hlstd{a} \hlkwb{<-} \hlstd{b} \hlkwb{<-} \hlstd{c} \hlkwb{<-} \hlnum{0.0}
\hlstd{a}
\hlstd{b}
\hlstd{c}
\hlnum{1} \hlkwb{->} \hlstd{a}
\hlstd{a}
\hlstd{a} \hlkwb{=} \hlnum{3}
\hlstd{a}
\end{alltt}
\end{kframe}
\end{knitrout}

\end{playground}

\begin{explainbox}
Here\index{mode of an object} I very briefly introduce the concept of \emph{mode} of an \Rlang object. In the case of \Rlang, numbers, belong to mode \Rclass{numeric}. We can query if the mode of an object is \Rclass{numeric} with function \Rfunction{is.numeric()}.

\begin{knitrout}\footnotesize
\definecolor{shadecolor}{rgb}{0.969, 0.969, 0.969}\color{fgcolor}\begin{kframe}
\begin{alltt}
\hlkwd{mode}\hlstd{(}\hlnum{1}\hlstd{)}
\end{alltt}
\begin{verbatim}
## [1] "numeric"
\end{verbatim}
\begin{alltt}
\hlstd{a} \hlkwb{<-} \hlnum{1}
\hlkwd{mode}\hlstd{(a)}
\end{alltt}
\begin{verbatim}
## [1] "numeric"
\end{verbatim}
\begin{alltt}
\hlkwd{is.numeric}\hlstd{(a)}
\end{alltt}
\begin{verbatim}
## [1] TRUE
\end{verbatim}
\end{kframe}
\end{knitrout}

One can think informally of a \emph{mode}, as a ``type'' or ``kind'' of objects. Constants like \code{1} or variables such as \code{a} in the code chunk above, belong to, or have a mode, that indicates that they are numbers. Other modes that we will use later in the present chapter are \Rclass{logical} and \Rclass{character} (We will discuss the concepts of \emph{mode} and \emph{class}, as used in \Rlang, in section \ref{sec:rlang:mode} on page \pageref{sec:rlang:mode}).

As in computers numbers can be stored in different formats requiring different amounts of memory, most computing languages implement several different types of numbers. In most cases \Rpgrm's \Rfunction{numeric()} can be used everywhere where a number is expected. In some cases it can be more efficient to explicitly indicate whether we will store or operate on integer numbers, in which case we can use class \Rclass{integer}, with integer constants indicated with a trailing capital `L', as in  \code{32L}.

Real numbers are a mathematical abstraction, and do not have an exact equivalent in computers. Instead of Real numbers, computers store and operate on numbers that are restricted to a broad but finite range values and have a finite resolution. They are called, \emph{floats} (or \emph{floating-point} numbers), in \Rlang they go by the name of \Rclass{double} and can be created with the constructor \Rfunction{double()}.

\begin{knitrout}\footnotesize
\definecolor{shadecolor}{rgb}{0.969, 0.969, 0.969}\color{fgcolor}\begin{kframe}
\begin{alltt}
\hlkwd{is.numeric}\hlstd{(}\hlnum{1L}\hlstd{)}
\end{alltt}
\begin{verbatim}
## [1] TRUE
\end{verbatim}
\begin{alltt}
\hlkwd{is.integer}\hlstd{(}\hlnum{1L}\hlstd{)}
\end{alltt}
\begin{verbatim}
## [1] TRUE
\end{verbatim}
\begin{alltt}
\hlkwd{is.double}\hlstd{(}\hlnum{1L}\hlstd{)}
\end{alltt}
\begin{verbatim}
## [1] FALSE
\end{verbatim}
\end{kframe}
\end{knitrout}

The name \code{double} originates from the \Clang language, in which there are different types of floats available. With the name \code{double} used to mean ``double-precision floating-point numbers''. Similarly, the use of \code{L} stems the \texttt{long} type in \Clang, meaning ``long integer numbers''.
\end{explainbox}

Numeric variables can contain more than one value. Even single numbers are in \Rlang \Rclass{vector}s of length one. We will later see why this is important. As you have seen above, the results of calculations were printed preceded with \code{[1]}. This is the index or position in the vector of the first number (or other value) displayed at the head of the current line.

One can use \Rmethod{c()} `concatenate' to create a vector from other vectors, including vectors of length 1, such as the \code{numeric} constants in the statements below.

\begin{knitrout}\footnotesize
\definecolor{shadecolor}{rgb}{0.969, 0.969, 0.969}\color{fgcolor}\begin{kframe}
\begin{alltt}
\hlstd{a} \hlkwb{<-} \hlkwd{c}\hlstd{(}\hlnum{3}\hlstd{,} \hlnum{1}\hlstd{,} \hlnum{2}\hlstd{)}
\hlstd{a}
\end{alltt}
\begin{verbatim}
## [1] 3 1 2
\end{verbatim}
\begin{alltt}
\hlstd{b} \hlkwb{<-} \hlkwd{c}\hlstd{(}\hlnum{4}\hlstd{,} \hlnum{5}\hlstd{,} \hlnum{0}\hlstd{)}
\hlstd{b}
\end{alltt}
\begin{verbatim}
## [1] 4 5 0
\end{verbatim}
\begin{alltt}
\hlstd{c} \hlkwb{<-} \hlkwd{c}\hlstd{(a, b)}
\hlstd{c}
\end{alltt}
\begin{verbatim}
## [1] 3 1 2 4 5 0
\end{verbatim}
\begin{alltt}
\hlstd{d} \hlkwb{<-} \hlkwd{c}\hlstd{(b, a)}
\hlstd{d}
\end{alltt}
\begin{verbatim}
## [1] 4 5 0 3 1 2
\end{verbatim}
\end{kframe}
\end{knitrout}

Method \code{c()} accepts as argument two or more vectors and concatenates them, one after another. Quite frequently we may need to insert one vector in the middle of another. For this operation, \code{c()} is not useful by itself. One could use indexing combined with \code{c()}, but this is not needed as \Rlang provides a function capable of directly doing this operation. Although it can be used to ``insert'' values it is named \code{append()}, and by default, it indeed appends one vector at the end of another.

\begin{knitrout}\footnotesize
\definecolor{shadecolor}{rgb}{0.969, 0.969, 0.969}\color{fgcolor}\begin{kframe}
\begin{alltt}
\hlkwd{append}\hlstd{(a, b)}
\end{alltt}
\begin{verbatim}
## [1] 3 1 2 4 5 0
\end{verbatim}
\end{kframe}
\end{knitrout}

The output above is the same as for \code{c(a, b)}, however, \Rfunction{append()} accepts as argument an index position after which to ``append'' its second argument. This results in an \emph{insert} operation when the index points at any position different from the end of the vector.

\begin{knitrout}\footnotesize
\definecolor{shadecolor}{rgb}{0.969, 0.969, 0.969}\color{fgcolor}\begin{kframe}
\begin{alltt}
\hlkwd{append}\hlstd{(a,} \hlkwc{values} \hlstd{= b,} \hlkwc{after} \hlstd{=} \hlnum{2L}\hlstd{)}
\end{alltt}
\begin{verbatim}
## [1] 3 1 4 5 0 2
\end{verbatim}
\end{kframe}
\end{knitrout}

Both \code{c()} and \code{append()} can be also used with lists.

\begin{playground}
One can create sequences\index{sequence} using function \Rfunction{seq()} or the operator \Roperator{:}, or repeat values using function \Rfunction{rep()}. In this case I leave to the reader to work out the rules by running these and his/her own examples, with the help of the documentation, available through \code{help(seq)} and \code{help(rep)}.

\begin{knitrout}\footnotesize
\definecolor{shadecolor}{rgb}{0.969, 0.969, 0.969}\color{fgcolor}\begin{kframe}
\begin{alltt}
\hlstd{a} \hlkwb{<-} \hlopt{-}\hlnum{1}\hlopt{:}\hlnum{5}
\hlstd{a}
\hlstd{b} \hlkwb{<-} \hlnum{5}\hlopt{:-}\hlnum{1}
\hlstd{b}
\hlstd{c} \hlkwb{<-} \hlkwd{seq}\hlstd{(}\hlkwc{from} \hlstd{=} \hlopt{-}\hlnum{1}\hlstd{,} \hlkwc{to} \hlstd{=} \hlnum{1}\hlstd{,} \hlkwc{by} \hlstd{=} \hlnum{0.1}\hlstd{)}
\hlstd{c}
\hlstd{d} \hlkwb{<-} \hlkwd{rep}\hlstd{(}\hlopt{-}\hlnum{5}\hlstd{,} \hlnum{4}\hlstd{)}
\hlstd{d}
\end{alltt}
\end{kframe}
\end{knitrout}

\end{playground}

Next something that makes \Rlang different from most other programming languages: vectorized arithmetic\index{vectorized arithmetic}. Operators and functions that are vectorized accept as argument vectors of arbitrary length, in which case the result returned is equivalent to having applied the same function or operator individually to each element of the vector.\label{par:vectorized:numeric}

\begin{knitrout}\footnotesize
\definecolor{shadecolor}{rgb}{0.969, 0.969, 0.969}\color{fgcolor}\begin{kframe}
\begin{alltt}
\hlstd{a} \hlopt{+} \hlnum{1} \hlcom{# we add one to vector a defined above}
\end{alltt}
\begin{verbatim}
## [1] 4 2 3
\end{verbatim}
\begin{alltt}
\hlstd{(a} \hlopt{+} \hlnum{1}\hlstd{)} \hlopt{*} \hlnum{2}
\end{alltt}
\begin{verbatim}
## [1] 8 4 6
\end{verbatim}
\begin{alltt}
\hlstd{a} \hlopt{+} \hlstd{b}
\end{alltt}
\begin{verbatim}
## [1] 7 6 2
\end{verbatim}
\begin{alltt}
\hlstd{a} \hlopt{-} \hlstd{a}
\end{alltt}
\begin{verbatim}
## [1] 0 0 0
\end{verbatim}
\end{kframe}
\end{knitrout}

As it can be seen in the first line above, another peculiarity of \Rpgrm, is what is frequently called ``recycling'' of arguments:\index{recycling@recycling of arguments} as vector \code{a} is of length 6, but the constant 1 is a vector of length 1, this 1 is extended by recycling into a vector of ones of the same length as the longest vector in the statement, in this case, \code{a}.\label{par:recycling:numeric}

Make sure you understand what calculations are taking place in the chunk above, and also the one below.

\begin{knitrout}\footnotesize
\definecolor{shadecolor}{rgb}{0.969, 0.969, 0.969}\color{fgcolor}\begin{kframe}
\begin{alltt}
\hlstd{a} \hlkwb{<-} \hlkwd{rep}\hlstd{(}\hlnum{1}\hlstd{,} \hlnum{6}\hlstd{)}
\hlstd{a}
\end{alltt}
\begin{verbatim}
## [1] 1 1 1 1 1 1
\end{verbatim}
\begin{alltt}
\hlstd{a} \hlopt{+} \hlnum{1}\hlopt{:}\hlnum{2}
\end{alltt}
\begin{verbatim}
## [1] 2 3 2 3 2 3
\end{verbatim}
\begin{alltt}
\hlstd{a} \hlopt{+} \hlnum{1}\hlopt{:}\hlnum{3}
\end{alltt}
\begin{verbatim}
## [1] 2 3 4 2 3 4
\end{verbatim}
\begin{alltt}
\hlstd{a} \hlopt{+} \hlnum{1}\hlopt{:}\hlnum{4}
\end{alltt}


{\ttfamily\noindent\color{warningcolor}{\#\# Warning in a + 1:4: longer object length is not a multiple of shorter object length}}\begin{verbatim}
## [1] 2 3 4 5 2 3
\end{verbatim}
\end{kframe}
\end{knitrout}

\begin{explainbox}
A useful thing to know: a vector can have length zero. Vectors of length zero may seem at first sight quite useless, but in fact they are very useful. They allow the handling of ``no input'' or ``nothing to do'' cases as normal cases, which in the absence of vectors of length zero would require to be treated as special cases. I describe here a useful function, \Rfunction{length()} which returns the length of a vector or list.

\begin{knitrout}\footnotesize
\definecolor{shadecolor}{rgb}{0.969, 0.969, 0.969}\color{fgcolor}\begin{kframe}
\begin{alltt}
\hlstd{z} \hlkwb{<-} \hlkwd{numeric}\hlstd{(}\hlnum{0}\hlstd{)}
\hlstd{z}
\end{alltt}
\begin{verbatim}
## numeric(0)
\end{verbatim}
\begin{alltt}
\hlkwd{length}\hlstd{(z)}
\end{alltt}
\begin{verbatim}
## [1] 0
\end{verbatim}
\end{kframe}
\end{knitrout}

Vectors and lists of length zero, behave in most cases, as expected---e.g.\ they can be concatenated as shown here.

\begin{knitrout}\footnotesize
\definecolor{shadecolor}{rgb}{0.969, 0.969, 0.969}\color{fgcolor}\begin{kframe}
\begin{alltt}
\hlkwd{length}\hlstd{(}\hlkwd{c}\hlstd{(a,} \hlkwd{numeric}\hlstd{(}\hlnum{0}\hlstd{), b))}
\end{alltt}
\begin{verbatim}
## [1] 9
\end{verbatim}
\begin{alltt}
\hlkwd{length}\hlstd{(}\hlkwd{c}\hlstd{(a, b))}
\end{alltt}
\begin{verbatim}
## [1] 9
\end{verbatim}
\end{kframe}
\end{knitrout}

Many functions, such as \Rlang's maths functions and operators, will accept numeric vectors of length zero as valid input, returning also a vector of length zero, issuing neither a warning nor an error message. In other words, \emph{these are valid operations} in \Rlang.

\begin{knitrout}\footnotesize
\definecolor{shadecolor}{rgb}{0.969, 0.969, 0.969}\color{fgcolor}\begin{kframe}
\begin{alltt}
\hlkwd{log}\hlstd{(}\hlkwd{numeric}\hlstd{(}\hlnum{0}\hlstd{))}
\end{alltt}
\begin{verbatim}
## numeric(0)
\end{verbatim}
\begin{alltt}
\hlnum{5} \hlopt{+} \hlkwd{numeric}\hlstd{(}\hlnum{0}\hlstd{)}
\end{alltt}
\begin{verbatim}
## numeric(0)
\end{verbatim}
\end{kframe}
\end{knitrout}

Even when of length zero, vectors do have to belong to a class acceptable for the operation.

\end{explainbox}

It\index{removing objects}\index{deleting objects|see {removing objects}} is possible to \emph{remove} variables from the workspace with \Rfunction{rm()}. Function \Rfunction{ls()} returns a \emph{list} all objects in the current environment, or by supplying a \code{pattern} argument, only the objects with names matching the \code{pattern}. The pattern is given as a regular expression, with \verb|[]| enclosing alternative matching characters, \verb|^| and \verb|$| indicating the extremes of the name (start and end, respectively). For example \verb|"^z$"| matches only the single character `z' while \verb|"^z"| matches any name starting with `z'. In contrast \verb|"^[zy]$"| matches both `z' and `y' but neither `zy' nor `yz', and \verb|"^[a-z]"| matches any name starting with a lower case ASCII letter. If you are using \pgrmname{RStudio}, all objects are listed in the Environment pane, and the search box of the panel can be used to find a given object.

\begin{knitrout}\footnotesize
\definecolor{shadecolor}{rgb}{0.969, 0.969, 0.969}\color{fgcolor}\begin{kframe}
\begin{alltt}
\hlkwd{ls}\hlstd{(}\hlkwc{pattern}\hlstd{=}\hlstr{"^z$"}\hlstd{)}
\end{alltt}
\begin{verbatim}
## [1] "z"
\end{verbatim}
\begin{alltt}
\hlkwd{rm}\hlstd{(z)}
\hlkwd{ls}\hlstd{(}\hlkwc{pattern}\hlstd{=}\hlstr{"^z$"}\hlstd{)}
\end{alltt}
\begin{verbatim}
## character(0)
\end{verbatim}
\end{kframe}
\end{knitrout}

There are some special values available for numbers. \Rconst{NA} meaning `not available' is used for missing values. Calculations can yield also the following values \Rconst{NaN} `not a number', \Rconst{Inf} and \Rconst{-Inf} for $\infty$ and $-\infty$. As you will see below, calculations yielding these values do \textbf{not} trigger errors or warnings, as they are arithmetically valid. \Rconst{Inf} and \Rconst{-Inf} are also valid numerical values for input and constants.

\begin{knitrout}\footnotesize
\definecolor{shadecolor}{rgb}{0.969, 0.969, 0.969}\color{fgcolor}\begin{kframe}
\begin{alltt}
\hlstd{a} \hlkwb{<-} \hlnum{NA}
\hlstd{a}
\end{alltt}
\begin{verbatim}
## [1] NA
\end{verbatim}
\begin{alltt}
\hlopt{-}\hlnum{1} \hlopt{/} \hlnum{0}
\end{alltt}
\begin{verbatim}
## [1] -Inf
\end{verbatim}
\begin{alltt}
\hlnum{1} \hlopt{/} \hlnum{0}
\end{alltt}
\begin{verbatim}
## [1] Inf
\end{verbatim}
\begin{alltt}
\hlnum{Inf} \hlopt{/} \hlnum{Inf}
\end{alltt}
\begin{verbatim}
## [1] NaN
\end{verbatim}
\begin{alltt}
\hlnum{Inf} \hlopt{+} \hlnum{4}
\end{alltt}
\begin{verbatim}
## [1] Inf
\end{verbatim}
\begin{alltt}
\hlstd{b} \hlkwb{<-} \hlopt{-}\hlnum{Inf}
\hlstd{b} \hlopt{* -}\hlnum{1}
\end{alltt}
\begin{verbatim}
## [1] Inf
\end{verbatim}
\end{kframe}
\end{knitrout}

Not available (\Rconst{NA}) values are very important in the analysis of experimental data, as frequently some observations are missing from an otherwise complete data set due to ``accidents'' during the course of an experiment. It is important to understand how to interpret \Rconst{NA}'s. They are simple place holders for something that is unavailable, in other words \emph{unknown}.

\begin{knitrout}\footnotesize
\definecolor{shadecolor}{rgb}{0.969, 0.969, 0.969}\color{fgcolor}\begin{kframe}
\begin{alltt}
\hlstd{A} \hlkwb{<-} \hlnum{NA}
\hlstd{A}
\end{alltt}
\begin{verbatim}
## [1] NA
\end{verbatim}
\begin{alltt}
\hlstd{A} \hlopt{+} \hlnum{1}
\end{alltt}
\begin{verbatim}
## [1] NA
\end{verbatim}
\begin{alltt}
\hlstd{A} \hlopt{+} \hlnum{Inf}
\end{alltt}
\begin{verbatim}
## [1] NA
\end{verbatim}
\end{kframe}
\end{knitrout}

\begin{playground}
\textbf{When to use vectors of length zero, and when \code{NA}s?}\index{zero length objects}\index{vectors!zero length} Make sure you understand the logic behind the different behaviour of functions and operators with respect to \code{NA} and \code{numeric()} or its equivalent \code{numeric(0)}. What do they represent? Why \Rconst{NA}s are not ignored, while vectors of length zero are?

\begin{knitrout}\footnotesize
\definecolor{shadecolor}{rgb}{0.969, 0.969, 0.969}\color{fgcolor}\begin{kframe}
\begin{alltt}
\hlnum{123} \hlopt{+} \hlkwd{numeric}\hlstd{()}
\hlnum{123} \hlopt{+} \hlnum{NA}
\end{alltt}
\end{kframe}
\end{knitrout}

\emph{Model answer:}
\Rconst{NA} is used to signal a value that ``was lost'' or ``was expected'' but is unavailable because of some accident. A vector of length zero, represents no values, but within the normal expectations. In particular, if vectors are expected to have a certain length, or if index positions along a vector are meaningful, then using \Rconst{NA} is a must.

\end{playground}

Any operation, even tests of equality, involving one or more \Rconst{NA}'s return an \Rconst{NA}. In other words when one input to a calculation is unknown, the result of the calculation is unknown. This means that a special function is needed for testing for the presence of \code{NA} values.

\begin{knitrout}\footnotesize
\definecolor{shadecolor}{rgb}{0.969, 0.969, 0.969}\color{fgcolor}\begin{kframe}
\begin{alltt}
\hlkwd{is.na}\hlstd{(}\hlkwd{c}\hlstd{(}\hlnum{NA}\hlstd{,} \hlnum{1}\hlstd{))}
\end{alltt}
\begin{verbatim}
## [1]  TRUE FALSE
\end{verbatim}
\end{kframe}
\end{knitrout}

In the example above, we can also see that \Rfunction{is.na()} is vectorized, and that it applies the test, to each of the to elements of the vector individually, returning the result as a new vector.

One thing\index{precision!math operations}\index{numbers!floating point} to be aware of are the consequences of the fact that numbers in computers are almost always stored with finite precision and/or range: the expectations derived from the mathematical definition of Real numbers are not always fulfilled. See box on page \pageref{box:floats} for an in-depth explanation.

\begin{knitrout}\footnotesize
\definecolor{shadecolor}{rgb}{0.969, 0.969, 0.969}\color{fgcolor}\begin{kframe}
\begin{alltt}
\hlnum{1} \hlopt{-} \hlnum{1e-20}
\end{alltt}
\begin{verbatim}
## [1] 1
\end{verbatim}
\end{kframe}
\end{knitrout}

When comparing \Rclass{integer}\index{numbers!whole}\index{numbers!integer} values these problems do not exist, as integer arithmetic is not affected by loss of precision in calculations restricted to integers (the \code{L} comes from `long', a name sometimes used for a machine representation of integers). Because of the way integers are stored in the memory of computers, within the representable range, they are stored exactly. One can think of computer integers as a subset of whole numbers restricted to a certain range of values.

\begin{knitrout}\footnotesize
\definecolor{shadecolor}{rgb}{0.969, 0.969, 0.969}\color{fgcolor}\begin{kframe}
\begin{alltt}
\hlnum{1L} \hlopt{+} \hlnum{3L}
\end{alltt}
\begin{verbatim}
## [1] 4
\end{verbatim}
\begin{alltt}
\hlnum{1L} \hlopt{*} \hlnum{3L}
\end{alltt}
\begin{verbatim}
## [1] 3
\end{verbatim}
\begin{alltt}
\hlnum{1L} \hlopt \hlnum{3L}
\end{alltt}
\begin{verbatim}
## [1] 0
\end{verbatim}
\begin{alltt}
\hlnum{1L} \hlopt \hlnum{3L}
\end{alltt}
\begin{verbatim}
## [1] 1
\end{verbatim}
\begin{alltt}
\hlnum{1L} \hlopt{/} \hlnum{3L}
\end{alltt}
\begin{verbatim}
## [1] 0.3333333
\end{verbatim}
\end{kframe}
\end{knitrout}

The last statement in the example immediately above, using the `usual' division operator yields a floating-point \code{double} result, while the integer division operator \Roperator{\%/\%} yields an \code{integer} result, and \Roperator{\%\%} returns the remainder from the integer division. If as a result of an operation the result falls outside the range of representable values, the returned value is \code{NA}.

\begin{knitrout}\footnotesize
\definecolor{shadecolor}{rgb}{0.969, 0.969, 0.969}\color{fgcolor}\begin{kframe}
\begin{alltt}
\hlnum{1000000L} \hlopt{*} \hlnum{1000000L}
\end{alltt}


{\ttfamily\noindent\color{warningcolor}{\#\# Warning in 1000000L * 1000000L: NAs produced by integer overflow}}\begin{verbatim}
## [1] NA
\end{verbatim}
\end{kframe}
\end{knitrout}

Both doubles and integers are considered numeric. In most situations conversion is automatic and we do not need to worry about the differences between these two types of numeric values. This last chunk shows returned values that are either \Rconst{TRUE} or \Rconst{FALSE}. These are \code{logical} values that will be discussed in the next section.\index{numbers!double}\index{numbers!integer}

\begin{knitrout}\footnotesize
\definecolor{shadecolor}{rgb}{0.969, 0.969, 0.969}\color{fgcolor}\begin{kframe}
\begin{alltt}
\hlkwd{is.numeric}\hlstd{(}\hlnum{1L}\hlstd{)}
\end{alltt}
\begin{verbatim}
## [1] TRUE
\end{verbatim}
\begin{alltt}
\hlkwd{is.integer}\hlstd{(}\hlnum{1L}\hlstd{)}
\end{alltt}
\begin{verbatim}
## [1] TRUE
\end{verbatim}
\begin{alltt}
\hlkwd{is.double}\hlstd{(}\hlnum{1L}\hlstd{)}
\end{alltt}
\begin{verbatim}
## [1] FALSE
\end{verbatim}
\begin{alltt}
\hlkwd{is.double}\hlstd{(}\hlnum{1L} \hlopt{/} \hlnum{3L}\hlstd{)}
\end{alltt}
\begin{verbatim}
## [1] TRUE
\end{verbatim}
\begin{alltt}
\hlkwd{is.numeric}\hlstd{(}\hlnum{1L} \hlopt{/} \hlnum{3L}\hlstd{)}
\end{alltt}
\begin{verbatim}
## [1] TRUE
\end{verbatim}
\end{kframe}
\end{knitrout}

\begin{advplayground}
Study the variations of the previous example shown below, and explain why the two statements return different values. Hint: 1 is a \code{double} constant. You can use \code{is.integer()} and \code{is.double()} in your explorations.

\begin{knitrout}\footnotesize
\definecolor{shadecolor}{rgb}{0.969, 0.969, 0.969}\color{fgcolor}\begin{kframe}
\begin{alltt}
\hlnum{1} \hlopt{*} \hlnum{1000000L} \hlopt{*} \hlnum{1000000L}
\hlnum{1000000L} \hlopt{*} \hlnum{1000000L} \hlopt{*} \hlnum{1}
\end{alltt}
\end{kframe}
\end{knitrout}
\end{advplayground}

Both when displaying numbers or as part of computations, we may want to decrease the number of significant digits or the number of digits after the decimal marker. Be aware that in the examples below even if printing is being done by default, these functions return \code{numeric} values that are different from their input and can be stored and used in computations. Function \Rfunction{round()} is used to round numbers to a certain number of decimal places after or before the decimal marker, while \Rfunction{signif()} rounds to the requested number of significant digits.

\begin{knitrout}\footnotesize
\definecolor{shadecolor}{rgb}{0.969, 0.969, 0.969}\color{fgcolor}\begin{kframe}
\begin{alltt}
\hlkwd{round}\hlstd{(}\hlnum{0.0124567}\hlstd{,} \hlkwc{digits} \hlstd{=} \hlnum{3}\hlstd{)}
\end{alltt}
\begin{verbatim}
## [1] 0.012
\end{verbatim}
\begin{alltt}
\hlkwd{signif}\hlstd{(}\hlnum{0.0124567}\hlstd{,} \hlkwc{digits} \hlstd{=} \hlnum{3}\hlstd{)}
\end{alltt}
\begin{verbatim}
## [1] 0.0125
\end{verbatim}
\begin{alltt}
\hlkwd{round}\hlstd{(}\hlnum{1789.1234}\hlstd{,} \hlkwc{digits} \hlstd{=} \hlnum{3}\hlstd{)}
\end{alltt}
\begin{verbatim}
## [1] 1789.123
\end{verbatim}
\begin{alltt}
\hlkwd{signif}\hlstd{(}\hlnum{1789.1234}\hlstd{,} \hlkwc{digits} \hlstd{=} \hlnum{3}\hlstd{)}
\end{alltt}
\begin{verbatim}
## [1] 1790
\end{verbatim}
\begin{alltt}
\hlkwd{round}\hlstd{(}\hlnum{1789.1234}\hlstd{,} \hlkwc{digits} \hlstd{=} \hlopt{-}\hlnum{1}\hlstd{)}
\end{alltt}
\begin{verbatim}
## [1] 1790
\end{verbatim}
\begin{alltt}
\hlstd{a} \hlkwb{<-} \hlnum{0.12345}
\hlstd{b} \hlkwb{<-} \hlkwd{round}\hlstd{(a,} \hlkwc{digits} \hlstd{=} \hlnum{2}\hlstd{)}
\hlstd{a} \hlopt{==} \hlstd{b}
\end{alltt}
\begin{verbatim}
## [1] FALSE
\end{verbatim}
\begin{alltt}
\hlstd{a} \hlopt{-} \hlstd{b}
\end{alltt}
\begin{verbatim}
## [1] 0.00345
\end{verbatim}
\begin{alltt}
\hlstd{b}
\end{alltt}
\begin{verbatim}
## [1] 0.12
\end{verbatim}
\end{kframe}
\end{knitrout}

\begin{explainbox}
Being \code{digits} the second parameter of these functions, the argument can be also passed by position. However, code is usually easier to understand for humans when parameter names are made explicit.

\begin{knitrout}\footnotesize
\definecolor{shadecolor}{rgb}{0.969, 0.969, 0.969}\color{fgcolor}\begin{kframe}
\begin{alltt}
\hlkwd{round}\hlstd{(}\hlnum{0.0124567}\hlstd{,} \hlkwc{digits} \hlstd{=} \hlnum{3}\hlstd{)}
\end{alltt}
\begin{verbatim}
## [1] 0.012
\end{verbatim}
\begin{alltt}
\hlkwd{round}\hlstd{(}\hlnum{0.0124567}\hlstd{,} \hlnum{3}\hlstd{)}
\end{alltt}
\begin{verbatim}
## [1] 0.012
\end{verbatim}
\end{kframe}
\end{knitrout}
\end{explainbox}

Functions \Rfunction{trunc()} and \Rfunction{ceiling()} return the non-fractional part of a numeric value as a new numeric value. They differ in how they handle negative values, and neither of them rounds the returned value to the nearest whole number.

\begin{playground}
What does value truncation mean? Function \Rfunction{trunc()} truncates a numeric value, but it does not return an \code{integer}.
\begin{itemize}
  \item Explore how \Rfunction{trunc()} and \Rfunction{ceiling()} differ. Test them both with positive and negative values.
  \item \textbf{Advanced} Use function \Rfunction{abs()} and operators \Roperator{+} and \Roperator{-} to recreate the output of \Rfunction{trunc()} and \Rfunction{ceiling()} for the different inputs.
  \item Can \Rfunction{trunc()} and \Rfunction{ceiling()} be considered type conversion functions in \Rlang?
\end{itemize}
\end{playground}

\index{classes and modes!numeric, integer, double|)}\index{numbers and their arithmetic|)}

\section{Logical values and Boolean algebra}\label{sec:calc:boolean}
\index{classes and modes!logical|(}\index{logical operators}\index{logical values and their algebra|(}\index{Boolean arithmetic}
What in Mathematics are usually called Boolean values, are called \Rclass{logical} values in \Rlang. They can have only two values \code{TRUE} and \code{FALSE}, in addition to \code{NA} (not available). They are vectors as all other atomic types in \Rlang (by \emph{atomic} we mean that each value is not composed of `parts'). There are also logical operators that allow Boolean algebra. In the chunk below we operate on \Rclass{logical} vectors of length one.

\begin{knitrout}\footnotesize
\definecolor{shadecolor}{rgb}{0.969, 0.969, 0.969}\color{fgcolor}\begin{kframe}
\begin{alltt}
\hlstd{a} \hlkwb{<-} \hlnum{TRUE}
\hlstd{b} \hlkwb{<-} \hlnum{FALSE}
\hlkwd{mode}\hlstd{(a)}
\end{alltt}
\begin{verbatim}
## [1] "logical"
\end{verbatim}
\begin{alltt}
\hlstd{a}
\end{alltt}
\begin{verbatim}
## [1] TRUE
\end{verbatim}
\begin{alltt}
\hlopt{!}\hlstd{a} \hlcom{# negation}
\end{alltt}
\begin{verbatim}
## [1] FALSE
\end{verbatim}
\begin{alltt}
\hlstd{a} \hlopt{&&} \hlstd{b} \hlcom{# logical AND}
\end{alltt}
\begin{verbatim}
## [1] FALSE
\end{verbatim}
\begin{alltt}
\hlstd{a} \hlopt{||} \hlstd{b} \hlcom{# logical OR}
\end{alltt}
\begin{verbatim}
## [1] TRUE
\end{verbatim}
\begin{alltt}
\hlkwd{xor}\hlstd{(a, b)} \hlcom{# exclusive OR}
\end{alltt}
\begin{verbatim}
## [1] TRUE
\end{verbatim}
\end{kframe}
\end{knitrout}

%%%% index operators using verb!!
As with arithmetic operators vectorization is available with \emph{some} logical operators. The availability of two kinds of logical operators is one of the most troublesome aspects of the \Rlang language for beginners. Pairs of `equivalent' logical operators behave differently, use similar syntax and use similar symbols! The vectorized operators have single-character names \Roperator{\&} and \Roperator{\textbar}, while the non vectorized ones have double-character names \Roperator{\&\&} and \Roperator{\textbar\textbar}. There is only one version of the negation operator \Roperator{!} that is vectorized. In some, but not all cases, a warning will indicate that there is a possible problem.

\begin{knitrout}\footnotesize
\definecolor{shadecolor}{rgb}{0.969, 0.969, 0.969}\color{fgcolor}\begin{kframe}
\begin{alltt}
\hlstd{a} \hlkwb{<-} \hlkwd{c}\hlstd{(}\hlnum{TRUE}\hlstd{,}\hlnum{FALSE}\hlstd{)}
\hlstd{b} \hlkwb{<-} \hlkwd{c}\hlstd{(}\hlnum{TRUE}\hlstd{,}\hlnum{TRUE}\hlstd{)}
\hlstd{a}
\end{alltt}
\begin{verbatim}
## [1]  TRUE FALSE
\end{verbatim}
\begin{alltt}
\hlstd{b}
\end{alltt}
\begin{verbatim}
## [1] TRUE TRUE
\end{verbatim}
\begin{alltt}
\hlstd{a} \hlopt{&} \hlstd{b} \hlcom{# vectorized AND}
\end{alltt}
\begin{verbatim}
## [1]  TRUE FALSE
\end{verbatim}
\begin{alltt}
\hlstd{a} \hlopt{|} \hlstd{b} \hlcom{# vectorized OR}
\end{alltt}
\begin{verbatim}
## [1] TRUE TRUE
\end{verbatim}
\begin{alltt}
\hlstd{a} \hlopt{&&} \hlstd{b} \hlcom{# not vectorized}
\end{alltt}
\begin{verbatim}
## [1] TRUE
\end{verbatim}
\begin{alltt}
\hlstd{a} \hlopt{||} \hlstd{b} \hlcom{# not vectorized}
\end{alltt}
\begin{verbatim}
## [1] TRUE
\end{verbatim}
\end{kframe}
\end{knitrout}

Functions \Rfunction{any()} and \Rfunction{all()} take a logical vector as argument, and return a single logical value `summarizing' the logical values in the vector. Function \Rfunction{all()} returns \code{TRUE} only if every value in the vector passed as argument is \code{TRUE}, and \Rfunction{any()} returns \code{TRUE} unless every value in its argument is \code{FALSE}.

\begin{knitrout}\footnotesize
\definecolor{shadecolor}{rgb}{0.969, 0.969, 0.969}\color{fgcolor}\begin{kframe}
\begin{alltt}
\hlkwd{any}\hlstd{(a)}
\end{alltt}
\begin{verbatim}
## [1] TRUE
\end{verbatim}
\begin{alltt}
\hlkwd{all}\hlstd{(a)}
\end{alltt}
\begin{verbatim}
## [1] FALSE
\end{verbatim}
\begin{alltt}
\hlkwd{any}\hlstd{(a} \hlopt{&} \hlstd{b)}
\end{alltt}
\begin{verbatim}
## [1] TRUE
\end{verbatim}
\begin{alltt}
\hlkwd{all}\hlstd{(a} \hlopt{&} \hlstd{b)}
\end{alltt}
\begin{verbatim}
## [1] FALSE
\end{verbatim}
\end{kframe}
\end{knitrout}

Another important thing to know about logical operators is that they `short-cut' evaluation. If the result is known from the first part of the statement, the rest of the statement is not evaluated. Try to understand what happens when you enter the following commands. Short-cut evaluation is useful, as the first condition can be used as a guard preventing a later condition to be evaluated when its computation would result in an error.

\begin{knitrout}\footnotesize
\definecolor{shadecolor}{rgb}{0.969, 0.969, 0.969}\color{fgcolor}\begin{kframe}
\begin{alltt}
\hlnum{TRUE} \hlopt{||} \hlnum{NA}
\end{alltt}
\begin{verbatim}
## [1] TRUE
\end{verbatim}
\begin{alltt}
\hlnum{FALSE} \hlopt{||} \hlnum{NA}
\end{alltt}
\begin{verbatim}
## [1] NA
\end{verbatim}
\begin{alltt}
\hlnum{TRUE} \hlopt{&&} \hlnum{NA}
\end{alltt}
\begin{verbatim}
## [1] NA
\end{verbatim}
\begin{alltt}
\hlnum{FALSE} \hlopt{&&} \hlnum{NA}
\end{alltt}
\begin{verbatim}
## [1] FALSE
\end{verbatim}
\begin{alltt}
\hlnum{TRUE} \hlopt{&&} \hlnum{FALSE} \hlopt{&&} \hlnum{NA}
\end{alltt}
\begin{verbatim}
## [1] FALSE
\end{verbatim}
\begin{alltt}
\hlnum{TRUE} \hlopt{&&} \hlnum{TRUE} \hlopt{&&} \hlnum{NA}
\end{alltt}
\begin{verbatim}
## [1] NA
\end{verbatim}
\end{kframe}
\end{knitrout}

When using the vectorized operators on vectors of length greater than one, `short-cut' evaluation still applies for the result obtained at each index position.

\begin{knitrout}\footnotesize
\definecolor{shadecolor}{rgb}{0.969, 0.969, 0.969}\color{fgcolor}\begin{kframe}
\begin{alltt}
\hlstd{a} \hlopt{&} \hlstd{b} \hlopt{&} \hlnum{NA}
\end{alltt}
\begin{verbatim}
## [1]    NA FALSE
\end{verbatim}
\begin{alltt}
\hlstd{a} \hlopt{&} \hlstd{b} \hlopt{&} \hlkwd{c}\hlstd{(}\hlnum{NA}\hlstd{,} \hlnum{NA}\hlstd{)}
\end{alltt}
\begin{verbatim}
## [1]    NA FALSE
\end{verbatim}
\begin{alltt}
\hlstd{a} \hlopt{|} \hlstd{b} \hlopt{|} \hlkwd{c}\hlstd{(}\hlnum{NA}\hlstd{,} \hlnum{NA}\hlstd{)}
\end{alltt}
\begin{verbatim}
## [1] TRUE TRUE
\end{verbatim}
\end{kframe}
\end{knitrout}

\begin{playground}
Based on the description of ``recycling'' presented on page \pageref{par:recycling:numeric} for \code{numeric} operators, explore how ``recycling'' works with vectorized logical operators. Create logical vectors of different lengths (including length one) and \emph{play} by writing several code statements with operations on them. To get you started, one example is given below. Execute this example, and then create and run your own, making sure that you understand why the values returned are what they are. Sometimes, you will need to device several examples or test cases to tease out of \Rlang an understanding of how a certain feature of the language works, so do not give up early, and make use of your imagination!

\begin{knitrout}\footnotesize
\definecolor{shadecolor}{rgb}{0.969, 0.969, 0.969}\color{fgcolor}\begin{kframe}
\begin{alltt}
\hlstd{x} \hlkwb{<-} \hlkwd{c}\hlstd{(}\hlnum{TRUE}\hlstd{,} \hlnum{FALSE}\hlstd{,} \hlnum{TRUE}\hlstd{,} \hlnum{NA}\hlstd{)}
\hlstd{x} \hlopt{&} \hlnum{FALSE}
\hlstd{x} \hlopt{|} \hlkwd{c}\hlstd{(}\hlnum{TRUE}\hlstd{,} \hlnum{FALSE}\hlstd{)}
\end{alltt}
\end{kframe}
\end{knitrout}

\end{playground}
\index{logical values and their algebra|)}
\section{Comparison operators and operations}
\index{comparison operators|(}\index{operators!comparison|(}\qRoperator{>}\qRoperator{<}\qRoperator{>=}\qRoperator{<=}\qRoperator{==}\qRoperator{!=}
Comparison operators yield as result vectors of \code{logical} values.

\begin{knitrout}\footnotesize
\definecolor{shadecolor}{rgb}{0.969, 0.969, 0.969}\color{fgcolor}\begin{kframe}
\begin{alltt}
\hlnum{1.2} \hlopt{>} \hlnum{1.0}
\end{alltt}
\begin{verbatim}
## [1] TRUE
\end{verbatim}
\begin{alltt}
\hlnum{1.2} \hlopt{>=} \hlnum{1.0}
\end{alltt}
\begin{verbatim}
## [1] TRUE
\end{verbatim}
\begin{alltt}
\hlnum{1.2} \hlopt{==} \hlnum{1.0} \hlcom{# be aware that here we use two = symbols}
\end{alltt}
\begin{verbatim}
## [1] FALSE
\end{verbatim}
\begin{alltt}
\hlnum{1.2} \hlopt{!=} \hlnum{1.0}
\end{alltt}
\begin{verbatim}
## [1] TRUE
\end{verbatim}
\begin{alltt}
\hlnum{1.2} \hlopt{<=} \hlnum{1.0}
\end{alltt}
\begin{verbatim}
## [1] FALSE
\end{verbatim}
\begin{alltt}
\hlnum{1.2} \hlopt{<} \hlnum{1.0}
\end{alltt}
\begin{verbatim}
## [1] FALSE
\end{verbatim}
\begin{alltt}
\hlstd{a} \hlkwb{<-} \hlnum{20}
\hlstd{a} \hlopt{<} \hlnum{100} \hlopt{&&} \hlstd{a} \hlopt{>} \hlnum{10}
\end{alltt}
\begin{verbatim}
## [1] TRUE
\end{verbatim}
\end{kframe}
\end{knitrout}

These operators can be used on vectors of any length, returning as result a logical vector. In other words, they behave in the same way as the arithmetic operators described on page \pageref{par:vectorized:numeric}: their arguments are recycled when needed. Hint: if you do not know what to expect as value for the vector returned by \code{1:10}, execute the statement \code{print(a)} after the first code statement below, or, alternatively, \code{1:10} without saving the result to a variable.

\begin{knitrout}\footnotesize
\definecolor{shadecolor}{rgb}{0.969, 0.969, 0.969}\color{fgcolor}\begin{kframe}
\begin{alltt}
\hlstd{a} \hlkwb{<-} \hlnum{1}\hlopt{:}\hlnum{10}
\hlstd{a} \hlopt{>} \hlnum{5}
\end{alltt}
\begin{verbatim}
##  [1] FALSE FALSE FALSE FALSE FALSE  TRUE  TRUE  TRUE  TRUE  TRUE
\end{verbatim}
\begin{alltt}
\hlstd{a} \hlopt{<} \hlnum{5}
\end{alltt}
\begin{verbatim}
##  [1]  TRUE  TRUE  TRUE  TRUE FALSE FALSE FALSE FALSE FALSE FALSE
\end{verbatim}
\begin{alltt}
\hlstd{a} \hlopt{==} \hlnum{5}
\end{alltt}
\begin{verbatim}
##  [1] FALSE FALSE FALSE FALSE  TRUE FALSE FALSE FALSE FALSE FALSE
\end{verbatim}
\begin{alltt}
\hlkwd{all}\hlstd{(a} \hlopt{>} \hlnum{5}\hlstd{)}
\end{alltt}
\begin{verbatim}
## [1] FALSE
\end{verbatim}
\begin{alltt}
\hlkwd{any}\hlstd{(a} \hlopt{>} \hlnum{5}\hlstd{)}
\end{alltt}
\begin{verbatim}
## [1] TRUE
\end{verbatim}
\begin{alltt}
\hlstd{b} \hlkwb{<-} \hlstd{a} \hlopt{>} \hlnum{5}
\hlstd{b}
\end{alltt}
\begin{verbatim}
##  [1] FALSE FALSE FALSE FALSE FALSE  TRUE  TRUE  TRUE  TRUE  TRUE
\end{verbatim}
\begin{alltt}
\hlkwd{any}\hlstd{(b)}
\end{alltt}
\begin{verbatim}
## [1] TRUE
\end{verbatim}
\begin{alltt}
\hlkwd{all}\hlstd{(b)}
\end{alltt}
\begin{verbatim}
## [1] FALSE
\end{verbatim}
\end{kframe}
\end{knitrout}

Precedence rules also apply to comparison operators and they can be overridden by means of parentheses.

\begin{knitrout}\footnotesize
\definecolor{shadecolor}{rgb}{0.969, 0.969, 0.969}\color{fgcolor}\begin{kframe}
\begin{alltt}
\hlstd{a} \hlopt{>} \hlnum{2} \hlopt{+} \hlnum{3}
\end{alltt}
\begin{verbatim}
##  [1] FALSE FALSE FALSE FALSE FALSE  TRUE  TRUE  TRUE  TRUE  TRUE
\end{verbatim}
\begin{alltt}
\hlstd{(a} \hlopt{>} \hlnum{2}\hlstd{)} \hlopt{+} \hlnum{3}
\end{alltt}
\begin{verbatim}
##  [1] 3 3 4 4 4 4 4 4 4 4
\end{verbatim}
\end{kframe}
\end{knitrout}

\begin{playground}
Use the statement below as a starting point in exploring how precedence works when logical and arithmetic operators are part of the same statement. \emph{Play} with the example by adding parentheses at different positions and based on the returned values, work out what is the default order of precedence used for the evaluation of the example given below.

\begin{knitrout}\footnotesize
\definecolor{shadecolor}{rgb}{0.969, 0.969, 0.969}\color{fgcolor}\begin{kframe}
\begin{alltt}
\hlstd{a} \hlkwb{<-} \hlnum{1}\hlopt{:}\hlnum{10}
\hlstd{a} \hlopt{>} \hlnum{3} \hlopt{|} \hlstd{a} \hlopt{+} \hlnum{2} \hlopt{<} \hlnum{3}
\end{alltt}
\end{kframe}
\end{knitrout}
\end{playground}

Be once more aware of `short-cut evaluation'. If the result would not be affected by the missing value then the result, \code{TRUE} or \code{FALSE} is returned. If the presence of the \code{NA} makes the end result unknown, then \code{NA} is returned.

\begin{knitrout}\footnotesize
\definecolor{shadecolor}{rgb}{0.969, 0.969, 0.969}\color{fgcolor}\begin{kframe}
\begin{alltt}
\hlstd{c} \hlkwb{<-} \hlkwd{c}\hlstd{(a,} \hlnum{NA}\hlstd{)}
\hlstd{c} \hlopt{>} \hlnum{5}
\end{alltt}
\begin{verbatim}
##  [1] FALSE FALSE FALSE FALSE FALSE  TRUE  TRUE  TRUE  TRUE  TRUE    NA
\end{verbatim}
\begin{alltt}
\hlkwd{all}\hlstd{(c} \hlopt{>} \hlnum{5}\hlstd{)}
\end{alltt}
\begin{verbatim}
## [1] FALSE
\end{verbatim}
\begin{alltt}
\hlkwd{any}\hlstd{(c} \hlopt{>} \hlnum{5}\hlstd{)}
\end{alltt}
\begin{verbatim}
## [1] TRUE
\end{verbatim}
\begin{alltt}
\hlkwd{all}\hlstd{(c} \hlopt{<} \hlnum{20}\hlstd{)}
\end{alltt}
\begin{verbatim}
## [1] NA
\end{verbatim}
\begin{alltt}
\hlkwd{any}\hlstd{(c} \hlopt{>} \hlnum{20}\hlstd{)}
\end{alltt}
\begin{verbatim}
## [1] NA
\end{verbatim}
\begin{alltt}
\hlkwd{is.na}\hlstd{(a)}
\end{alltt}
\begin{verbatim}
##  [1] FALSE FALSE FALSE FALSE FALSE FALSE FALSE FALSE FALSE FALSE
\end{verbatim}
\begin{alltt}
\hlkwd{is.na}\hlstd{(c)}
\end{alltt}
\begin{verbatim}
##  [1] FALSE FALSE FALSE FALSE FALSE FALSE FALSE FALSE FALSE FALSE  TRUE
\end{verbatim}
\begin{alltt}
\hlkwd{any}\hlstd{(}\hlkwd{is.na}\hlstd{(c))}
\end{alltt}
\begin{verbatim}
## [1] TRUE
\end{verbatim}
\begin{alltt}
\hlkwd{all}\hlstd{(}\hlkwd{is.na}\hlstd{(c))}
\end{alltt}
\begin{verbatim}
## [1] FALSE
\end{verbatim}
\end{kframe}
\end{knitrout}

The behaviour of many base \Rlang's functions when \code{NA}s are present in their input arguments can be modified. \code{TRUE} passed as argument to parameter \code{na.rm}, results in \code{NA} values being \emph{removed} from the input \textbf{before} the function is applied.

\begin{knitrout}\footnotesize
\definecolor{shadecolor}{rgb}{0.969, 0.969, 0.969}\color{fgcolor}\begin{kframe}
\begin{alltt}
\hlkwd{all}\hlstd{(c} \hlopt{<} \hlnum{20}\hlstd{)}
\end{alltt}
\begin{verbatim}
## [1] NA
\end{verbatim}
\begin{alltt}
\hlkwd{any}\hlstd{(c} \hlopt{>} \hlnum{20}\hlstd{)}
\end{alltt}
\begin{verbatim}
## [1] NA
\end{verbatim}
\begin{alltt}
\hlkwd{all}\hlstd{(c} \hlopt{<} \hlnum{20}\hlstd{,} \hlkwc{na.rm}\hlstd{=}\hlnum{TRUE}\hlstd{)}
\end{alltt}
\begin{verbatim}
## [1] TRUE
\end{verbatim}
\begin{alltt}
\hlkwd{any}\hlstd{(c} \hlopt{>} \hlnum{20}\hlstd{,} \hlkwc{na.rm}\hlstd{=}\hlnum{TRUE}\hlstd{)}
\end{alltt}
\begin{verbatim}
## [1] FALSE
\end{verbatim}
\end{kframe}
\end{knitrout}

\begin{explainbox}
\label{box:floats} \label{par:float}\index{floating point numbers!arithmetic|(}\index{machine arithmetic!precision|(}
\index{floats|see{floating point numbers}}\index{machine arithmetic!rounding errors}\index{Real numbers and computers}
\index{EPS ($\epsilon$)|see{machine arithmetic precision}}%
You may skip this box on first reading. See also page \pageref{par:float}. Here I give some examples for which the finite resolution of computer machine floats, as compared to Real numbers as defined in mathematics makes an important difference.

In \Rpgrm the usual numbers are stored as \emph{double-precision floats}, which have limits to the largest and smallest numbers that can be represented, and the number of significant digits that can be stored, usually described by $\epsilon$, abbreviated \emph{eps}, and defined as the largest number for which $1 + \epsilon = 1$. This can be important, and can generate unexpected results in some cases, especially when testing for equality. In the example below, the result of the subtraction is still exactly 1 due to insufficient resolution.

\begin{knitrout}\footnotesize
\definecolor{shadecolor}{rgb}{0.969, 0.969, 0.969}\color{fgcolor}\begin{kframe}
\begin{alltt}
\hlnum{0} \hlopt{-} \hlnum{1e-20}
\end{alltt}
\begin{verbatim}
## [1] -1e-20
\end{verbatim}
\begin{alltt}
\hlnum{1} \hlopt{-} \hlnum{1e-20}
\end{alltt}
\begin{verbatim}
## [1] 1
\end{verbatim}
\end{kframe}
\end{knitrout}

The finiteness of floats also affects tests of equality, which is more likely to result in errors of important consequences.

\begin{knitrout}\footnotesize
\definecolor{shadecolor}{rgb}{0.969, 0.969, 0.969}\color{fgcolor}\begin{kframe}
\begin{alltt}
\hlnum{1e20} \hlopt{==} \hlnum{1} \hlopt{+} \hlnum{1e20}
\end{alltt}
\begin{verbatim}
## [1] TRUE
\end{verbatim}
\begin{alltt}
\hlnum{1} \hlopt{==} \hlnum{1} \hlopt{+} \hlnum{1e-20}
\end{alltt}
\begin{verbatim}
## [1] TRUE
\end{verbatim}
\begin{alltt}
\hlnum{0} \hlopt{==} \hlnum{1e-20}
\end{alltt}
\begin{verbatim}
## [1] FALSE
\end{verbatim}
\end{kframe}
\end{knitrout}

As \Rpgrm can run on different types of computer hardware, the actual machine limits for storing numbers in memory may vary depending on the type of processor and even compiler used to build the \Rpgrm program executable. However, it is possible to obtain these values at run time from the variable \code{.Machine}, which is part of the \Rlang language. Please, see the help page for \code{.Machine} for a detailed, and up-to-date, description of the available constants.\qRconst{.Machine\$double.eps}\qRconst{.Machine\$double.neg.eps}\qRconst{.Machine\$double.max}\qRconst{.Machine\$double.min}

\begin{knitrout}\footnotesize
\definecolor{shadecolor}{rgb}{0.969, 0.969, 0.969}\color{fgcolor}\begin{kframe}
\begin{alltt}
\hlstd{.Machine}\hlopt{$}\hlstd{double.eps}
\end{alltt}
\begin{verbatim}
## [1] 2.220446e-16
\end{verbatim}
\begin{alltt}
\hlstd{.Machine}\hlopt{$}\hlstd{double.neg.eps}
\end{alltt}
\begin{verbatim}
## [1] 1.110223e-16
\end{verbatim}
\begin{alltt}
\hlstd{.Machine}\hlopt{$}\hlstd{double.max}
\end{alltt}
\begin{verbatim}
## [1] 1024
\end{verbatim}
\begin{alltt}
\hlstd{.Machine}\hlopt{$}\hlstd{double.min}
\end{alltt}
\begin{verbatim}
## [1] -1022
\end{verbatim}
\end{kframe}
\end{knitrout}

The last two values refer to the exponents of 10, rather than the maximum and minimum size of numbers that can be handled as objects of class \Rclass{double}. Values outside these limits are stored as \Rconst{-Inf} or \Rconst{Inf} and enter arithmetic as infinite values according the mathematical rules.

\begin{knitrout}\footnotesize
\definecolor{shadecolor}{rgb}{0.969, 0.969, 0.969}\color{fgcolor}\begin{kframe}
\begin{alltt}
\hlnum{1e1026}
\end{alltt}
\begin{verbatim}
## [1] Inf
\end{verbatim}
\begin{alltt}
\hlnum{1e-1026}
\end{alltt}
\begin{verbatim}
## [1] 0
\end{verbatim}
\begin{alltt}
\hlnum{Inf} \hlopt{+} \hlnum{1}
\end{alltt}
\begin{verbatim}
## [1] Inf
\end{verbatim}
\begin{alltt}
\hlopt{-}\hlnum{Inf} \hlopt{+} \hlnum{1}
\end{alltt}
\begin{verbatim}
## [1] -Inf
\end{verbatim}
\end{kframe}
\end{knitrout}

As \Rclass{integer} values are stored in machine memory without loss of precision, epsilon is not defined for \Rclass{integer} values.\qRconst{.Machine\$integer.max}

\begin{knitrout}\footnotesize
\definecolor{shadecolor}{rgb}{0.969, 0.969, 0.969}\color{fgcolor}\begin{kframe}
\begin{alltt}
\hlstd{.Machine}\hlopt{$}\hlstd{integer.max}
\end{alltt}
\begin{verbatim}
## [1] 2147483647
\end{verbatim}
\begin{alltt}
\hlnum{2147483699L}
\end{alltt}
\begin{verbatim}
## [1] 2147483699
\end{verbatim}
\end{kframe}
\end{knitrout}

In those statements in the chunk below where at least one operand is \Rclass{double} the \Rclass{integer} operands are \emph{promoted} to \Rclass{double} before computation. A similar promotion does not take place when operations are among \Rclass{integer} values, resulting in \emph{overflow}\index{arithmetic overflow}\index{overflow!see{arithmetic overflow}}, meaning numbers that are too big to be represented as \Rclass{integer} values.

\begin{knitrout}\footnotesize
\definecolor{shadecolor}{rgb}{0.969, 0.969, 0.969}\color{fgcolor}\begin{kframe}
\begin{alltt}
\hlnum{2147483600L} \hlopt{+} \hlnum{99L}
\end{alltt}


{\ttfamily\noindent\color{warningcolor}{\#\# Warning in 2147483600L + 99L: NAs produced by integer overflow}}\begin{verbatim}
## [1] NA
\end{verbatim}
\begin{alltt}
\hlnum{2147483600L} \hlopt{+} \hlnum{99}
\end{alltt}
\begin{verbatim}
## [1] 2147483699
\end{verbatim}
\begin{alltt}
\hlnum{2147483600L} \hlopt{*} \hlnum{2147483600L}
\end{alltt}


{\ttfamily\noindent\color{warningcolor}{\#\# Warning in 2147483600L * 2147483600L: NAs produced by integer overflow}}\begin{verbatim}
## [1] NA
\end{verbatim}
\begin{alltt}
\hlnum{2147483600L} \hlopt{*} \hlnum{2147483600}
\end{alltt}
\begin{verbatim}
## [1] 4.611686e+18
\end{verbatim}
\end{kframe}
\end{knitrout}

We see next that the exponentiation operator \Roperator{\^{}} forces the promotion\index{type promotion}\index{arithmetic overflow!type promotion} of its arguments to \Rclass{double}, resulting in no overflow. In contrast, as seen above, the multiplication operator \Roperator{*} operates on integers resulting in overflow.

\begin{knitrout}\footnotesize
\definecolor{shadecolor}{rgb}{0.969, 0.969, 0.969}\color{fgcolor}\begin{kframe}
\begin{alltt}
\hlnum{2147483600L} \hlopt{*} \hlnum{2147483600L}
\end{alltt}


{\ttfamily\noindent\color{warningcolor}{\#\# Warning in 2147483600L * 2147483600L: NAs produced by integer overflow}}\begin{verbatim}
## [1] NA
\end{verbatim}
\begin{alltt}
\hlnum{2147483600L}\hlopt{^}\hlnum{2L}
\end{alltt}
\begin{verbatim}
## [1] 4.611686e+18
\end{verbatim}
\end{kframe}
\end{knitrout}
\index{floating point numbers!arithmetic|)}\index{machine arithmetic!precision|)}
\end{explainbox}

\begin{warningbox}
\index{comparison of floating point numbers|(}\index{inequality and equality tests|(}\index{loss of numeric precision}\index{}In many situations, when writing programs one should avoid testing for equality of floating point numbers (`floats'). Here we show how to handle gracefully rounding errors. As the example shows, rounding errors may accumulate, and in practice \verb|.Machine$double.eps| is not always a good value to safely use in tests for ``zero'', a larger value may be needed. Whenever possible according to the logic of the calculations, it is best to test for inequalities, for example using \verb|x <= 1.0| instead of \verb|x == 1.0|. If this is not possible, then the tests should be done replacing tests like \verb|x == 1.0| with \verb|abs(x - 1.0) < eps|. Function \Rfunction{abs()} returns the absolute value, in simple words, makes all values positive or zero, by changing the sign of negative values, or in mathematical notation $|x| = |-x|$.

\begin{knitrout}\footnotesize
\definecolor{shadecolor}{rgb}{0.969, 0.969, 0.969}\color{fgcolor}\begin{kframe}
\begin{alltt}
\hlstd{a} \hlopt{==} \hlnum{0.0} \hlcom{# may not always work}
\end{alltt}
\begin{verbatim}
##  [1] FALSE FALSE FALSE FALSE FALSE FALSE FALSE FALSE FALSE FALSE
\end{verbatim}
\begin{alltt}
\hlkwd{abs}\hlstd{(a)} \hlopt{<} \hlnum{1e-15} \hlcom{# is safer}
\end{alltt}
\begin{verbatim}
##  [1] FALSE FALSE FALSE FALSE FALSE FALSE FALSE FALSE FALSE FALSE
\end{verbatim}
\begin{alltt}
\hlkwd{sin}\hlstd{(pi)} \hlopt{==} \hlnum{0.0} \hlcom{# angle in radians, not degrees!}
\end{alltt}
\begin{verbatim}
## [1] FALSE
\end{verbatim}
\begin{alltt}
\hlkwd{sin}\hlstd{(}\hlnum{2} \hlopt{*} \hlstd{pi)} \hlopt{==} \hlnum{0.0}
\end{alltt}
\begin{verbatim}
## [1] FALSE
\end{verbatim}
\begin{alltt}
\hlkwd{abs}\hlstd{(}\hlkwd{sin}\hlstd{(pi))} \hlopt{<} \hlnum{1e-15}
\end{alltt}
\begin{verbatim}
## [1] TRUE
\end{verbatim}
\begin{alltt}
\hlkwd{abs}\hlstd{(}\hlkwd{sin}\hlstd{(}\hlnum{2} \hlopt{*} \hlstd{pi))} \hlopt{<} \hlnum{1e-15}
\end{alltt}
\begin{verbatim}
## [1] TRUE
\end{verbatim}
\begin{alltt}
\hlkwd{sin}\hlstd{(pi)}
\end{alltt}
\begin{verbatim}
## [1] 1.224606e-16
\end{verbatim}
\begin{alltt}
\hlkwd{sin}\hlstd{(}\hlnum{2} \hlopt{*} \hlstd{pi)}
\end{alltt}
\begin{verbatim}
## [1] -2.449213e-16
\end{verbatim}
\end{kframe}
\end{knitrout}
\index{comparison of floating point numbers|(}\index{inequality and equality tests|(}
\end{warningbox}

\index{comparison operators|)}\index{operators!comparison|)}
\index{classes and modes!logical|)}

\section{Sets and set operations}
\index{sets|(}\index{algebra of sets}\index{operators!set|(}

The \Rlang language supports set operations on vectors. They can be useful in many different contexts when manipulating and comparing vectors of values. In Bioinformatics it is usual, for example, to have character vectors of gene tags. We may have a vector for each of a set of different samples, and need to compare them. However, we start by using a more mundane example, everyday shopping.

\begin{knitrout}\footnotesize
\definecolor{shadecolor}{rgb}{0.969, 0.969, 0.969}\color{fgcolor}\begin{kframe}
\begin{alltt}
\hlstd{fruits} \hlkwb{<-} \hlkwd{c}\hlstd{(}\hlstr{"apple"}\hlstd{,} \hlstr{"pear"}\hlstd{,} \hlstr{"orange"}\hlstd{,} \hlstr{"lemon"}\hlstd{,} \hlstr{"tangerine"}\hlstd{)}
\hlstd{bakery} \hlkwb{<-} \hlkwd{c}\hlstd{(}\hlstr{"bread"}\hlstd{,} \hlstr{"buns"}\hlstd{,} \hlstr{"cake"}\hlstd{,} \hlstr{"cookies"}\hlstd{)}
\hlstd{dairy} \hlkwb{<-} \hlkwd{c}\hlstd{(}\hlstr{"milk"}\hlstd{,} \hlstr{"butter"}\hlstd{,} \hlstr{"cheese"}\hlstd{)}
\hlstd{shopping} \hlkwb{<-} \hlkwd{c}\hlstd{(}\hlstr{"bread"}\hlstd{,} \hlstr{"butter"}\hlstd{,} \hlstr{"apple"}\hlstd{,} \hlstr{"cheese"}\hlstd{,} \hlstr{"orange"}\hlstd{)}
\hlkwd{intersect}\hlstd{(fruits, shopping)}
\end{alltt}
\begin{verbatim}
## [1] "apple"  "orange"
\end{verbatim}
\begin{alltt}
\hlkwd{intersect}\hlstd{(bakery, shopping)}
\end{alltt}
\begin{verbatim}
## [1] "bread"
\end{verbatim}
\begin{alltt}
\hlkwd{intersect}\hlstd{(dairy, shopping)}
\end{alltt}
\begin{verbatim}
## [1] "butter" "cheese"
\end{verbatim}
\begin{alltt}
\hlstr{"lemon"} \hlopt \hlstd{dairy}
\end{alltt}
\begin{verbatim}
## [1] FALSE
\end{verbatim}
\begin{alltt}
\hlstr{"lemon"} \hlopt \hlstd{fruits}
\end{alltt}
\begin{verbatim}
## [1] TRUE
\end{verbatim}
\begin{alltt}
\hlkwd{setdiff}\hlstd{(}\hlkwd{union}\hlstd{(bakery, dairy), shopping)}
\end{alltt}
\begin{verbatim}
## [1] "buns"    "cake"    "cookies" "milk"
\end{verbatim}
\end{kframe}
\end{knitrout}

And continue with abstract (symbolic) examples.

\begin{knitrout}\footnotesize
\definecolor{shadecolor}{rgb}{0.969, 0.969, 0.969}\color{fgcolor}\begin{kframe}
\begin{alltt}
\hlstd{my.set} \hlkwb{<-} \hlkwd{c}\hlstd{(}\hlstr{"a"}\hlstd{,} \hlstr{"b"}\hlstd{,} \hlstr{"c"}\hlstd{,} \hlstr{"b"}\hlstd{)}
\end{alltt}
\end{kframe}
\end{knitrout}

To test if a given value belongs to a set, we use operator \Roperator{\%in\%}. In the algebra of sets notation this is written $a \in A$, where $A$ is a set and $a$ a member. The second statement shows, that the \code{\%in\%} operator is vectorized on its left-hand-side (lhs) operand, returning a logical vector.

\begin{knitrout}\footnotesize
\definecolor{shadecolor}{rgb}{0.969, 0.969, 0.969}\color{fgcolor}\begin{kframe}
\begin{alltt}
\hlstr{"a"} \hlopt \hlstd{my.set}
\end{alltt}
\begin{verbatim}
## [1] TRUE
\end{verbatim}
\begin{alltt}
\hlkwd{c}\hlstd{(}\hlstr{"a"}\hlstd{,} \hlstr{"a"}\hlstd{,} \hlstr{"z"}\hlstd{)} \hlopt \hlstd{my.set}
\end{alltt}
\begin{verbatim}
## [1]  TRUE  TRUE FALSE
\end{verbatim}
\end{kframe}
\end{knitrout}

The negation of inclusion is $a \not\in A$, and coded in \Rlang by applying the negation operator \Roperator{!} to the result of the test done with \Roperator{\%in\%}.

\begin{knitrout}\footnotesize
\definecolor{shadecolor}{rgb}{0.969, 0.969, 0.969}\color{fgcolor}\begin{kframe}
\begin{alltt}
\hlopt{!}\hlstr{"a"} \hlopt \hlstd{my.set}
\end{alltt}
\begin{verbatim}
## [1] FALSE
\end{verbatim}
\begin{alltt}
\hlopt{!}\hlkwd{c}\hlstd{(}\hlstr{"a"}\hlstd{,} \hlstr{"a"}\hlstd{,} \hlstr{"z"}\hlstd{)} \hlopt \hlstd{my.set}
\end{alltt}
\begin{verbatim}
## [1] FALSE FALSE  TRUE
\end{verbatim}
\end{kframe}
\end{knitrout}

Although inclusion is a set operation, it is also very useful for the simplification of \code{if()\ldots else} statements by replacing multiple tests for alternative constant values of the same \code{mode} chained by multiple \Roperator{|} operators.

\begin{playground}
Use operator \Roperator{\%in\%} to simplify the following comparison.

\begin{knitrout}\footnotesize
\definecolor{shadecolor}{rgb}{0.969, 0.969, 0.969}\color{fgcolor}\begin{kframe}
\begin{alltt}
\hlstd{x} \hlkwb{<-} \hlkwd{c}\hlstd{(}\hlstr{"a"}\hlstd{,} \hlstr{"a"}\hlstd{,} \hlstr{"z"}\hlstd{)}
\hlstd{x} \hlopt{==} \hlstr{"a"} \hlopt{|} \hlstd{x} \hlopt{==} \hlstr{"b"} \hlopt{|} \hlstd{x} \hlopt{==} \hlstr{"c"} \hlopt{|} \hlstd{x} \hlopt{==} \hlstr{"d"}
\end{alltt}
\end{kframe}
\end{knitrout}
\end{playground}

With \Rfunction{unique()} we convert a vector of, possibly repeated, values into a set of unique values. In algebra of sets a certain object belongs or not to a set. Consequently, in a set multiple copies of the same object or value are meaningless.

\begin{knitrout}\footnotesize
\definecolor{shadecolor}{rgb}{0.969, 0.969, 0.969}\color{fgcolor}\begin{kframe}
\begin{alltt}
\hlkwd{unique}\hlstd{(my.set)}
\end{alltt}
\begin{verbatim}
## [1] "a" "b" "c"
\end{verbatim}
\begin{alltt}
\hlkwd{c}\hlstd{(}\hlstr{"a"}\hlstd{,} \hlstr{"a"}\hlstd{,} \hlstr{"z"}\hlstd{)} \hlopt \hlkwd{unique}\hlstd{(my.set)}
\end{alltt}
\begin{verbatim}
## [1]  TRUE  TRUE FALSE
\end{verbatim}
\end{kframe}
\end{knitrout}

In the notation used in algebra of sets, the set union operator is $\cup$ while the intersection operator is $\cap$. If we have sets $A$ and $B$, their union is given by $A \cup B$---In the next three examples, \code{c("a", "a", "z")} is a constant, while \code{my.set} is a variable.

\begin{knitrout}\footnotesize
\definecolor{shadecolor}{rgb}{0.969, 0.969, 0.969}\color{fgcolor}\begin{kframe}
\begin{alltt}
\hlkwd{union}\hlstd{(}\hlkwd{c}\hlstd{(}\hlstr{"a"}\hlstd{,} \hlstr{"a"}\hlstd{,} \hlstr{"z"}\hlstd{), my.set)}
\end{alltt}
\begin{verbatim}
## [1] "a" "z" "b" "c"
\end{verbatim}
\end{kframe}
\end{knitrout}

If we have sets $A$ and $B$, their intersection is given by $A \cap B$.

\begin{knitrout}\footnotesize
\definecolor{shadecolor}{rgb}{0.969, 0.969, 0.969}\color{fgcolor}\begin{kframe}
\begin{alltt}
\hlkwd{intersect}\hlstd{(}\hlkwd{c}\hlstd{(}\hlstr{"a"}\hlstd{,} \hlstr{"a"}\hlstd{,} \hlstr{"z"}\hlstd{), my.set)}
\end{alltt}
\begin{verbatim}
## [1] "a"
\end{verbatim}
\end{kframe}
\end{knitrout}

\begin{playground}
What do you expect to be the difference between the values returned by the three statements in the code chunk below? Before running them, write down your expectations about the value each one will return. Only then run the code. Independently of whether your predictions were correct or not, write down an explanation of what each statement's operation is.

\begin{knitrout}\footnotesize
\definecolor{shadecolor}{rgb}{0.969, 0.969, 0.969}\color{fgcolor}\begin{kframe}
\begin{alltt}
\hlkwd{union}\hlstd{(}\hlkwd{c}\hlstd{(}\hlstr{"a"}\hlstd{,} \hlstr{"a"}\hlstd{,} \hlstr{"z"}\hlstd{), my.set)}
\hlkwd{c}\hlstd{(}\hlkwd{c}\hlstd{(}\hlstr{"a"}\hlstd{,} \hlstr{"a"}\hlstd{,} \hlstr{"z"}\hlstd{), my.set)}
\hlkwd{c}\hlstd{(}\hlstr{"a"}\hlstd{,} \hlstr{"a"}\hlstd{,} \hlstr{"z"}\hlstd{, my.set)}
\end{alltt}
\end{kframe}
\end{knitrout}

In the algebra of sets notation $A \subseteq B$ where $A$ and $B$ are sets indicates that $A$ is a subset or equal to $B$. For a true subset, notation is $A \subset B$. The operators with the reverse direction are $\supseteq$ and $\supset$. Implement these four operations in four \Rlang statements, and test them on sets (represented by \Rlang vectors) with different ``overlap'' among set members.

\end{playground}

\begin{explainbox}
All set algebra examples above use character vectors and character constants. This is just the most frequent use case. Sets operations are valid on vectors of any atomic class, including \code{integer}, and computed values can be part of statements. In the second and third statements in the next chunk, we need to use additional parentheses to alter the default order of precedence between arithmetic and set operators.

\begin{knitrout}\footnotesize
\definecolor{shadecolor}{rgb}{0.969, 0.969, 0.969}\color{fgcolor}\begin{kframe}
\begin{alltt}
\hlnum{9L} \hlopt \hlnum{2L}\hlopt{:}\hlnum{4L}
\end{alltt}
\begin{verbatim}
## [1] FALSE
\end{verbatim}
\begin{alltt}
\hlnum{9L} \hlopt \hlstd{((}\hlnum{2L}\hlopt{:}\hlnum{4L}\hlstd{)} \hlopt{*} \hlstd{(}\hlnum{2L}\hlopt{:}\hlnum{4L}\hlstd{))}
\end{alltt}
\begin{verbatim}
## [1] TRUE
\end{verbatim}
\begin{alltt}
\hlkwd{c}\hlstd{(}\hlnum{1L}\hlstd{,} \hlnum{16L}\hlstd{)} \hlopt \hlstd{((}\hlnum{2L}\hlopt{:}\hlnum{4L}\hlstd{)} \hlopt{*} \hlstd{(}\hlnum{2L}\hlopt{:}\hlnum{4L}\hlstd{))}
\end{alltt}
\begin{verbatim}
## [1] FALSE  TRUE
\end{verbatim}
\end{kframe}
\end{knitrout}

\emph{Empty sets} are an important component of the algebra of sets, in \Rlang they are represented as vectors of zero length. Vectors and lists of zero length, which the \Rlang language fully supports, can be used to ``encode'' emptiness also in other contexts. These vectors do belong to a class such as \Rclass{numeric} or \Rclass{character} and must be compatible with other operands in an expression. By default, constructors for vectors, construct empty vectors.

\begin{knitrout}\footnotesize
\definecolor{shadecolor}{rgb}{0.969, 0.969, 0.969}\color{fgcolor}\begin{kframe}
\begin{alltt}
\hlkwd{length}\hlstd{(}\hlkwd{integer}\hlstd{())}
\end{alltt}
\begin{verbatim}
## [1] 0
\end{verbatim}
\begin{alltt}
\hlnum{1L} \hlopt \hlkwd{integer}\hlstd{()}
\end{alltt}
\begin{verbatim}
## [1] FALSE
\end{verbatim}
\begin{alltt}
\hlkwd{setdiff}\hlstd{(}\hlnum{1L}\hlopt{:}\hlnum{4L}\hlstd{,} \hlkwd{union}\hlstd{(}\hlnum{1L}\hlopt{:}\hlnum{4L}\hlstd{,} \hlkwd{integer}\hlstd{()))}
\end{alltt}
\begin{verbatim}
## integer(0)
\end{verbatim}
\end{kframe}
\end{knitrout}

Although set operators are defined for \Rclass{numeric} vectors, rounding errors in `floats' can result in unexpected results (see section \ref{box:floats} on page \pageref{box:floats}). The next two examples do, however, return the correct answers.\qRoperator{\%in\%}

\begin{knitrout}\footnotesize
\definecolor{shadecolor}{rgb}{0.969, 0.969, 0.969}\color{fgcolor}\begin{kframe}
\begin{alltt}
\hlnum{9} \hlopt \hlstd{(}\hlnum{2}\hlopt{:}\hlnum{4}\hlstd{)}\hlopt{^}\hlnum{2}
\end{alltt}
\begin{verbatim}
## [1] TRUE
\end{verbatim}
\begin{alltt}
\hlkwd{c}\hlstd{(}\hlnum{1}\hlstd{,} \hlnum{5}\hlstd{)} \hlopt \hlstd{(}\hlnum{1}\hlopt{:}\hlnum{10}\hlstd{)}\hlopt{^}\hlnum{2}
\end{alltt}
\begin{verbatim}
## [1]  TRUE FALSE
\end{verbatim}
\end{kframe}
\end{knitrout}

\end{explainbox}
\index{operators!set|)}
\index{sets|)}

\section{Character values}\label{sec:calc:character}
\index{character strings}\index{classes and modes!character|(}\qRclass{character}
Character variables can be used to store any character. Character constants are written by enclosing characters in quotes. There are three types of quotes in the ASCII character set, double quotes \code{"}, single quotes \code{'}, and back ticks \code{`}. The first two types of quotes can be used for delimiting \code{character} constants.

\begin{knitrout}\footnotesize
\definecolor{shadecolor}{rgb}{0.969, 0.969, 0.969}\color{fgcolor}\begin{kframe}
\begin{alltt}
\hlstd{a} \hlkwb{<-} \hlstr{"A"}
\hlstd{a}
\end{alltt}
\begin{verbatim}
## [1] "A"
\end{verbatim}
\begin{alltt}
\hlstd{b} \hlkwb{<-} \hlstr{'A'}
\hlstd{b}
\end{alltt}
\begin{verbatim}
## [1] "A"
\end{verbatim}
\begin{alltt}
\hlstd{a} \hlopt{==} \hlstd{b}
\end{alltt}
\begin{verbatim}
## [1] TRUE
\end{verbatim}
\end{kframe}
\end{knitrout}

There are in \Rlang two predefined vectors with characters for the 26 letters used in English, stored in alphabetical order: \Rconst{letters} and \Rconst{LETTERS}.

\begin{knitrout}\footnotesize
\definecolor{shadecolor}{rgb}{0.969, 0.969, 0.969}\color{fgcolor}\begin{kframe}
\begin{alltt}
\hlstd{a} \hlkwb{<-} \hlstr{"A"}
\hlstd{b} \hlkwb{<-} \hlstd{letters[}\hlnum{2}\hlstd{]}
\hlstd{c} \hlkwb{<-} \hlstd{letters[}\hlnum{1}\hlstd{]}
\hlstd{a}
\end{alltt}
\begin{verbatim}
## [1] "A"
\end{verbatim}
\begin{alltt}
\hlstd{b}
\end{alltt}
\begin{verbatim}
## [1] "b"
\end{verbatim}
\begin{alltt}
\hlstd{c}
\end{alltt}
\begin{verbatim}
## [1] "a"
\end{verbatim}
\begin{alltt}
\hlstd{d} \hlkwb{<-} \hlkwd{c}\hlstd{(a, b, c)}
\hlstd{d}
\end{alltt}
\begin{verbatim}
## [1] "A" "b" "a"
\end{verbatim}
\begin{alltt}
\hlstd{e} \hlkwb{<-} \hlkwd{c}\hlstd{(a, b,} \hlstr{"c"}\hlstd{)}
\hlstd{e}
\end{alltt}
\begin{verbatim}
## [1] "A" "b" "c"
\end{verbatim}
\begin{alltt}
\hlstd{h} \hlkwb{<-} \hlstr{"1"}
\hlkwd{try}\hlstd{(h} \hlopt{+} \hlnum{2}\hlstd{)}
\end{alltt}
\begin{verbatim}
## Error in h + 2 : non-numeric argument to binary operator
\end{verbatim}
\end{kframe}
\end{knitrout}

\begin{explainbox}
In many computer languages vectors of characters are distinct from vectors of character strings. In these languages, character vectors store at each index position a single character, while vectors of character strings store at each index position strings of characters of various lengths, such as words or sentences. \Rlang's \code{character} vectors are vectors of character strings, and there is no predefined class for vectors of individual characters. In \Rlang character string constants can be enclosed either in double or single quotes. If you are familiar with \Clang or \Cpplang you need to keep in mind that \Clang's \code{char} and \Rlang's \code{character} are not equivalent.
\end{explainbox}

One can use the `other' type of quotes as delimiter when one wants to include quotes within a string.

\begin{knitrout}\footnotesize
\definecolor{shadecolor}{rgb}{0.969, 0.969, 0.969}\color{fgcolor}\begin{kframe}
\begin{alltt}
\hlstd{a} \hlkwb{<-} \hlstr{"He said 'hello' when he came in"}
\hlstd{a}
\end{alltt}
\begin{verbatim}
## [1] "He said 'hello' when he came in"
\end{verbatim}
\begin{alltt}
\hlstd{b} \hlkwb{<-} \hlstr{'He said "hello" when he came in'}
\hlstd{b}
\end{alltt}
\begin{verbatim}
## [1] "He said \"hello\" when he came in"
\end{verbatim}
\end{kframe}
\end{knitrout}

The\index{character string delimiters} outer quotes are not part of the string, they are `delimiters' used to mark the boundaries. As you can see when \code{b} is printed special characters can be represented using `escape sequences'. There are several of them, and here we will show just four, new line (\verb|\n|) and tab (\verb|\t|), \verb|\"| the escape code for a quotation mark within a string and \verb|\\| the escape code for a single backslash \verb|\|. We also show here the different behaviour of \Rfunction{print()} and \Rfunction{cat()}, with \Rfunction{cat()} \emph{interpreting} the escape sequences and \Rfunction{print()} displaying them as entered.

\begin{knitrout}\footnotesize
\definecolor{shadecolor}{rgb}{0.969, 0.969, 0.969}\color{fgcolor}\begin{kframe}
\begin{alltt}
\hlstd{c} \hlkwb{<-} \hlstr{"abc\textbackslash{}ndef\textbackslash{}tx\textbackslash{}"yz\textbackslash{}"\textbackslash{}\textbackslash{}\textbackslash{}tm"}
\hlkwd{print}\hlstd{(c)}
\end{alltt}
\begin{verbatim}
## [1] "abc\ndef\tx\"yz\"\\\tm"
\end{verbatim}
\begin{alltt}
\hlkwd{cat}\hlstd{(c)}
\end{alltt}
\begin{verbatim}
## abc
## def	x"yz"\	m
\end{verbatim}
\end{kframe}
\end{knitrout}

The \textit{escape codes}\index{character escape codes} work only in some contexts, as when using \Rfunction{cat()} to generate the output. For example, the new-line escape (\verb|\n|) can be embedded in strings used for axis-label, title or label in a plot to split them over two or more lines.
\index{classes and modes!character|)}

\section{The `mode' and `class' of objects}\label{sec:rlang:mode}
\index{objects!mode}
Variables have a \emph{mode} that depends on what is stored in them. But differently to other languages, assignment to a variable of a different mode is allowed and in most cases its mode changes together with its contents. However, there is a restriction that all elements in a vector, array or matrix, must be of the same mode. While this is not required for lists, which can be heterogenous. In practice this means that we can assign an object, such as a vector, with a different \code{mode} to a name already in use, but, we cannot use indexing to assign an object of a different mode, to certain members of a vector, matrix or array. Functions with names starting with \code{is.} are tests returning a logical value, \code{TRUE}, \code{FALSE} or \code{NA}. Function \Rfunction{mode()} returns the mode of an object, as a character string.\qRfunction{is.character()}\qRfunction{is.numeric()}\qRfunction{is.logical()}

\begin{knitrout}\footnotesize
\definecolor{shadecolor}{rgb}{0.969, 0.969, 0.969}\color{fgcolor}\begin{kframe}
\begin{alltt}
\hlstd{my_var} \hlkwb{<-} \hlnum{1}\hlopt{:}\hlnum{5}
\hlkwd{mode}\hlstd{(my_var)}
\end{alltt}
\begin{verbatim}
## [1] "numeric"
\end{verbatim}
\begin{alltt}
\hlkwd{is.numeric}\hlstd{(my_var)}
\end{alltt}
\begin{verbatim}
## [1] TRUE
\end{verbatim}
\begin{alltt}
\hlkwd{is.logical}\hlstd{(my_var)}
\end{alltt}
\begin{verbatim}
## [1] FALSE
\end{verbatim}
\begin{alltt}
\hlkwd{is.character}\hlstd{(my_var)}
\end{alltt}
\begin{verbatim}
## [1] FALSE
\end{verbatim}
\begin{alltt}
\hlstd{my_var} \hlkwb{<-} \hlstr{"abc"}
\hlkwd{mode}\hlstd{(my_var)}
\end{alltt}
\begin{verbatim}
## [1] "character"
\end{verbatim}
\end{kframe}
\end{knitrout}

While \emph{mode} is a fundamental property, and limited to those modes defined as part of the \Rlang language, the concept of \emph{class}, is different in that new classes can be defined in user code. In particular, different \Rlang objects of a given mode, such as \code{numeric}, can belong to different \code{class}es. The use of classes for dispatching functions is discussed in section \ref{sec:script:objects:classes:methods} on page \pageref{sec:script:objects:classes:methods}, in relation to object oriented programming in \Rlang. Method \Rfunction{class()} is used to query the class of an object, and method \Rfunction{inherits()} is used to test if an object belongs to a specific class or not (including ``parent'' classes, to be later described).

\begin{knitrout}\footnotesize
\definecolor{shadecolor}{rgb}{0.969, 0.969, 0.969}\color{fgcolor}\begin{kframe}
\begin{alltt}
\hlkwd{class}\hlstd{(my_var)}
\end{alltt}
\begin{verbatim}
## [1] "character"
\end{verbatim}
\begin{alltt}
\hlkwd{inherits}\hlstd{(my_var,} \hlstr{"character"}\hlstd{)}
\end{alltt}
\begin{verbatim}
## [1] TRUE
\end{verbatim}
\begin{alltt}
\hlkwd{inherits}\hlstd{(my_var,} \hlstr{"numeric"}\hlstd{)}
\end{alltt}
\begin{verbatim}
## [1] FALSE
\end{verbatim}
\end{kframe}
\end{knitrout}

\section{`Type' conversions}
\index{type conversion|(}
The least intuitive type conversiosn are those related to logical values. All others are as one would expect. By convention, functions used to convert objects from one mode to a different one have names starting with \code{as.}\footnote{Except for some packages in the \pkgnameNI{tidyverse} that use names starting with \code{as\_} instead of \code{as.}.}.\qRfunction{as.character()}\qRfunction{as.numeric()}\qRfunction{as.logical()}

\begin{knitrout}\footnotesize
\definecolor{shadecolor}{rgb}{0.969, 0.969, 0.969}\color{fgcolor}\begin{kframe}
\begin{alltt}
\hlkwd{as.character}\hlstd{(}\hlnum{1}\hlstd{)}
\end{alltt}
\begin{verbatim}
## [1] "1"
\end{verbatim}
\begin{alltt}
\hlkwd{as.numeric}\hlstd{(}\hlstr{"1"}\hlstd{)}
\end{alltt}
\begin{verbatim}
## [1] 1
\end{verbatim}
\begin{alltt}
\hlkwd{as.logical}\hlstd{(}\hlstr{"TRUE"}\hlstd{)}
\end{alltt}
\begin{verbatim}
## [1] TRUE
\end{verbatim}
\begin{alltt}
\hlkwd{as.logical}\hlstd{(}\hlstr{"NA"}\hlstd{)}
\end{alltt}
\begin{verbatim}
## [1] NA
\end{verbatim}
\end{kframe}
\end{knitrout}

Conversion takes place automatically in arithmetic and logical expressions.

\begin{knitrout}\footnotesize
\definecolor{shadecolor}{rgb}{0.969, 0.969, 0.969}\color{fgcolor}\begin{kframe}
\begin{alltt}
\hlnum{TRUE} \hlopt{+} \hlnum{10}
\end{alltt}
\begin{verbatim}
## [1] 11
\end{verbatim}
\begin{alltt}
\hlnum{1} \hlopt{||} \hlnum{0}
\end{alltt}
\begin{verbatim}
## [1] TRUE
\end{verbatim}
\begin{alltt}
\hlnum{FALSE} \hlopt{| -}\hlnum{2}\hlopt{:}\hlnum{2}
\end{alltt}
\begin{verbatim}
## [1]  TRUE  TRUE FALSE  TRUE  TRUE
\end{verbatim}
\end{kframe}
\end{knitrout}

\begin{playground}
There is some flexibility in the conversion from character strings into \code{numeric} and \code{logical} values. Use the examples below plus your own variations to get an idea of what strings are acceptable and correctly converted and which are not. Do also pay attention at the conversion between \code{numeric} and \code{logical} values.\qRfunction{as.character()}\qRfunction{as.numeric()}\qRfunction{as.logical()}

\begin{knitrout}\footnotesize
\definecolor{shadecolor}{rgb}{0.969, 0.969, 0.969}\color{fgcolor}\begin{kframe}
\begin{alltt}
\hlkwd{as.character}\hlstd{(}\hlnum{3.0e10}\hlstd{)}
\hlkwd{as.numeric}\hlstd{(}\hlstr{"5E+5"}\hlstd{)}
\hlkwd{as.numeric}\hlstd{(}\hlstr{"A"}\hlstd{)}
\hlkwd{as.numeric}\hlstd{(}\hlnum{TRUE}\hlstd{)}
\hlkwd{as.numeric}\hlstd{(}\hlnum{FALSE}\hlstd{)}
\hlkwd{as.logical}\hlstd{(}\hlstr{"T"}\hlstd{)}
\hlkwd{as.logical}\hlstd{(}\hlstr{"t"}\hlstd{)}
\hlkwd{as.logical}\hlstd{(}\hlstr{"true"}\hlstd{)}
\hlkwd{as.logical}\hlstd{(}\hlnum{100}\hlstd{)}
\hlkwd{as.logical}\hlstd{(}\hlnum{0}\hlstd{)}
\hlkwd{as.logical}\hlstd{(}\hlopt{-}\hlnum{1}\hlstd{)}
\end{alltt}
\end{kframe}
\end{knitrout}

\end{playground}

\begin{playground}
Compare the values returned by \Rfunction{trunc()} and \Rfunction{as.integer()} when applied to a floating point number, such as \code{12.34}. Check for the equality of values, and for the \emph{class} of the returned objects.
\end{playground}

\begin{explainbox}
With conversions, it becomes obvious the difference between the length of a \code{character} vector and the number of characters composing each member ``string'' within a vector.\qRfunction{length()}\qRfunction{as.numeric()}

\begin{knitrout}\footnotesize
\definecolor{shadecolor}{rgb}{0.969, 0.969, 0.969}\color{fgcolor}\begin{kframe}
\begin{alltt}
\hlstd{f} \hlkwb{<-} \hlkwd{c}\hlstd{(}\hlstr{"1"}\hlstd{,} \hlstr{"2"}\hlstd{,} \hlstr{"3"}\hlstd{)}
\hlkwd{length}\hlstd{(f)}
\end{alltt}
\begin{verbatim}
## [1] 3
\end{verbatim}
\begin{alltt}
\hlstd{g} \hlkwb{<-} \hlstr{"123"}
\hlkwd{length}\hlstd{(g)}
\end{alltt}
\begin{verbatim}
## [1] 1
\end{verbatim}
\begin{alltt}
\hlkwd{as.numeric}\hlstd{(f)}
\end{alltt}
\begin{verbatim}
## [1] 1 2 3
\end{verbatim}
\begin{alltt}
\hlkwd{as.numeric}\hlstd{(g)}
\end{alltt}
\begin{verbatim}
## [1] 123
\end{verbatim}
\end{kframe}
\end{knitrout}
\end{explainbox}

\sloppy
Other\index{formatted character strings from numbers} functions relevant to the ``conversion'' of numbers and other values are \Rfunction{format()}, and \Rfunction{sprintf()}. These two functions return \Rclass{character} strings, instead of \code{numeric} or other values, and are useful for printing output. One could think of these functions as advanced conversion functions returning formatted, and possibly combined and annotated, character strings. However, they are usually not considered normal conversion functions, as they are very rarely used in a way that preserves the original precision of the input values. We show here the use of \Rfunction{format()} and \Rfunction{sprintf()} with \code{numeric} values, but they can be used also with values of other types.

When using \Rfunction{format()} the format used to display numbers is set by passing arguments to several different parameters. As \Rfunction{print()} calls \Rfunction{format()} to make numbers \emph{pretty} it accepts the same options.

\begin{knitrout}\footnotesize
\definecolor{shadecolor}{rgb}{0.969, 0.969, 0.969}\color{fgcolor}\begin{kframe}
\begin{alltt}
\hlstd{x} \hlkwb{=} \hlkwd{c}\hlstd{(}\hlnum{123.4567890}\hlstd{,} \hlnum{1.0}\hlstd{)}
\hlkwd{format}\hlstd{(x)} \hlcom{# using defaults}
\end{alltt}
\begin{verbatim}
## [1] "123.4568" "  1.0000"
\end{verbatim}
\begin{alltt}
\hlkwd{format}\hlstd{(x[}\hlnum{1}\hlstd{])} \hlcom{# using defaults}
\end{alltt}
\begin{verbatim}
## [1] "123.4568"
\end{verbatim}
\begin{alltt}
\hlkwd{format}\hlstd{(x[}\hlnum{2}\hlstd{])} \hlcom{# using defaults}
\end{alltt}
\begin{verbatim}
## [1] "1"
\end{verbatim}
\begin{alltt}
\hlkwd{format}\hlstd{(x,} \hlkwc{digits} \hlstd{=} \hlnum{3}\hlstd{,} \hlkwc{nsmall} \hlstd{=} \hlnum{1}\hlstd{)}
\end{alltt}
\begin{verbatim}
## [1] "123.5" "  1.0"
\end{verbatim}
\begin{alltt}
\hlkwd{format}\hlstd{(x[}\hlnum{1}\hlstd{],} \hlkwc{digits} \hlstd{=} \hlnum{3}\hlstd{,} \hlkwc{nsmall} \hlstd{=} \hlnum{1}\hlstd{)}
\end{alltt}
\begin{verbatim}
## [1] "123.5"
\end{verbatim}
\begin{alltt}
\hlkwd{format}\hlstd{(x[}\hlnum{2}\hlstd{],} \hlkwc{digits} \hlstd{=} \hlnum{3}\hlstd{,} \hlkwc{nsmall} \hlstd{=} \hlnum{1}\hlstd{)}
\end{alltt}
\begin{verbatim}
## [1] "1.0"
\end{verbatim}
\begin{alltt}
\hlkwd{format}\hlstd{(x,} \hlkwc{digits} \hlstd{=} \hlnum{3}\hlstd{,} \hlkwc{scientific} \hlstd{=} \hlnum{TRUE}\hlstd{)}
\end{alltt}
\begin{verbatim}
## [1] "1.23e+02" "1.00e+00"
\end{verbatim}
\end{kframe}
\end{knitrout}

Function \Rfunction{sprintf()} is similar to \Clang's function of the same name. The user interface is rather unusual, but very powerful, once one learns the syntax. All the formatting is specified using a \code{character} string as template. In this template placeholders for data and the formatting instructions are embedded using special codes. These codes start with a percent character. We show in the example below the use of some of these: \code{f} is used for \code{numeric} values to be formatted according to ``fixed point'', while \code{g} is used when we set the number of significant digits and \code{e} for exponential or \emph{scientific} notation.

\begin{knitrout}\footnotesize
\definecolor{shadecolor}{rgb}{0.969, 0.969, 0.969}\color{fgcolor}\begin{kframe}
\begin{alltt}
\hlstd{x} \hlkwb{=} \hlkwd{c}\hlstd{(}\hlnum{123.4567890}\hlstd{,} \hlnum{1.0}\hlstd{)}
\hlkwd{sprintf}\hlstd{(}\hlstr{"The numbers are: %4.2f and %.0f"}\hlstd{, x[}\hlnum{1}\hlstd{], x[}\hlnum{2}\hlstd{])}
\end{alltt}
\begin{verbatim}
## [1] "The numbers are: 123.46 and 1"
\end{verbatim}
\begin{alltt}
\hlkwd{sprintf}\hlstd{(}\hlstr{"The numbers are: %.4g and %.2g"}\hlstd{, x[}\hlnum{1}\hlstd{], x[}\hlnum{2}\hlstd{])}
\end{alltt}
\begin{verbatim}
## [1] "The numbers are: 123.5 and 1"
\end{verbatim}
\begin{alltt}
\hlkwd{sprintf}\hlstd{(}\hlstr{"The numbers are: %4.2e and %.0e"}\hlstd{, x[}\hlnum{1}\hlstd{], x[}\hlnum{2}\hlstd{])}
\end{alltt}
\begin{verbatim}
## [1] "The numbers are: 1.23e+02 and 1e+00"
\end{verbatim}
\end{kframe}
\end{knitrout}

\begin{playground}
Function \Rfunction{format()} may be easier to use, in some cases, but \Rfunction{sprintf()} is more flexible and powerful. Those with experience in the use of the \Clang language will already know about \Rfunction{sprintf()} and its use of templates for formatting output. Even if you are familiar with  \Clang, look up the help pages for both functions, and practice, by trying to create the same formatted output by means of the two functions. Do also play with these functions with other types of data like \code{integer} and \code{character}.
\end{playground}

\begin{explainbox}
We have above described \Rconst{NA} as a single value ignoring modes, but in reality \Rconst{NA}s come in various flavours. \Rconst{NA\_real\_}, \Rconst{NA\_character\_}, etc. and \Rconst{NA} defaults to an \Rconst{NA} of class \Rclass{logical}. Frequently on-the-fly conversion does what one expects when entering \Rconst{NA}.

\begin{knitrout}\footnotesize
\definecolor{shadecolor}{rgb}{0.969, 0.969, 0.969}\color{fgcolor}\begin{kframe}
\begin{alltt}
\hlstd{a} \hlkwb{<-} \hlkwd{c}\hlstd{(}\hlnum{1}\hlstd{,} \hlnum{NA}\hlstd{)}
\hlkwd{is.numeric}\hlstd{(a[}\hlnum{2}\hlstd{])}
\end{alltt}
\begin{verbatim}
## [1] TRUE
\end{verbatim}
\begin{alltt}
\hlkwd{is.numeric}\hlstd{(}\hlnum{NA}\hlstd{)}
\end{alltt}
\begin{verbatim}
## [1] FALSE
\end{verbatim}
\begin{alltt}
\hlstd{b} \hlkwb{<-} \hlkwd{c}\hlstd{(}\hlstr{"abc"}\hlstd{,} \hlnum{NA}\hlstd{)}
\hlkwd{is.character}\hlstd{(b[}\hlnum{2}\hlstd{])}
\end{alltt}
\begin{verbatim}
## [1] TRUE
\end{verbatim}
\begin{alltt}
\hlkwd{is.character}\hlstd{(}\hlnum{NA}\hlstd{)}
\end{alltt}
\begin{verbatim}
## [1] FALSE
\end{verbatim}
\begin{alltt}
\hlkwd{class}\hlstd{(}\hlnum{NA}\hlstd{)}
\end{alltt}
\begin{verbatim}
## [1] "logical"
\end{verbatim}
\end{kframe}
\end{knitrout}

Even the following works:

\begin{knitrout}\footnotesize
\definecolor{shadecolor}{rgb}{0.969, 0.969, 0.969}\color{fgcolor}\begin{kframe}
\begin{alltt}
\hlstd{a[}\hlnum{3}\hlstd{]} \hlkwb{<-} \hlstd{b[}\hlnum{2}\hlstd{]}
\end{alltt}
\end{kframe}
\end{knitrout}

Problems may occur if we test for class before testing for \Rconst{NA} with \Rfunction{is.na()}.

\begin{knitrout}\footnotesize
\definecolor{shadecolor}{rgb}{0.969, 0.969, 0.969}\color{fgcolor}\begin{kframe}
\begin{alltt}
\hlcom{# case 1}
\hlstd{a} \hlkwb{<-} \hlnum{1}\hlopt{:}\hlnum{3}
\hlkwa{if} \hlstd{(}\hlkwd{is.numeric}\hlstd{(a))} \hlkwd{sum}\hlstd{(a)}
\end{alltt}
\begin{verbatim}
## [1] 6
\end{verbatim}
\begin{alltt}
\hlcom{# case 2}
\hlstd{a} \hlkwb{<-} \hlkwd{c}\hlstd{(}\hlnum{1}\hlstd{,}\hlnum{NA}\hlstd{,}\hlnum{3}\hlstd{)}
\hlkwa{if} \hlstd{(}\hlkwd{is.numeric}\hlstd{(a))} \hlkwd{sum}\hlstd{(a)}
\end{alltt}
\begin{verbatim}
## [1] NA
\end{verbatim}
\begin{alltt}
\hlcom{# case 3}
\hlstd{a} \hlkwb{<-} \hlnum{NA}
\hlkwa{if} \hlstd{(}\hlkwd{is.numeric}\hlstd{(a))} \hlkwd{sum}\hlstd{(a)}
\end{alltt}
\end{kframe}
\end{knitrout}
\end{explainbox}

\index{type conversion|(}

\section{Vector manipulation}\label{sec:vectors}\label{sec:calc:indexing}
\index{vectors!indexing|(}\index{vectors!member extraction}
If you have read earlier sections of this chapter, you already know how to create a vector. \Rlang's vectors are equivalent to what would be written in mathematical notation as $x_{1\ldots n} = a_1, a_2, \ldots, a_i, \ldots, a_n$, they are not the equivalent to the vectors, common in Physics, which are symbolized with an arrow as ``accent'', such as $\overrightarrow{\mathbf{F}}$.

In this section we are going to see how to extract or retrieve, replace, and move elements such as $a_2$ from a vector. Elements are extracted using an index enclosed in single square brackets. The index indicates the position in the vector, starting from one, following the usual mathematical tradition. What in maths would be $a_i$ for a vector $a_{1\ldots n}$, in \Rpgrm is represented as \code{a[i]} and the whole vector as earlier seen as \code{a}.

\begin{knitrout}\footnotesize
\definecolor{shadecolor}{rgb}{0.969, 0.969, 0.969}\color{fgcolor}\begin{kframe}
\begin{alltt}
\hlstd{a} \hlkwb{<-} \hlstd{letters[}\hlnum{1}\hlopt{:}\hlnum{10}\hlstd{]}
\hlstd{a}
\end{alltt}
\begin{verbatim}
##  [1] "a" "b" "c" "d" "e" "f" "g" "h" "i" "j"
\end{verbatim}
\begin{alltt}
\hlstd{a[}\hlnum{2}\hlstd{]}
\end{alltt}
\begin{verbatim}
## [1] "b"
\end{verbatim}
\end{kframe}
\end{knitrout}

\begin{explainbox}
Four constant vectors are available in \Rlang: \Rconst{letters}, \Rconst{LETTERS}, \Rconst{month.name} and  \Rconst{month.abb}, of which we used \code{letters} in the example above. These vectors are always for English, irrespective of the locale.
\end{explainbox}

\begin{warningbox}
In \Rlang indexes always start from one, while in some other programming languages such as \Clang and \Cpplang, indexes start from zero. It is important to be aware of this difference, as many computation algorithms are valid only under a given indexing convention.
\end{warningbox}

It is possible to extract a subset of the elements of a vector in a single operation, using a vector of indexes. The positions of the extracted elements in the result (``returned value'') are determined by the ordering of the members of the vector of indexes---easier to demonstrate than to explain.

\begin{knitrout}\footnotesize
\definecolor{shadecolor}{rgb}{0.969, 0.969, 0.969}\color{fgcolor}\begin{kframe}
\begin{alltt}
\hlstd{a[}\hlkwd{c}\hlstd{(}\hlnum{3}\hlstd{,}\hlnum{2}\hlstd{)]}
\end{alltt}
\begin{verbatim}
## [1] "c" "b"
\end{verbatim}
\begin{alltt}
\hlstd{a[}\hlnum{10}\hlopt{:}\hlnum{1}\hlstd{]}
\end{alltt}
\begin{verbatim}
##  [1] "j" "i" "h" "g" "f" "e" "d" "c" "b" "a"
\end{verbatim}
\end{kframe}
\end{knitrout}

\begin{playground}
The length of the indexing vector is not restricted by the length of the indexed vector. However, only numerical indexes that match positions present in the indexed vector can extract values. Those values in the indexing vector pointing to positions that are not present in the indexed vector, result in \code{NA}s. This is easier to learn by \emph{playing} with \Rlang, than from explanations. Play with \Rlang, using the following examples as starting point.

\begin{knitrout}\footnotesize
\definecolor{shadecolor}{rgb}{0.969, 0.969, 0.969}\color{fgcolor}\begin{kframe}
\begin{alltt}
\hlkwd{length}\hlstd{(a)}
\hlstd{a[}\hlkwd{c}\hlstd{(}\hlnum{3}\hlstd{,}\hlnum{3}\hlstd{,}\hlnum{3}\hlstd{,}\hlnum{3}\hlstd{)]}
\hlstd{a[}\hlkwd{c}\hlstd{(}\hlnum{10}\hlopt{:}\hlnum{1}\hlstd{,} \hlnum{1}\hlopt{:}\hlnum{10}\hlstd{)]}
\hlstd{a[}\hlkwd{c}\hlstd{(}\hlnum{1}\hlstd{,}\hlnum{11}\hlstd{)]}
\hlstd{a[}\hlnum{11}\hlstd{]}
\end{alltt}
\end{kframe}
\end{knitrout}

Have you tried some of your own examples? If not yet, do \emph{play} with additional variations of your own before continuing.

\end{playground}

Negative indexes have a special meaning, they indicate the positions at which values should be excluded. Be aware that it is \emph{illegal} to mix positive and negative values in the same indexing operation.

\begin{knitrout}\footnotesize
\definecolor{shadecolor}{rgb}{0.969, 0.969, 0.969}\color{fgcolor}\begin{kframe}
\begin{alltt}
\hlstd{a[}\hlopt{-}\hlnum{2}\hlstd{]}
\end{alltt}
\begin{verbatim}
## [1] "a" "c" "d" "e" "f" "g" "h" "i" "j"
\end{verbatim}
\begin{alltt}
\hlstd{a[}\hlopt{-}\hlkwd{c}\hlstd{(}\hlnum{3}\hlstd{,}\hlnum{2}\hlstd{)]}
\end{alltt}
\begin{verbatim}
## [1] "a" "d" "e" "f" "g" "h" "i" "j"
\end{verbatim}
\begin{alltt}
\hlstd{a[}\hlopt{-}\hlnum{3}\hlopt{:-}\hlnum{2}\hlstd{]}
\end{alltt}
\begin{verbatim}
## [1] "a" "d" "e" "f" "g" "h" "i" "j"
\end{verbatim}
\begin{alltt}
\hlcom{# a[c(-3,2)]}
\end{alltt}
\end{kframe}
\end{knitrout}

\begin{playground}
Results from indexing with special values may be surprising. Try to build a logic explanation from the examples below, a logic that will help you remember what to expect next time you are confronted with similar statements---this is likely to happen sooner or later as these special values can be returned by different \Rlang expressions in certain circumstances, some of them described earlier in this chapter.

\begin{knitrout}\footnotesize
\definecolor{shadecolor}{rgb}{0.969, 0.969, 0.969}\color{fgcolor}\begin{kframe}
\begin{alltt}
\hlstd{a[ ]}
\hlstd{a[}\hlnum{0}\hlstd{]}
\hlstd{a[}\hlkwd{numeric}\hlstd{(}\hlnum{0}\hlstd{)]}
\hlstd{a[}\hlnum{NA}\hlstd{]}
\hlstd{a[}\hlkwd{c}\hlstd{(}\hlnum{1}\hlstd{,} \hlnum{NA}\hlstd{)]}
\hlstd{a[}\hlkwa{NULL}\hlstd{]}
\hlstd{a[}\hlkwd{c}\hlstd{(}\hlnum{1}\hlstd{,} \hlkwa{NULL}\hlstd{)]}
\end{alltt}
\end{kframe}
\end{knitrout}
\end{playground}

Another way of indexing, which is very handy, but not available in most other programming languages, is indexing with a vector of \code{logical} values. The \code{logical} vector used for `indexing' is usually of the same length as the vector from which elements are going to be selected. However, this is not a requirement, because if the \code{logical} vector of indexes is shorter than the indexed vector it is `recycled' as discussed above in relation to operators.

\begin{knitrout}\footnotesize
\definecolor{shadecolor}{rgb}{0.969, 0.969, 0.969}\color{fgcolor}\begin{kframe}
\begin{alltt}
\hlstd{a[}\hlnum{TRUE}\hlstd{]}
\end{alltt}
\begin{verbatim}
##  [1] "a" "b" "c" "d" "e" "f" "g" "h" "i" "j"
\end{verbatim}
\begin{alltt}
\hlstd{a[}\hlnum{FALSE}\hlstd{]}
\end{alltt}
\begin{verbatim}
## character(0)
\end{verbatim}
\begin{alltt}
\hlstd{a[}\hlkwd{c}\hlstd{(}\hlnum{TRUE}\hlstd{,} \hlnum{FALSE}\hlstd{)]}
\end{alltt}
\begin{verbatim}
## [1] "a" "c" "e" "g" "i"
\end{verbatim}
\begin{alltt}
\hlstd{a[}\hlkwd{c}\hlstd{(}\hlnum{FALSE}\hlstd{,} \hlnum{TRUE}\hlstd{)]}
\end{alltt}
\begin{verbatim}
## [1] "b" "d" "f" "h" "j"
\end{verbatim}
\begin{alltt}
\hlstd{a} \hlopt{>} \hlstr{"c"}
\end{alltt}
\begin{verbatim}
##  [1] FALSE FALSE FALSE  TRUE  TRUE  TRUE  TRUE  TRUE  TRUE  TRUE
\end{verbatim}
\begin{alltt}
\hlstd{a[a} \hlopt{>} \hlstr{"c"}\hlstd{]}
\end{alltt}
\begin{verbatim}
## [1] "d" "e" "f" "g" "h" "i" "j"
\end{verbatim}
\end{kframe}
\end{knitrout}

Indexing with logical vectors is very frequently used in \Rlang because comparison operators are vectorized. Comparison operators  when applied to a vector return a \code{logical} vector, a vector that can be used to extract the elements for which the result of the comparison test was \code{TRUE}.

\begin{playground}
The following examples demonstrate further uses of logical vectors: 1) the logical vector returned by a vectorized comparison can be stored in a variable, and the variable used as a ``selector'' for extracting a subset of values from the same vector, or from a different vector.

\begin{knitrout}\footnotesize
\definecolor{shadecolor}{rgb}{0.969, 0.969, 0.969}\color{fgcolor}\begin{kframe}
\begin{alltt}
\hlstd{a} \hlkwb{<-} \hlstd{letters[}\hlnum{1}\hlopt{:}\hlnum{10}\hlstd{]}
\hlstd{b} \hlkwb{<-} \hlnum{1}\hlopt{:}\hlnum{10}
\hlstd{selector} \hlkwb{<-} \hlstd{a} \hlopt{>} \hlstr{"c"}
\hlstd{selector}
\hlstd{a[selector]}
\hlstd{b[selector]}
\end{alltt}
\end{kframe}
\end{knitrout}

Numerical indexes can be obtained from a logical vector by means of function \code{which()}.

\begin{knitrout}\footnotesize
\definecolor{shadecolor}{rgb}{0.969, 0.969, 0.969}\color{fgcolor}\begin{kframe}
\begin{alltt}
\hlstd{indexes} \hlkwb{<-} \hlkwd{which}\hlstd{(a} \hlopt{>} \hlstr{"c"}\hlstd{)}
\hlstd{indexes}
\hlstd{a[indexes]}
\hlstd{b[indexes]}
\end{alltt}
\end{kframe}
\end{knitrout}

Make sure to understand the examples above. These constructs are very widely used in \Rlang because they allow for concise code that is easy to understand once you are familiar with the indexing rules. However, if you do not command these rules, many of these `terse' statements will be unintelligible to you.
\end{playground}

Indexing can be used on both sides of an assignment. This may look rather esoteric at first sight, but it is just a simple extension of the logic of indexing described above.

\begin{knitrout}\footnotesize
\definecolor{shadecolor}{rgb}{0.969, 0.969, 0.969}\color{fgcolor}\begin{kframe}
\begin{alltt}
\hlstd{a} \hlkwb{<-} \hlnum{1}\hlopt{:}\hlnum{10}
\hlstd{a}
\end{alltt}
\begin{verbatim}
##  [1]  1  2  3  4  5  6  7  8  9 10
\end{verbatim}
\begin{alltt}
\hlstd{a[}\hlnum{1}\hlstd{]} \hlkwb{<-} \hlnum{99}
\hlstd{a}
\end{alltt}
\begin{verbatim}
##  [1] 99  2  3  4  5  6  7  8  9 10
\end{verbatim}
\begin{alltt}
\hlstd{a[}\hlkwd{c}\hlstd{(}\hlnum{2}\hlstd{,}\hlnum{4}\hlstd{)]} \hlkwb{<-} \hlopt{-}\hlnum{99} \hlcom{# recycling}
\hlstd{a}
\end{alltt}
\begin{verbatim}
##  [1]  99 -99   3 -99   5   6   7   8   9  10
\end{verbatim}
\begin{alltt}
\hlstd{a[}\hlkwd{c}\hlstd{(}\hlnum{2}\hlstd{,}\hlnum{4}\hlstd{)]} \hlkwb{<-} \hlkwd{c}\hlstd{(}\hlopt{-}\hlnum{99}\hlstd{,} \hlnum{99}\hlstd{)}
\hlstd{a}
\end{alltt}
\begin{verbatim}
##  [1]  99 -99   3  99   5   6   7   8   9  10
\end{verbatim}
\begin{alltt}
\hlstd{a[}\hlnum{TRUE}\hlstd{]} \hlkwb{<-} \hlnum{1}
\hlstd{a}
\end{alltt}
\begin{verbatim}
##  [1] 1 1 1 1 1 1 1 1 1 1
\end{verbatim}
\begin{alltt}
\hlstd{a} \hlkwb{<-} \hlnum{1}
\hlstd{a}
\end{alltt}
\begin{verbatim}
## [1] 1
\end{verbatim}
\end{kframe}
\end{knitrout}

We can also have subscripting on both sides.

\begin{knitrout}\footnotesize
\definecolor{shadecolor}{rgb}{0.969, 0.969, 0.969}\color{fgcolor}\begin{kframe}
\begin{alltt}
\hlstd{a} \hlkwb{<-} \hlstd{letters[}\hlnum{1}\hlopt{:}\hlnum{10}\hlstd{]}
\hlstd{a}
\end{alltt}
\begin{verbatim}
##  [1] "a" "b" "c" "d" "e" "f" "g" "h" "i" "j"
\end{verbatim}
\begin{alltt}
\hlstd{a[}\hlnum{1}\hlstd{]} \hlkwb{<-} \hlstd{a[}\hlnum{10}\hlstd{]}
\hlstd{a}
\end{alltt}
\begin{verbatim}
##  [1] "j" "b" "c" "d" "e" "f" "g" "h" "i" "j"
\end{verbatim}
\begin{alltt}
\hlstd{a} \hlkwb{<-} \hlstd{a[}\hlnum{10}\hlopt{:}\hlnum{1}\hlstd{]}
\hlstd{a}
\end{alltt}
\begin{verbatim}
##  [1] "j" "i" "h" "g" "f" "e" "d" "c" "b" "j"
\end{verbatim}
\begin{alltt}
\hlstd{a[}\hlnum{10}\hlopt{:}\hlnum{1}\hlstd{]} \hlkwb{<-} \hlstd{a}
\hlstd{a}
\end{alltt}
\begin{verbatim}
##  [1] "j" "b" "c" "d" "e" "f" "g" "h" "i" "j"
\end{verbatim}
\begin{alltt}
\hlstd{a[}\hlnum{5}\hlopt{:}\hlnum{1}\hlstd{]} \hlkwb{<-} \hlstd{a[}\hlkwd{c}\hlstd{(}\hlnum{TRUE}\hlstd{,}\hlnum{FALSE}\hlstd{)]} \hlcom{# What?? It does work!}
\hlstd{a}
\end{alltt}
\begin{verbatim}
##  [1] "i" "g" "e" "c" "j" "f" "g" "h" "i" "j"
\end{verbatim}
\end{kframe}
\end{knitrout}

\begin{playground}
Do play with subscripts to your heart's content, really grasping how they work and how they can be used, will be very useful in anything you do in the future with \Rlang. Even the most contrived example above follows the same simple rules, just study it bit by bit.
\end{playground}

\begin{explainbox}\label{box:vec:sort}
In \Rlang indexing with positional indexes can be done with \Rclass{integer} or \Rclass{numeric} values. Numeric values can be ``floats'', but for indexing only integer values are meaningful. Consequently, \Rclass{double} values are converted into \code{integer} values when used as indexes. The conversion is done invisibly, but it does slow down computations slightly. When working on big data sets, explicitly using \code{integer} values can improve performance.

\begin{knitrout}\footnotesize
\definecolor{shadecolor}{rgb}{0.969, 0.969, 0.969}\color{fgcolor}\begin{kframe}
\begin{alltt}
\hlstd{b} \hlkwb{<-} \hlstd{LETTERS[}\hlnum{1}\hlopt{:}\hlnum{10}\hlstd{]}
\hlstd{b[}\hlnum{1}\hlstd{]}
\end{alltt}
\begin{verbatim}
## [1] "A"
\end{verbatim}
\begin{alltt}
\hlstd{b[}\hlnum{1.1}\hlstd{]}
\end{alltt}
\begin{verbatim}
## [1] "A"
\end{verbatim}
\begin{alltt}
\hlstd{b[}\hlnum{1.9999}\hlstd{]} \hlcom{# surprise!!}
\end{alltt}
\begin{verbatim}
## [1] "A"
\end{verbatim}
\begin{alltt}
\hlstd{b[}\hlnum{2}\hlstd{]}
\end{alltt}
\begin{verbatim}
## [1] "B"
\end{verbatim}
\end{kframe}
\end{knitrout}

From this experiment, we can learn that if indexes are not whole numbers, they are truncated to the next smaller integer. This example, also shows how one can tease out of \Rlang its rules through experimentation.

\end{explainbox}

A\index{vectors!sorting} frequent operation on vectors is sorting them into an increasing or decreasing ordering. The most direct approach it to use \Rfunction{sort()}.

\begin{knitrout}\footnotesize
\definecolor{shadecolor}{rgb}{0.969, 0.969, 0.969}\color{fgcolor}\begin{kframe}
\begin{alltt}
\hlstd{my.vector} \hlkwb{<-} \hlkwd{c}\hlstd{(}\hlnum{10}\hlstd{,} \hlnum{4}\hlstd{,} \hlnum{22}\hlstd{,} \hlnum{1}\hlstd{,} \hlnum{4}\hlstd{)}
\hlkwd{sort}\hlstd{(my.vector)}
\end{alltt}
\begin{verbatim}
## [1]  1  4  4 10 22
\end{verbatim}
\begin{alltt}
\hlkwd{sort}\hlstd{(my.vector,} \hlkwc{decreasing} \hlstd{=} \hlnum{TRUE}\hlstd{)}
\end{alltt}
\begin{verbatim}
## [1] 22 10  4  4  1
\end{verbatim}
\end{kframe}
\end{knitrout}

An indirect way of sorting a vector, possibly based on a different vector, is to generate with \Rfunction{order()} a vector of numerical indexes that can be used to achieve the ordering.

\begin{knitrout}\footnotesize
\definecolor{shadecolor}{rgb}{0.969, 0.969, 0.969}\color{fgcolor}\begin{kframe}
\begin{alltt}
\hlkwd{order}\hlstd{(my.vector)}
\end{alltt}
\begin{verbatim}
## [1] 4 2 5 1 3
\end{verbatim}
\begin{alltt}
\hlstd{my.vector[}\hlkwd{order}\hlstd{(my.vector)]}
\end{alltt}
\begin{verbatim}
## [1]  1  4  4 10 22
\end{verbatim}
\begin{alltt}
\hlstd{another.vector} \hlkwb{<-} \hlkwd{c}\hlstd{(}\hlstr{"ab"}\hlstd{,} \hlstr{"aa"}\hlstd{,} \hlstr{"c"}\hlstd{,} \hlstr{"zy"}\hlstd{,} \hlstr{"e"}\hlstd{)}
\hlstd{another.vector[}\hlkwd{order}\hlstd{(my.vector)]}
\end{alltt}
\begin{verbatim}
## [1] "zy" "aa" "e"  "ab" "c"
\end{verbatim}
\end{kframe}
\end{knitrout}

\begin{explainbox}
A problem linked to sorting that we may face is counting how many copies of each value are present in a vector. We need to use two functions \Rfunction{sort()} and \Rfunction{rle()}\index{vector!run length encoding}. The second of these functions computes \emph{run length} as used in \emph{run length encoding} of which \emph{rle} is an abbreviation. A \emph{run} is a series of consecutive identical values. As the objective is to count the number of copies of each value present, we need first to sort the vector.

\begin{knitrout}\footnotesize
\definecolor{shadecolor}{rgb}{0.969, 0.969, 0.969}\color{fgcolor}\begin{kframe}
\begin{alltt}
\hlstd{my.letters} \hlkwb{<-} \hlstd{letters[}\hlkwd{c}\hlstd{(}\hlnum{1}\hlstd{,}\hlnum{5}\hlstd{,}\hlnum{10}\hlstd{,}\hlnum{3}\hlstd{,}\hlnum{1}\hlstd{,}\hlnum{4}\hlstd{,}\hlnum{21}\hlstd{,}\hlnum{1}\hlstd{,}\hlnum{10}\hlstd{)]}
\hlstd{my.letters}
\end{alltt}
\begin{verbatim}
## [1] "a" "e" "j" "c" "a" "d" "u" "a" "j"
\end{verbatim}
\begin{alltt}
\hlkwd{sort}\hlstd{(my.letters)}
\end{alltt}
\begin{verbatim}
## [1] "a" "a" "a" "c" "d" "e" "j" "j" "u"
\end{verbatim}
\begin{alltt}
\hlkwd{rle}\hlstd{(}\hlkwd{sort}\hlstd{(my.letters))}
\end{alltt}
\begin{verbatim}
## Run Length Encoding
##   lengths: int [1:6] 3 1 1 1 2 1
##   values : chr [1:6] "a" "c" "d" "e" "j" "u"
\end{verbatim}
\end{kframe}
\end{knitrout}

The second and third statements are only to demonstrate what is the effect of each step. The last statement uses nested function calls to compute the number of copies of each value in the vector.
\end{explainbox}



\index{vectors!indexing|)}

\section{Matrices and multidimensional arrays}\label{sec:matrix:array}
\index{matrices|(}\index{arrays|(}\qRclass{matrix}\qRclass{array}

Vectors have a single dimension, and, as we saw above we can query their length with method \Rfunction{length()}. Matrices have two dimensions, which can be queried with \Rfunction{dim()}, \Rfunction{ncol()} and \Rfunction{nrow()}. \Rlang arrays can have any number of dimensions, even a single dimension, which can be queried with method \Rfunction{dim()}. As expected \Rfunction{is.vector()}, \Rfunction{is.matrix()} and \Rfunction{is.array()} can be used to query the class.

We can create a new matrix using the \Rfunction{matrix()} or \Rfunction{as.matrix()} constructors. The first argument of \Rfunction{matrix()} is a vector. In the same way as vectors, matrices are homogeneous, all elements are of the same type.

\begin{knitrout}\footnotesize
\definecolor{shadecolor}{rgb}{0.969, 0.969, 0.969}\color{fgcolor}\begin{kframe}
\begin{alltt}
\hlkwd{matrix}\hlstd{(}\hlnum{1}\hlopt{:}\hlnum{15}\hlstd{,} \hlkwc{ncol} \hlstd{=} \hlnum{3}\hlstd{)}
\end{alltt}
\begin{verbatim}
##      [,1] [,2] [,3]
## [1,]    1    6   11
## [2,]    2    7   12
## [3,]    3    8   13
## [4,]    4    9   14
## [5,]    5   10   15
\end{verbatim}
\begin{alltt}
\hlkwd{matrix}\hlstd{(}\hlnum{1}\hlopt{:}\hlnum{15}\hlstd{,} \hlkwc{nrow} \hlstd{=} \hlnum{3}\hlstd{)}
\end{alltt}
\begin{verbatim}
##      [,1] [,2] [,3] [,4] [,5]
## [1,]    1    4    7   10   13
## [2,]    2    5    8   11   14
## [3,]    3    6    9   12   15
\end{verbatim}
\end{kframe}
\end{knitrout}

When a vector is converted to a matrix, \Rlang's default is to allocate the values in the vector to the matrix starting from the leftmost column, and within the column, down from the top. Once the first column is filled, the process continues from the top of the next column, as can be seen above. This order can be changed as you will discover in the playground below.

\begin{playground}
Check in the help page for the \code{matrix}\qRfunction{matrix()} constructor how to use the \code{byrow} parameter to alter the default order in which the elements of the vector are allocated to columns and rows of the new matrix.

\begin{knitrout}\footnotesize
\definecolor{shadecolor}{rgb}{0.969, 0.969, 0.969}\color{fgcolor}\begin{kframe}
\begin{alltt}
\hlkwd{help}\hlstd{(matrix)}
\end{alltt}
\end{kframe}
\end{knitrout}

While you are looking at the help page, also consider the default number of columns and rows.

\begin{knitrout}\footnotesize
\definecolor{shadecolor}{rgb}{0.969, 0.969, 0.969}\color{fgcolor}\begin{kframe}
\begin{alltt}
\hlkwd{matrix}\hlstd{(}\hlnum{1}\hlopt{:}\hlnum{15}\hlstd{)}
\end{alltt}
\end{kframe}
\end{knitrout}

And to start getting a sense of how to interpret error messages, try the code below and make sure you understand what is the problem.

\begin{knitrout}\footnotesize
\definecolor{shadecolor}{rgb}{0.969, 0.969, 0.969}\color{fgcolor}\begin{kframe}
\begin{alltt}
\hlkwd{matrix}\hlstd{(}\hlnum{1}\hlopt{:}\hlnum{15}\hlstd{,} \hlkwc{ncol} \hlstd{=} \hlnum{2}\hlstd{)}
\end{alltt}
\end{kframe}
\end{knitrout}

\end{playground}

Subscripting of matrices and arrays is consistent with that used for vectors, we only need to supply an indexing vector, or leave a blank space, for each dimension. A matrix has two dimensions, so to access any element or group of elements, we use two indices. The only complication is that there are two possible orders in which, in principle, indexes could be supplied. In \Rlang, indexes for matrices are written ``row-first'', in simpler words the first index value selects rows, and the second one columns.

\begin{knitrout}\footnotesize
\definecolor{shadecolor}{rgb}{0.969, 0.969, 0.969}\color{fgcolor}\begin{kframe}
\begin{alltt}
\hlstd{A} \hlkwb{<-} \hlkwd{matrix}\hlstd{(}\hlnum{1}\hlopt{:}\hlnum{20}\hlstd{,} \hlkwc{ncol} \hlstd{=} \hlnum{4}\hlstd{)}
\hlstd{A}
\end{alltt}
\begin{verbatim}
##      [,1] [,2] [,3] [,4]
## [1,]    1    6   11   16
## [2,]    2    7   12   17
## [3,]    3    8   13   18
## [4,]    4    9   14   19
## [5,]    5   10   15   20
\end{verbatim}
\begin{alltt}
\hlstd{A[}\hlnum{1}\hlstd{,} \hlnum{1}\hlstd{]}
\end{alltt}
\begin{verbatim}
## [1] 1
\end{verbatim}
\end{kframe}
\end{knitrout}

Remind yourself of how indexing of vectors works in \Rlang (see section \ref{sec:vectors} on page \pageref{sec:vectors}). We will now apply the same rules in two dimensions.

\begin{knitrout}\footnotesize
\definecolor{shadecolor}{rgb}{0.969, 0.969, 0.969}\color{fgcolor}\begin{kframe}
\begin{alltt}
\hlstd{A[}\hlnum{1}\hlstd{, ]}
\end{alltt}
\begin{verbatim}
## [1]  1  6 11 16
\end{verbatim}
\begin{alltt}
\hlstd{A[ ,} \hlnum{1}\hlstd{]}
\end{alltt}
\begin{verbatim}
## [1] 1 2 3 4 5
\end{verbatim}
\begin{alltt}
\hlstd{A[}\hlnum{2}\hlopt{:}\hlnum{3}\hlstd{,} \hlkwd{c}\hlstd{(}\hlnum{1}\hlstd{,}\hlnum{3}\hlstd{)]}
\end{alltt}
\begin{verbatim}
##      [,1] [,2]
## [1,]    2   12
## [2,]    3   13
\end{verbatim}
\begin{alltt}
\hlstd{A[}\hlnum{3}\hlstd{,} \hlnum{4}\hlstd{]} \hlkwb{<-} \hlnum{99}
\hlstd{A}
\end{alltt}
\begin{verbatim}
##      [,1] [,2] [,3] [,4]
## [1,]    1    6   11   16
## [2,]    2    7   12   17
## [3,]    3    8   13   99
## [4,]    4    9   14   19
## [5,]    5   10   15   20
\end{verbatim}
\begin{alltt}
\hlstd{A[}\hlnum{4}\hlopt{:}\hlnum{3}\hlstd{,} \hlnum{2}\hlopt{:}\hlnum{1}\hlstd{]} \hlkwb{<-} \hlstd{A[}\hlnum{3}\hlopt{:}\hlnum{4}\hlstd{,} \hlnum{1}\hlopt{:}\hlnum{2}\hlstd{]}
\hlstd{A}
\end{alltt}
\begin{verbatim}
##      [,1] [,2] [,3] [,4]
## [1,]    1    6   11   16
## [2,]    2    7   12   17
## [3,]    9    4   13   99
## [4,]    8    3   14   19
## [5,]    5   10   15   20
\end{verbatim}
\end{kframe}
\end{knitrout}

\begin{explainbox}
In \Rlang, a \Rclass{matrix} can have a single row, a single column, a single element or no elements. However, in all cases a \code{matrix} will have \emph{dimensions} of length two defined and stored as an attribute.

\begin{knitrout}\footnotesize
\definecolor{shadecolor}{rgb}{0.969, 0.969, 0.969}\color{fgcolor}\begin{kframe}
\begin{alltt}
\hlstd{my.vector} \hlkwb{<-} \hlnum{1}\hlopt{:}\hlnum{6}
\hlkwd{dim}\hlstd{(my.vector)}
\end{alltt}
\begin{verbatim}
## NULL
\end{verbatim}
\end{kframe}
\end{knitrout}

\begin{knitrout}\footnotesize
\definecolor{shadecolor}{rgb}{0.969, 0.969, 0.969}\color{fgcolor}\begin{kframe}
\begin{alltt}
\hlstd{one.col.matrix} \hlkwb{<-} \hlkwd{matrix}\hlstd{(}\hlnum{1}\hlopt{:}\hlnum{6}\hlstd{,} \hlkwc{ncol} \hlstd{=} \hlnum{1}\hlstd{)}
\hlkwd{dim}\hlstd{(one.col.matrix)}
\end{alltt}
\begin{verbatim}
## [1] 6 1
\end{verbatim}
\begin{alltt}
\hlstd{two.col.matrix} \hlkwb{<-} \hlkwd{matrix}\hlstd{(}\hlnum{1}\hlopt{:}\hlnum{6}\hlstd{,} \hlkwc{ncol} \hlstd{=} \hlnum{2}\hlstd{)}
\hlkwd{dim}\hlstd{(two.col.matrix)}
\end{alltt}
\begin{verbatim}
## [1] 3 2
\end{verbatim}
\begin{alltt}
\hlstd{one.elem.matrix} \hlkwb{<-} \hlkwd{matrix}\hlstd{(}\hlnum{1}\hlstd{,} \hlkwc{ncol} \hlstd{=} \hlnum{1}\hlstd{)}
\hlkwd{dim}\hlstd{(one.elem.matrix)}
\end{alltt}
\begin{verbatim}
## [1] 1 1
\end{verbatim}
\begin{alltt}
\hlstd{no.elem.matrix} \hlkwb{<-} \hlkwd{matrix}\hlstd{(}\hlkwd{numeric}\hlstd{(),} \hlkwc{ncol} \hlstd{=} \hlnum{0}\hlstd{)}
\hlkwd{dim}\hlstd{(no.elem.matrix)}
\end{alltt}
\begin{verbatim}
## [1] 0 0
\end{verbatim}
\end{kframe}
\end{knitrout}

\end{explainbox}

Arrays\index{matrix!dimensions}\index{arrays!dimensions} are similar to matrices, but can have more than two dimensions, which are specified with the \code{dim} argument to the \Rfunction{array()} constructor.

\begin{knitrout}\footnotesize
\definecolor{shadecolor}{rgb}{0.969, 0.969, 0.969}\color{fgcolor}\begin{kframe}
\begin{alltt}
\hlstd{B} \hlkwb{<-} \hlkwd{array}\hlstd{(}\hlnum{1}\hlopt{:}\hlnum{27}\hlstd{,} \hlkwc{dim} \hlstd{=} \hlkwd{c}\hlstd{(}\hlnum{3}\hlstd{,} \hlnum{3}\hlstd{,} \hlnum{3}\hlstd{))}
\hlstd{B}
\end{alltt}
\begin{verbatim}
## , , 1
## 
##      [,1] [,2] [,3]
## [1,]    1    4    7
## [2,]    2    5    8
## [3,]    3    6    9
## 
## , , 2
## 
##      [,1] [,2] [,3]
## [1,]   10   13   16
## [2,]   11   14   17
## [3,]   12   15   18
## 
## , , 3
## 
##      [,1] [,2] [,3]
## [1,]   19   22   25
## [2,]   20   23   26
## [3,]   21   24   27
\end{verbatim}
\begin{alltt}
\hlstd{B[}\hlnum{2}\hlstd{,} \hlnum{2}\hlstd{,} \hlnum{2}\hlstd{]}
\end{alltt}
\begin{verbatim}
## [1] 14
\end{verbatim}
\end{kframe}
\end{knitrout}

In the chunk above, the length of the supplied vector is the product of the dimensions, $27 = 3 \times 3 \times 3$.

\begin{playground}
  How do you use indexes to extract the second element of the original vector, in each of the following matrices and arrays?

\begin{knitrout}\footnotesize
\definecolor{shadecolor}{rgb}{0.969, 0.969, 0.969}\color{fgcolor}\begin{kframe}
\begin{alltt}
\hlstd{v} \hlkwb{<-} \hlnum{1}\hlopt{:}\hlnum{10}
\hlstd{m2c} \hlkwb{<-} \hlkwd{matrix}\hlstd{(v,} \hlkwc{ncol} \hlstd{=} \hlnum{2}\hlstd{)}
\hlstd{m2cr} \hlkwb{<-} \hlkwd{matrix}\hlstd{(v,} \hlkwc{ncol} \hlstd{=} \hlnum{2}\hlstd{,} \hlkwc{byrow} \hlstd{=} \hlnum{TRUE}\hlstd{)}
\hlstd{m2r} \hlkwb{<-} \hlkwd{matrix}\hlstd{(v,} \hlkwc{nrow} \hlstd{=} \hlnum{2}\hlstd{)}
\hlstd{m2rc} \hlkwb{<-} \hlkwd{matrix}\hlstd{(v,} \hlkwc{nrow} \hlstd{=} \hlnum{2}\hlstd{,} \hlkwc{byrow} \hlstd{=} \hlnum{TRUE}\hlstd{)}
\end{alltt}
\end{kframe}
\end{knitrout}

\begin{knitrout}\footnotesize
\definecolor{shadecolor}{rgb}{0.969, 0.969, 0.969}\color{fgcolor}\begin{kframe}
\begin{alltt}
\hlstd{v} \hlkwb{<-} \hlnum{1}\hlopt{:}\hlnum{10}
\hlstd{a2c} \hlkwb{<-} \hlkwd{array}\hlstd{(v,} \hlkwc{dim} \hlstd{=} \hlkwd{c}\hlstd{(}\hlnum{5}\hlstd{,} \hlnum{2}\hlstd{))}
\hlstd{a2c} \hlkwb{<-} \hlkwd{array}\hlstd{(v,} \hlkwc{dim} \hlstd{=} \hlkwd{c}\hlstd{(}\hlnum{5}\hlstd{,} \hlnum{2}\hlstd{),} \hlkwc{dimnames} \hlstd{=} \hlkwd{list}\hlstd{(}\hlkwa{NULL}\hlstd{,} \hlkwd{c}\hlstd{(}\hlstr{"c1"}\hlstd{,} \hlstr{"c2"}\hlstd{)))}
\hlstd{a2r} \hlkwb{<-} \hlkwd{array}\hlstd{(v,} \hlkwc{dim} \hlstd{=} \hlkwd{c}\hlstd{(}\hlnum{2}\hlstd{,} \hlnum{5}\hlstd{))}
\end{alltt}
\end{kframe}
\end{knitrout}

Be aware that vectors and one-dimensional arrays are not the same thing, while two dimensional arrays are matrices.
\begin{enumerate}
      \item Use the different constructors and query methods to explore this, and its consequences.
      \item Convert a matrix into a vector using \Rfunction{unlist()} and \Rfunction{as.vector()} and compare the returned values.
\end{enumerate}

\end{playground}

Operators for matrices are available in \Rlang, as matrices are used in many statistical algorithms. We will not describe them all here, only \Rfunction{t()} and some specializations of arithmetic operators. Function \Rfunction{t()} transposes a matrix, by swapping columns and rows.

\begin{knitrout}\footnotesize
\definecolor{shadecolor}{rgb}{0.969, 0.969, 0.969}\color{fgcolor}\begin{kframe}
\begin{alltt}
\hlstd{A} \hlkwb{<-} \hlkwd{matrix}\hlstd{(}\hlnum{1}\hlopt{:}\hlnum{20}\hlstd{,} \hlkwc{ncol} \hlstd{=} \hlnum{4}\hlstd{)}
\hlstd{A}
\end{alltt}
\begin{verbatim}
##      [,1] [,2] [,3] [,4]
## [1,]    1    6   11   16
## [2,]    2    7   12   17
## [3,]    3    8   13   18
## [4,]    4    9   14   19
## [5,]    5   10   15   20
\end{verbatim}
\begin{alltt}
\hlkwd{t}\hlstd{(A)}
\end{alltt}
\begin{verbatim}
##      [,1] [,2] [,3] [,4] [,5]
## [1,]    1    2    3    4    5
## [2,]    6    7    8    9   10
## [3,]   11   12   13   14   15
## [4,]   16   17   18   19   20
\end{verbatim}
\end{kframe}
\end{knitrout}

As with vectors, recycling applies to arithmetic operators when applied to matrices.

\begin{knitrout}\footnotesize
\definecolor{shadecolor}{rgb}{0.969, 0.969, 0.969}\color{fgcolor}\begin{kframe}
\begin{alltt}
\hlstd{A} \hlopt{+} \hlnum{2}
\end{alltt}
\begin{verbatim}
##      [,1] [,2] [,3] [,4]
## [1,]    3    8   13   18
## [2,]    4    9   14   19
## [3,]    5   10   15   20
## [4,]    6   11   16   21
## [5,]    7   12   17   22
\end{verbatim}
\begin{alltt}
\hlstd{A} \hlopt{*} \hlnum{0}\hlopt{:}\hlnum{1}
\end{alltt}
\begin{verbatim}
##      [,1] [,2] [,3] [,4]
## [1,]    0    6    0   16
## [2,]    2    0   12    0
## [3,]    0    8    0   18
## [4,]    4    0   14    0
## [5,]    0   10    0   20
\end{verbatim}
\begin{alltt}
\hlstd{A} \hlopt{*} \hlnum{1}\hlopt{:}\hlnum{0}
\end{alltt}
\begin{verbatim}
##      [,1] [,2] [,3] [,4]
## [1,]    1    0   11    0
## [2,]    0    7    0   17
## [3,]    3    0   13    0
## [4,]    0    9    0   19
## [5,]    5    0   15    0
\end{verbatim}
\end{kframe}
\end{knitrout}

In the examples above with the usual multiplication operator \code{*}, the operation described is not a matrix product, but instead the products between individual elements of the matrix and vectors. Matrix multiplication is indicated by operator \Roperator{\%*\%}.

\begin{knitrout}\footnotesize
\definecolor{shadecolor}{rgb}{0.969, 0.969, 0.969}\color{fgcolor}\begin{kframe}
\begin{alltt}
\hlstd{B} \hlkwb{<-} \hlkwd{matrix}\hlstd{(}\hlnum{1}\hlopt{:}\hlnum{16}\hlstd{,} \hlkwc{ncol} \hlstd{=} \hlnum{4}\hlstd{)}
\hlstd{B} \hlopt{*} \hlstd{B}
\end{alltt}
\begin{verbatim}
##      [,1] [,2] [,3] [,4]
## [1,]    1   25   81  169
## [2,]    4   36  100  196
## [3,]    9   49  121  225
## [4,]   16   64  144  256
\end{verbatim}
\begin{alltt}
\hlstd{B} \hlopt \hlstd{B}
\end{alltt}
\begin{verbatim}
##      [,1] [,2] [,3] [,4]
## [1,]   90  202  314  426
## [2,]  100  228  356  484
## [3,]  110  254  398  542
## [4,]  120  280  440  600
\end{verbatim}
\end{kframe}
\end{knitrout}

Other operators and functions for matrix algebra like cross-product (\Rfunction{crossprod()}), extracting or replacing the diagonal (\Rfunction{diag()}) are available in base R. Packages, including \pkgname{matrixStats}, provide additional functions and operators for matrices.



\index{matrices|)}\index{arrays|)}

\section{Factors}\label{sec:calc:factors}
\index{factors|(}
\index{categorical variables|see{factors}}\qRclass{factor}
Factors are used to indicate categories, most frequently the factors describing the treatments in an experiment, or categories in a survey. They can be created either from numerical or character vectors. The different possible values are called \emph{levels}. Normal factors created with \Rfunction{factor()} are unordered or categorical. \Rlang also supports ordered factors that can be created with function \Rfunction{ordered()}.\index{factors!ordered}

\begin{knitrout}\footnotesize
\definecolor{shadecolor}{rgb}{0.969, 0.969, 0.969}\color{fgcolor}\begin{kframe}
\begin{alltt}
\hlstd{my.vector} \hlkwb{<-} \hlkwd{c}\hlstd{(}\hlstr{"treated"}\hlstd{,} \hlstr{"treated"}\hlstd{,} \hlstr{"control"}\hlstd{,} \hlstr{"control"}\hlstd{,} \hlstr{"control"}\hlstd{,} \hlstr{"treated"}\hlstd{)}
\hlstd{my.factor} \hlkwb{<-} \hlkwd{factor}\hlstd{(my.vector)}
\hlstd{my.factor}
\end{alltt}
\begin{verbatim}
## [1] treated treated control control control treated
## Levels: control treated
\end{verbatim}
\begin{alltt}
\hlstd{my.factor} \hlkwb{<-} \hlkwd{factor}\hlstd{(}\hlkwc{x} \hlstd{= my.vector,} \hlkwc{levels} \hlstd{=} \hlkwd{c}\hlstd{(}\hlstr{"treated"}\hlstd{,} \hlstr{"control"}\hlstd{))}
\hlstd{my.factor}
\end{alltt}
\begin{verbatim}
## [1] treated treated control control control treated
## Levels: treated control
\end{verbatim}
\end{kframe}
\end{knitrout}

The\index{factors!labels}\index{factors!levels} labels (``names'') of the levels can be set when the factor is created. In this case, both \code{levels} and \code{labels} should be set, and levels and matching labels must be in the same position of the two argument vectors. The argument passed to \code{levels} determines the order of the levels, and the argument passed to \code{labels} gives new names to the levels. The argument passed to \code{levels} must contain the same values as present in the vector passed as first argument.

\begin{knitrout}\footnotesize
\definecolor{shadecolor}{rgb}{0.969, 0.969, 0.969}\color{fgcolor}\begin{kframe}
\begin{alltt}
\hlstd{my.vector} \hlkwb{<-} \hlkwd{c}\hlstd{(}\hlnum{1}\hlstd{,} \hlnum{1}\hlstd{,} \hlnum{0}\hlstd{,} \hlnum{0}\hlstd{,} \hlnum{0}\hlstd{,} \hlnum{1}\hlstd{)}
\hlstd{my.factor} \hlkwb{<-} \hlkwd{factor}\hlstd{(}\hlkwc{x} \hlstd{= my.vector,} \hlkwc{levels} \hlstd{=} \hlkwd{c}\hlstd{(}\hlnum{1}\hlstd{,} \hlnum{0}\hlstd{),} \hlkwc{labels} \hlstd{=} \hlkwd{c}\hlstd{(}\hlstr{"treated"}\hlstd{,} \hlstr{"control"}\hlstd{))}
\hlstd{my.factor}
\end{alltt}
\begin{verbatim}
## [1] treated treated control control control treated
## Levels: treated control
\end{verbatim}
\end{kframe}
\end{knitrout}

It is always preferable to use meaningful labels for levels, although it is also possible to use numbers.

In the examples above we passed a numeric vector or a character vector as argument for parameter \code{x} of function \Rfunction{factor()}. It is also possible to pass a \code{factor} as argument for parameter \code{x}. We use indexing with a test returning a logical vector to extract all ``controls''. We use function \Rfunction{levels()} to look at the levels of the factors.

\begin{knitrout}\footnotesize
\definecolor{shadecolor}{rgb}{0.969, 0.969, 0.969}\color{fgcolor}\begin{kframe}
\begin{alltt}
\hlkwd{levels}\hlstd{(my.factor)}
\end{alltt}
\begin{verbatim}
## [1] "treated" "control"
\end{verbatim}
\begin{alltt}
\hlstd{control.factor} \hlkwb{<-} \hlstd{my.factor[my.factor} \hlopt{==} \hlstr{"control"}\hlstd{]}
\hlkwd{levels}\hlstd{(control.factor)}
\end{alltt}
\begin{verbatim}
## [1] "treated" "control"
\end{verbatim}
\begin{alltt}
\hlstd{control.factor} \hlkwb{<-} \hlkwd{factor}\hlstd{(control.factor)}
\hlkwd{levels}\hlstd{(control.factor)}
\end{alltt}
\begin{verbatim}
## [1] "control"
\end{verbatim}
\end{kframe}
\end{knitrout}

It can be seen above that subsetting does not drop unused factor levels, and that \code{factor()} can be used to explicitly drop the unused factor levels.\index{factors!drop unused levels}

\begin{explainbox}
When the pattern of levels is regular it is possible to use function \Rfunction{gl()} to \emph{generate levels} in a factor. Nowadays, it is more usual to read data into \Rlang from files in which the treatment codes are already available as character strings or numeric values, however, when we need to create a factor within R, \Rfunction{gl()} can save some typing.

\begin{knitrout}\footnotesize
\definecolor{shadecolor}{rgb}{0.969, 0.969, 0.969}\color{fgcolor}\begin{kframe}
\begin{alltt}
\hlkwd{gl}\hlstd{(}\hlnum{2}\hlstd{,} \hlnum{5}\hlstd{,} \hlkwc{labels} \hlstd{=} \hlkwd{c}\hlstd{(}\hlstr{"A"}\hlstd{,} \hlstr{"B"}\hlstd{))}
\end{alltt}
\begin{verbatim}
##  [1] A A A A A B B B B B
## Levels: A B
\end{verbatim}
\end{kframe}
\end{knitrout}
\end{explainbox}

Converting factors into numbers is not intuitive, even in the case a factor was created from a \code{numeric} vector.

\begin{knitrout}\footnotesize
\definecolor{shadecolor}{rgb}{0.969, 0.969, 0.969}\color{fgcolor}\begin{kframe}
\begin{alltt}
\hlstd{my.vector2} \hlkwb{<-} \hlkwd{rep}\hlstd{(}\hlnum{3}\hlopt{:}\hlnum{5}\hlstd{,} \hlnum{4}\hlstd{)}
\hlstd{my.factor2} \hlkwb{<-} \hlkwd{factor}\hlstd{(my.vector2)}
\hlkwd{as.numeric}\hlstd{(my.factor2)}
\end{alltt}
\begin{verbatim}
##  [1] 1 2 3 1 2 3 1 2 3 1 2 3
\end{verbatim}
\begin{alltt}
\hlkwd{as.numeric}\hlstd{(}\hlkwd{as.character}\hlstd{(my.factor2))}
\end{alltt}
\begin{verbatim}
##  [1] 3 4 5 3 4 5 3 4 5 3 4 5
\end{verbatim}
\end{kframe}
\end{knitrout}

\begin{explainbox}
\textbf{Why is a double conversion needed?}\index{factors!convert to numeric} Internally factor levels are stored as running integers starting from one, and those are the numbers returned by \Rfunction{as.numeric()} when applied to a factor. The labels of the factor levels are always stored as character strings, even when these characters are digits. In contrast to \Rfunction{as.numeric()}, \Rfunction{as.character()} returns the character labels of the levels. If these character strings represent numbers, they can be converted, in a second step, using \Rfunction{as.numeric()} into the original numeric values.

\begin{knitrout}\footnotesize
\definecolor{shadecolor}{rgb}{0.969, 0.969, 0.969}\color{fgcolor}\begin{kframe}
\begin{alltt}
\hlkwd{class}\hlstd{(my.factor2)}
\end{alltt}
\begin{verbatim}
## [1] "factor"
\end{verbatim}
\begin{alltt}
\hlkwd{mode}\hlstd{(my.factor2)}
\end{alltt}
\begin{verbatim}
## [1] "numeric"
\end{verbatim}
\end{kframe}
\end{knitrout}
\end{explainbox}

\begin{playground}
Create a factor with levels labeled with words. Create another factor with the levels labeled with the same words, but ordered differently. After this convert both factors to numeric vectors using \Rfunction{as.numeric()}. Explain why the two numeric vectors differ or not from each other.
\end{playground}

Factors are very important in \Rlang. In contrast to other statistical software in which the role of a variable is set when defining a model to be fitted or when setting up a test, in \Rlang models are specified exactly in the same way for ANOVA and regression analysis, as \emph{linear models}. What `decides' what type of model is fitted is whether the explanatory variable is a factor (giving ANOVA) or a numerical variable (giving regression). This makes a lot of sense, as in most cases, considering an explanatory variable as categorical or not, depends on the design of the experiment or survey, in other words, is a property of the data and the experiment or survey that gave origin to them, rather than of the data analysis.

The order of the levels in a \code{factor} does not affect simple calculations or the values plotted, but it does affect how the output is printed, the order of the levels in the scales of plots, and in some cases the contrasts in significance tests. The default ordering is alphabetical, and is established at the time a factor is created. Consequently, rather frequently the default ordering of levels is not the one needed. As shown above, parameter \code{levels} in the constructor makes it possible to set the order of the levels. It is also possible to change the ordering of an existing factor.

\begin{explainbox}
\textbf{Renaming factor levels.}\index{factors!rename levels} Most direct way is using \Rfunction{levels()<-} as shown below, but it is also possible to use \Rfunction{factor()}. The difference is that \code{factor()} drops the unused levels and \code{levels()} only renames existing levels. All of them by position. (Although we here use \code{character} strings only one character long, longer strings can also be set as labels in exactly the same way.

\begin{knitrout}\footnotesize
\definecolor{shadecolor}{rgb}{0.969, 0.969, 0.969}\color{fgcolor}\begin{kframe}
\begin{alltt}
\hlstd{my.factor1} \hlkwb{<-} \hlkwd{gl}\hlstd{(}\hlnum{4}\hlstd{,} \hlnum{3}\hlstd{,} \hlkwc{labels} \hlstd{=} \hlkwd{c}\hlstd{(}\hlstr{"A"}\hlstd{,} \hlstr{"F"}\hlstd{,} \hlstr{"B"}\hlstd{,} \hlstr{"Z"}\hlstd{))}
\hlstd{my.factor1}
\end{alltt}
\begin{verbatim}
##  [1] A A A F F F B B B Z Z Z
## Levels: A F B Z
\end{verbatim}
\begin{alltt}
\hlkwd{levels}\hlstd{(my.factor1)} \hlkwb{<-} \hlkwd{c}\hlstd{(}\hlstr{"a"}\hlstd{,} \hlstr{"b"}\hlstd{,} \hlstr{"c"}\hlstd{,} \hlstr{"d"}\hlstd{)}
\hlstd{my.factor1}
\end{alltt}
\begin{verbatim}
##  [1] a a a b b b c c c d d d
## Levels: a b c d
\end{verbatim}
\end{kframe}
\end{knitrout}

Or more safely by matching names---i.e.\ order in the list of replacement values is irrelevant.

\begin{knitrout}\footnotesize
\definecolor{shadecolor}{rgb}{0.969, 0.969, 0.969}\color{fgcolor}\begin{kframe}
\begin{alltt}
\hlstd{my.factor1} \hlkwb{<-} \hlkwd{gl}\hlstd{(}\hlnum{4}\hlstd{,} \hlnum{3}\hlstd{,} \hlkwc{labels} \hlstd{=} \hlkwd{c}\hlstd{(}\hlstr{"A"}\hlstd{,} \hlstr{"F"}\hlstd{,} \hlstr{"B"}\hlstd{,} \hlstr{"Z"}\hlstd{))}
\hlstd{my.factor1}
\end{alltt}
\begin{verbatim}
##  [1] A A A F F F B B B Z Z Z
## Levels: A F B Z
\end{verbatim}
\begin{alltt}
\hlkwd{levels}\hlstd{(my.factor1)} \hlkwb{<-} \hlkwd{list}\hlstd{(}\hlstr{"a"} \hlstd{=} \hlstr{"A"}\hlstd{,} \hlstr{"d"} \hlstd{=} \hlstr{"Z"}\hlstd{,} \hlstr{"c"} \hlstd{=} \hlstr{"B"}\hlstd{,} \hlstr{"b"} \hlstd{=} \hlstr{"F"}\hlstd{)}
\hlstd{my.factor1}
\end{alltt}
\begin{verbatim}
##  [1] a a a b b b c c c d d d
## Levels: a d c b
\end{verbatim}
\end{kframe}
\end{knitrout}

Or alternatively by position changing only some level labels---i.e.\ rather unsafe.

\begin{knitrout}\footnotesize
\definecolor{shadecolor}{rgb}{0.969, 0.969, 0.969}\color{fgcolor}\begin{kframe}
\begin{alltt}
\hlstd{my.factor1} \hlkwb{<-} \hlkwd{gl}\hlstd{(}\hlnum{4}\hlstd{,} \hlnum{3}\hlstd{,} \hlkwc{labels} \hlstd{=} \hlkwd{c}\hlstd{(}\hlstr{"A"}\hlstd{,} \hlstr{"F"}\hlstd{,} \hlstr{"B"}\hlstd{,} \hlstr{"Z"}\hlstd{))}
\hlstd{my.factor1}
\end{alltt}
\begin{verbatim}
##  [1] A A A F F F B B B Z Z Z
## Levels: A F B Z
\end{verbatim}
\begin{alltt}
\hlkwd{levels}\hlstd{(my.factor1)[}\hlkwd{c}\hlstd{(}\hlnum{1}\hlstd{,} \hlnum{4}\hlstd{)]} \hlkwb{<-} \hlkwd{c}\hlstd{(}\hlstr{"a"}\hlstd{,} \hlstr{"d"}\hlstd{)}
\hlstd{my.factor1}
\end{alltt}
\begin{verbatim}
##  [1] a a a F F F B B B d d d
## Levels: a F B d
\end{verbatim}
\end{kframe}
\end{knitrout}

\end{explainbox}

\begin{explainbox}
\textbf{Merging factor levels.}\index{factors!merge levels} We use \Rfunction{factor()} as shown below, setting the same label for the levels we want to merge.

\begin{knitrout}\footnotesize
\definecolor{shadecolor}{rgb}{0.969, 0.969, 0.969}\color{fgcolor}\begin{kframe}
\begin{alltt}
\hlstd{my.factor1} \hlkwb{<-} \hlkwd{gl}\hlstd{(}\hlnum{4}\hlstd{,} \hlnum{3}\hlstd{,} \hlkwc{labels} \hlstd{=} \hlkwd{c}\hlstd{(}\hlstr{"A"}\hlstd{,} \hlstr{"F"}\hlstd{,} \hlstr{"B"}\hlstd{,} \hlstr{"Z"}\hlstd{))}
\hlstd{my.factor1}
\end{alltt}
\begin{verbatim}
##  [1] A A A F F F B B B Z Z Z
## Levels: A F B Z
\end{verbatim}
\begin{alltt}
\hlkwd{factor}\hlstd{(my.factor1,}
       \hlkwc{levels} \hlstd{=} \hlkwd{c}\hlstd{(}\hlstr{"A"}\hlstd{,} \hlstr{"B"}\hlstd{,} \hlstr{"F"}\hlstd{,} \hlstr{"Z"}\hlstd{),}
       \hlkwc{labels} \hlstd{=} \hlkwd{c}\hlstd{(}\hlstr{"A"}\hlstd{,} \hlstr{"B"}\hlstd{,} \hlstr{"C"}\hlstd{,} \hlstr{"C"}\hlstd{))}
\end{alltt}
\begin{verbatim}
##  [1] A A A C C C B B B C C C
## Levels: A B C
\end{verbatim}
\end{kframe}
\end{knitrout}
\end{explainbox}

\begin{explainbox}
\textbf{Reordering factor levels.}\index{factors!reorder levels} The simplest approach is to use \Rfunction{factor()} and its \code{levels} parameter. The only complication is that the names of the existing levels and those passed as argument need to match, and typing mistakes can cause bugs. To avoid the error-prone step, in all examples except the first, we use \Rfunction{levels()} to retrieve the names of the levels from the factor itself.

\begin{knitrout}\footnotesize
\definecolor{shadecolor}{rgb}{0.969, 0.969, 0.969}\color{fgcolor}\begin{kframe}
\begin{alltt}
\hlkwd{levels}\hlstd{(my.factor2)}
\end{alltt}
\begin{verbatim}
## [1] "3" "4" "5"
\end{verbatim}
\begin{alltt}
\hlstd{my.factor2} \hlkwb{<-} \hlkwd{factor}\hlstd{(my.factor2,} \hlkwc{levels} \hlstd{=} \hlkwd{c}\hlstd{(}\hlstr{"5"}\hlstd{,} \hlstr{"3"}\hlstd{,} \hlstr{"4"}\hlstd{))}
\hlkwd{levels}\hlstd{(my.factor2)}
\end{alltt}
\begin{verbatim}
## [1] "5" "3" "4"
\end{verbatim}
\begin{alltt}
\hlstd{my.factor2} \hlkwb{<-} \hlkwd{factor}\hlstd{(my.factor2,} \hlkwc{levels} \hlstd{=} \hlkwd{rev}\hlstd{(}\hlkwd{levels}\hlstd{(my.factor2)))}
\hlkwd{levels}\hlstd{(my.factor2)}
\end{alltt}
\begin{verbatim}
## [1] "4" "3" "5"
\end{verbatim}
\begin{alltt}
\hlstd{my.factor2} \hlkwb{<-} \hlkwd{factor}\hlstd{(my.factor2,} \hlkwc{levels} \hlstd{=} \hlkwd{sort}\hlstd{(}\hlkwd{levels}\hlstd{(my.factor2),} \hlkwc{decreasing} \hlstd{=} \hlnum{TRUE}\hlstd{))}
\hlkwd{levels}\hlstd{(my.factor2)}
\end{alltt}
\begin{verbatim}
## [1] "5" "4" "3"
\end{verbatim}
\begin{alltt}
\hlstd{my.factor2} \hlkwb{<-} \hlkwd{factor}\hlstd{(my.factor2,} \hlkwc{levels} \hlstd{=} \hlkwd{levels}\hlstd{(my.factor2)[}\hlkwd{c}\hlstd{(}\hlnum{2}\hlstd{,} \hlnum{1}\hlstd{,} \hlnum{3}\hlstd{)])}
\hlkwd{levels}\hlstd{(my.factor2)}
\end{alltt}
\begin{verbatim}
## [1] "4" "5" "3"
\end{verbatim}
\end{kframe}
\end{knitrout}

Reordering the levels of a factor based on summary quantities from data is very useful, especially when plotting. Function \Rfunction{reorder()} can be used in this case. It defaults to using \code{mean()} for summaries, but other suitable functions can be supplied in its place.

\begin{knitrout}\footnotesize
\definecolor{shadecolor}{rgb}{0.969, 0.969, 0.969}\color{fgcolor}\begin{kframe}
\begin{alltt}
\hlstd{my.factor3} \hlkwb{<-} \hlkwd{gl}\hlstd{(}\hlnum{2}\hlstd{,} \hlnum{5}\hlstd{,} \hlkwc{labels} \hlstd{=} \hlkwd{c}\hlstd{(}\hlstr{"A"}\hlstd{,} \hlstr{"B"}\hlstd{))}
\hlstd{my.vector3} \hlkwb{<-} \hlkwd{c}\hlstd{(}\hlnum{5.6}\hlstd{,} \hlnum{7.3}\hlstd{,} \hlnum{3.1}\hlstd{,} \hlnum{8.7}\hlstd{,} \hlnum{6.9}\hlstd{,} \hlnum{2.4}\hlstd{,} \hlnum{4.5}\hlstd{,} \hlnum{2.1}\hlstd{,} \hlnum{1.4}\hlstd{,} \hlnum{2.0}\hlstd{)}
\hlkwd{levels}\hlstd{(my.factor3)}
\end{alltt}
\begin{verbatim}
## [1] "A" "B"
\end{verbatim}
\begin{alltt}
\hlstd{my.factor3ord} \hlkwb{<-} \hlkwd{reorder}\hlstd{(my.factor3, my.vector3)}
\hlkwd{levels}\hlstd{(my.factor3ord)}
\end{alltt}
\begin{verbatim}
## [1] "B" "A"
\end{verbatim}
\begin{alltt}
\hlstd{my.factor3rev} \hlkwb{<-} \hlkwd{reorder}\hlstd{(my.factor3,} \hlopt{-}\hlstd{my.vector3)} \hlcom{# a simple trick}
\hlkwd{levels}\hlstd{(my.factor3rev)}
\end{alltt}
\begin{verbatim}
## [1] "A" "B"
\end{verbatim}
\end{kframe}
\end{knitrout}

In the last statement, using the unary negation operator, which is vectorized, allows us to easily reverse the ordering of the levels, while still using the default function, \code{mean()}, to summarize the data.

\end{explainbox}

\begin{advplayground}\label{calc:ADVPG:order:sort}
\textbf{Reordering factor values.}\index{factors!reorder values}\index{factors!arrange values} It is possible to arrange the values stored in a factor either alphabetically according to the labels of the levels or according to the order of the levels.

\begin{knitrout}\footnotesize
\definecolor{shadecolor}{rgb}{0.969, 0.969, 0.969}\color{fgcolor}\begin{kframe}
\begin{alltt}
\hlcom{# gl() keeps order of levels}
\hlstd{my.factor4} \hlkwb{<-} \hlkwd{gl}\hlstd{(}\hlnum{4}\hlstd{,} \hlnum{3}\hlstd{,} \hlkwc{labels} \hlstd{=} \hlkwd{c}\hlstd{(}\hlstr{"A"}\hlstd{,} \hlstr{"F"}\hlstd{,} \hlstr{"B"}\hlstd{,} \hlstr{"Z"}\hlstd{))}
\hlstd{my.factor4}
\hlkwd{as.integer}\hlstd{(my.factor4)}
\hlcom{# factor() orders levels alphabetically}
\hlstd{my.factor5} \hlkwb{<-} \hlkwd{factor}\hlstd{(}\hlkwd{rep}\hlstd{(}\hlkwd{c}\hlstd{(}\hlstr{"A"}\hlstd{,} \hlstr{"F"}\hlstd{,} \hlstr{"B"}\hlstd{,} \hlstr{"Z"}\hlstd{),} \hlkwd{rep}\hlstd{(}\hlnum{3}\hlstd{,}\hlnum{4}\hlstd{)))}
\hlstd{my.factor5}
\hlkwd{as.integer}\hlstd{(my.factor5)}
\end{alltt}
\end{kframe}
\end{knitrout}

We see above that the integer values are equivalent to indexes into the vector of labels. Function \Rfunction{sort()} operates on the values' underlying integers and sorts according to the order of the levels while \Rfunction{order()} operates on the values' labels and returns a vector of indexes that arrange the values alphabetically.

\begin{knitrout}\footnotesize
\definecolor{shadecolor}{rgb}{0.969, 0.969, 0.969}\color{fgcolor}\begin{kframe}
\begin{alltt}
\hlkwd{sort}\hlstd{(my.factor4)}
\hlstd{my.factor4[}\hlkwd{order}\hlstd{(my.factor4)]}
\hlstd{my.factor4[}\hlkwd{order}\hlstd{(}\hlkwd{as.integer}\hlstd{(my.factor4))]}
\end{alltt}
\end{kframe}
\end{knitrout}

Run the examples in the chunk above and work out why the results differ.
\end{advplayground}


\index{factors|)}

\section{Lists}\label{sec:calc:lists}
\index{lists|(}\qRclass{list}
\emph{Lists'} main difference to other collections is, in \Rlang, that they can be heterogeneous. In \Rlang, the members of a list can be considered as following a sequence, and accessible through numerical indexes, the same as vectors. However, frequently members of a list are given names, and retrieved (indexed) through these names. Lists are created using function \Rfunction{list()}.

\begin{knitrout}\footnotesize
\definecolor{shadecolor}{rgb}{0.969, 0.969, 0.969}\color{fgcolor}\begin{kframe}
\begin{alltt}
\hlstd{a.list} \hlkwb{<-} \hlkwd{list}\hlstd{(}\hlkwc{x} \hlstd{=} \hlnum{1}\hlopt{:}\hlnum{6}\hlstd{,} \hlkwc{y} \hlstd{=} \hlstr{"a"}\hlstd{,} \hlkwc{z} \hlstd{=} \hlkwd{c}\hlstd{(}\hlnum{TRUE}\hlstd{,} \hlnum{FALSE}\hlstd{))}
\hlstd{a.list}
\end{alltt}
\begin{verbatim}
## $x
## [1] 1 2 3 4 5 6
## 
## $y
## [1] "a"
## 
## $z
## [1]  TRUE FALSE
\end{verbatim}
\end{kframe}
\end{knitrout}

Some examples of extraction of list members using indexes.\index{lists!member extraction|(}\index{lists!member indexing|see{lists!member extraction}}\index{lists!indexes|see{lists!member extraction}}

\begin{knitrout}\footnotesize
\definecolor{shadecolor}{rgb}{0.969, 0.969, 0.969}\color{fgcolor}\begin{kframe}
\begin{alltt}
\hlstd{a.list}\hlopt{$}\hlstd{x}
\end{alltt}
\begin{verbatim}
## [1] 1 2 3 4 5 6
\end{verbatim}
\begin{alltt}
\hlstd{a.list[[}\hlstr{"x"}\hlstd{]]}
\end{alltt}
\begin{verbatim}
## [1] 1 2 3 4 5 6
\end{verbatim}
\begin{alltt}
\hlstd{a.list[[}\hlnum{1}\hlstd{]]}
\end{alltt}
\begin{verbatim}
## [1] 1 2 3 4 5 6
\end{verbatim}
\begin{alltt}
\hlstd{a.list[}\hlstr{"x"}\hlstd{]}
\end{alltt}
\begin{verbatim}
## $x
## [1] 1 2 3 4 5 6
\end{verbatim}
\begin{alltt}
\hlstd{a.list[}\hlnum{1}\hlstd{]}
\end{alltt}
\begin{verbatim}
## $x
## [1] 1 2 3 4 5 6
\end{verbatim}
\begin{alltt}
\hlstd{a.list[}\hlkwd{c}\hlstd{(}\hlnum{1}\hlstd{,}\hlnum{3}\hlstd{)]}
\end{alltt}
\begin{verbatim}
## $x
## [1] 1 2 3 4 5 6
## 
## $z
## [1]  TRUE FALSE
\end{verbatim}
\begin{alltt}
\hlkwd{try}\hlstd{(a.list[[}\hlkwd{c}\hlstd{(}\hlnum{1}\hlstd{,}\hlnum{3}\hlstd{)]])}
\end{alltt}
\begin{verbatim}
## [1] 3
\end{verbatim}
\end{kframe}
\end{knitrout}

\begin{explainbox}
\emph{Lists} as usually defined in languages like \Clang are based on pointers stored at each node, and these pointers chain or link the different member nodes. In such implementations, indexing by position is not possible, or at least requires ``walking'' down the list, node by node. In \Rlang, \code{list} members can be accessed through positional indexes. Of course, insertions and deletions in the middle of a list, whatever their implementation, alter (or \emph{invalidate}) any position-based indexes.
\end{explainbox}

To investigate the returned values, function \Rfunction{str()} for \emph{structure} is of help, especially when the lists have many members, as it formats lists more compactly than function \code{print()}.

\begin{knitrout}\footnotesize
\definecolor{shadecolor}{rgb}{0.969, 0.969, 0.969}\color{fgcolor}\begin{kframe}
\begin{alltt}
\hlkwd{str}\hlstd{(a.list)}
\end{alltt}
\begin{verbatim}
## List of 3
##  $ x: int [1:6] 1 2 3 4 5 6
##  $ y: chr "a"
##  $ z: logi [1:2] TRUE FALSE
\end{verbatim}
\end{kframe}
\end{knitrout}

Using\qRoperator{[[]]} double square brackets for indexing a list extracts the element stored in the list, in its original mode, in the example above, \code{a.list[["x"]]} returns a numeric vector, while \code{a.list[1]} returns a list containing the numeric vector \code{x}. \code{a.list\$x} returns the same value as \code{a.list[["x"]]}, a numeric vector. While \code{a.list[c(1,3)]} returns a list of length two, while \code{a.list[[c(1,3)]]} is an error.
\index{lists!member extraction|)}

In other languages the two most common operations on lists are insertions and deletions.\index{lists!insert into}\index{lists!append to} In \Rlang function \Rfunction{append()} can be used both to append elements at the end of a list and insert elements into the head or any position in the middle of a list. To delete a member from a list we assign \code{NULL} to it.

\begin{knitrout}\footnotesize
\definecolor{shadecolor}{rgb}{0.969, 0.969, 0.969}\color{fgcolor}\begin{kframe}
\begin{alltt}
\hlstd{another.list} \hlkwb{<-} \hlkwd{append}\hlstd{(a.list,} \hlkwd{list}\hlstd{(}\hlkwc{yy} \hlstd{=} \hlnum{1}\hlopt{:}\hlnum{10}\hlstd{,} \hlkwc{zz} \hlstd{= letters[}\hlnum{5}\hlopt{:}\hlnum{1}\hlstd{]),} \hlnum{2L}\hlstd{)}
\hlstd{another.list}
\end{alltt}
\begin{verbatim}
## $x
## [1] 1 2 3 4 5 6
## 
## $y
## [1] "a"
## 
## $yy
##  [1]  1  2  3  4  5  6  7  8  9 10
## 
## $zz
## [1] "e" "d" "c" "b" "a"
## 
## $z
## [1]  TRUE FALSE
\end{verbatim}
\end{kframe}
\end{knitrout}

\begin{knitrout}\footnotesize
\definecolor{shadecolor}{rgb}{0.969, 0.969, 0.969}\color{fgcolor}\begin{kframe}
\begin{alltt}
\hlstd{a.list}\hlopt{$}\hlstd{y} \hlkwb{<-} \hlkwa{NULL}
\hlstd{a.list}
\end{alltt}
\begin{verbatim}
## $x
## [1] 1 2 3 4 5 6
## 
## $z
## [1]  TRUE FALSE
\end{verbatim}
\end{kframe}
\end{knitrout}

Lists can be also nested---i.e.\ lists of lists.\index{lists!nested}

\begin{knitrout}\footnotesize
\definecolor{shadecolor}{rgb}{0.969, 0.969, 0.969}\color{fgcolor}\begin{kframe}
\begin{alltt}
\hlstd{a.list} \hlkwb{<-} \hlkwd{list}\hlstd{(}\hlstr{"a"}\hlstd{,} \hlstr{"aa"}\hlstd{,} \hlstr{"aaa"}\hlstd{)}
\hlstd{b.list} \hlkwb{<-} \hlkwd{list}\hlstd{(}\hlstr{"b"}\hlstd{,} \hlstr{"bb"}\hlstd{)}
\hlstd{nested.list} \hlkwb{<-} \hlkwd{list}\hlstd{(}\hlkwc{A} \hlstd{= a.list,} \hlkwc{B} \hlstd{= b.list)}
\hlkwd{str}\hlstd{(nested.list)}
\end{alltt}
\begin{verbatim}
## List of 2
##  $ A:List of 3
##   ..$ : chr "a"
##   ..$ : chr "aa"
##   ..$ : chr "aaa"
##  $ B:List of 2
##   ..$ : chr "b"
##   ..$ : chr "bb"
\end{verbatim}
\end{kframe}
\end{knitrout}

\begin{explainbox}\index{lists!structure}
When dealing with deep lists, it is sometimes useful to limit the number of levels of nesting returned by \Rfunction{str()} by means of a \code{numeric} argument passed to parameter \code{max.levels}.

\begin{knitrout}\footnotesize
\definecolor{shadecolor}{rgb}{0.969, 0.969, 0.969}\color{fgcolor}\begin{kframe}
\begin{alltt}
\hlkwd{str}\hlstd{(nested.list,} \hlkwc{max.level} \hlstd{=} \hlnum{1}\hlstd{)}
\end{alltt}
\begin{verbatim}
## List of 2
##  $ A:List of 3
##  $ B:List of 2
\end{verbatim}
\end{kframe}
\end{knitrout}

\end{explainbox}

The nested\index{lists!nested} can be also contructed within a single statement in which several member lists are created. Here we combine the first three statements in the earlier chunk into a single one.

\begin{knitrout}\footnotesize
\definecolor{shadecolor}{rgb}{0.969, 0.969, 0.969}\color{fgcolor}\begin{kframe}
\begin{alltt}
\hlstd{nested.list} \hlkwb{<-} \hlkwd{list}\hlstd{(}\hlkwc{A} \hlstd{=} \hlkwd{list}\hlstd{(}\hlstr{"a"}\hlstd{,} \hlstr{"aa"}\hlstd{,} \hlstr{"aaa"}\hlstd{),} \hlkwc{B} \hlstd{=} \hlkwd{list}\hlstd{(}\hlstr{"b"}\hlstd{,} \hlstr{"bb"}\hlstd{))}
\hlkwd{str}\hlstd{(nested.list)}
\end{alltt}
\begin{verbatim}
## List of 2
##  $ A:List of 3
##   ..$ : chr "a"
##   ..$ : chr "aa"
##   ..$ : chr "aaa"
##  $ B:List of 2
##   ..$ : chr "b"
##   ..$ : chr "bb"
\end{verbatim}
\end{kframe}
\end{knitrout}

\begin{playground}
What\index{lists!nested} do you expect each of the statements below to return? \emph{Before running the code} predict what value and of which mode each statement will return. You may use implicit, or explicit, calls to \Rfunction{print()}, or calls to \Rfunction{str()} to visualize the structure of the different objects.

\begin{knitrout}\footnotesize
\definecolor{shadecolor}{rgb}{0.969, 0.969, 0.969}\color{fgcolor}\begin{kframe}
\begin{alltt}
\hlstd{nested.list} \hlkwb{<-} \hlkwd{list}\hlstd{(}\hlkwc{A} \hlstd{=} \hlkwd{list}\hlstd{(}\hlstr{"a"}\hlstd{,} \hlstr{"aa"}\hlstd{,} \hlstr{"aaa"}\hlstd{),} \hlkwc{B} \hlstd{=} \hlkwd{list}\hlstd{(}\hlstr{"b"}\hlstd{,} \hlstr{"bb"}\hlstd{))}
\hlkwd{str}\hlstd{(nested.list)}
\hlstd{nested.list[}\hlnum{2}\hlopt{:}\hlnum{1}\hlstd{]}
\hlstd{nested.list[}\hlnum{1}\hlstd{]}
\hlstd{nested.list[[}\hlnum{1}\hlstd{]][}\hlnum{2}\hlstd{]}
\hlstd{nested.list[[}\hlnum{1}\hlstd{]][[}\hlnum{2}\hlstd{]]}
\hlstd{nested.list[}\hlnum{2}\hlstd{]}
\hlstd{nested.list[}\hlnum{2}\hlstd{][[}\hlnum{1}\hlstd{]]}
\end{alltt}
\end{kframe}
\end{knitrout}

\end{playground}

Sometimes we need to flatten a list\index{lists!flattening}\index{lists!nested}, or a nested structure of lists within lists. Function \Rfunction{unlist()} is what should be normally used in such cases.

The list \code{nested.list} is a nested system of lists, but all the ``terminal'' members are character strings. In other words, terminal nodes are all of the same mode.

\begin{knitrout}\footnotesize
\definecolor{shadecolor}{rgb}{0.969, 0.969, 0.969}\color{fgcolor}\begin{kframe}
\begin{alltt}
\hlstd{c.vec} \hlkwb{<-} \hlkwd{unlist}\hlstd{(nested.list)}
\hlstd{c.vec}
\end{alltt}
\begin{verbatim}
##    A1    A2    A3    B1    B2 
##   "a"  "aa" "aaa"   "b"  "bb"
\end{verbatim}
\begin{alltt}
\hlkwd{is.list}\hlstd{(nested.list)}
\end{alltt}
\begin{verbatim}
## [1] TRUE
\end{verbatim}
\begin{alltt}
\hlkwd{is.list}\hlstd{(c.vec)}
\end{alltt}
\begin{verbatim}
## [1] FALSE
\end{verbatim}
\begin{alltt}
\hlkwd{mode}\hlstd{(nested.list)}
\end{alltt}
\begin{verbatim}
## [1] "list"
\end{verbatim}
\begin{alltt}
\hlkwd{mode}\hlstd{(c.vec)}
\end{alltt}
\begin{verbatim}
## [1] "character"
\end{verbatim}
\begin{alltt}
\hlkwd{names}\hlstd{(nested.list)}
\end{alltt}
\begin{verbatim}
## [1] "A" "B"
\end{verbatim}
\begin{alltt}
\hlkwd{names}\hlstd{(c.vec)}
\end{alltt}
\begin{verbatim}
## [1] "A1" "A2" "A3" "B1" "B2"
\end{verbatim}
\end{kframe}
\end{knitrout}

The returned value is a vector with named member elements. Function \Rfunction{str()}---for \emph{structure}--- helps, as seen before, figure out what this object looks like. The names, in this case are based in the names of list elements when available, but numbers are used for anonymous nodes in the list. We can access the members of the vector either through numeric indexes, or names.

\begin{knitrout}\footnotesize
\definecolor{shadecolor}{rgb}{0.969, 0.969, 0.969}\color{fgcolor}\begin{kframe}
\begin{alltt}
\hlkwd{str}\hlstd{(c.vec)}
\end{alltt}
\begin{verbatim}
##  Named chr [1:5] "a" "aa" "aaa" "b" "bb"
##  - attr(*, "names")= chr [1:5] "A1" "A2" "A3" "B1" ...
\end{verbatim}
\begin{alltt}
\hlstd{c.vec[}\hlnum{2}\hlstd{]}
\end{alltt}
\begin{verbatim}
##   A2 
## "aa"
\end{verbatim}
\begin{alltt}
\hlstd{c.vec[}\hlstr{"A2"}\hlstd{]}
\end{alltt}
\begin{verbatim}
##   A2 
## "aa"
\end{verbatim}
\end{kframe}
\end{knitrout}

\begin{playground}
Function \Rfunction{unlist()}\index{lists!convert into vector}, has two additional parameters, with default argument values, which we did not modify in the example above. These are \code{recursive} and \code{use.names}, both of them expecting a \code{logical} value as argument. Modify the statement \code{c.vec <- unlist(c.list)}, by passing \code{FALSE} as argument to these two parameters, in turn, and in each case study the value returned and how it differs with respect to the one obtained above.
\end{playground}


\index{lists|)}

\section{Data frames}\label{sec:R:data:frames}
\index{data frames|(}\qRclass{data.frame}
\index{worksheet@`worksheet'|see{data frame}}
Data frames are a special type of list, in which each element is a vector or a factor of the same length. They are created with function \Rfunction{data.frame()} with a syntax similar to that used for lists---in object-oriented programming we say that data frames are derived from list. As the expectation is equal length, if vectors of different lengths are supplied as arguments, the shorter vector(s) is/are recycled, possibly several times, until the full length required is reached.

\begin{knitrout}\footnotesize
\definecolor{shadecolor}{rgb}{0.969, 0.969, 0.969}\color{fgcolor}\begin{kframe}
\begin{alltt}
\hlstd{a.df} \hlkwb{<-} \hlkwd{data.frame}\hlstd{(}\hlkwc{x} \hlstd{=} \hlnum{1}\hlopt{:}\hlnum{6}\hlstd{,} \hlkwc{y} \hlstd{=} \hlstr{"a"}\hlstd{,} \hlkwc{z} \hlstd{=} \hlkwd{c}\hlstd{(}\hlnum{TRUE}\hlstd{,} \hlnum{FALSE}\hlstd{))}
\hlstd{a.df}
\end{alltt}
\begin{verbatim}
##   x y     z
## 1 1 a  TRUE
## 2 2 a FALSE
## 3 3 a  TRUE
## 4 4 a FALSE
## 5 5 a  TRUE
## 6 6 a FALSE
\end{verbatim}
\begin{alltt}
\hlkwd{str}\hlstd{(a.df)}
\end{alltt}
\begin{verbatim}
## 'data.frame':	6 obs. of  3 variables:
##  $ x: int  1 2 3 4 5 6
##  $ y: Factor w/ 1 level "a": 1 1 1 1 1 1
##  $ z: logi  TRUE FALSE TRUE FALSE TRUE FALSE
\end{verbatim}
\begin{alltt}
\hlkwd{class}\hlstd{(a.df)}
\end{alltt}
\begin{verbatim}
## [1] "data.frame"
\end{verbatim}
\begin{alltt}
\hlkwd{mode}\hlstd{(a.df)}
\end{alltt}
\begin{verbatim}
## [1] "list"
\end{verbatim}
\begin{alltt}
\hlkwd{is.data.frame}\hlstd{(a.df)}
\end{alltt}
\begin{verbatim}
## [1] TRUE
\end{verbatim}
\begin{alltt}
\hlkwd{is.list}\hlstd{(a.df)}
\end{alltt}
\begin{verbatim}
## [1] TRUE
\end{verbatim}
\end{kframe}
\end{knitrout}

Indexing of data frames is similar to that of the underlying list, but not exactly equivalent. We can index with operator \Roperator{[[]]} to extract individual variables, thought as being the columns in a matrix-like list or ``worksheet''.

\begin{knitrout}\footnotesize
\definecolor{shadecolor}{rgb}{0.969, 0.969, 0.969}\color{fgcolor}\begin{kframe}
\begin{alltt}
\hlstd{a.df}\hlopt{$}\hlstd{x}
\end{alltt}
\begin{verbatim}
## [1] 1 2 3 4 5 6
\end{verbatim}
\begin{alltt}
\hlstd{a.df[[}\hlstr{"x"}\hlstd{]]}
\end{alltt}
\begin{verbatim}
## [1] 1 2 3 4 5 6
\end{verbatim}
\begin{alltt}
\hlstd{a.df[[}\hlnum{1}\hlstd{]]}
\end{alltt}
\begin{verbatim}
## [1] 1 2 3 4 5 6
\end{verbatim}
\begin{alltt}
\hlkwd{class}\hlstd{(a.df)}
\end{alltt}
\begin{verbatim}
## [1] "data.frame"
\end{verbatim}
\end{kframe}
\end{knitrout}

With function \Rfunction{class()} we can query the class of an \Rlang object (see section \ref{sec:rlang:mode} on page \pageref{sec:rlang:mode}). As we saw in the two previous chunks, \code{list} and \code{data.frame} objects belong to two different classes. However, their relationship is based on a hierarchy of classes. We say that class \Rclass{data.frame} is derived from class \code{list}. Consequently, data frames inherit the methods and characteristics of lists, as long as they have not been hideen by new ones defined for data frames.

In the same way as with vectors, we can add members to lists and data frames.

\begin{knitrout}\footnotesize
\definecolor{shadecolor}{rgb}{0.969, 0.969, 0.969}\color{fgcolor}\begin{kframe}
\begin{alltt}
\hlstd{a.df}\hlopt{$}\hlstd{x2} \hlkwb{<-} \hlnum{6}\hlopt{:}\hlnum{1}
\hlstd{a.df}\hlopt{$}\hlstd{x3} \hlkwb{<-} \hlstr{"b"}
\hlkwd{str}\hlstd{(a.df)}
\end{alltt}
\begin{verbatim}
## 'data.frame':	6 obs. of  5 variables:
##  $ x : int  1 2 3 4 5 6
##  $ y : Factor w/ 1 level "a": 1 1 1 1 1 1
##  $ z : logi  TRUE FALSE TRUE FALSE TRUE FALSE
##  $ x2: int  6 5 4 3 2 1
##  $ x3: chr  "b" "b" "b" "b" ...
\end{verbatim}
\end{kframe}
\end{knitrout}

We have added two columns to the data frame, and in the case of column \code{x3} recycling took place. This is where lists and data frames differ substantially in their behaviour. In a data frame, although class and mode can be different for different variables (columns), they are required to be vectors or factors of the same length. In the case of lists, there is no such requirement, and recycling never takes place when adding a node. Compare the values returned below for \code{a.ls}, to those in the example above for \code{a.df}.

\begin{knitrout}\footnotesize
\definecolor{shadecolor}{rgb}{0.969, 0.969, 0.969}\color{fgcolor}\begin{kframe}
\begin{alltt}
\hlstd{a.ls} \hlkwb{<-} \hlkwd{list}\hlstd{(}\hlkwc{x} \hlstd{=} \hlnum{1}\hlopt{:}\hlnum{6}\hlstd{,} \hlkwc{y} \hlstd{=} \hlstr{"a"}\hlstd{,} \hlkwc{z} \hlstd{=} \hlkwd{c}\hlstd{(}\hlnum{TRUE}\hlstd{,} \hlnum{FALSE}\hlstd{))}
\hlkwd{str}\hlstd{(a.ls)}
\end{alltt}
\begin{verbatim}
## List of 3
##  $ x: int [1:6] 1 2 3 4 5 6
##  $ y: chr "a"
##  $ z: logi [1:2] TRUE FALSE
\end{verbatim}
\begin{alltt}
\hlstd{a.ls}\hlopt{$}\hlstd{x2} \hlkwb{<-} \hlnum{6}\hlopt{:}\hlnum{1}
\hlstd{a.ls}\hlopt{$}\hlstd{x3} \hlkwb{<-} \hlstr{"b"}
\hlkwd{str}\hlstd{(a.ls)}
\end{alltt}
\begin{verbatim}
## List of 5
##  $ x : int [1:6] 1 2 3 4 5 6
##  $ y : chr "a"
##  $ z : logi [1:2] TRUE FALSE
##  $ x2: int [1:6] 6 5 4 3 2 1
##  $ x3: chr "b"
\end{verbatim}
\end{kframe}
\end{knitrout}

Data frames are extremely important to anyone analysing or plotting data using \Rlang. One can think of data frames as tightly structured work-sheets, or as lists. As you may have guessed from the examples earlier in this section, there are several different ways of accessing columns, rows, and individual observations stored in a data frame. The columns can be treated as members in a list, and can be accessed both by name or index (position). When accessed by name, using \Roperator{\$} or double square brackets a single column is returned as a vector or factor. In contrast to lists, data frames are `rectangular' and for this reason the values stored can be also accessed in a way similar to how elements in a matrix are accessed, using two indexes. As we saw for vectors, indexes can be vectors of integer numbers or vectors of logical values. For columns they can in addition be vectors of character strings matching the names of the columns. When using indexes it is extremely important to remember that the indexes are always given \textbf{row first}.

\begin{explainbox}
Indexing of data frames can in all cases be done as if they were lists, which is preferable, as it ensures compatibility with regular \Rlang lists and with newer implementations of data-frame-like structures like those defined in package \pkgname{tibble}. Using this approach, extracting two values from the second and third positions in the first column of \code{a.df} is done as follows, using numerical indexes.

\begin{knitrout}\footnotesize
\definecolor{shadecolor}{rgb}{0.969, 0.969, 0.969}\color{fgcolor}\begin{kframe}
\begin{alltt}
\hlstd{a.df[[}\hlnum{1}\hlstd{]][}\hlnum{2}\hlopt{:}\hlnum{3}\hlstd{]}
\end{alltt}
\begin{verbatim}
## [1] 2 3
\end{verbatim}
\end{kframe}
\end{knitrout}

Or using the column name.

\begin{knitrout}\footnotesize
\definecolor{shadecolor}{rgb}{0.969, 0.969, 0.969}\color{fgcolor}\begin{kframe}
\begin{alltt}
\hlstd{a.df[[}\hlstr{"x"}\hlstd{]][}\hlnum{2}\hlopt{:}\hlnum{3}\hlstd{]}
\end{alltt}
\begin{verbatim}
## [1] 2 3
\end{verbatim}
\end{kframe}
\end{knitrout}

The less portable, matrix-like indexing is done as follows, with the first index indicating rows and the second one indicating columns. This notation allows simultaneous extraction from multiple columns, which is not possible with lists. The value returned is a ``smaller'' data frame.

\begin{knitrout}\footnotesize
\definecolor{shadecolor}{rgb}{0.969, 0.969, 0.969}\color{fgcolor}\begin{kframe}
\begin{alltt}
\hlstd{a.df[}\hlnum{2}\hlopt{:}\hlnum{3}\hlstd{,} \hlnum{1}\hlopt{:}\hlnum{2}\hlstd{]}
\end{alltt}
\begin{verbatim}
##   x y
## 2 2 a
## 3 3 a
\end{verbatim}
\end{kframe}
\end{knitrout}

If the length of the column indexing vector is one, the returned value is a vector, which is not consistent with the previous example which returned a data frame. This is not only surprising in everyday use, but can be the source of bugs when coding algorithms in which the length of the second index vector cannot be guaranteed to be always more than one.

\begin{knitrout}\footnotesize
\definecolor{shadecolor}{rgb}{0.969, 0.969, 0.969}\color{fgcolor}\begin{kframe}
\begin{alltt}
\hlstd{a.df[}\hlnum{2}\hlopt{:}\hlnum{3}\hlstd{,} \hlnum{1}\hlstd{]}
\end{alltt}
\begin{verbatim}
## [1] 2 3
\end{verbatim}
\end{kframe}
\end{knitrout}

In contrast, indexing of tibbles---defined in package \pkgname{tibble}---is always consistent, independently of the length of the second indexing vector (see chapter \ref{chap:R:data}).
\end{explainbox}


\begin{knitrout}\footnotesize
\definecolor{shadecolor}{rgb}{0.969, 0.969, 0.969}\color{fgcolor}\begin{kframe}
\begin{alltt}
\hlcom{# first column, a.df[[1]] preferred}
\hlstd{a.df[ ,} \hlnum{1}\hlstd{]}
\end{alltt}
\begin{verbatim}
## [1] 1 2 3 4 5 6
\end{verbatim}
\begin{alltt}
\hlcom{# first column, a.df[["x"]] or a.df$x preferred}
\hlstd{a.df[ ,} \hlstr{"x"}\hlstd{]}
\end{alltt}
\begin{verbatim}
## [1] 1 2 3 4 5 6
\end{verbatim}
\begin{alltt}
\hlcom{# first row}
\hlstd{a.df[}\hlnum{1}\hlstd{, ]}
\end{alltt}
\begin{verbatim}
##   x y    z x2 x3
## 1 1 a TRUE  6  b
\end{verbatim}
\begin{alltt}
\hlcom{# first two rows of the third and fourth columns}
\hlstd{a.df[}\hlnum{1}\hlopt{:}\hlnum{2}\hlstd{,} \hlkwd{c}\hlstd{(}\hlnum{FALSE}\hlstd{,} \hlnum{FALSE}\hlstd{,} \hlnum{TRUE}\hlstd{,} \hlnum{TRUE}\hlstd{,} \hlnum{FALSE}\hlstd{)]}
\end{alltt}
\begin{verbatim}
##       z x2
## 1  TRUE  6
## 2 FALSE  5
\end{verbatim}
\begin{alltt}
\hlcom{# the rows for which z is true}
\hlstd{a.df[a.df}\hlopt{$}\hlstd{z , ]}
\end{alltt}
\begin{verbatim}
##   x y    z x2 x3
## 1 1 a TRUE  6  b
## 3 3 a TRUE  4  b
## 5 5 a TRUE  2  b
\end{verbatim}
\begin{alltt}
\hlcom{# the rows for which x > 3 keeping all columns except the third one}
\hlstd{a.df[a.df}\hlopt{$}\hlstd{x} \hlopt{>} \hlnum{3}\hlstd{,} \hlopt{-}\hlnum{3}\hlstd{]}
\end{alltt}
\begin{verbatim}
##   x y x2 x3
## 4 4 a  3  b
## 5 5 a  2  b
## 6 6 a  1  b
\end{verbatim}
\end{kframe}
\end{knitrout}

As earlier explained for vectors (see section \ref{sec:calc:indexing} on page \pageref{sec:calc:indexing}), indexing can be present both on the right-hand side and left-hand-side of an assignment.
The next few examples do assignments to ``cells'' of \code{a.df}, either to one whole column, or individual values. The last statement in the chunk below copies a number from one location to another by using indexing of the same data frame both on the `right side' and `left side' of the assignment.\qRoperator{[[]]}\qRoperator{[]}

\begin{knitrout}\footnotesize
\definecolor{shadecolor}{rgb}{0.969, 0.969, 0.969}\color{fgcolor}\begin{kframe}
\begin{alltt}
\hlstd{a.df[}\hlnum{1}\hlstd{,} \hlnum{1}\hlstd{]} \hlkwb{<-} \hlnum{99}
\hlstd{a.df}
\end{alltt}
\begin{verbatim}
##    x y     z x2 x3
## 1 99 a  TRUE  6  b
## 2  2 a FALSE  5  b
## 3  3 a  TRUE  4  b
## 4  4 a FALSE  3  b
## 5  5 a  TRUE  2  b
## 6  6 a FALSE  1  b
\end{verbatim}
\begin{alltt}
\hlstd{a.df[ ,} \hlnum{1}\hlstd{]} \hlkwb{<-} \hlopt{-}\hlnum{99}
\hlstd{a.df}
\end{alltt}
\begin{verbatim}
##     x y     z x2 x3
## 1 -99 a  TRUE  6  b
## 2 -99 a FALSE  5  b
## 3 -99 a  TRUE  4  b
## 4 -99 a FALSE  3  b
## 5 -99 a  TRUE  2  b
## 6 -99 a FALSE  1  b
\end{verbatim}
\begin{alltt}
\hlstd{a.df[[}\hlstr{"x"}\hlstd{]]} \hlkwb{<-} \hlnum{123}
\hlstd{a.df}
\end{alltt}
\begin{verbatim}
##     x y     z x2 x3
## 1 123 a  TRUE  6  b
## 2 123 a FALSE  5  b
## 3 123 a  TRUE  4  b
## 4 123 a FALSE  3  b
## 5 123 a  TRUE  2  b
## 6 123 a FALSE  1  b
\end{verbatim}
\begin{alltt}
\hlstd{a.df[}\hlnum{1}\hlstd{,} \hlnum{1}\hlstd{]} \hlkwb{<-} \hlstd{a.df[}\hlnum{6}\hlstd{,} \hlnum{4}\hlstd{]}
\hlstd{a.df}
\end{alltt}
\begin{verbatim}
##     x y     z x2 x3
## 1   1 a  TRUE  6  b
## 2 123 a FALSE  5  b
## 3 123 a  TRUE  4  b
## 4 123 a FALSE  3  b
## 5 123 a  TRUE  2  b
## 6 123 a FALSE  1  b
\end{verbatim}
\end{kframe}
\end{knitrout}

\begin{warningbox}
We mentioned above that indexing by name can be done either with double square brackets, \Roperator{[[]]}, or with \Roperator{\$}. In the first case the name of the variable or column is given as a character string, enclosed in quotation marks, or as a variable with mode \code{character}. When using \Roperator{\$}, the name is entered as a constant, without quotation marks, and cannot be a variable.

\begin{knitrout}\footnotesize
\definecolor{shadecolor}{rgb}{0.969, 0.969, 0.969}\color{fgcolor}\begin{kframe}
\begin{alltt}
\hlstd{x.list} \hlkwb{<-} \hlkwd{list}\hlstd{(}\hlkwc{abcd} \hlstd{=} \hlnum{123}\hlstd{,} \hlkwc{xyzw} \hlstd{=} \hlnum{789}\hlstd{)}
\hlstd{x.list[[}\hlstr{"abcd"}\hlstd{]]}
\end{alltt}
\begin{verbatim}
## [1] 123
\end{verbatim}
\begin{alltt}
\hlstd{a.var} \hlkwb{<-} \hlstr{"abcd"}
\hlstd{x.list[[a.var]]}
\end{alltt}
\begin{verbatim}
## [1] 123
\end{verbatim}
\begin{alltt}
\hlstd{x.list}\hlopt{$}\hlstd{abcd}
\end{alltt}
\begin{verbatim}
## [1] 123
\end{verbatim}
\begin{alltt}
\hlstd{x.list}\hlopt{$}\hlstd{ab}
\end{alltt}
\begin{verbatim}
## [1] 123
\end{verbatim}
\begin{alltt}
\hlstd{x.list}\hlopt{$}\hlstd{a}
\end{alltt}
\begin{verbatim}
## [1] 123
\end{verbatim}
\end{kframe}
\end{knitrout}

Both in the case of lists and data frames, when using double square brackets, an exact match is required between the name in the object and the name used for indexing. In contrast, with \Roperator{\$} any unambiguous partial match will be accepted. For interactive use, partial matching is helpful in reducing typing. However, in scripts, and especially \Rlang code in packages it is best to avoid the use of \Roperator{\$} as partial matching to a wrong variable present at a later time, e.g.\ when someone else revises the script, can lead to very difficult to diagnose errors. In addition, as \Roperator{\$} is implemented by first attempting a match to the name and then calling \Roperator{[[]]}, using \Roperator{\$} for indexing can result in slightly slower performance compared to using \Roperator{[[]]}. It is possible to set an \Rlang option so that partial matching triggers a warning, which can be very useful when debugging.
\end{warningbox}

\subsection{Operating within data frames}\label{sec:calc:df:with}
\index{data frames!operating within}
When\index{data frames!subsetting}\index{data frames!``filtering rows''} the names of data frames are long, complex conditions become awkward to write using indexing---i.e.\ subscripts. In such cases \Rfunction{subset()} is handy because evaluation is done in the `environment' of the data frame, i.e.\ the names of the columns are recognized if entered directly when writing the condition. Function  \Rfunction{subset()} ``filters'' rows, usually corresponding to observations or experimental units. The condition is computed for each row, and if it returns \code{TRUE} the row is included in the returned data frame, and excluded if \code{FALSE}.

\begin{knitrout}\footnotesize
\definecolor{shadecolor}{rgb}{0.969, 0.969, 0.969}\color{fgcolor}\begin{kframe}
\begin{alltt}
\hlstd{a.df} \hlkwb{<-} \hlkwd{data.frame}\hlstd{(}\hlkwc{x} \hlstd{=} \hlnum{1}\hlopt{:}\hlnum{6}\hlstd{,} \hlkwc{y} \hlstd{=} \hlstr{"a"}\hlstd{,} \hlkwc{z} \hlstd{=} \hlkwd{c}\hlstd{(}\hlnum{TRUE}\hlstd{,} \hlnum{FALSE}\hlstd{))}
\hlkwd{subset}\hlstd{(a.df, x} \hlopt{>} \hlnum{3}\hlstd{)}
\end{alltt}
\begin{verbatim}
##   x y     z
## 4 4 a FALSE
## 5 5 a  TRUE
## 6 6 a FALSE
\end{verbatim}
\end{kframe}
\end{knitrout}

\begin{playground}
What is the behaviour of \code{subset()} when the condition is \code{NA}? Find the answer by writing code to test this, for a case where tests for different rows return \code{NA}, \code{TRUE} and \code{FALSE}.
\end{playground}

When calling functions that return a vector, data frame, or other structure, the square brackets can be appended to the rightmost parenthesis of the function call, in the same way as to the name of a variable holding the same data.

\begin{knitrout}\footnotesize
\definecolor{shadecolor}{rgb}{0.969, 0.969, 0.969}\color{fgcolor}\begin{kframe}
\begin{alltt}
\hlkwd{subset}\hlstd{(a.df, x} \hlopt{>} \hlnum{3}\hlstd{)[ ,} \hlopt{-}\hlnum{3}\hlstd{]}
\end{alltt}
\begin{verbatim}
##   x y
## 4 4 a
## 5 5 a
## 6 6 a
\end{verbatim}
\begin{alltt}
\hlkwd{subset}\hlstd{(a.df, x} \hlopt{>} \hlnum{3}\hlstd{)}\hlopt{$}\hlstd{x}
\end{alltt}
\begin{verbatim}
## [1] 4 5 6
\end{verbatim}
\end{kframe}
\end{knitrout}

None of the examples in the last three code chunks alter the original data frame \code{a.df}. We can store the returned value using a new name, if we want to preserve \code{a.df} unchanged, or we can assign the result to \code{a.df} deleting in the process the original one.
\begin{warningbox}
  In the example above the names in the expression passed as second argument to \code{subset()} are first searched within \code{ad.df} but if not found, searched in the environment. There being no variable \code{A} in the data frame \code{a.df}, vector \code{A} from the environment is silently used in the expression resulting in a returned data frame with no rows.

\begin{knitrout}\footnotesize
\definecolor{shadecolor}{rgb}{0.969, 0.969, 0.969}\color{fgcolor}\begin{kframe}
\begin{alltt}
\hlstd{A} \hlkwb{<-} \hlnum{1}
\hlkwd{subset}\hlstd{(a.df, A} \hlopt{>} \hlnum{3}\hlstd{)}
\end{alltt}
\begin{verbatim}
## [1] x y z
## <0 rows> (or 0-length row.names)
\end{verbatim}
\end{kframe}
\end{knitrout}

The use of \Rfunction{subset()} is convenient, but more prone to result in bugs compared to directly using the extraction operator \code{[]}. This same ``cost'' to achieving convenience applies to functions like \Rfunction{attach()} and \Rfunction{with()} described below. The longer time a script is expected to be used, adapted and reused, the more careful we should be when using any of these functions. An alternative way of avoiding excessive verbosity is to keep the names of data frames short.
\end{warningbox}

A frequently used way of deleting a column by name from a data frame is to assign \code{NULL} to it ---i.e.\ in the same way as members are deleted from \code{list}s. This approach modifies \code{a.df} in place.

\begin{knitrout}\footnotesize
\definecolor{shadecolor}{rgb}{0.969, 0.969, 0.969}\color{fgcolor}\begin{kframe}
\begin{alltt}
\hlstd{aa.df} \hlkwb{<-} \hlstd{a.df}
\hlkwd{colnames}\hlstd{(aa.df)}
\end{alltt}
\begin{verbatim}
## [1] "x" "y" "z"
\end{verbatim}
\begin{alltt}
\hlstd{aa.df[[}\hlstr{"y"}\hlstd{]]} \hlkwb{<-} \hlkwa{NULL}
\hlkwd{colnames}\hlstd{(aa.df)}
\end{alltt}
\begin{verbatim}
## [1] "x" "z"
\end{verbatim}
\end{kframe}
\end{knitrout}

\begin{explainbox}
Alternatively we can use negative indexing to remove columns from a copy of a data frame. In this example we remove a single column. As base \Rlang does not support negative indexing by name, we need to find the numerical index to the column to delete.

\begin{knitrout}\footnotesize
\definecolor{shadecolor}{rgb}{0.969, 0.969, 0.969}\color{fgcolor}\begin{kframe}
\begin{alltt}
\hlstd{a.df[ ,} \hlopt{-}\hlkwd{which}\hlstd{(}\hlkwd{colnames}\hlstd{(a.df)} \hlopt{==} \hlstr{"y"}\hlstd{)]}
\end{alltt}
\begin{verbatim}
##   x     z
## 1 1  TRUE
## 2 2 FALSE
## 3 3  TRUE
## 4 4 FALSE
## 5 5  TRUE
## 6 6 FALSE
\end{verbatim}
\end{kframe}
\end{knitrout}
Instead of using the equality test we can use the operator \code{\%in\%} or function \code{grepl()} to delete multiple columns in a single statement.
\end{explainbox}

\begin{playground}
In the previous code chuck we deleted the last two columns of the data frame \code{a.df}.
Here is an esoteric trick for you to first untangle and then think how and why it can be useful.

\begin{knitrout}\footnotesize
\definecolor{shadecolor}{rgb}{0.969, 0.969, 0.969}\color{fgcolor}\begin{kframe}
\begin{alltt}
\hlstd{a.df[}\hlnum{1}\hlopt{:}\hlnum{6}\hlstd{,} \hlkwd{c}\hlstd{(}\hlnum{1}\hlstd{,}\hlnum{3}\hlstd{)]} \hlkwb{<-} \hlstd{a.df[}\hlnum{6}\hlopt{:}\hlnum{1}\hlstd{,} \hlkwd{c}\hlstd{(}\hlnum{3}\hlstd{,}\hlnum{1}\hlstd{)]}
\hlstd{a.df}
\end{alltt}
\end{kframe}
\end{knitrout}
\end{playground}

\begin{warningbox}
Although in this last example we used numeric indexes to make it more interesting, in practice, especially in scripts or other code that will be reused, do use column or member names instead of positional indexes whenever possible. This makes code much more reliable, as changes elsewhere in the script could alter the order of columns and \emph{invalidate} numerical indexes. In addition, using meaningful names makes programmers' intentions easier to understand.
\end{warningbox}

\begin{explainbox}
It is sometimes inconvenient to have to pre-pend the name of a \emph{container} such as a list or data frame to the name of each member variable being accessed. There are functions in \Rlang that allow us to change where \Rlang looks for the names of objects we include in a code statement. Here I describe the use of \Rscoping{attach()} and its matching \Rscoping{detach()}, and \Rscoping{with()} and \Rscoping{within()} to access members of a data frame. They can be used as well with lists and classes derived from \code{list}.

As we can see below, when using a rather long name for a data frame, entering a simple calculation can easily result in a long and difficult to read statement. (Method \code{head()} is used here to limit the displayed value to the first two rows---\code{head()} is described in section \ref{sec:calc:looking:at:data} on page \pageref{sec:calc:looking:at:data}.)

\begin{knitrout}\footnotesize
\definecolor{shadecolor}{rgb}{0.969, 0.969, 0.969}\color{fgcolor}\begin{kframe}
\begin{alltt}
\hlstd{my_data_frame.df} \hlkwb{<-} \hlkwd{data.frame}\hlstd{(}\hlkwc{A} \hlstd{=} \hlnum{1}\hlopt{:}\hlnum{10}\hlstd{,} \hlkwc{B} \hlstd{=} \hlnum{3}\hlstd{)}
\hlstd{my_data_frame.df}\hlopt{$}\hlstd{C} \hlkwb{<-}
  \hlstd{(my_data_frame.df}\hlopt{$}\hlstd{A} \hlopt{+} \hlstd{my_data_frame.df}\hlopt{$}\hlstd{B)} \hlopt{/} \hlstd{my_data_frame.df}\hlopt{$}\hlstd{A}
\hlkwd{head}\hlstd{(my_data_frame.df,} \hlnum{2}\hlstd{)}
\end{alltt}
\begin{verbatim}
##   A B   C
## 1 1 3 4.0
## 2 2 3 2.5
\end{verbatim}
\end{kframe}
\end{knitrout}

Using\index{data frames!attaching} \Rscoping{attach()} we can alter how \Rlang looks up names and consequently simplify the statement. With \Rscoping{detach()} we can restore the original state. It is important to remember that here we can only simplify the right hand side of the assignment, while the ``destination'' of the result of the computation needs still to be fully specified on the left-hand side of the assignment operator. We show above only one statement between \Rscoping{attach()} and \Rscoping{detach()} but multiple statements are also allowed. Furthermore, if variables with the same name as the columns exist, these will take precedence, something that can result in bugs or crashes depending on what variables are present in the \Rlang environment at run time.

\begin{knitrout}\footnotesize
\definecolor{shadecolor}{rgb}{0.969, 0.969, 0.969}\color{fgcolor}\begin{kframe}
\begin{alltt}
\hlstd{my_data_frame.df}\hlopt{$}\hlstd{C} \hlkwb{<-} \hlkwa{NULL}
\hlkwd{attach}\hlstd{(my_data_frame.df)}
\end{alltt}


{\ttfamily\noindent\itshape\color{messagecolor}{\#\# The following object is masked \_by\_ .GlobalEnv:\\\#\# \\\#\#\ \ \ \  A}}\begin{alltt}
\hlstd{my_data_frame.df}\hlopt{$}\hlstd{C} \hlkwb{<-} \hlstd{(A} \hlopt{+} \hlstd{B)} \hlopt{/} \hlstd{A}
\hlkwd{detach}\hlstd{(my_data_frame.df)}
\hlkwd{head}\hlstd{(my_data_frame.df,} \hlnum{2}\hlstd{)}
\end{alltt}
\begin{verbatim}
##   A B C
## 1 1 3 4
## 2 2 3 4
\end{verbatim}
\end{kframe}
\end{knitrout}

In the case of \Rscoping{with()} only one, possibly compound, code statement is affected and this statement is passed as an argument. As before, we need to fully specify the left hand side of the assignment. The value returned is the one returned by the statement passed as argument, in the case of compound statements, the value returned by the last contained simple code statement to be executed. Consequently, if the intent is to modify the container, assignment to an individual member variable (column in this case) is required.

\begin{knitrout}\footnotesize
\definecolor{shadecolor}{rgb}{0.969, 0.969, 0.969}\color{fgcolor}\begin{kframe}
\begin{alltt}
\hlstd{my_data_frame.df}\hlopt{$}\hlstd{C} \hlkwb{<-} \hlkwa{NULL}
\hlstd{my_data_frame.df}\hlopt{$}\hlstd{C} \hlkwb{<-} \hlkwd{with}\hlstd{(my_data_frame.df, (A} \hlopt{+} \hlstd{B)} \hlopt{/} \hlstd{A)}
\hlkwd{head}\hlstd{(my_data_frame.df,} \hlnum{2}\hlstd{)}
\end{alltt}
\begin{verbatim}
##   A B   C
## 1 1 3 4.0
## 2 2 3 2.5
\end{verbatim}
\end{kframe}
\end{knitrout}

In the case of \Rscoping{within()} assignments in the argument to its second parameter affect the object returned, which is a copy of the container (in this case a whole data frame), which still needs to be saved trough assignment. Here the intention is to modify it, so we assign it back to the same name, but it could have been assigned to a different name so as not to overwrite the original data frame.

\begin{knitrout}\footnotesize
\definecolor{shadecolor}{rgb}{0.969, 0.969, 0.969}\color{fgcolor}\begin{kframe}
\begin{alltt}
\hlstd{my_data_frame.df}\hlopt{$}\hlstd{C} \hlkwb{<-} \hlkwa{NULL}
\hlstd{my_data_frame.df} \hlkwb{<-} \hlkwd{within}\hlstd{(my_data_frame.df,  C} \hlkwb{<-} \hlstd{(A} \hlopt{+} \hlstd{B)} \hlopt{/} \hlstd{A)}
\hlkwd{head}\hlstd{(my_data_frame.df,} \hlnum{2}\hlstd{)}
\end{alltt}
\begin{verbatim}
##   A B   C
## 1 1 3 4.0
## 2 2 3 2.5
\end{verbatim}
\end{kframe}
\end{knitrout}
In the example above \code{within()} makes little difference compared to using \Rscoping{with()} with respect to the amount of typing or clarity, but with multiple member variables being operated upon, as shown below, \Rscoping{within()} has an advantage resulting in more concise and easier to understand code.

\begin{knitrout}\footnotesize
\definecolor{shadecolor}{rgb}{0.969, 0.969, 0.969}\color{fgcolor}\begin{kframe}
\begin{alltt}
\hlstd{my_data_frame.df}\hlopt{$}\hlstd{C} \hlkwb{<-} \hlkwa{NULL}
\hlstd{my_data_frame.df} \hlkwb{<-} \hlkwd{within}\hlstd{(my_data_frame.df,}
                           \hlstd{\{C} \hlkwb{<-} \hlstd{(A} \hlopt{+} \hlstd{B)} \hlopt{/} \hlstd{A}
                            \hlstd{D} \hlkwb{<-} \hlstd{A} \hlopt{*} \hlstd{B}
                            \hlstd{E} \hlkwb{<-} \hlstd{A} \hlopt{/} \hlstd{B} \hlopt{+} \hlnum{1}\hlstd{\}}
                           \hlstd{)}
\hlkwd{head}\hlstd{(my_data_frame.df,} \hlnum{2}\hlstd{)}
\end{alltt}
\begin{verbatim}
##   A B        E D   C
## 1 1 3 1.333333 3 4.0
## 2 2 3 1.666667 6 2.5
\end{verbatim}
\end{kframe}
\end{knitrout}

Use of \Rscoping{attach()} and \Rscoping{detach()}, which function as a pair of ON and OFF switches, can result in an undesired after-effect on name lookup if the script terminates after \Rscoping{attach()} but before \Rscoping{detach()} are executed, as cleanup is not enforced. In contrast, \Rscoping{with()} and \Rscoping{within()} being self-contained guarantee that clean up takes place. Consequently, the usual recommendation is to give preference to the use of \Rscoping{with()} and \Rscoping{within()} over \Rscoping{attach()} and \Rscoping{detach()}. Use of these functions not only saves typing but also makes code more readable.
\end{explainbox}

\subsection{Re-arranging columns and rows}
\index{data frames!ordering rows}\index{data frames!ordering columns}
The most direct way of changing the order of columns and/or rows in data frames (and matrices and arrays) is to use subscripting as described above. Once we know the original position and target position we can use numerical indexes on both right hand side and left hand side of an assignment.

\begin{warningbox}
When using the extraction operator \Roperator{[]} on both the left-hand-side and right-hand-side to swap columns, the vectors or factors are swapped, while the names of the columns are not! The same applies to rownames, which makes storing important information in them inconvenient and error prone.
\end{warningbox}

To retain the correct naming after the column swap, we need to separately swap the names of the columns.
\begin{knitrout}\footnotesize
\definecolor{shadecolor}{rgb}{0.969, 0.969, 0.969}\color{fgcolor}\begin{kframe}
\begin{alltt}
\hlstd{my_data_frame.df} \hlkwb{<-} \hlkwd{data.frame}\hlstd{(}\hlkwc{A} \hlstd{=} \hlnum{1}\hlopt{:}\hlnum{10}\hlstd{,} \hlkwc{B} \hlstd{=} \hlnum{3}\hlstd{)}
\hlkwd{head}\hlstd{(my_data_frame.df,} \hlnum{2}\hlstd{)}
\end{alltt}
\begin{verbatim}
##   A B
## 1 1 3
## 2 2 3
\end{verbatim}
\begin{alltt}
\hlstd{my_data_frame.df[ ,} \hlnum{1}\hlopt{:}\hlnum{2}\hlstd{]} \hlkwb{<-} \hlstd{my_data_frame.df[ ,} \hlnum{2}\hlopt{:}\hlnum{1}\hlstd{]}
\hlkwd{head}\hlstd{(my_data_frame.df,} \hlnum{2}\hlstd{)}
\end{alltt}
\begin{verbatim}
##   A B
## 1 3 1
## 2 3 2
\end{verbatim}
\begin{alltt}
\hlkwd{colnames}\hlstd{(my_data_frame.df)[}\hlnum{1}\hlopt{:}\hlnum{2}\hlstd{]} \hlkwb{<-} \hlkwd{colnames}\hlstd{(my_data_frame.df)[}\hlnum{2}\hlopt{:}\hlnum{1}\hlstd{]}
\hlkwd{head}\hlstd{(my_data_frame.df,} \hlnum{2}\hlstd{)}
\end{alltt}
\begin{verbatim}
##   B A
## 1 3 1
## 2 3 2
\end{verbatim}
\end{kframe}
\end{knitrout}

Taking into account that \Rfunction{order()} returns the indexes needed to sort a vector, we can use order to generate the indexes to use through subscripting. When we want to sort rows the argument to \Rfunction{order()} is usually a column of the data frame being arranged. Order can be applied to any vector of suitable length, including the result of applying a function to one or more columns. (The use of \code{order()} for sorting vectors is described on page \pageref{box:vec:sort}.)

\begin{playground}\index{data frames!ordering rows}
The first task to be completed is to sort a data frame based on the values in one column, using indexing and \Rfunction{order()}. Create a new data frame and with three numeric columns with three different haphazard sequences of values. Call these columns A, B and C. 1) Sort the rows of the data frame so that the values in A are in decreasing order. 2) Sort the rows of the data frame according to increasing values of the sum of A and B without adding a new column to the data frame or storing the sums in a vector. In other words do the sorting based on sums calculated ``on-the-fly''.
\end{playground}

\begin{advplayground}\index{data frames!ordering rows}
Repeat the tasks in the playground immediately above but using factors instead of numeric vectors as columns in the data frame. Hint: Revisit the exercise on page \pageref{calc:ADVPG:order:sort} were the use of \Rfunction{order()} on factors is described.
\end{advplayground}

\index{data frames|)}



\section{Attributes of R objects}\label{sec:calc:attributes}
\index{attributes|(}

\Rlang objects can have attributes. Attributes are normally used to store ancillary data. They are used by \Rlang itself to store things like column names in data frames and labels of factor levels. All these attributes are visible to user code, and user code can read and write objects' attributes. Of the attributes defined in \Rlang the one that is expected to be set by users is \code{"comment"}. We use it for this first example as comments can be very useful to store metadata together with data in a given object.\qRfunction{comment()}\qRfunction{comment()<-}

\begin{knitrout}\footnotesize
\definecolor{shadecolor}{rgb}{0.969, 0.969, 0.969}\color{fgcolor}\begin{kframe}
\begin{alltt}
\hlstd{a.df} \hlkwb{<-} \hlkwd{data.frame}\hlstd{(}\hlkwc{x} \hlstd{=} \hlnum{1}\hlopt{:}\hlnum{6}\hlstd{,} \hlkwc{y} \hlstd{=} \hlstr{"a"}\hlstd{,} \hlkwc{z} \hlstd{=} \hlkwd{c}\hlstd{(}\hlnum{TRUE}\hlstd{,} \hlnum{FALSE}\hlstd{))}
\hlkwd{comment}\hlstd{(a.df)}
\end{alltt}
\begin{verbatim}
## NULL
\end{verbatim}
\begin{alltt}
\hlkwd{comment}\hlstd{(a.df)} \hlkwb{<-} \hlstr{"this is stored as a comment"}
\hlkwd{comment}\hlstd{(a.df)}
\end{alltt}
\begin{verbatim}
## [1] "this is stored as a comment"
\end{verbatim}
\end{kframe}
\end{knitrout}

Methods like \Rfunction{names()}, \Rfunction{dim()} or \Rfunction{levels()} return values retrieved from attributes stored in \Rlang objects, and methods like \Rfunction{names()<-}, \Rfunction{dim()<-} or \Rfunction{levels()<-} set (or unset with \code{NULL}) the value of the respective attributes. Specific query and set methods do not exist for all attributes. Methods \Rfunction{attr()}, \Rfunction{attr()<-} and \Rfunction{attributes()} can be used with any attribute. In addition, method \Rfunction{str()} displays all components of \Rlang objects including their attributes.

\begin{knitrout}\footnotesize
\definecolor{shadecolor}{rgb}{0.969, 0.969, 0.969}\color{fgcolor}\begin{kframe}
\begin{alltt}
\hlkwd{names}\hlstd{(a.df)}
\end{alltt}
\begin{verbatim}
## [1] "x" "y" "z"
\end{verbatim}
\begin{alltt}
\hlkwd{names}\hlstd{(a.df)} \hlkwb{<-} \hlkwd{toupper}\hlstd{(}\hlkwd{names}\hlstd{(a.df))}
\hlkwd{names}\hlstd{(a.df)}
\end{alltt}
\begin{verbatim}
## [1] "X" "Y" "Z"
\end{verbatim}
\begin{alltt}
\hlkwd{attr}\hlstd{(a.df,} \hlstr{"names"}\hlstd{)} \hlcom{# same as previous line}
\end{alltt}
\begin{verbatim}
## [1] "X" "Y" "Z"
\end{verbatim}
\begin{alltt}
\hlkwd{attr}\hlstd{(a.df,} \hlstr{"my.attribute"}\hlstd{)} \hlkwb{<-} \hlstr{"this is stored in my attribute"}
\hlkwd{attributes}\hlstd{(a.df)}
\end{alltt}
\begin{verbatim}
## $names
## [1] "X" "Y" "Z"
## 
## $class
## [1] "data.frame"
## 
## $row.names
## [1] 1 2 3 4 5 6
## 
## $comment
## [1] "this is stored as a comment"
## 
## $my.attribute
## [1] "this is stored in my attribute"
\end{verbatim}
\end{kframe}
\end{knitrout}

\begin{warningbox}
There is no restriction to the creation, setting, resetting and reading of attributes, but not all methods and operators that can be used to modify objects will preserve non-standard attributes. So, using private attributes is a double edged sword that usually is worthwhile considering only when designing a new class together with the corresponding methods for it. A good example of extensive use of class-specific attributes are the values returned by model fitting functions like \Rfunction{lm()} described in section \ref{sec:stat:LM} on page \pageref{sec:stat:LM}.

Even the class of S3 objects is stored as an attribute that is accessible as any other attribute---this is in contrast to the mode and atomic class of an object. Object-oriented programming in \Rlang in described in section \ref{sec:script:objects:classes:methods} on page \ref{sec:script:objects:classes:methods}.

\begin{knitrout}\footnotesize
\definecolor{shadecolor}{rgb}{0.969, 0.969, 0.969}\color{fgcolor}\begin{kframe}
\begin{alltt}
\hlstd{numbers} \hlkwb{<-} \hlnum{1}\hlopt{:}\hlnum{10}
\hlkwd{class}\hlstd{(numbers)}
\end{alltt}
\begin{verbatim}
## [1] "integer"
\end{verbatim}
\begin{alltt}
\hlkwd{attributes}\hlstd{(numbers)}
\end{alltt}
\begin{verbatim}
## NULL
\end{verbatim}
\end{kframe}
\end{knitrout}

\begin{knitrout}\footnotesize
\definecolor{shadecolor}{rgb}{0.969, 0.969, 0.969}\color{fgcolor}\begin{kframe}
\begin{alltt}
\hlstd{a.factor} \hlkwb{<-} \hlkwd{factor}\hlstd{(numbers)}
\hlkwd{class}\hlstd{(a.factor)}
\end{alltt}
\begin{verbatim}
## [1] "factor"
\end{verbatim}
\begin{alltt}
\hlkwd{attributes}\hlstd{(a.factor)}
\end{alltt}
\begin{verbatim}
## $levels
##  [1] "1"  "2"  "3"  "4"  "5"  "6"  "7"  "8"  "9"  "10"
## 
## $class
## [1] "factor"
\end{verbatim}
\end{kframe}
\end{knitrout}
\end{warningbox}



\index{attributes|)}

\section{Loading data}

\subsection{Data sets in R and packages}
\index{data!loading data sets|(}
To be able to present more meaningful examples, we need some data. Here we use \code{cars}, one of the many data sets included in base \Rpgrm. Function \Rfunction{data()} is used load data objects saved in a file format used by \Rlang. It is also possible to `import' data saved in files of \textit{foreign} formats, defined by other programs. Several contributed packages allow importing data from other statistic and data analysis applications and from standard data exchange formats. How to read or import ``foreign'' data is discussed in \Rlang documentation in \emph{R Data Import/Export}, and in this book, in chapter \ref{chap:R:data:io} starting on page \pageref{chap:R:data:io}. It is also good to keep in mind that in \Rlang urls are accepted as arguments to the \code{file} or \code{path} argument of many functions (see section \ref{sec:files:remote} starting on page \pageref{sec:files:remote}).

In the next example we use data included in R, as \Rlang objects, which can be loaded with function \Rfunction{data()}. Object \code{cars} is a data frame.

\begin{knitrout}\footnotesize
\definecolor{shadecolor}{rgb}{0.969, 0.969, 0.969}\color{fgcolor}\begin{kframe}
\begin{alltt}
\hlkwd{data}\hlstd{(cars)}
\end{alltt}
\end{kframe}
\end{knitrout}

Once we have a data set available, the first step is usually to explore it, and we do this with \code{cars} in section \ref{sec:calc:looking:at:data} on page \pageref{sec:calc:looking:at:data}.
\index{data!loading data sets|)}

\subsection{.rda files}\label{sec:data:rda}

By default, at the end of a session, the current workspace containing the results of your work has been saved into a file called \code{.RData}. In addition to saving the whole workspace, it is possible to save one or more \Rlang objects present in the workspace to disk using the same file format (with file name tag \code{.rda} or \code{.Rda}). One or more objects, belonging to any mode or class can be saved into a single file using function \Rfunction{save()}. Reading the file restores all the saved objects into the current workspace with their original names. These files are portable across most \Rlang versions---i.e.\ old formats can be read and written by newer versions of R, although the newer, default format may be not readable with earlier version. Whether compression is used, and whether the ``binary'' data is encoded into ASCII characters---allowing maximum portability at the expense of increased size can be controlled by passing suitable arguments to \Rfunction{save()}.

We create data frame object and then save it to a file.

\begin{knitrout}\footnotesize
\definecolor{shadecolor}{rgb}{0.969, 0.969, 0.969}\color{fgcolor}\begin{kframe}
\begin{alltt}
\hlstd{my.df} \hlkwb{<-} \hlkwd{data.frame}\hlstd{(}\hlkwc{x} \hlstd{=} \hlnum{1}\hlopt{:}\hlnum{5}\hlstd{,} \hlkwc{y} \hlstd{=} \hlnum{5}\hlopt{:}\hlnum{1}\hlstd{)}
\hlstd{my.df}
\end{alltt}
\begin{verbatim}
##   x y
## 1 1 5
## 2 2 4
## 3 3 3
## 4 4 2
## 5 5 1
\end{verbatim}
\begin{alltt}
\hlkwd{save}\hlstd{(my.df,} \hlkwc{file} \hlstd{=} \hlstr{"my-df.rda"}\hlstd{)}
\end{alltt}
\end{kframe}
\end{knitrout}

We delete the data frame object and confirm that it is no longer present in the workspace.

\begin{knitrout}\footnotesize
\definecolor{shadecolor}{rgb}{0.969, 0.969, 0.969}\color{fgcolor}\begin{kframe}
\begin{alltt}
\hlkwd{rm}\hlstd{(my.df)}
\hlkwd{ls}\hlstd{(}\hlkwc{pattern} \hlstd{=} \hlstr{"my.df"}\hlstd{)}
\end{alltt}
\begin{verbatim}
## character(0)
\end{verbatim}
\end{kframe}
\end{knitrout}

We read the file we earlier saved to restore the object.\qRfunction{load()}

\begin{knitrout}\footnotesize
\definecolor{shadecolor}{rgb}{0.969, 0.969, 0.969}\color{fgcolor}\begin{kframe}
\begin{alltt}
\hlkwd{load}\hlstd{(}\hlkwc{file} \hlstd{=} \hlstr{"my-df.rda"}\hlstd{)}
\hlkwd{ls}\hlstd{(}\hlkwc{pattern} \hlstd{=} \hlstr{"my.df"}\hlstd{)}
\end{alltt}
\begin{verbatim}
## [1] "my.df"
\end{verbatim}
\begin{alltt}
\hlstd{my.df}
\end{alltt}
\begin{verbatim}
##   x y
## 1 1 5
## 2 2 4
## 3 3 3
## 4 4 2
## 5 5 1
\end{verbatim}
\end{kframe}
\end{knitrout}

The default format used is binary and compressed, which results in smaller files.

\begin{playground}
In the example above, only one object was saved, but one can simply give the names of additional objects as arguments. Just try saving, more than one data frame to the same file. Then the data frames plus a few vectors. Then define a simple function and save it. After saving each file, clear the workspace and then load the objects you save from the file.
\end{playground}

Sometimes it is easier to supply the names of the objects to be saved as a vector of character strings through an argument to parameter \code{list}. One case is when wanting to save a group of objects based on their names. We can use \Rfunction{ls()} to list the names of objects matching a simple \code{pattern} or a complex regular expression. The example below does this in two steps saving the character vector first, and then using this saved object as argument to \code{save}'s \code{list} parameter.

\begin{knitrout}\footnotesize
\definecolor{shadecolor}{rgb}{0.969, 0.969, 0.969}\color{fgcolor}\begin{kframe}
\begin{alltt}
\hlstd{objcts} \hlkwb{<-} \hlkwd{ls}\hlstd{(}\hlkwc{pattern} \hlstd{=} \hlstr{"*.df"}\hlstd{)}
\hlkwd{save}\hlstd{(}\hlkwc{list} \hlstd{= objcts,} \hlkwc{file} \hlstd{=} \hlstr{"my-df1.rda"}\hlstd{)}
\end{alltt}
\end{kframe}
\end{knitrout}

The intermediate step can be skipped.

\begin{knitrout}\footnotesize
\definecolor{shadecolor}{rgb}{0.969, 0.969, 0.969}\color{fgcolor}\begin{kframe}
\begin{alltt}
\hlkwd{save}\hlstd{(}\hlkwc{list} \hlstd{=} \hlkwd{ls}\hlstd{(}\hlkwc{pattern} \hlstd{=} \hlstr{"*.df"}\hlstd{),} \hlkwc{file} \hlstd{=} \hlstr{"my-df1.rda"}\hlstd{)}
\end{alltt}
\end{kframe}
\end{knitrout}

\begin{playground}
Practice using different patterns with \Rfunction{ls()}. You do not need to save the objects to a file. Just have a look at the list of object names returned.
\end{playground}

As a coda, we show how to cleanup by deleting the two files we created. Function \Rfunction{unlink()} can also be used to delete files.

\begin{knitrout}\footnotesize
\definecolor{shadecolor}{rgb}{0.969, 0.969, 0.969}\color{fgcolor}\begin{kframe}
\begin{alltt}
\hlkwd{unlink}\hlstd{(}\hlkwd{c}\hlstd{(}\hlstr{"my-df.rda"}\hlstd{,} \hlstr{"my-df1.rda"}\hlstd{))}
\end{alltt}
\end{kframe}
\end{knitrout}

\subsection{.rds files}\label{sec:data:rds}

The RDS format can be used to save individual objects instead of multiple objects (usually using file name tag \code{.rds}). They are read and saved with functions \Rfunction{readRDS()} and \Rfunction{saveRDS()}, respectively. Contrary to with RDA format, which restore objects to the working environment using their original names, in this case we have to assign a name, possibly different from the original one.

\begin{knitrout}\footnotesize
\definecolor{shadecolor}{rgb}{0.969, 0.969, 0.969}\color{fgcolor}\begin{kframe}
\begin{alltt}
\hlkwd{saveRDS}\hlstd{(my.df,} \hlstr{"my-df.rds"}\hlstd{)}
\end{alltt}
\end{kframe}
\end{knitrout}

We assign the read object to a different name for this example, and check that the object read is identical to the one saved.
\begin{knitrout}\footnotesize
\definecolor{shadecolor}{rgb}{0.969, 0.969, 0.969}\color{fgcolor}\begin{kframe}
\begin{alltt}
\hlstd{my_read.df} \hlkwb{<-} \hlkwd{readRDS}\hlstd{(}\hlstr{"my-df.rds"}\hlstd{)}
\hlkwd{identical}\hlstd{(my.df, my_read.df)}
\end{alltt}
\begin{verbatim}
## [1] TRUE
\end{verbatim}
\end{kframe}
\end{knitrout}

\begin{knitrout}\footnotesize
\definecolor{shadecolor}{rgb}{0.969, 0.969, 0.969}\color{fgcolor}\begin{kframe}
\begin{alltt}
\hlkwd{unlink}\hlstd{(}\hlstr{"my-df.rds"}\hlstd{)}
\end{alltt}
\end{kframe}
\end{knitrout}

\section{Looking at data}\label{sec:calc:looking:at:data}
\index{data!exploration at the R console|(}
There are several functions in \Rlang that let us obtain different `views' into objects. Function \Rfunction{print()} is useful for small data sets, or objects. Especially in the case of large data frames, we need to explore them step by step. In the case of named components, we can obtain their names, with \Rfunction{colnames()}, \Rfunction{rownames()} and \Rfunction{names()}. If a data frame contains many rows of observations, \Rfunction{head()} and \Rfunction{tail()} allow us to easily restrict the number of rows printed. Functions \Rfunction{nrow()} and \Rfunction{ncol()} return the number of rows and columns in the data frame (also applicable to matrices but are not to lists or vectors with which we use \Rfunction{length()}). As earlier mentioned, function \Rfunction{str()} concisely displays the structure of \Rlang objects.

\begin{knitrout}\footnotesize
\definecolor{shadecolor}{rgb}{0.969, 0.969, 0.969}\color{fgcolor}\begin{kframe}
\begin{alltt}
\hlkwd{class}\hlstd{(cars)}
\end{alltt}
\begin{verbatim}
## [1] "data.frame"
\end{verbatim}
\begin{alltt}
\hlkwd{nrow}\hlstd{(cars)}
\end{alltt}
\begin{verbatim}
## [1] 50
\end{verbatim}
\begin{alltt}
\hlkwd{ncol}\hlstd{(cars)}
\end{alltt}
\begin{verbatim}
## [1] 2
\end{verbatim}
\begin{alltt}
\hlkwd{names}\hlstd{(cars)}
\end{alltt}
\begin{verbatim}
## [1] "speed" "dist"
\end{verbatim}
\begin{alltt}
\hlkwd{head}\hlstd{(cars)}
\end{alltt}
\begin{verbatim}
##   speed dist
## 1     4    2
## 2     4   10
## 3     7    4
## 4     7   22
## 5     8   16
## 6     9   10
\end{verbatim}
\begin{alltt}
\hlkwd{tail}\hlstd{(cars)}
\end{alltt}
\begin{verbatim}
##    speed dist
## 45    23   54
## 46    24   70
## 47    24   92
## 48    24   93
## 49    24  120
## 50    25   85
\end{verbatim}
\begin{alltt}
\hlkwd{str}\hlstd{(cars)}
\end{alltt}
\begin{verbatim}
## 'data.frame':	50 obs. of  2 variables:
##  $ speed: num  4 4 7 7 8 9 10 10 10 11 ...
##  $ dist : num  2 10 4 22 16 10 18 26 34 17 ...
\end{verbatim}
\end{kframe}
\end{knitrout}

\begin{playground}
Look up the help pages for \Rfunction{head()} and \Rfunction{tail()}, and edit the code above to print only the first line, or only the last line of \code{cars}, respectively. As a second exercise print the 25 topmost rows of \code{cars}.
\end{playground}

Data frames consist in columns of equal length (see section \ref{sec:R:data:frames} on page \pageref{sec:R:data:frames}). The different columns of a data frame be factors or vectors of different classes (e.g.\ numeric, logical, character, etc.).

To explore the class of the columns of \code{cars}, we can use an \emph{apply} function. In the present case, we want to apply function \code{class()} to each column of the data frame \code{cars}. (Apply functions are described in section \ref{sec:data:apply} on page \pageref{sec:data:apply}.)
\qRloop{sapply}

\begin{knitrout}\footnotesize
\definecolor{shadecolor}{rgb}{0.969, 0.969, 0.969}\color{fgcolor}\begin{kframe}
\begin{alltt}
\hlkwd{sapply}\hlstd{(}\hlkwc{X} \hlstd{= cars,} \hlkwc{FUN} \hlstd{= class)}
\end{alltt}
\begin{verbatim}
##     speed      dist 
## "numeric" "numeric"
\end{verbatim}
\end{kframe}
\end{knitrout}

The statement above returns a vector of character strings, with the class of each column. Each element of the vector is named according to the name of the corresponding ``column'' in the data frame. For this same statement to be used with any other data frame or list, we need only to substitute the name of the object, the argument to the first parameter called \code{X}, to the one of current interest.

\begin{playground}
Data set \code{airquality} contains data from air quality measurements in New York, and, being included in the \Rpgrm distribution, can be loaded with \code{data(airquality)}. Load it, and repeat the steps above, to learn what variables are included, their classes, the number of rows, etc.
\end{playground}

Function \Rfunction{summary()} can be used to obtain a summary from objects of most \Rlang classes, including data frames. We can also use \Rloop{sapply()}, \Rloop{lapply()} or \Rloop{vapply()} to apply any suitable function to individual columns.

\begin{knitrout}\footnotesize
\definecolor{shadecolor}{rgb}{0.969, 0.969, 0.969}\color{fgcolor}\begin{kframe}
\begin{alltt}
\hlkwd{summary}\hlstd{(cars)}
\end{alltt}
\begin{verbatim}
##      speed           dist       
##  Min.   : 4.0   Min.   :  2.00  
##  1st Qu.:12.0   1st Qu.: 26.00  
##  Median :15.0   Median : 36.00  
##  Mean   :15.4   Mean   : 42.98  
##  3rd Qu.:19.0   3rd Qu.: 56.00  
##  Max.   :25.0   Max.   :120.00
\end{verbatim}
\begin{alltt}
\hlkwd{sapply}\hlstd{(cars, range)}
\end{alltt}
\begin{verbatim}
##      speed dist
## [1,]     4    2
## [2,]    25  120
\end{verbatim}
\end{kframe}
\end{knitrout}

\begin{advplayground}
Obtain the summary of \code{airquality} with function \Rfunction{summary()}, but in addition, write code with an \emph{apply} function to count the number of non-missing values in each column. Hint: using \code{sum()} on a \code{logical} vector returns the count of \code{TRUE} values as \code{TRUE} is converted to numeric 1.
\end{advplayground}

\section{Plotting}
\index{plots!base R graphics}
Base \Rlang generic method \Rfunction{plot()} can be used to plot different data. It is a generic method that has specializations suitable for different kinds of objects (see section \ref{sec:script:objects:classes:methods} on page \pageref{sec:script:objects:classes:methods} for a brief introduction to objects, classes and methods). In this section we only very briefly demonstrate the use of the most common base \langname{R}'s graphics functions. They are well described in the book \citebooktitle{Murrell2011} \autocite{Murrell2011}. We will not describe the Lattice (based on S's Trellis) approach to plotting \autocite{Sarkar2008}. Instead we describe in detail the use of the \emph{grammar of graphics} and plotting with package \ggplot in chapter \ref{chap:R:plotting} starting on page \pageref{chap:R:plotting}.

It is possible to pass two variables (here columns from a data frame) directly as arguments to the \code{x} and \code{y} parameters of \Rfunction{plot()}.



\begin{knitrout}\footnotesize
\definecolor{shadecolor}{rgb}{0.969, 0.969, 0.969}\color{fgcolor}\begin{kframe}
\begin{alltt}
\hlkwd{plot}\hlstd{(}\hlkwc{x} \hlstd{= cars}\hlopt{$}\hlstd{speed,} \hlkwc{y} \hlstd{= cars}\hlopt{$}\hlstd{dist)}
\end{alltt}
\end{kframe}

{\centering \includegraphics[width=.54\textwidth]{figure/pos-plot-1-1} 

}



\end{knitrout}

It is also possible, and usually more convenient, to use a \emph{formula} to specify the variables to be plotted on the $x$ and $y$ axes, passing additionally as argument to  parameter \code{data} the name of the data frame containing these variables. The formula \code{dist \textasciitilde\ speed}, is read as \code{dist} explained by \code{speed}---i.e.\ \code{dist} is mapped to the $y$-axis as the dependent variable and \code{speed} to the $x$-axis as the independent variable.

\begin{knitrout}\footnotesize
\definecolor{shadecolor}{rgb}{0.969, 0.969, 0.969}\color{fgcolor}\begin{kframe}
\begin{alltt}
\hlkwd{plot}\hlstd{(dist} \hlopt{~} \hlstd{speed,} \hlkwc{data} \hlstd{= cars)}
\end{alltt}
\end{kframe}

{\centering \includegraphics[width=.54\textwidth]{figure/pos-plot-2-1} 

}



\end{knitrout}

Within \Rlang there exist different specializations, or ``flavours'', of method \Rfunction{plot()} that become active depending on the class of the variables passed as arguments: passing two numerical variables results in a scatter plot as seen above. In contrast passing one factor and one numeric variable to \code{plot()} results in a box-and-whiskers plot being produced. To exemplify this we need to use a different data set, here \code{chickwts} as \code{cars} does not contain any factors. Use \code{help("chickwts")} to learn more about this data set, also included in \Rpgrm .

\begin{knitrout}\footnotesize
\definecolor{shadecolor}{rgb}{0.969, 0.969, 0.969}\color{fgcolor}\begin{kframe}
\begin{alltt}
\hlkwd{plot}\hlstd{(weight} \hlopt{~} \hlstd{feed,} \hlkwc{data} \hlstd{= chickwts)}
\end{alltt}
\end{kframe}

{\centering \includegraphics[width=.54\textwidth]{figure/pos-plot-3-1} 

}



\end{knitrout}

Method \Rfunction{plot()} and variants defined in \Rlang, when used for plotting return their graphical output to a \emph{graphical output device}. In \Rlang, graphical devices are very frequently not real physical devices like a printer, but virtual devices implemented fully in software that translate the plotting commands into a specific graphical file format. Several different graphical devices are available in \Rlang and they differ in the kind of output they produce: raster files (e.g.\ TIFF, PNG and JPEG formats), vector graphics files (e.g.\ SVG, EPS and PDF) or output to a physical device like a window in the screen of a computer. Additional devices are available through contributed \Rlang packages.

Devices follow the paradigm of ON and OFF switches. Some devices producing a file as output, save this output only when the device is closed. When opening a device the user supplies additional information. For the PDF device that produces output in a vector-graphics format, width and height of the output are specified in \emph{inches}. A default file name is used unless we pass a \code{character} string as argument to parameter \code{file}.

\begin{knitrout}\footnotesize
\definecolor{shadecolor}{rgb}{0.969, 0.969, 0.969}\color{fgcolor}\begin{kframe}
\begin{alltt}
\hlkwd{pdf}\hlstd{(}\hlkwc{file} \hlstd{=} \hlstr{"output/my-file.pdf"}\hlstd{,} \hlkwc{width} \hlstd{=} \hlnum{6}\hlstd{,} \hlkwc{height} \hlstd{=} \hlnum{5}\hlstd{,} \hlkwc{onefile} \hlstd{=} \hlnum{TRUE}\hlstd{)}
\hlkwd{plot}\hlstd{(dist} \hlopt{~} \hlstd{speed,} \hlkwc{data} \hlstd{= cars)}
\hlkwd{plot}\hlstd{(weight} \hlopt{~} \hlstd{feed,} \hlkwc{data} \hlstd{= chickwts)}
\hlkwd{dev.off}\hlstd{()}
\end{alltt}
\begin{verbatim}
## pdf 
##   2
\end{verbatim}
\end{kframe}
\end{knitrout}

Raster devices return bitmaps and \code{width} and \code{height} are specified in \emph{pixels}.

\begin{knitrout}\footnotesize
\definecolor{shadecolor}{rgb}{0.969, 0.969, 0.969}\color{fgcolor}\begin{kframe}
\begin{alltt}
\hlkwd{png}\hlstd{(}\hlkwc{file} \hlstd{=} \hlstr{"output/my-file.png"}\hlstd{,} \hlkwc{width} \hlstd{=} \hlnum{600}\hlstd{,} \hlkwc{height} \hlstd{=} \hlnum{500}\hlstd{)}
\hlkwd{plot}\hlstd{(weight} \hlopt{~} \hlstd{feed,} \hlkwc{data} \hlstd{= chickwts)}
\hlkwd{dev.off}\hlstd{()}
\end{alltt}
\begin{verbatim}
## pdf 
##   2
\end{verbatim}
\end{kframe}
\end{knitrout}

When \Rlang is used interactively, a device to output the graphical output to a display device is open automatically. The name of the device may depend on the operating system used (e.g.\ \osname{MS-Windows} or \osname{Linux}) or an IDE---e.g.\ \RStudio defines its own graphic device for output to the Plots pane of its user interface.

\begin{warningbox}
This approach of direct output to a device, and addition of plot components as show below directly on the output device itself is not the only approach available. As we will see in chapter \ref{chap:R:plotting} starting on page \pageref{chap:R:plotting} an alternative approach is to built a \emph{plot object} as a list of member components that is later rendered as a whole on a graphical device by calling \code{print()} once.

\begin{knitrout}\footnotesize
\definecolor{shadecolor}{rgb}{0.969, 0.969, 0.969}\color{fgcolor}\begin{kframe}
\begin{alltt}
\hlkwd{png}\hlstd{(}\hlkwc{file} \hlstd{=} \hlstr{"output/my-file.png"}\hlstd{,} \hlkwc{width} \hlstd{=} \hlnum{600}\hlstd{,} \hlkwc{height} \hlstd{=} \hlnum{500}\hlstd{)}
\hlkwd{plot}\hlstd{(dist} \hlopt{~} \hlstd{speed,} \hlkwc{data} \hlstd{= cars)}
\hlkwd{text}\hlstd{(}\hlkwc{x} \hlstd{=} \hlnum{10}\hlstd{,} \hlkwc{y} \hlstd{=} \hlnum{110}\hlstd{,} \hlkwc{labels} \hlstd{=} \hlstr{"some texts to be added"}\hlstd{)}
\hlkwd{dev.off}\hlstd{()}
\end{alltt}
\begin{verbatim}
## pdf 
##   2
\end{verbatim}
\end{kframe}
\end{knitrout}
\end{warningbox}



\index{data!exploration at the R console|)}
















\chapter{Further reading about R}\label{chap:R:readings}

\begin{VF}
Before you become too entranced with gorgeous gadgets and mesmerizing video displays, let me remind you that information is not knowledge, knowledge is not wisdom, and wisdom is not foresight. Each grows out of the other, and we need them all.

\VA{Arthur C. Clarke}{Official website at \url{http://arthurcclarke.org}}
\end{VF}

%\dictum[Arthur C. Clarke]{Before you become too entranced with gorgeous gadgets and mesmerizing video displays, let me remind you that information is not knowledge, knowledge is not wisdom, and wisdom is not foresight. Each grows out of the other, and we need them all.}\vskip2ex

\begin{warningbox}
  This list will be expanded and more importantly reorganized and short comments added for book or group of books.
\end{warningbox}

\section{Introductory texts}

\cite{Allerhand2011,Dalgaard2008,Zuur2009,Teetor2011,Peng2017,Paradis2005,Peng2016}

\section{Texts on specific aspects}

\cite{Chang2013,Fox2002,Fox2010,Faraway2004,Faraway2006,Everitt2011,Wickham2017}

\section{Advanced texts}

\cite{Xie2013,Chambers2016,Wickham2015,Wickham2014advanced,Wickham2016,Pinheiro2000,Murrell2011,Matloff2011,Ihaka1996,Venables2000}

\section{Texts for S/R wisdom}

\cite{Burns1998,Burns2011,Burns2012,Bentley1986,Bentley1988}

\backmatter

\printbibliography

\printindex

\printindex[rcatsidx]

\printindex[rindex]

\end{document}

\appendix

\chapter{Build information}

\begin{knitrout}\footnotesize
\definecolor{shadecolor}{rgb}{0.969, 0.969, 0.969}\color{fgcolor}\begin{kframe}
\begin{alltt}
\hlkwd{Sys.info}\hlstd{()}
\end{alltt}
\begin{verbatim}
##        sysname        release        version       nodename        machine 
##      "Windows"       "10 x64"  "build 18362"        "MUSTI"       "x86-64" 
##          login           user effective_user 
##       "aphalo"       "aphalo"       "aphalo"
\end{verbatim}
\end{kframe}
\end{knitrout}



\begin{knitrout}\footnotesize
\definecolor{shadecolor}{rgb}{0.969, 0.969, 0.969}\color{fgcolor}\begin{kframe}
\begin{alltt}
\hlkwd{sessionInfo}\hlstd{()}
\end{alltt}
\begin{verbatim}
## R version 3.6.1 (2019-07-05)
## Platform: x86_64-w64-mingw32/x64 (64-bit)
## Running under: Windows 10 x64 (build 18362)
## 
## Matrix products: default
## 
## locale:
## [1] LC_COLLATE=English_United Kingdom.1252 
## [2] LC_CTYPE=English_United Kingdom.1252   
## [3] LC_MONETARY=English_United Kingdom.1252
## [4] LC_NUMERIC=C                           
## [5] LC_TIME=English_United Kingdom.1252    
## 
## attached base packages:
## [1] tools     stats     graphics  grDevices utils     datasets  methods  
## [8] base     
## 
## other attached packages:
## [1] svglite_1.2.2 stringr_1.4.0 knitr_1.23   
## 
## loaded via a namespace (and not attached):
## [1] compiler_3.6.1 magrittr_1.5   Rcpp_1.0.1     gdtools_0.1.9 
## [5] stringi_1.4.3  highr_0.8      xfun_0.8       evaluate_0.14
\end{verbatim}
\end{kframe}
\end{knitrout}

%

\end{document}


